%%%%%%%%%%%%%%%%%%%%% chapter.tex %%%%%%%%%%%%%%%%%%%%%%%%%%%%%%%%%
%
% sample chapter
%
% Use this file as a template for your own input.
%
%%%%%%%%%%%%%%%%%%%%%%%% Springer-Verlag %%%%%%%%%%%%%%%%%%%%%%%%%%
%\motto{Use the template \emph{chapter.tex} to style the various elements of your chapter content.}

\chapter{Rosetta Code Tasks starting with J}

\section*{JSON}

Load a \href{http://en.wikipedia.org/wiki/JSON}{JSON} string into a data
structure. Also create a new data structure and serialize it into JSON.
Use objects and arrays (as appropriate for your language) and make sure
your JSON is valid
(\href{http://www.jsonlint.com/}{http://www.jsonlint.com/}). However,
beware that this web application accepts extra commas, so {[}1,2,3,{]}
validates as correct, which is at odds with the grammar given at
json.org \href{http://www.json.org/fatfree.html}{{[}1{]}}.

\begin{wideverbatim}

PicoLisp has no JSON library, but it is easy enough to write one. The following
supports only fixpoint numbers (no floating point, as it doesn't exist in
PicoLisp), no arrays (uses lists instead), and no unicode escapes (UTF-8 can be
used directly):

(de checkJson (X Item)
   (unless (= X Item)
      (quit "Bad JSON" Item) ) )

(de readJson ()
   (case (read "_")
      ("{"
         (make
            (for (X (readJson) (not (= "}" X)) (readJson))
               (checkJson ":" (readJson))
               (link (cons X (readJson)))
               (T (= "}" (setq X (readJson))))
               (checkJson "," X) ) ) )
      ("["
         (make
            (link T)  # Array marker
            (for (X (readJson) (not (= "]" X)) (readJson))
               (link X)
               (T (= "]" (setq X (readJson))))
               (checkJson "," X) ) ) )
      (T
         (let X @
            (if (and (= "-" X) (format (peek)))
               (- (read))
               X ) ) ) ) )

(de printJson (Item)  # For simplicity, without indentation
   (cond
      ((atom Item) (if Item (print @) (prin "{}")))
      ((=T (car Item))
         (prin "[")
         (map
            '((X)
               (printJson (car X))
               (and (cdr X) (prin ", ")) )
            (cdr Item) )
         (prin "]") )
      (T
         (prin "{")
         (map
            '((X)
               (print (caar X))
               (prin ": ")
               (printJson (cdar X))
               (and (cdr X) (prin ", ")) )
            Item )
         (prin "}") ) ) )

\end{wideverbatim}

\begin{wideverbatim}

This reads/prints JSON from/to files, pipes, sockets etc. To read from a string,
a pipe can be used:

: (pipe (prinl "{ \"foo\": 1, \"bar\": [10, \"apples\"] }")
   (readJson) )
-> (("foo" . 1) ("bar" T 10 "apples"))

: (printJson
   (quote
      ("name" . "Smith")
      ("age" . 25)
      ("address"
         ("street" . "21 2nd Street")
         ("city" . "New York")
         ("state" . "NY")
         ("zip" . "10021") )
      ("phone" T "212 555-1234" "646 555-4567") ) )
{"name": "Smith", "age": 25, ... {"street": ... "phone": ["212 555-1234", ...

\end{wideverbatim}

\pagebreak{}
\section*{Jensen's Device}

This task is an exercise in
\href{http://en.wikipedia.org/wiki/Call-by-name\#Call\_by\_name}{call
  by name}.

\textbf{Jensen's Device} is a computer programming technique devised
by Danish computer scientist
\href{http://en.wikipedia.org/wiki/J\%C3\%B8rn\_Jensen}{Jørn Jensen}
after studying the \emph{ALGOL 60} Report.

The following program was proposed to illustrate the technique. It
computes the 100th
\href{http://en.wikipedia.org/wiki/Harmonic\_number}{harmonic number}:

\begin{wideverbatim}
begin
   integer i;
   real procedure sum (i, lo, hi, term);
      value lo, hi;
      integer i, lo, hi;
      real term;
      comment term is passed by-name, and so is i;
   begin
      real temp;
      temp := 0;
      for i := lo step 1 until hi do
         temp := temp + term;
      sum := temp
   end;
   comment note the correspondence between 
           the mathematical notation and the call to sum;
   print (sum (i, 1, 100, 1/i))
end
\end{wideverbatim}

The above exploits
\href{http://en.wikipedia.org/wiki/Call-by-name\#Call\_by\_name}{call
  by name} to produce the correct answer (5.187\ldots{}). It depends
on the assumption that an expression passed as an actual parameter to
a procedure would be re-evaluated every time the corresponding formal
parameter's value was required. If the last parameter to \emph{sum}
had been passed by value, and assuming the initial value of \emph{i}
were 1, the result would have been 100 × 1/1 = 100.

Moreover, the \emph{first} parameter to \emph{sum}, representing the
``bound'' variable of the summation, must also be passed by name,
otherwise it would not be possible to compute the values to be added.
(On the other hand, the global variable does not have to use the same
identifier, in this case \emph{i}, as the formal parameter.)

\href{http://en.wikipedia.org/wiki/Donald\_Knuth}{Donald Knuth} later
proposed the \emph{Man or Boy Test} as a more
rigorous exercise.


\begin{wideverbatim}

(scl 6)

(de jensen (I Lo Hi Term)
   (let Temp 0
      (set I Lo)
      (while (>= Hi (val I))
         (inc 'Temp (Term))
         (inc I) )
      Temp ) )

(let I (box)  # Create indirect reference
   (format
      (jensen I 1 100 '(() (*/ 1.0 (val I))))
      *Scl ) )

Output:

-> "5.187383"

\end{wideverbatim}

\pagebreak{}
\section*{Joystick position}

The task is to determine the joystick position and represent this on
the display via a crosshair. For a centred joystick, the crosshair
should appear in the centre of the screen. If the joystick is pushed
left or right, then the cross hair should move left or right according
to the extent that the joystick is pushed. If the joystick is pushed
forward or pulled back, then the crosshair should move up or down
according to the extent that that joystick is pushed or pulled. The
edges of the display represent maximum extents for joystick movement.
For example, a joystick pushed fully forward would raise the crosshair
to the top centre of the screen. A joystick pulled backwards and to
the right would move the crosshair to the bottom right of the screen
(except for a small area reserved to show joystick status).
Implementations can use a graphical display method to produce the
crosshair, or alternatively represent the crosshair using a plus
symbol on a terminal, and move the plus symbol position according to
the joystick. The bottom part of the display can hide or show an
alphanumeric sequence to represent the buttons pressed. For example,
if pushbuttons 1,4 and 10 are depressed, we could display ``1 4 A''.
The implemented code should continue to redraw the crosshair according
to the joystick position and show the current pushbutton statuses
until the task is terminated. Digital joysticks that produce no extent
data, should have their position indicated as full extent movement of
the crosshair.

For the purpose of this task, we assume that the joystick is calibrated
and that the first joystick is being used. The task implementer could at
their option provide a solution that includes a joystick selection
facility, enabling the user to choose which joystick is to be used for
this task.


\begin{wideverbatim}

This is for the 64-bit version.

Note: The code is not yet tested with a real joystick (I don't have one), it was
just simulated with dummy functions. Can somebody having a joystick please test
it, and remove this message?

(load "@lib/openGl.l")

(setq *JoyX 0.0  *JoyY 0.0)

(glutInit)
(glutInitDisplayMode (| GLUT_RGBA GLUT_DOUBLE GLUT_ALPHA GLUT_DEPTH))
(glutInitWindowSize 400 400)
(glutCreateWindow "Joystick")

(glClearColor 0.3 0.3 0.5 0)

(displayPrg
   (glClear GL_COLOR_BUFFER_BIT)
   (glBegin GL_LINES)
   (glVertex2f *JoyX (- *JoyY 0.1))  # Draw crosshair
   (glVertex2f *JoyX (+ *JoyY 0.1))
   (glVertex2f (- *JoyX 0.1) *JoyY)
   (glVertex2f (+ *JoyX 0.1) *JoyY)
   (glEnd)
   (glFlush)
   (glutSwapBuffers) )

# Track joystick position
(native `*GlutLib "glutJoystickFunc" NIL
   (lisp 'joystickFunc
      '((Btn X Y Z)
         (msg                          # Display buttons
            (make
               (for (B 1 (n0 Btn) (inc B))
                  (and (bit? 1 Btn) (link B))
                  (setq Btn (>> 1 Btn)) ) ) )
         (setq                         # Move crosshair
            *JoyX (*/ X 1.0 1000)
            *JoyY (*/ Y 1.0 1000) )
         (glutPostRedisplay) ) )
   100 )

# Exit upon mouse click
(mouseFunc '((Btn State X Y) (bye)))
(glutMainLoop)

\end{wideverbatim}

\pagebreak{}
\section*{Jump anywhere}

\emph{Imperative programs} like to jump around, but some languages
restrict these jumps. Many structured languages restrict their
\emph{conditional structures} and \emph{loops} to \emph{local jumps}
within a function. Some assembly languages limit certain jumps or
branches to a small range.

This task is demonstrate a local jump and a global jump and the
various other types of jumps that the language supports. For the
purpose of this task, the jumps need not be used for a single purpose
and you have the freedom to use these jumps for different purposes.
You may also defer to more specific tasks, like \emph{Exceptions} or
\emph{Generator}. This task provides a ``grab bag'' for several types
of jumps. There are \emph{non-local jumps} across function calls, or
\emph{long jumps} to anywhere within a program. Anywhere means not
only to the tops of functions!

\begin{itemize}
\item
  Some languages can \emph{go to} any global label in a program.
\item
  Some languages can break multiple function calls, also known as
  \emph{unwinding the call stack}.
\item
  Some languages can save a \emph{continuation}. The program can later
  continue from the same place. So you can jump anywhere, but only if
  you have a previous visit there (to save the continuation).
\end{itemize}

These jumps are not all alike. A simple \emph{goto} never touches the
call stack. A continuation saves the call stack, so you can continue a
function call after it ends.

Use your language to demonstrate the various types of jumps that it
supports. Because the possibilities vary by language, this task is not
specific. You have the freedom to use these jumps for different
purposes. You may also defer to more specific tasks, like
\emph{Exceptions} or \emph{Generator}.

\begin{wideverbatim}

PicoLisp supports non-local jumps to a previously setup environment (see
[[Exceptions#PicoLisp|exceptions]]) via
'[http://software-lab.de/doc/refC.html#catch catch]' and
'[http://software-lab.de/doc/refT.html#throw throw]', or to some location in
another coroutine with '[http://software-lab.de/doc/refY.html#yield yield]' (see
[[Generator#PicoLisp|generator]]).

'[http://software-lab.de/doc/refQ.html#quit quit]' is similar to 'throw', but
doesn't require a corresponding 'catch', as it directly jumps to the error
handler (where the program may catch that error again).

There is no 'go' or 'goto' function in PicoLisp, but it can be emulated with
normal list processing functions. This allows "jumps" to arbitrary locations
within (the same or other) functions. The following example implements a "loop":

(de foo (N)
   (prinl "This is 'foo'")
   (printsp N)
   (or (=0 (dec 'N)) (run (cddr foo))) )

Test:

: (foo 7)
This is 'foo'
7 6 5 4 3 2 1 -> 0

\end{wideverbatim}



% %%%%%%%%%%%%%%%%%%%%%%%% referenc.tex %%%%%%%%%%%%%%%%%%%%%%%%%%%%%%
% sample references
% %
% Use this file as a template for your own input.
%
%%%%%%%%%%%%%%%%%%%%%%%% Springer-Verlag %%%%%%%%%%%%%%%%%%%%%%%%%%
%
% BibTeX users please use
% \bibliographystyle{}
% \bibliography{}
%
\biblstarthook{In view of the parallel print and (chapter-wise) online publication of your book at \url{www.springerlink.com} it has been decided that -- as a genreral rule --  references should be sorted chapter-wise and placed at the end of the individual chapters. However, upon agreement with your contact at Springer you may list your references in a single seperate chapter at the end of your book. Deactivate the class option \texttt{sectrefs} and the \texttt{thebibliography} environment will be put out as a chapter of its own.\\\indent
References may be \textit{cited} in the text either by number (preferred) or by author/year.\footnote{Make sure that all references from the list are cited in the text. Those not cited should be moved to a separate \textit{Further Reading} section or chapter.} The reference list should ideally be \textit{sorted} in alphabetical order -- even if reference numbers are used for the their citation in the text. If there are several works by the same author, the following order should be used: 
\begin{enumerate}
\item all works by the author alone, ordered chronologically by year of publication
\item all works by the author with a coauthor, ordered alphabetically by coauthor
\item all works by the author with several coauthors, ordered chronologically by year of publication.
\end{enumerate}
The \textit{styling} of references\footnote{Always use the standard abbreviation of a journal's name according to the ISSN \textit{List of Title Word Abbreviations}, see \url{http://www.issn.org/en/node/344}} depends on the subject of your book:
\begin{itemize}
\item The \textit{two} recommended styles for references in books on \textit{mathematical, physical, statistical and computer sciences} are depicted in ~\cite{science-contrib, science-online, science-mono, science-journal, science-DOI} and ~\cite{phys-online, phys-mono, phys-journal, phys-DOI, phys-contrib}.
\item Examples of the most commonly used reference style in books on \textit{Psychology, Social Sciences} are~\cite{psysoc-mono, psysoc-online,psysoc-journal, psysoc-contrib, psysoc-DOI}.
\item Examples for references in books on \textit{Humanities, Linguistics, Philosophy} are~\cite{humlinphil-journal, humlinphil-contrib, humlinphil-mono, humlinphil-online, humlinphil-DOI}.
\item Examples of the basic Springer style used in publications on a wide range of subjects such as \textit{Computer Science, Economics, Engineering, Geosciences, Life Sciences, Medicine, Biomedicine} are ~\cite{basic-contrib, basic-online, basic-journal, basic-DOI, basic-mono}. 
\end{itemize}
}

\begin{thebibliography}{99.}%
% and use \bibitem to create references.
%
% Use the following syntax and markup for your references if 
% the subject of your book is from the field 
% "Mathematics, Physics, Statistics, Computer Science"
%
% Contribution 
\bibitem{science-contrib} Broy, M.: Software engineering --- from auxiliary to key technologies. In: Broy, M., Dener, E. (eds.) Software Pioneers, pp. 10-13. Springer, Heidelberg (2002)
%
% Online Document
\bibitem{science-online} Dod, J.: Effective substances. In: The Dictionary of Substances and Their Effects. Royal Society of Chemistry (1999) Available via DIALOG. \\
\url{http://www.rsc.org/dose/title of subordinate document. Cited 15 Jan 1999}
%
% Monograph
\bibitem{science-mono} Geddes, K.O., Czapor, S.R., Labahn, G.: Algorithms for Computer Algebra. Kluwer, Boston (1992) 
%
% Journal article
\bibitem{science-journal} Hamburger, C.: Quasimonotonicity, regularity and duality for nonlinear systems of partial differential equations. Ann. Mat. Pura. Appl. \textbf{169}, 321--354 (1995)
%
% Journal article by DOI
\bibitem{science-DOI} Slifka, M.K., Whitton, J.L.: Clinical implications of dysregulated cytokine production. J. Mol. Med. (2000) doi: 10.1007/s001090000086 
%
\bigskip

% Use the following (APS) syntax and markup for your references if 
% the subject of your book is from the field 
% "Mathematics, Physics, Statistics, Computer Science"
%
% Online Document
\bibitem{phys-online} J. Dod, in \textit{The Dictionary of Substances and Their Effects}, Royal Society of Chemistry. (Available via DIALOG, 1999), 
\url{http://www.rsc.org/dose/title of subordinate document. Cited 15 Jan 1999}
%
% Monograph
\bibitem{phys-mono} H. Ibach, H. L\"uth, \textit{Solid-State Physics}, 2nd edn. (Springer, New York, 1996), pp. 45-56 
%
% Journal article
\bibitem{phys-journal} S. Preuss, A. Demchuk Jr., M. Stuke, Appl. Phys. A \textbf{61}
%
% Journal article by DOI
\bibitem{phys-DOI} M.K. Slifka, J.L. Whitton, J. Mol. Med., doi: 10.1007/s001090000086
%
% Contribution 
\bibitem{phys-contrib} S.E. Smith, in \textit{Neuromuscular Junction}, ed. by E. Zaimis. Handbook of Experimental Pharmacology, vol 42 (Springer, Heidelberg, 1976), p. 593
%
\bigskip
%
% Use the following syntax and markup for your references if 
% the subject of your book is from the field 
% "Psychology, Social Sciences"
%
%
% Monograph
\bibitem{psysoc-mono} Calfee, R.~C., \& Valencia, R.~R. (1991). \textit{APA guide to preparing manuscripts for journal publication.} Washington, DC: American Psychological Association.
%
% Online Document
\bibitem{psysoc-online} Dod, J. (1999). Effective substances. In: The dictionary of substances and their effects. Royal Society of Chemistry. Available via DIALOG. \\
\url{http://www.rsc.org/dose/Effective substances.} Cited 15 Jan 1999.
%
% Journal article
\bibitem{psysoc-journal} Harris, M., Karper, E., Stacks, G., Hoffman, D., DeNiro, R., Cruz, P., et al. (2001). Writing labs and the Hollywood connection. \textit{J Film} Writing, 44(3), 213--245.
%
% Contribution 
\bibitem{psysoc-contrib} O'Neil, J.~M., \& Egan, J. (1992). Men's and women's gender role journeys: Metaphor for healing, transition, and transformation. In B.~R. Wainrig (Ed.), \textit{Gender issues across the life cycle} (pp. 107--123). New York: Springer.
%
% Journal article by DOI
\bibitem{psysoc-DOI}Kreger, M., Brindis, C.D., Manuel, D.M., Sassoubre, L. (2007). Lessons learned in systems change initiatives: benchmarks and indicators. \textit{American Journal of Community Psychology}, doi: 10.1007/s10464-007-9108-14.
%
%
% Use the following syntax and markup for your references if 
% the subject of your book is from the field 
% "Humanities, Linguistics, Philosophy"
%
\bigskip
%
% Journal article
\bibitem{humlinphil-journal} Alber John, Daniel C. O'Connell, and Sabine Kowal. 2002. Personal perspective in TV interviews. \textit{Pragmatics} 12:257--271
%
% Contribution 
\bibitem{humlinphil-contrib} Cameron, Deborah. 1997. Theoretical debates in feminist linguistics: Questions of sex and gender. In \textit{Gender and discourse}, ed. Ruth Wodak, 99--119. London: Sage Publications.
%
% Monograph
\bibitem{humlinphil-mono} Cameron, Deborah. 1985. \textit{Feminism and linguistic theory.} New York: St. Martin's Press.
%
% Online Document
\bibitem{humlinphil-online} Dod, Jake. 1999. Effective substances. In: The dictionary of substances and their effects. Royal Society of Chemistry. Available via DIALOG. \\
http://www.rsc.org/dose/title of subordinate document. Cited 15 Jan 1999
%
% Journal article by DOI
\bibitem{humlinphil-DOI} Suleiman, Camelia, Daniel C. O�Connell, and Sabine Kowal. 2002. `If you and I, if we, in this later day, lose that sacred fire...�': Perspective in political interviews. \textit{Journal of Psycholinguistic Research}. doi: 10.1023/A:1015592129296.
%
%
%
\bigskip
%
%
% Use the following syntax and markup for your references if 
% the subject of your book is from the field 
% "Computer Science, Economics, Engineering, Geosciences, Life Sciences"
%
%
% Contribution 
\bibitem{basic-contrib} Brown B, Aaron M (2001) The politics of nature. In: Smith J (ed) The rise of modern genomics, 3rd edn. Wiley, New York 
%
% Online Document
\bibitem{basic-online} Dod J (1999) Effective Substances. In: The dictionary of substances and their effects. Royal Society of Chemistry. Available via DIALOG. \\
\url{http://www.rsc.org/dose/title of subordinate document. Cited 15 Jan 1999}
%
% Journal article by DOI
\bibitem{basic-DOI} Slifka MK, Whitton JL (2000) Clinical implications of dysregulated cytokine production. J Mol Med, doi: 10.1007/s001090000086
%
% Journal article
\bibitem{basic-journal} Smith J, Jones M Jr, Houghton L et al (1999) Future of health insurance. N Engl J Med 965:325--329
%
% Monograph
\bibitem{basic-mono} South J, Blass B (2001) The future of modern genomics. Blackwell, London 
%
\end{thebibliography}

