%%%%%%%%%%%%%%%%%%%%% chapter.tex %%%%%%%%%%%%%%%%%%%%%%%%%%%%%%%%%
%
% sample chapter
%
% Use this file as a template for your own input.
%
%%%%%%%%%%%%%%%%%%%%%%%% Springer-Verlag %%%%%%%%%%%%%%%%%%%%%%%%%%
%\motto{Use the template \emph{chapter.tex} to style the various elements of your chapter content.}



\chapter{Symbols starting with G}
\label{sec:func-ref-G-}


 
\section*{\texttt{(gc ['cnt]) -> cnt | NIL}}
\label{sec:func-ref-G-(gc ['cnt]) -> cnt | NIL}


Forces a garbage collection. When \texttt{cnt} is given, so many megabytes of
free cells are reserved, increasing the heap size if necessary. If \texttt{cnt}
is zero, all currently unused heap blocks are purged, decreasing the
heap size if possible. See also \texttt{heap}.


\begin{wideverbatim}
: (gc)
-> NIL
: (heap)
-> 2
: (gc 4)
-> 4
: (heap)
-> 5
\end{wideverbatim}

 
\section*{\texttt{(ge0 'any) -> num | NIL}}
\label{sec:func-ref-G-(ge0 'any) -> num | NIL}


Returns \texttt{num} when the argument is a number and greater or equal zero,
otherwise \texttt{NIL}. See also \texttt{lt0}, \texttt{le0}, \texttt{gt0}, \texttt{=0} and \texttt{n0}.


\begin{wideverbatim}
: (ge0 -2)
-> NIL
: (ge0 3)
-> 3
: (ge0 0)
-> 0
\end{wideverbatim}

 
\section*{\texttt{(genKey 'var 'cls ['hook ['num1 ['num2]]]) -> num}}
\label{sec:func-ref-G-(genKey 'var 'cls ['hook ['num1 ['num2]]]) -> num}


Generates a key for a database tree. If a minimal key \texttt{num1} and/or a
maximal key \texttt{num2} is given, the next free number in that range is
returned. Otherwise, the current maximal key plus one is returned. See
also \texttt{useKey}, \texttt{genStrKey} and \texttt{maxKey}.


\begin{wideverbatim}
: (maxKey (tree 'nr '+Item))
-> 8
: (genKey 'nr '+Item)
-> 9
\end{wideverbatim}

 
\section*{\texttt{(genStrKey 'sym 'var 'cls ['hook]) -> sym}}
\label{sec:func-ref-G-(genStrKey 'sym 'var 'cls ['hook]) -> sym}


Generates a unique string for a database tree, by prepending as many ``\#''
sequences as necessary. See also \texttt{genKey}.


\begin{wideverbatim}
: (genStrKey "ben" 'nm '+User)
-> "# ben"
\end{wideverbatim}

 
\section*{\texttt{(get 'sym1|lst ['sym2|cnt ..]) -> any}}
\label{sec:func-ref-G-(get 'sym1|lst ['sym2|cnt ..]) -> any}


Fetches a value \texttt{any} from the properties of a symbol, or from a list.
From the first argument \texttt{sym1|lst}, values are retrieved in successive
steps by either extracting the value (if the next argument is zero) or a
property from a symbol, the \texttt{asoq}ed element (if the next argument is a
symbol), the n'th element (if the next argument is a positive number) or
the n'th CDR (if the next argument is a negative number) from a list.
See also \texttt{put}, \texttt{;} and \texttt{:}.


\begin{wideverbatim}
: (put 'X 'a 1)
-> 1
: (get 'X 'a)
-> 1
: (put 'Y 'link 'X)
-> X
: (get 'Y 'link)
-> X
: (get 'Y 'link 'a)
-> 1
: (get '((a (b . 1) (c . 2)) (d (e . 3) (f . 4))) 'a 'b)
-> 1
: (get '((a (b . 1) (c . 2)) (d (e . 3) (f . 4))) 'd 'f)
-> 4
: (get '(X Y Z) 2)
-> Y
: (get '(X Y Z) 2 'link 'a)
-> 1
\end{wideverbatim}

 
\section*{\texttt{(getd 'any) -> fun | NIL}}
\label{sec:func-ref-G-(getd 'any) -> fun | NIL}


Returns \texttt{fun} if \texttt{any} is a symbol that has a function definition,
otherwise \texttt{NIL}. See also \texttt{fun?}.


\begin{wideverbatim}
: (getd '+)
-> 67327232
: (getd 'script)
-> ((File . @) (load File))
: (getd 1)
-> NIL
\end{wideverbatim}

 
\section*{\texttt{(getl 'sym1|lst1 ['sym2|cnt ..]) -> lst}}
\label{sec:func-ref-G-(getl 'sym1|lst1 ['sym2|cnt ..]) -> lst}


Fetches the complete property list \texttt{lst} from a symbol. That symbol is
\texttt{sym1} (if no other arguments are given), or a symbol found by applying
the \texttt{get} algorithm to \texttt{sym1|lst1} and the following arguments. See also
\texttt{putl} and \texttt{maps}.


\begin{wideverbatim}
: (put 'X 'a 1)
-> 1
: (put 'X 'b 2)
-> 2
: (put 'X 'flg T)
-> T
: (getl 'X)
-> (flg (2 . b) (1 . a))
\end{wideverbatim}

 
\section*{\texttt{(glue 'any 'lst) -> sym}}
\label{sec:func-ref-G-(glue 'any 'lst) -> sym}


Builds a new transient symbol (string) by \texttt{pack}ing the
\texttt{any} argument between the individual elements of \texttt{lst}.
See also \texttt{text}.


\begin{wideverbatim}
: (glue "," '(a b c d))
-> "a,b,c,d"
\end{wideverbatim}

 
\section*{\texttt{(goal '([pat 'any ..] . lst) ['sym 'any ..]) -> lst}}
\label{sec:func-ref-G-(goal '([pat 'any ..] . lst) ['sym 'any ..]) -> lst}


Constructs a \emph{Pilog} query list from the list of
clauses \texttt{lst}. The head of the argument list may consist of a sequence
of pattern symbols (Pilog variables) and expressions, which are used
together with the optional \texttt{sym} and \texttt{any} arguments to form an initial
environment. See also \texttt{prove} and \texttt{fail}.


\begin{wideverbatim}
: (goal '((likes John @X)))
-> (((1 (0) NIL ((likes John @X)) NIL T)))
: (goal '(@X 'John (likes @X @Y)))
-> (((1 (0) NIL ((likes @X @Y)) NIL ((0 . @X) 1 . John) T)))
\end{wideverbatim}

 
\section*{\texttt{(group 'lst) -> lst}}
\label{sec:func-ref-G-(group 'lst) -> lst}


Builds a list of lists, by grouping all elements of \texttt{lst} with the same
CAR into a common sublist. See also \emph{Comparing}, \texttt{by},
\texttt{sort} and \texttt{uniq}.


\begin{wideverbatim}
: (group '((1 . a) (1 . b) (1 . c) (2 . d) (2 . e) (2 . f)))
-> ((1 a b c) (2 d e f))
: (by name group '("x" "x" "y" "z" "x" "z")))
-> (("x" "x" "x") ("y") ("z" "z"))
: (by length group '(123 (1 2) "abcd" "xyz" (1 2 3 4) "XY"))
-> ((123 "xyz") ((1 2) "XY") ("abcd" (1 2 3 4))
\end{wideverbatim}

 
\section*{\texttt{(gt0 'any) -> num | NIL}}
\label{sec:func-ref-G-(gt0 'any) -> num | NIL}


Returns \texttt{num} when the argument is a number and greater than
zero, otherwise \texttt{NIL}. See also \texttt{lt0}, \texttt{le0},
\texttt{ge0}, \texttt{=0} and \texttt{n0}.


\begin{wideverbatim}
: (gt0 -2)
-> NIL
: (gt0 3)
-> 3
\end{wideverbatim}




% %%%%%%%%%%%%%%%%%%%%%%%% referenc.tex %%%%%%%%%%%%%%%%%%%%%%%%%%%%%%
% sample references
% %
% Use this file as a template for your own input.
%
%%%%%%%%%%%%%%%%%%%%%%%% Springer-Verlag %%%%%%%%%%%%%%%%%%%%%%%%%%
%
% BibTeX users please use
% \bibliographystyle{}
% \bibliography{}
%
\biblstarthook{In view of the parallel print and (chapter-wise) online publication of your book at \url{www.springerlink.com} it has been decided that -- as a genreral rule --  references should be sorted chapter-wise and placed at the end of the individual chapters. However, upon agreement with your contact at Springer you may list your references in a single seperate chapter at the end of your book. Deactivate the class option \texttt{sectrefs} and the \texttt{thebibliography} environment will be put out as a chapter of its own.\\\indent
References may be \textit{cited} in the text either by number (preferred) or by author/year.\footnote{Make sure that all references from the list are cited in the text. Those not cited should be moved to a separate \textit{Further Reading} section or chapter.} The reference list should ideally be \textit{sorted} in alphabetical order -- even if reference numbers are used for the their citation in the text. If there are several works by the same author, the following order should be used: 
\begin{enumerate}
\item all works by the author alone, ordered chronologically by year of publication
\item all works by the author with a coauthor, ordered alphabetically by coauthor
\item all works by the author with several coauthors, ordered chronologically by year of publication.
\end{enumerate}
The \textit{styling} of references\footnote{Always use the standard abbreviation of a journal's name according to the ISSN \textit{List of Title Word Abbreviations}, see \url{http://www.issn.org/en/node/344}} depends on the subject of your book:
\begin{itemize}
\item The \textit{two} recommended styles for references in books on \textit{mathematical, physical, statistical and computer sciences} are depicted in ~\cite{science-contrib, science-online, science-mono, science-journal, science-DOI} and ~\cite{phys-online, phys-mono, phys-journal, phys-DOI, phys-contrib}.
\item Examples of the most commonly used reference style in books on \textit{Psychology, Social Sciences} are~\cite{psysoc-mono, psysoc-online,psysoc-journal, psysoc-contrib, psysoc-DOI}.
\item Examples for references in books on \textit{Humanities, Linguistics, Philosophy} are~\cite{humlinphil-journal, humlinphil-contrib, humlinphil-mono, humlinphil-online, humlinphil-DOI}.
\item Examples of the basic Springer style used in publications on a wide range of subjects such as \textit{Computer Science, Economics, Engineering, Geosciences, Life Sciences, Medicine, Biomedicine} are ~\cite{basic-contrib, basic-online, basic-journal, basic-DOI, basic-mono}. 
\end{itemize}
}

\begin{thebibliography}{99.}%
% and use \bibitem to create references.
%
% Use the following syntax and markup for your references if 
% the subject of your book is from the field 
% "Mathematics, Physics, Statistics, Computer Science"
%
% Contribution 
\bibitem{science-contrib} Broy, M.: Software engineering --- from auxiliary to key technologies. In: Broy, M., Dener, E. (eds.) Software Pioneers, pp. 10-13. Springer, Heidelberg (2002)
%
% Online Document
\bibitem{science-online} Dod, J.: Effective substances. In: The Dictionary of Substances and Their Effects. Royal Society of Chemistry (1999) Available via DIALOG. \\
\url{http://www.rsc.org/dose/title of subordinate document. Cited 15 Jan 1999}
%
% Monograph
\bibitem{science-mono} Geddes, K.O., Czapor, S.R., Labahn, G.: Algorithms for Computer Algebra. Kluwer, Boston (1992) 
%
% Journal article
\bibitem{science-journal} Hamburger, C.: Quasimonotonicity, regularity and duality for nonlinear systems of partial differential equations. Ann. Mat. Pura. Appl. \textbf{169}, 321--354 (1995)
%
% Journal article by DOI
\bibitem{science-DOI} Slifka, M.K., Whitton, J.L.: Clinical implications of dysregulated cytokine production. J. Mol. Med. (2000) doi: 10.1007/s001090000086 
%
\bigskip

% Use the following (APS) syntax and markup for your references if 
% the subject of your book is from the field 
% "Mathematics, Physics, Statistics, Computer Science"
%
% Online Document
\bibitem{phys-online} J. Dod, in \textit{The Dictionary of Substances and Their Effects}, Royal Society of Chemistry. (Available via DIALOG, 1999), 
\url{http://www.rsc.org/dose/title of subordinate document. Cited 15 Jan 1999}
%
% Monograph
\bibitem{phys-mono} H. Ibach, H. L\"uth, \textit{Solid-State Physics}, 2nd edn. (Springer, New York, 1996), pp. 45-56 
%
% Journal article
\bibitem{phys-journal} S. Preuss, A. Demchuk Jr., M. Stuke, Appl. Phys. A \textbf{61}
%
% Journal article by DOI
\bibitem{phys-DOI} M.K. Slifka, J.L. Whitton, J. Mol. Med., doi: 10.1007/s001090000086
%
% Contribution 
\bibitem{phys-contrib} S.E. Smith, in \textit{Neuromuscular Junction}, ed. by E. Zaimis. Handbook of Experimental Pharmacology, vol 42 (Springer, Heidelberg, 1976), p. 593
%
\bigskip
%
% Use the following syntax and markup for your references if 
% the subject of your book is from the field 
% "Psychology, Social Sciences"
%
%
% Monograph
\bibitem{psysoc-mono} Calfee, R.~C., \& Valencia, R.~R. (1991). \textit{APA guide to preparing manuscripts for journal publication.} Washington, DC: American Psychological Association.
%
% Online Document
\bibitem{psysoc-online} Dod, J. (1999). Effective substances. In: The dictionary of substances and their effects. Royal Society of Chemistry. Available via DIALOG. \\
\url{http://www.rsc.org/dose/Effective substances.} Cited 15 Jan 1999.
%
% Journal article
\bibitem{psysoc-journal} Harris, M., Karper, E., Stacks, G., Hoffman, D., DeNiro, R., Cruz, P., et al. (2001). Writing labs and the Hollywood connection. \textit{J Film} Writing, 44(3), 213--245.
%
% Contribution 
\bibitem{psysoc-contrib} O'Neil, J.~M., \& Egan, J. (1992). Men's and women's gender role journeys: Metaphor for healing, transition, and transformation. In B.~R. Wainrig (Ed.), \textit{Gender issues across the life cycle} (pp. 107--123). New York: Springer.
%
% Journal article by DOI
\bibitem{psysoc-DOI}Kreger, M., Brindis, C.D., Manuel, D.M., Sassoubre, L. (2007). Lessons learned in systems change initiatives: benchmarks and indicators. \textit{American Journal of Community Psychology}, doi: 10.1007/s10464-007-9108-14.
%
%
% Use the following syntax and markup for your references if 
% the subject of your book is from the field 
% "Humanities, Linguistics, Philosophy"
%
\bigskip
%
% Journal article
\bibitem{humlinphil-journal} Alber John, Daniel C. O'Connell, and Sabine Kowal. 2002. Personal perspective in TV interviews. \textit{Pragmatics} 12:257--271
%
% Contribution 
\bibitem{humlinphil-contrib} Cameron, Deborah. 1997. Theoretical debates in feminist linguistics: Questions of sex and gender. In \textit{Gender and discourse}, ed. Ruth Wodak, 99--119. London: Sage Publications.
%
% Monograph
\bibitem{humlinphil-mono} Cameron, Deborah. 1985. \textit{Feminism and linguistic theory.} New York: St. Martin's Press.
%
% Online Document
\bibitem{humlinphil-online} Dod, Jake. 1999. Effective substances. In: The dictionary of substances and their effects. Royal Society of Chemistry. Available via DIALOG. \\
http://www.rsc.org/dose/title of subordinate document. Cited 15 Jan 1999
%
% Journal article by DOI
\bibitem{humlinphil-DOI} Suleiman, Camelia, Daniel C. O�Connell, and Sabine Kowal. 2002. `If you and I, if we, in this later day, lose that sacred fire...�': Perspective in political interviews. \textit{Journal of Psycholinguistic Research}. doi: 10.1023/A:1015592129296.
%
%
%
\bigskip
%
%
% Use the following syntax and markup for your references if 
% the subject of your book is from the field 
% "Computer Science, Economics, Engineering, Geosciences, Life Sciences"
%
%
% Contribution 
\bibitem{basic-contrib} Brown B, Aaron M (2001) The politics of nature. In: Smith J (ed) The rise of modern genomics, 3rd edn. Wiley, New York 
%
% Online Document
\bibitem{basic-online} Dod J (1999) Effective Substances. In: The dictionary of substances and their effects. Royal Society of Chemistry. Available via DIALOG. \\
\url{http://www.rsc.org/dose/title of subordinate document. Cited 15 Jan 1999}
%
% Journal article by DOI
\bibitem{basic-DOI} Slifka MK, Whitton JL (2000) Clinical implications of dysregulated cytokine production. J Mol Med, doi: 10.1007/s001090000086
%
% Journal article
\bibitem{basic-journal} Smith J, Jones M Jr, Houghton L et al (1999) Future of health insurance. N Engl J Med 965:325--329
%
% Monograph
\bibitem{basic-mono} South J, Blass B (2001) The future of modern genomics. Blackwell, London 
%
\end{thebibliography}

