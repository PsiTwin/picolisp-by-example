%%%%%%%%%%%%%%%%%%%%% chapter.tex %%%%%%%%%%%%%%%%%%%%%%%%%%%%%%%%%
%
% sample chapter
%
% Use this file as a template for your own input.
%
%%%%%%%%%%%%%%%%%%%%%%%% Springer-Verlag %%%%%%%%%%%%%%%%%%%%%%%%%%
%\motto{Use the template \emph{chapter.tex} to style the various elements of your chapter content.}


\chapter{Symbols starting with O}
\label{cha:func-ref-O-functions-starting-with-O}
 
\section*{\texttt{*Once}}
\label{sec:func-ref-O-*Once}


Holds an \texttt{idx} tree of already \texttt{load}ed source locations (as returned by \texttt{file}) See also \texttt{once}.


\begin{wideverbatim}
: *Once
-> (("lib/" "misc.l" . 11) (("lib/" "http.l" . 9) (("lib/" "form.l" . 11))))
\end{wideverbatim}

 
\section*{\texttt{*OS}}
\label{sec:func-ref-O-*OS}


A global constant holding the name of the operating system. Possible
values include \texttt{''Linux''}, \texttt{''FreeBSD''},
\texttt{''Darwin''} or \texttt{''Cygwin''}.


\begin{wideverbatim}
: *OS
-> "Linux"
\end{wideverbatim}

 
\section*{\texttt{(obj (typ var [hook] val ..) var2 val2 ..) -> obj}}
\label{sec:func-ref-O-(obj (typ var [hook] val ..) var2 val2 ..) -> obj}


Finds or creates a database object (using \texttt{request}) corresponding to
\texttt{(typ var [hook] val ..)}, and initializes additional properties using
the \texttt{varN} and \texttt{valN} arguments.


\begin{wideverbatim}
: (obj ((+Item) nr 2) nm "Spare Part" sup `(db 'nr '+CuSu 2) inv 100 pr 1250)
-> {3-2}
\end{wideverbatim}

 
\section*{\texttt{(object 'sym 'any ['sym2 'any2 ..]) -> obj}}
\label{sec:func-ref-O-(object 'sym 'any ['sym2 'any2 ..]) -> obj}


Defines \texttt{sym} to be an object with the value (or type) \texttt{any}. The
property list is initialized with all optionally supplied key-value
pairs. See also \texttt{OO Concepts}, \texttt{new}, \texttt{type} and \texttt{isa}.


\begin{wideverbatim}
: (object 'Obj '(+A +B +C) 'a 1 'b 2 'c 3)
-> Obj
: (show 'Obj)
Obj (+A +B +C)
   c 3
   b 2
   a 1
-> Obj
\end{wideverbatim}

 
\section*{\texttt{(oct 'num ['num]) -> sym}}
\label{sec:func-ref-O-(oct 'num ['num]) -> sym}


\texttt{(oct 'sym) -> num}

Converts a number \texttt{num} to an octal string, or an octal string \texttt{sym} to
a number. In the first case, if the second argument is given, the result
is separated by spaces into groups of such many digits. See also \texttt{bin},
\texttt{hex}, \texttt{fmt64}, \texttt{hax} and \texttt{format}.


\begin{wideverbatim}
: (oct 73)
-> "111"
: (oct "111")
-> 73
: (oct 1234567 3)
-> "4 553 207"
\end{wideverbatim}

 
\section*{\texttt{(off var ..) -> NIL}}
\label{sec:func-ref-O-(off var ..) -> NIL}


Stores \texttt{NIL} in all \texttt{var} arguments. See also \texttt{on}, \texttt{onOff}, \texttt{zero} and
\texttt{one}.


\begin{wideverbatim}
: (off A B)
-> NIL
: A
-> NIL
: B
-> NIL
\end{wideverbatim}

 
\section*{\texttt{(offset 'lst1 'lst2) -> cnt | NIL}}
\label{sec:func-ref-O-(offset 'lst1 'lst2) -> cnt | NIL}


Returns the \texttt{cnt} position of the tail list \texttt{lst1} in \texttt{lst2}, or \texttt{NIL}
if it is not found. See also \texttt{index} and \texttt{tail}.


\begin{wideverbatim}
: (offset '(c d e f) '(a b c d e f))
-> 3
: (offset '(c d e) '(a b c d e f))
-> NIL
\end{wideverbatim}

 
\section*{\texttt{(on var ..) -> T}}
\label{sec:func-ref-O-(on var ..) -> T}


Stores \texttt{T} in all \texttt{var} arguments. See also \texttt{off}, \texttt{onOff}, \texttt{zero} and
\texttt{one}.


\begin{wideverbatim}
: (on A B)
-> T
: A
-> T
: B
-> T
\end{wideverbatim}

 
\section*{\texttt{(once . prg) -> any}}
\label{sec:func-ref-O-(once . prg) -> any}


Executes \texttt{prg} once, when the current file is \texttt{load}ed the first time. Subsequent loads at a later time will not execute \texttt{prg}, and \texttt{once}
returns \texttt{NIL}. See also \texttt{*Once}.


\begin{wideverbatim}
(once
   (zero *Cnt1 *Cnt2)  # Init counters
   (load "file1.l" "file2.l") )  # Load other files
\end{wideverbatim}

 
\section*{\texttt{(one var ..) -> 1}}
\label{sec:func-ref-O-(one var ..) -> 1}


Stores \texttt{1} in all \texttt{var} arguments. See also \texttt{zero}, \texttt{on}, \texttt{off} and
\texttt{onOff}.


\begin{wideverbatim}
: (one A B)
-> 1
: A
-> 1
: B
-> 1
\end{wideverbatim}

 
\section*{\texttt{(onOff var ..) -> flg}}
\label{sec:func-ref-O-(onOff var ..) -> flg}


Logically negates the values of all \texttt{var} arguments. Returns the new
value of the last symbol. See also \texttt{on}, \texttt{off}, \texttt{zero} and \texttt{one}.


\begin{wideverbatim}
: (onOff A B)
-> T
: A
-> T
: B
-> T
: (onOff A B)
-> NIL
: A
-> NIL
: B
-> NIL
\end{wideverbatim}

 
\section*{\texttt{(open 'any ['flg]) -> cnt | NIL}}
\label{sec:func-ref-O-(open 'any ['flg]) -> cnt | NIL}


Opens the file with the name \texttt{any} in read/write mode (or read-only if
\texttt{flg} is non-\texttt{NIL}), and returns a file descriptor \texttt{cnt} (or \texttt{NIL} on
error). A leading ``\texttt{@}'' character in \texttt{any} is substituted with the
PicoLisp Home Directory, as it was remembered during interpreter
startup. If \texttt{flg} is \texttt{NIL} and the file does not exist, it is created.
The file descriptor can be used in subsequent calls to \texttt{in} and \texttt{out}.
See also \texttt{close} and \texttt{poll}.


\begin{wideverbatim}
: (open "x")
-> 3
\end{wideverbatim}

 
\section*{\texttt{(opid) -> pid | NIL}}
\label{sec:func-ref-O-(opid) -> pid | NIL}


Returns the corresponding process ID when the current output channel is
writing to a pipe, otherwise \texttt{NIL}. See also \texttt{ipid} and \texttt{out}.


\begin{wideverbatim}
: (out '(cat) (call 'ps "-p" (opid)))
  PID TTY          TIME CMD
 7127 pts/3    00:00:00 cat
-> T
\end{wideverbatim}

 
\section*{\texttt{(opt) -> sym}}
\label{sec:func-ref-O-(opt) -> sym}


Return the next command line argument (``option'', as would be processed
by \texttt{load}) as a string, and remove it from the remaining command line
arguments. See also \emph{Invocation} and \texttt{argv}.


\begin{wideverbatim}
$ pil  -"de f () (println 'opt (opt))"  -f abc  -bye
opt "abc"
\end{wideverbatim}

 
\section*{\texttt{(or 'any ..) -> any}}
\label{sec:func-ref-O-(or 'any ..) -> any}


Logical OR. The expressions \texttt{any} are evaluated from left to right. If a
non-\texttt{NIL} value is encountered, it is returned immediately. Else the
result of the last expression is returned.


\begin{wideverbatim}
: (or (= 3 3) (read))
-> T
: (or (= 3 4) (read))
abc
-> abc
\end{wideverbatim}

 
\section*{\texttt{or/2}}
\label{sec:func-ref-O-or/2}


\emph{Pilog} predicate that takes an arbitrary number of
clauses, and succeeds if one of them can be proven. See also \texttt{not/1}.


\begin{wideverbatim}
: (?
   (or
      ((equal 3 @X) (equal @X 4))
      ((equal 7 @X) (equal @X 7)) ) )
 @X=7
-> NIL
\end{wideverbatim}

 
\section*{\texttt{(out 'any . prg) -> any}}
\label{sec:func-ref-O-(out 'any . prg) -> any}


Opens \texttt{any} as output channel during the execution of
\texttt{prg}. The current output channel will be saved and restored
appropriately. If the argument is \texttt{NIL}, standard output is
used. If the argument is a symbol, it is used as a file name (opened
in ``append'' mode if the first character is ``\texttt{+}''). If it is
a positve number, it is used as the descriptor of an open file. If it
is a negative number, the saved output channel such many levels above
the current one is used. Otherwise (if it is a list), it is taken as a
command with arguments, and a pipe is opened for output. See also
\texttt{opid}, \texttt{ call}, \texttt{in}, \texttt{err},
\texttt{ctl}, \texttt{pipe}, \texttt{ poll}, \texttt{close} and
\texttt{load}.


\begin{wideverbatim}
: (out "a" (println 123 '(a b c) 'def))  # Write one line to file "a"
-> def
\end{wideverbatim}




% %%%%%%%%%%%%%%%%%%%%%%%% referenc.tex %%%%%%%%%%%%%%%%%%%%%%%%%%%%%%
% sample references
% %
% Use this file as a template for your own input.
%
%%%%%%%%%%%%%%%%%%%%%%%% Springer-Verlag %%%%%%%%%%%%%%%%%%%%%%%%%%
%
% BibTeX users please use
% \bibliographystyle{}
% \bibliography{}
%
\biblstarthook{In view of the parallel print and (chapter-wise) online publication of your book at \url{www.springerlink.com} it has been decided that -- as a genreral rule --  references should be sorted chapter-wise and placed at the end of the individual chapters. However, upon agreement with your contact at Springer you may list your references in a single seperate chapter at the end of your book. Deactivate the class option \texttt{sectrefs} and the \texttt{thebibliography} environment will be put out as a chapter of its own.\\\indent
References may be \textit{cited} in the text either by number (preferred) or by author/year.\footnote{Make sure that all references from the list are cited in the text. Those not cited should be moved to a separate \textit{Further Reading} section or chapter.} The reference list should ideally be \textit{sorted} in alphabetical order -- even if reference numbers are used for the their citation in the text. If there are several works by the same author, the following order should be used: 
\begin{enumerate}
\item all works by the author alone, ordered chronologically by year of publication
\item all works by the author with a coauthor, ordered alphabetically by coauthor
\item all works by the author with several coauthors, ordered chronologically by year of publication.
\end{enumerate}
The \textit{styling} of references\footnote{Always use the standard abbreviation of a journal's name according to the ISSN \textit{List of Title Word Abbreviations}, see \url{http://www.issn.org/en/node/344}} depends on the subject of your book:
\begin{itemize}
\item The \textit{two} recommended styles for references in books on \textit{mathematical, physical, statistical and computer sciences} are depicted in ~\cite{science-contrib, science-online, science-mono, science-journal, science-DOI} and ~\cite{phys-online, phys-mono, phys-journal, phys-DOI, phys-contrib}.
\item Examples of the most commonly used reference style in books on \textit{Psychology, Social Sciences} are~\cite{psysoc-mono, psysoc-online,psysoc-journal, psysoc-contrib, psysoc-DOI}.
\item Examples for references in books on \textit{Humanities, Linguistics, Philosophy} are~\cite{humlinphil-journal, humlinphil-contrib, humlinphil-mono, humlinphil-online, humlinphil-DOI}.
\item Examples of the basic Springer style used in publications on a wide range of subjects such as \textit{Computer Science, Economics, Engineering, Geosciences, Life Sciences, Medicine, Biomedicine} are ~\cite{basic-contrib, basic-online, basic-journal, basic-DOI, basic-mono}. 
\end{itemize}
}

\begin{thebibliography}{99.}%
% and use \bibitem to create references.
%
% Use the following syntax and markup for your references if 
% the subject of your book is from the field 
% "Mathematics, Physics, Statistics, Computer Science"
%
% Contribution 
\bibitem{science-contrib} Broy, M.: Software engineering --- from auxiliary to key technologies. In: Broy, M., Dener, E. (eds.) Software Pioneers, pp. 10-13. Springer, Heidelberg (2002)
%
% Online Document
\bibitem{science-online} Dod, J.: Effective substances. In: The Dictionary of Substances and Their Effects. Royal Society of Chemistry (1999) Available via DIALOG. \\
\url{http://www.rsc.org/dose/title of subordinate document. Cited 15 Jan 1999}
%
% Monograph
\bibitem{science-mono} Geddes, K.O., Czapor, S.R., Labahn, G.: Algorithms for Computer Algebra. Kluwer, Boston (1992) 
%
% Journal article
\bibitem{science-journal} Hamburger, C.: Quasimonotonicity, regularity and duality for nonlinear systems of partial differential equations. Ann. Mat. Pura. Appl. \textbf{169}, 321--354 (1995)
%
% Journal article by DOI
\bibitem{science-DOI} Slifka, M.K., Whitton, J.L.: Clinical implications of dysregulated cytokine production. J. Mol. Med. (2000) doi: 10.1007/s001090000086 
%
\bigskip

% Use the following (APS) syntax and markup for your references if 
% the subject of your book is from the field 
% "Mathematics, Physics, Statistics, Computer Science"
%
% Online Document
\bibitem{phys-online} J. Dod, in \textit{The Dictionary of Substances and Their Effects}, Royal Society of Chemistry. (Available via DIALOG, 1999), 
\url{http://www.rsc.org/dose/title of subordinate document. Cited 15 Jan 1999}
%
% Monograph
\bibitem{phys-mono} H. Ibach, H. L\"uth, \textit{Solid-State Physics}, 2nd edn. (Springer, New York, 1996), pp. 45-56 
%
% Journal article
\bibitem{phys-journal} S. Preuss, A. Demchuk Jr., M. Stuke, Appl. Phys. A \textbf{61}
%
% Journal article by DOI
\bibitem{phys-DOI} M.K. Slifka, J.L. Whitton, J. Mol. Med., doi: 10.1007/s001090000086
%
% Contribution 
\bibitem{phys-contrib} S.E. Smith, in \textit{Neuromuscular Junction}, ed. by E. Zaimis. Handbook of Experimental Pharmacology, vol 42 (Springer, Heidelberg, 1976), p. 593
%
\bigskip
%
% Use the following syntax and markup for your references if 
% the subject of your book is from the field 
% "Psychology, Social Sciences"
%
%
% Monograph
\bibitem{psysoc-mono} Calfee, R.~C., \& Valencia, R.~R. (1991). \textit{APA guide to preparing manuscripts for journal publication.} Washington, DC: American Psychological Association.
%
% Online Document
\bibitem{psysoc-online} Dod, J. (1999). Effective substances. In: The dictionary of substances and their effects. Royal Society of Chemistry. Available via DIALOG. \\
\url{http://www.rsc.org/dose/Effective substances.} Cited 15 Jan 1999.
%
% Journal article
\bibitem{psysoc-journal} Harris, M., Karper, E., Stacks, G., Hoffman, D., DeNiro, R., Cruz, P., et al. (2001). Writing labs and the Hollywood connection. \textit{J Film} Writing, 44(3), 213--245.
%
% Contribution 
\bibitem{psysoc-contrib} O'Neil, J.~M., \& Egan, J. (1992). Men's and women's gender role journeys: Metaphor for healing, transition, and transformation. In B.~R. Wainrig (Ed.), \textit{Gender issues across the life cycle} (pp. 107--123). New York: Springer.
%
% Journal article by DOI
\bibitem{psysoc-DOI}Kreger, M., Brindis, C.D., Manuel, D.M., Sassoubre, L. (2007). Lessons learned in systems change initiatives: benchmarks and indicators. \textit{American Journal of Community Psychology}, doi: 10.1007/s10464-007-9108-14.
%
%
% Use the following syntax and markup for your references if 
% the subject of your book is from the field 
% "Humanities, Linguistics, Philosophy"
%
\bigskip
%
% Journal article
\bibitem{humlinphil-journal} Alber John, Daniel C. O'Connell, and Sabine Kowal. 2002. Personal perspective in TV interviews. \textit{Pragmatics} 12:257--271
%
% Contribution 
\bibitem{humlinphil-contrib} Cameron, Deborah. 1997. Theoretical debates in feminist linguistics: Questions of sex and gender. In \textit{Gender and discourse}, ed. Ruth Wodak, 99--119. London: Sage Publications.
%
% Monograph
\bibitem{humlinphil-mono} Cameron, Deborah. 1985. \textit{Feminism and linguistic theory.} New York: St. Martin's Press.
%
% Online Document
\bibitem{humlinphil-online} Dod, Jake. 1999. Effective substances. In: The dictionary of substances and their effects. Royal Society of Chemistry. Available via DIALOG. \\
http://www.rsc.org/dose/title of subordinate document. Cited 15 Jan 1999
%
% Journal article by DOI
\bibitem{humlinphil-DOI} Suleiman, Camelia, Daniel C. O�Connell, and Sabine Kowal. 2002. `If you and I, if we, in this later day, lose that sacred fire...�': Perspective in political interviews. \textit{Journal of Psycholinguistic Research}. doi: 10.1023/A:1015592129296.
%
%
%
\bigskip
%
%
% Use the following syntax and markup for your references if 
% the subject of your book is from the field 
% "Computer Science, Economics, Engineering, Geosciences, Life Sciences"
%
%
% Contribution 
\bibitem{basic-contrib} Brown B, Aaron M (2001) The politics of nature. In: Smith J (ed) The rise of modern genomics, 3rd edn. Wiley, New York 
%
% Online Document
\bibitem{basic-online} Dod J (1999) Effective Substances. In: The dictionary of substances and their effects. Royal Society of Chemistry. Available via DIALOG. \\
\url{http://www.rsc.org/dose/title of subordinate document. Cited 15 Jan 1999}
%
% Journal article by DOI
\bibitem{basic-DOI} Slifka MK, Whitton JL (2000) Clinical implications of dysregulated cytokine production. J Mol Med, doi: 10.1007/s001090000086
%
% Journal article
\bibitem{basic-journal} Smith J, Jones M Jr, Houghton L et al (1999) Future of health insurance. N Engl J Med 965:325--329
%
% Monograph
\bibitem{basic-mono} South J, Blass B (2001) The future of modern genomics. Blackwell, London 
%
\end{thebibliography}

