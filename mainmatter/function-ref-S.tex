%%%%%%%%%%%%%%%%%%%%% chapter.tex %%%%%%%%%%%%%%%%%%%%%%%%%%%%%%%%%
%
% sample chapter
%
% Use this file as a template for your own input.
%
%%%%%%%%%%%%%%%%%%%%%%%% Springer-Verlag %%%%%%%%%%%%%%%%%%%%%%%%%%
%\motto{Use the template \emph{chapter.tex} to style the various elements of your chapter content.}



\chapter{Symbols starting with S}
\label{cha:func-ref-S-functions-starting-with-S}
 
\section*{\texttt{*Scl}}
\label{sec:func-ref-S-*Scl}


A global variable holding the current fixpoint input scale. See also
\emph{Numbers} and \texttt{scl}.


\begin{wideverbatim}
: (str "123.45")  # Default value of '*Scl' is 0
-> (123)
: (setq *Scl 3)
-> 3
: (str "123.45")
-> (123450)
\end{wideverbatim}

 
\section*{\texttt{*Sig1}}
\label{sec:func-ref-S-*Sig1}


\texttt{*Sig2}

Global variables holding (possibly empty) \texttt{prg} bodies, which will be
executed when a SIGUSR1 signal (or a SIGUSR2 signal, respectively) is
sent to the current process. See also \texttt{alarm}, \texttt{sigio} and \texttt{*Hup}.


\begin{wideverbatim}
: (de *Sig1 (msg 'SIGUSR1))
-> *Sig1
\end{wideverbatim}

 
\section*{\texttt{*Solo}}
\label{sec:func-ref-S-*Solo}


A global variable indicating exclusive database access. Its value is \texttt{0}
initially, set to \texttt{T} (or \texttt{NIL}) during cooperative database locks when
\texttt{lock} is successfully called with a \texttt{NIL} (or non-\texttt{NIL}) argument. See
also \texttt{*Zap}.


\begin{wideverbatim}
: *Solo
-> 0
: (lock *DB)
-> NIL
: *Solo
-> NIL
: (rollback)
-> T
: *Solo
-> 0
: (lock)
-> NIL
: *Solo
-> T
: (rollback)
-> T
: *Solo
-> T
\end{wideverbatim}

 
\section*{\texttt{+Sn}}
\label{sec:func-ref-S-+Sn}


Prefix class for maintaining indexes according to a modified soundex
algorithm, for tolerant name searches, to \texttt{+String} relations. Typically
used in combination with the \texttt{+Idx} prefix class. See also \texttt{Database}.


\begin{wideverbatim}
(rel nm (+Sn +Idx +String))  # Name
\end{wideverbatim}

 
\section*{\texttt{+String}}
\label{sec:func-ref-S-+String}


Class for string (transient symbol) relations, a subclass of \texttt{+Symbol}.
Accepts an optional argument for the string length (currently not used).
See also \texttt{Database}.


\begin{wideverbatim}
(rel nm (+Sn +Idx +String))  # Name, indexed by soundex and substrings
\end{wideverbatim}

 
\section*{\texttt{+Symbol}}
\label{sec:func-ref-S-+Symbol}


Class for symbolic relations, a subclass of \texttt{+relation}. Objects of that
class typically maintain internal symbols, as opposed to the more
often-used \texttt{+String} for transient symbols. See also \texttt{Database}.


\begin{wideverbatim}
(rel perm (+List +Symbol))  # Permission list
\end{wideverbatim}

 
\section*{\texttt{same/3}}
\label{sec:func-ref-S-same/3}


\emph{Pilog} predicate that succeeds if the first argument
matches the result of applying the \texttt{get} algorithm to the following
arguments. Typically used as filter predicate in \texttt{select/3} database
queries. See also \texttt{isa/2}, \texttt{bool/3}, \texttt{range/3}, \texttt{head/3}, \texttt{fold/3},
\texttt{part/3} and \texttt{tolr/3}.


\begin{wideverbatim}
: (?
   @Nr 2
   @Nm "Spare"
   (select (@Item)
      ((nr +Item @Nr) (nm +Item @Nm))
      (same @Nr @Item nr)
      (head @Nm @Item nm) ) )
 @Nr=2 @Nm="Spare" @Item={3-2}
\end{wideverbatim}

 
\section*{\texttt{(scan 'tree ['fun] ['any1] ['any2] ['flg])}}
\label{sec:func-ref-S-(scan 'tree ['fun] ['any1] ['any2] ['flg])}


Scans through a database tree by applying \texttt{fun} to all key-value pairs.
\texttt{fun} should be a function accepting two arguments for key and value. It
defaults to \texttt{println}. \texttt{any1} and \texttt{any2} may specify a range of keys. If
\texttt{any2} is greater than \texttt{any1}, the traversal will be in opposite
direction. Note that the keys need not to be atomic, depending on the
application's index structure. If \texttt{flg} is non-\texttt{NIL}, partial keys are
skipped. See also \texttt{tree}, \texttt{iter}, \texttt{init} and \texttt{step}.


\begin{wideverbatim}
: (scan (tree 'nm '+Item))
("ASLRSNSTRSTN" {3-3} . T) {3-3}
("Additive" {3-4}) {3-4}
("Appliance" {3-6}) {3-6}
("Auxiliary Construction" . {3-3}) {3-3}
("Construction" {3-3}) {3-3}
("ENNSNNTTTF" {3-4} . T) {3-4}
("Enhancement Additive" . {3-4}) {3-4}
("Fittings" {3-5}) {3-5}
("GTSTFLNS" {3-6} . T) {3-6}
("Gadget Appliance" . {3-6}) {3-6}
...

: (scan (tree 'nm '+Item) println NIL T T)  # 'flg' is non-NIL
("Auxiliary Construction" . {3-3}) {3-3}
("Enhancement Additive" . {3-4}) {3-4}
("Gadget Appliance" . {3-6}) {3-6}
("Main Part" . {3-1}) {3-1}
("Metal Fittings" . {3-5}) {3-5}
("Spare Part" . {3-2}) {3-2}
("Testartikel" . {3-8}) {3-8}
-> {7-6}
\end{wideverbatim}

 
\section*{\texttt{(scl 'num) -> num}}
\label{sec:func-ref-S-(scl 'num) -> num}


Sets \texttt{*Scl} globally to \texttt{num}. See also \emph{Numbers}.


\begin{wideverbatim}
: (scl 0)
-> 0
: (str "123.45")
-> (123)
: (scl 1)
-> 1
: (read)
123.45
-> 1235
: (scl 3)
-> 3
: (str "123.45")
-> (123450)
\end{wideverbatim}

 
\section*{\texttt{(script 'any ..) -> any}}
\label{sec:func-ref-S-(script 'any ..) -> any}


The first \texttt{any} argument is \texttt{load}ed, with the remaining arguments \texttt{pass}ed as variable arguments. They can be accessed with \texttt{next}, \texttt{arg},
\texttt{args} and \texttt{rest}.


\begin{wideverbatim}
$ cat x
(* (next) (next))

$ pil +
: (script "x" 3 4)
-> 12
\end{wideverbatim}

 
\section*{\texttt{(sect 'lst 'lst) -> lst}}
\label{sec:func-ref-S-(sect 'lst 'lst) -> lst}


Returns the intersection of the \texttt{lst} arguments. See also \texttt{diff}.


\begin{wideverbatim}
: (sect (1 2 3 4) (3 4 5 6))
-> (3 4)
: (sect (1 2 3) (4 5 6))
-> NIL
\end{wideverbatim}

 
\section*{\texttt{(seed 'any) -> cnt}}
\label{sec:func-ref-S-(seed 'any) -> cnt}


Initializes the random generator's seed, and returns a pseudo random
number in the range --2147483648 .. +2147483647. See also \texttt{rand} and
\texttt{hash}.


\begin{wideverbatim}
: (seed "init string")
-> 2015582081
: (rand)
-> -706917003
: (rand)
-> 1224196082

: (seed (time))
-> 128285383
\end{wideverbatim}

 
\section*{\texttt{(seek 'fun 'lst ..) -> lst}}
\label{sec:func-ref-S-(seek 'fun 'lst ..) -> lst}


Applies \texttt{fun} to \texttt{lst} and all successive CDRs, until non-\texttt{NIL} is
returned. Returns the tail of \texttt{lst} starting with that element, or \texttt{NIL}
if \texttt{fun} did not return non-\texttt{NIL} for any element of \texttt{lst}. When
additional \texttt{lst} arguments are given, they are passed to \texttt{fun} in the
same way. See also \texttt{find}, \texttt{pick}.


\begin{wideverbatim}
: (seek '((X) (> (car X) 9)) (1 5 8 12 19 22))
-> (12 19 22)
\end{wideverbatim}

 
\section*{\texttt{(select [var ..] cls [hook|T] [var val ..]) -> obj | NIL}}
\label{sec:func-ref-S-(select [var ..] cls [hook|T] [var val ..]) -> obj | NIL}


Interactive database function, loosely modelled after the SQL `\texttt{SELECT}'
command. A (limited) front-end to the Pilog \texttt{select/3} predicate. When
called with only a \texttt{cls} argument, \texttt{select} steps through all objects of
that class, and \texttt{show}s their complete contents (this is analog to 'SELECT * from CLS'). If \texttt{cls} is followed by attribute/value
specifications, the search is limited to these values (this is analog to
`SELECT * from CLS where VAR = VAL'). If between the \texttt{select} function
and \texttt{cls} one or several attribute names are supplied, only these
attribute (instead of the full \texttt{show}) are printed. These attribute
specifications may also be lists, then the \texttt{get} algorithm will be used
to retrieve related data. See also \texttt{update}, \texttt{Database} and
\emph{Pilog}.


\begin{wideverbatim}
: (select +Item)                       # Show all items
{3-1} (+Item)
   nr 1
   pr 29900
   inv 100
   sup {2-1}
   nm "Main Part"

{3-2} (+Item)
   nr 2
   pr 1250
   inv 100
   sup {2-2}
   nm "Spare Part"
.                                      # Stop
-> {3-2}

: (select +Item nr 3)                  # Show only item 3
{3-3} (+Item)
   nr 3
   sup {2-1}
   pr 15700
   nm "Auxiliary Construction"
   inv 100
.                                      # Stop
-> {3-3}

# Show selected attributes for items 3 through 3
: (select nr nm pr (sup nm) +Item nr (3 . 5))
3 "Auxiliary Construction" 157.00 "Active Parts Inc." {3-3}
4 "Enhancement Additive" 9.99 "Seven Oaks Ltd." {3-4}
5 "Metal Fittings" 79.80 "Active Parts Inc." {3-5}
-> NIL
\end{wideverbatim}

 
\section*{\texttt{select/3}}
\label{sec:func-ref-S-select/3}


\emph{Pilog} database predicate that allows combined
searches over \texttt{+index} and other relations. It takes a list of Pilog
variables, a list of generator clauses, and an arbitrary number of
filter clauses. The functionality is described in detail in
\emph{The `select' Predicate}. See also \texttt{db/3}, \texttt{isa/2},
\texttt{same/3}, \texttt{bool/3}, \texttt{range/3}, \texttt{head/3}, \texttt{fold/3}, \texttt{part/3}, \texttt{tolr/3}
and \texttt{remote/2}.


\begin{wideverbatim}
: (?
   @Nr (2 . 5)          # Select all items with numbers between 2 and 5
   @Sup "Active"        # and suppliers matching "Active"
   (select (@Item)                                  # Bind results to '@Item"
      ((nr +Item @Nr) (nm +CuSu @Sup (sup +Item)))  # Generator clauses
      (range @Nr @Item nr)                          # Filter clauses
      (part @Sup @Item sup nm) ) )
 @Nr=(2 . 5) @Sup="Active" @Item={3-3}
 @Nr=(2 . 5) @Sup="Active" @Item={3-5}
-> NIL
\end{wideverbatim}

 
\section*{\texttt{(send 'msg 'obj ['any ..]) -> any}}
\label{sec:func-ref-S-(send 'msg 'obj ['any ..]) -> any}


Sends the message \texttt{msg} to the object \texttt{obj}, optionally with arguments
\texttt{any}. If the message cannot be located in \texttt{obj}, its classes and
superclasses, an error \texttt{''Bad message''} is issued. See also
\texttt{OO Concepts}, \texttt{try}, \texttt{method}, \texttt{meth}, \texttt{super} and \texttt{extra}.


\begin{wideverbatim}
: (send 'stop> Dlg)  # Equivalent to (stop> Dlg)
-> NIL
\end{wideverbatim}

 
\section*{\texttt{(seq 'cnt|sym1) -> sym | NIL}}
\label{sec:func-ref-S-(seq 'cnt|sym1) -> sym | NIL}


Sequential single step: Returns the \emph{first} external symbol in the
\texttt{cnt}'th database file, or the \emph{next} external symbol following \texttt{sym1}
in the database, or \texttt{NIL} when the end of the database is reached. See
also \texttt{free}.


\begin{wideverbatim}
: (pool "db")
-> T
: (seq *DB)
-> {2}
: (seq @)
-> {3}
\end{wideverbatim}

 
\section*{\texttt{(set 'var 'any ..) -> any}}
\label{sec:func-ref-S-(set 'var 'any ..) -> any}


Stores new values \texttt{any} in the \texttt{var} arguments. See also \texttt{setq}, \texttt{val},
\texttt{con} and \texttt{def}.


\begin{wideverbatim}
: (set 'L '(a b c)  (cdr L) '999)
-> 999
: L
-> (a 999 c)
\end{wideverbatim}

 
\section*{\texttt{(set! 'obj 'any) -> any}}
\label{sec:func-ref-S-(set! 'obj 'any) -> any}


\emph{Transaction} wrapper function for \texttt{set}. Note that
for setting the value of entities typically the \texttt{set!>} message is used.
See also \texttt{new!}, \texttt{put!} and \texttt{inc!}.


\begin{wideverbatim}
(set! Obj (* Count Size))  # Setting a non-entity object to a numeric value
\end{wideverbatim}

 
\section*{\texttt{(setq var 'any ..) -> any}}
\label{sec:func-ref-S-(setq var 'any ..) -> any}


Stores new values \texttt{any} in the \texttt{var} arguments. See also \texttt{set}, \texttt{val}
and \texttt{def}.


\begin{wideverbatim}
: (setq  A 123  B (list A A))  # Set 'A' to 123, then 'B' to (123 123)
-> (123 123)
\end{wideverbatim}

 
\section*{\texttt{(show 'any ['sym|cnt ..]) -> any}}
\label{sec:func-ref-S-(show 'any ['sym|cnt ..]) -> any}


Shows the name, value and property list of a symbol found by applying
the \texttt{get} algorithm to \texttt{any} and the following arguments. See also
\texttt{edit} and \texttt{view}.


\begin{wideverbatim}
: (setq A 123456)
-> 123456
: (put 'A 'x 1)
-> 1
: (put 'A 'lst (9 8 7))
-> (9 8 7)
: (put 'A 'flg T)
-> T

: (show 'A)
A 123456
   flg
   lst (9 8 7)
   x 1
-> A

: (show 'A 'lst 2)
-> 8
\end{wideverbatim}

 
\section*{\texttt{show/1}}
\label{sec:func-ref-S-show/1}


\emph{Pilog} predicate that always succeeds, and shows the
name, value and property list of the argument symbol. See also \texttt{show}.


\begin{wideverbatim}
: (? (db nr +Item 2 @Item) (show @Item))
{3-2} (+Item)
   nm "Spare Part"
   nr 2
   pr 1250
   inv 100
   sup {2-2}
 @Item={3-2}
-> NIL
\end{wideverbatim}

 
\section*{\texttt{(sigio ['cnt [. prg]]) -> cnt | prg}}
\label{sec:func-ref-S-(sigio ['cnt [. prg]]) -> cnt | prg}


Sets a signal handler \texttt{prg} for SIGIO on the file descriptor \texttt{cnt}. If
called without arguments, the currently installed handler is returned.
See also \texttt{alarm}, \texttt{*Hup} and \texttt{*Sig[12]}.


\begin{wideverbatim}
# First session
: (sigio (setq *SigSock (port T 4444))  # Register signal handler at UDP port
   (while (udp *SigSock)                # Queue all received data
      (fifo '*SigQueue @) ) )
-> 3

# Second session
: (for I 7 (udp "localhost" 4444 I))  # Send numbers to first session

# First session
: (fifo '*SigQueue)
-> 1
: (fifo '*SigQueue)
-> 2
\end{wideverbatim}

 
\section*{\texttt{(size 'any) -> cnt}}
\label{sec:func-ref-S-(size 'any) -> cnt}


Returns the ``size'' of \texttt{any}. For numbers this is the number of bytes
needed for the value, for external symbols it is the number of bytes it
would occupy in the database, for other symbols it is the number of
bytes occupied by the UTF--8 representation of the name, and for lists
it is the total number of cells in this list and all its sublists. See
also \texttt{length}.


\begin{wideverbatim}
: (size "abc")
-> 3
: (size "äbc")
-> 4
: (size 127)  # One byte
-> 1
: (size 128)  # Two bytes (eight bits plus sign bit!)
-> 2
: (size (1 (2) 3))
-> 4
: (size (1 2 3 .))
-> 3
\end{wideverbatim}

 
\section*{\texttt{(skip ['any]) -> sym}}
\label{sec:func-ref-S-(skip ['any]) -> sym}


Skips all whitespace (and comments if \texttt{any} is given) in the input
stream. Returns the next available character, or \texttt{NIL} upon end of file.
See also \texttt{peek} and \texttt{eof}.


\begin{wideverbatim}
$ cat a
# Comment
abcd
$ pil +
: (in "a" (skip "#"))
-> "a"
\end{wideverbatim}

 
\section*{\texttt{(solve 'lst [. prg]) -> lst}}
\label{sec:func-ref-S-(solve 'lst [. prg]) -> lst}


Evaluates a \emph{Pilog} query and, returns the list of
result sets. If \texttt{prg} is given, it is executed for each result set, with
all Pilog variables bound to their matching values, and returns a list
of the results. See also \texttt{pilog}, \texttt{?}, \texttt{goal} and \texttt{prove}.


\begin{wideverbatim}
: (solve '((append @X @Y (a b c))))
-> (((@X) (@Y a b c)) ((@X a) (@Y b c)) ((@X a b) (@Y c)) ((@X a b c) (@Y)))

: (solve '((append @X @Y (a b c))) @X)
-> (NIL (a) (a b) (a b c))
\end{wideverbatim}

 
\section*{\texttt{(sort 'lst ['fun]) -> lst}}
\label{sec:func-ref-S-(sort 'lst ['fun]) -> lst}


Sorts \texttt{lst} by destructively exchanging its elements. If \texttt{fun} is given,
it is used as a ``less than'' predicate for comparisons. Typically, \texttt{sort}
is used in combination with \emph{by}, giving shorter and
often more efficient solutions than with the predicate function. See
also \emph{Comparing}, \texttt{group}, \texttt{maxi}, \texttt{mini} and \texttt{uniq}.


\begin{wideverbatim}
: (sort '(a 3 1 (1 2 3) d b 4 T NIL (a b c) (x y z) c 2))
-> (NIL 1 2 3 4 a b c d (1 2 3) (a b c) (x y z) T)
: (sort '(a 3 1 (1 2 3) d b 4 T NIL (a b c) (x y z) c 2) >)
-> (T (x y z) (a b c) (1 2 3) d c b a 4 3 2 1 NIL)
: (by cadr sort '((1 4 3) (5 1 3) (1 2 4) (3 8 5) (6 4 5)))
-> ((5 1 3) (1 2 4) (1 4 3) (6 4 5) (3 8 5))
\end{wideverbatim}

 
\section*{\texttt{(space ['cnt]) -> cnt}}
\label{sec:func-ref-S-(space ['cnt]) -> cnt}


Prints \texttt{cnt} spaces, or a single space when \texttt{cnt} is not given.


\begin{wideverbatim}
: (space)
 -> 1
: (space 1)
 -> 1
: (space 2)
  -> 2
\end{wideverbatim}

 
\section*{\texttt{(sp? 'any) -> flg}}
\label{sec:func-ref-S-(sp? 'any) -> flg}


Returns \texttt{T} when the argument \texttt{any} is \texttt{NIL}, or if it is a string
(symbol) that consists only of whitespace characters.


\begin{wideverbatim}
: (sp? "  ")
-> T
: (sp? "ABC")
-> NIL
: (sp? 123)
-> NIL
\end{wideverbatim}

 
\section*{\texttt{(split 'lst 'any ..) -> lst}}
\label{sec:func-ref-S-(split 'lst 'any ..) -> lst}


Splits \texttt{lst} at all places containing an element \texttt{any} and returns the
resulting list of sublists. See also \texttt{stem}.


\begin{wideverbatim}
: (split (1 a 2 b 3 c 4 d 5 e 6) 'e 3 'a)
-> ((1) (2 b) (c 4 d 5) (6))
: (mapcar pack (split (chop "The quick brown fox") " "))
-> ("The" "quick" "brown" "fox")
\end{wideverbatim}

 
\section*{\texttt{(sqrt 'num ['flg]) -> num}}
\label{sec:func-ref-S-(sqrt 'num ['flg]) -> num}


Returns the square root of the \texttt{num} argument. If \texttt{flg} is given and
non-\texttt{NIL}, the result will be rounded.


\begin{wideverbatim}
: (sqrt 64)
-> 8
: (sqrt 1000)
-> 31
: (sqrt 1000 T)
-> 32
: (sqrt 10000000000000000000000000000000000000000)
-> 100000000000000000000
\end{wideverbatim}

 
\section*{\texttt{(stack ['cnt]) -> cnt | (.. sym . cnt)}}
\label{sec:func-ref-S-(stack ['cnt]) -> cnt | (.. sym . cnt)}


(64-bit version only) Maintains the stack segment size for coroutines.
If called without a \texttt{cnt} argument, or if already one or more
\emph{coroutines} are running, the current size in
megabytes is returned. Otherwise, the stack segment size is set to the
new value (default 4 MB). If there are running coroutines, their tags
will be \texttt{cons}ed in front of the size. See also \texttt{heap}.


\begin{wideverbatim}
: (stack)         # Get current stack segment size
-> 4
: (stack 10)      # Set to 10 MB
-> 10
: (let N 0 (recur (N) (recurse (inc N))))
!? (recurse (inc N))
Stack overflow
? N
-> 109181
?

: (co "routine" (yield 7))  # Create two coroutines
-> 7
: (co "routine2" (yield 8))
-> 8
: (stack)
-> ("routine2" "routine" . 4)
\end{wideverbatim}

 
\section*{\texttt{(stamp ['dat 'tim]|['T]) -> sym}}
\label{sec:func-ref-S-(stamp ['dat 'tim]|['T]) -> sym}


Returns a date-time string in the form ``YYYY-MM-DD HH:MM:SS''. If \texttt{dat}
and \texttt{tim} is missing, the current date and time is used. If \texttt{T} is
passed, the current Coordinated Universal Time (UTC) is used instead.
See also \texttt{date} and \texttt{time}.


\begin{wideverbatim}
: (stamp)
-> "2000-09-12 07:48:04"
: (stamp (date) 0)
-> "2000-09-12 00:00:00"
: (stamp (date 2000 1 1) (time 12 0 0))
-> "2000-01-01 12:00:00"
\end{wideverbatim}

 
\section*{\texttt{(state 'var (sym|lst exe [. prg]) ..) -> any}}
\label{sec:func-ref-S-(state 'var (sym|lst exe [. prg]) ..) -> any}


Implements a finite state machine. The variable \texttt{var} holds the current
state as a symbolic value. When a clause is found that contains the
current state in its CAR \texttt{sym|lst} value, and where the \texttt{exe} in its
CADR evaluates to non-\texttt{NIL}, the current state will be set to that
value, the body \texttt{prg} in the CDDR will be executed, and the result
returned. \texttt{T} is a catch-all for any state. If no state-condition
matches, \texttt{NIL} is returned. See also \texttt{case}, \texttt{cond} and \texttt{job}.


\begin{wideverbatim}
: (de tst ()
   (job '((Cnt . 4))
      (state '(start)
         (start 'run
            (printsp 'start) )
         (run (and (gt0 (dec 'Cnt)) 'run)
            (printsp 'run) )
         (run 'stop
            (printsp 'run) )
         (stop 'start
            (setq Cnt 4)
            (println 'stop) ) ) ) )
-> tst
: (do 12 (tst))
start run run run run stop
start run run run run stop
-> stop
: (pp 'tst)
(de tst NIL
   (job '((Cnt . 4))
      (state '(start)
      ...
-> tst
: (do 3 (tst))
start run run -> run
: (pp 'tst)
(de tst NIL
   (job '((Cnt . 2))
      (state '(run)
      ...
-> tst
\end{wideverbatim}

 
\section*{\texttt{(stem 'lst 'any ..) -> lst}}
\label{sec:func-ref-S-(stem 'lst 'any ..) -> lst}


Returns the tail of \texttt{lst} that does not contain any of the \texttt{any}
arguments. \texttt{(stem 'lst 'any ..)} is equivalent to
\texttt{(last (split 'lst 'any ..))}. See also \texttt{tail} and \texttt{split}.


\begin{wideverbatim}
: (stem (chop "abc/def\\ghi") "/" "\\")
-> ("g" "h" "i")
\end{wideverbatim}

 
\section*{\texttt{(step 'lst ['flg]) -> any}}
\label{sec:func-ref-S-(step 'lst ['flg]) -> any}


Single-steps iteratively through a database tree. \texttt{lst} is a structure
as received from \texttt{init}. If \texttt{flg} is non-\texttt{NIL}, partial keys are
skipped. See also \texttt{tree}, \texttt{scan}, \texttt{iter}, \texttt{leaf} and \texttt{fetch}.


\begin{wideverbatim}
: (setq Q (init (tree 'nr '+Item) 3 5))
-> (((3 . 5) ((3 NIL . {3-3}) (4 NIL . {3-4}) (5 NIL . {3-5}) (6 NIL . {3-6}) (7 NIL . {3-8}))))
: (get (step Q) 'nr)
-> 3
: (get (step Q) 'nr)
-> 4
: (get (step Q) 'nr)
-> 5
: (get (step Q) 'nr)
-> NIL
\end{wideverbatim}

 
\section*{\texttt{(store 'tree 'any1 'any2 ['(num1 . num2)])}}
\label{sec:func-ref-S-(store 'tree 'any1 'any2 ['(num1 . num2)])}


Stores a value \texttt{any2} for the key \texttt{any1} in a database tree. \texttt{num1} is a
database file number, as used in \texttt{new} (defaulting to 1), and \texttt{num2} a
database block size (defaulting to 256). When \texttt{any2} is \texttt{NIL}, the
corresponding entry is deleted from the tree. See also \texttt{tree} and
\texttt{fetch}.


\begin{wideverbatim}
: (store (tree 'nr '+Item) 2 '{3-2})
\end{wideverbatim}

 
\section*{\texttt{(str 'sym ['sym1]) -> lst}}
\label{sec:func-ref-S-(str 'sym ['sym1]) -> lst}


\texttt{(str 'lst) -> sym}

In the first form, the string \texttt{sym} is parsed into a list. This
mechanism is also used by \texttt{load}. If \texttt{sym1} is given, it should specify
a set of characters, and \texttt{str} will then return a list of tokens analog
to \texttt{read}. The second form does the reverse operation by building a
string from a list. See also \texttt{any}, \texttt{name} and \texttt{sym}.


\begin{wideverbatim}
: (str "a (1 2) b")
-> (a (1 2) b)
: (str '(a "Hello" DEF))
-> "a \"Hello\" DEF"
: (str "a*3+b*4" "_")
-> (a "*" 3 "+" b "*" 4)
\end{wideverbatim}

 
\section*{\texttt{(strDat 'sym) -> dat}}
\label{sec:func-ref-S-(strDat 'sym) -> dat}


Converts a string \texttt{sym} in the date format of the current \texttt{locale} to a
\texttt{date}. See also \texttt{expDat}, \texttt{\$dat} and \texttt{datStr}.


\begin{wideverbatim}
: (strDat "2007-06-01")
-> 733134
: (strDat "01.06.2007")
-> NIL
: (locale "DE" "de")
-> NIL
: (strDat "01.06.2007")
-> 733134
: (strDat "1.6.2007")
-> 733134
\end{wideverbatim}

 
\section*{\texttt{(strip 'any) -> any}}
\label{sec:func-ref-S-(strip 'any) -> any}


Strips all leading \texttt{quote} symbols from \texttt{any}.


\begin{wideverbatim}
: (strip 123)
-> 123
: (strip '''(a))
-> (a)
: (strip (quote quote a b c))
-> (a b c)
\end{wideverbatim}

 
\section*{\texttt{(str? 'any) -> sym | NIL}}
\label{sec:func-ref-S-(str? 'any) -> sym | NIL}


Returns the argument \texttt{any} when it is a transient symbol (string),
otherwise \texttt{NIL}. See also \texttt{sym?}, \texttt{box?} and \texttt{ext?}.


\begin{wideverbatim}
: (str? 123)
-> NIL
: (str? '{ABC})
-> NIL
: (str? 'abc)
-> NIL
: (str? "abc")
-> "abc"
\end{wideverbatim}

 
\section*{\texttt{(sub? 'any1 'any2) -> any2 | NIL}}
\label{sec:func-ref-S-(sub? 'any1 'any2) -> any2 | NIL}


Returns \texttt{any2} when the string representation of \texttt{any1} is a substring
of the string representation of \texttt{any2}. See also \texttt{pre?}.


\begin{wideverbatim}
: (sub? "def" "abcdef")
-> T
: (sub? "abb" "abcdef")
-> NIL
: (sub? NIL "abcdef")
-> T
\end{wideverbatim}

 
\section*{\texttt{(subr 'sym) -> num}}
\label{sec:func-ref-S-(subr 'sym) -> num}


Converts a Lisp-function that was previously converted with \texttt{expr} back
to a C-function.


\begin{wideverbatim}
: car
-> 67313448
: (expr 'car)
-> (@ (pass $385260187))
: (subr 'car)
-> 67313448
: car
-> 67313448
\end{wideverbatim}

 
\section*{\texttt{(sum 'fun 'lst ..) -> num}}
\label{sec:func-ref-S-(sum 'fun 'lst ..) -> num}


Applies \texttt{fun} to each element of \texttt{lst}. When additional \texttt{lst} arguments
are given, their elements are also passed to \texttt{fun}. Returns the sum of
all numeric values returned from \texttt{fun}.


\begin{wideverbatim}
: (setq A 1  B 2  C 3)
-> 3
: (sum val '(A B C))
-> 6
: (sum                           # Total size of symbol list values
   '((X)
      (and (pair (val X)) (size @)) )
   (what) )
-> 32021
\end{wideverbatim}

 
\section*{\texttt{(super ['any ..]) -> any}}
\label{sec:func-ref-S-(super ['any ..]) -> any}


Can only be used inside methods. Sends the current message to the
current object \texttt{This}, this time starting the search for a method at the
superclass(es) of the class where the current method was found. See also
\texttt{OO Concepts}, \texttt{extra}, \texttt{method}, \texttt{meth}, \texttt{send} and \texttt{try}.


\begin{wideverbatim}
(dm stop> ()         # 'stop>' method of current class
   (super)           # Call the 'stop>' method of the superclass
   ... )             # other things
\end{wideverbatim}

 
\section*{\texttt{(sym 'any) -> sym}}
\label{sec:func-ref-S-(sym 'any) -> sym}


Generate the printed representation of \texttt{any} into the name of a new
symbol \texttt{sym}. This is the reverse operation of \texttt{any}. See also \texttt{name}
and \texttt{str}.


\begin{wideverbatim}
: (sym '(abc "Hello" 123))
-> "(abc \"Hello\" 123)"
\end{wideverbatim}

 
\section*{\texttt{(sym? 'any) -> flg}}
\label{sec:func-ref-S-(sym? 'any) -> flg}


Returns \texttt{T} when the argument \texttt{any} is a symbol. See also \texttt{str?}, \texttt{box?}
and \texttt{ext?}.


\begin{wideverbatim}
: (sym? 'a)
-> T
: (sym? NIL)
-> T
: (sym? 123)
-> NIL
: (sym? '(a b))
-> NIL
\end{wideverbatim}

 
\section*{\texttt{(symbols) -> sym}}
\label{sec:func-ref-S-(symbols) -> sym}


\texttt{(symbols 'sym1) -> sym2}

\texttt{(symbols 'sym1 'sym2) -> sym3}

(64-bit version only) Creates and manages namespaces of internal
symbols: In the first form, the current namespace is returned. In the
second form, the current namespace is set to \texttt{sym1}, and the previous
namespace \texttt{sym2} is returned. In the third form, \texttt{sym1} is assigned a
\texttt{balance}d copy of an existing namespace \texttt{sym2} and becomes the new
current namespace, returning the previous namespace \texttt{sym3}. See also
\texttt{pico}, \texttt{local}, \texttt{import} and \texttt{intern}.


\begin{wideverbatim}
: (symbols 'myLib 'pico)
-> pico
: (de foo (X)
   (bar (inx X)) )
-> foo

: (symbols 'pico)
-> myLib
: (pp 'foo)
(de foo . NIL)
-> foo
: (pp 'myLib~foo)
(de "foo" (X)
   ("bar" ("inx" X)) )
-> "foo"

: (symbols 'myLib)
-> pico
: (pp 'foo)
(de foo (X)
   (bar (inx X)) )
-> foo
\end{wideverbatim}

 
\section*{\texttt{(sync) -> flg}}
\label{sec:func-ref-S-(sync) -> flg}


Waits for pending data from all family processes. While other processes
are still sending data (via the \texttt{tell} mechanism), a \texttt{select} system
call is executed for all file descriptors and timers in the \texttt{VAL} of the
global variable \texttt{*Run}. When used in a non-database context, \texttt{(tell)}
should be called in the end to inform the parent process that it may
grant synchronization to other processes waiting for \texttt{sync}. In a
database context, where \texttt{sync} is usually called by \texttt{dbSync}, this is
not necessary because it is done internally by \texttt{commit} or \texttt{rollback}.
See also \texttt{key} and \texttt{wait}.


\begin{wideverbatim}
: (or (lock) (sync))       # Ensure database consistency
-> T                       # (numeric process-id if lock failed)
\end{wideverbatim}

 
\section*{\texttt{(sys 'any ['any]) -> sym}}
\label{sec:func-ref-S-(sys 'any ['any]) -> sym}


Returns or sets a system environment variable.


\begin{wideverbatim}
: (sys "TERM")  # Get current value
-> "xterm"
: (sys "TERM" "vt100")  # Set new value
-> "vt100"
: (sys "TERM")
-> "vt100"
\end{wideverbatim}




% %%%%%%%%%%%%%%%%%%%%%%%% referenc.tex %%%%%%%%%%%%%%%%%%%%%%%%%%%%%%
% sample references
% %
% Use this file as a template for your own input.
%
%%%%%%%%%%%%%%%%%%%%%%%% Springer-Verlag %%%%%%%%%%%%%%%%%%%%%%%%%%
%
% BibTeX users please use
% \bibliographystyle{}
% \bibliography{}
%
\biblstarthook{In view of the parallel print and (chapter-wise) online publication of your book at \url{www.springerlink.com} it has been decided that -- as a genreral rule --  references should be sorted chapter-wise and placed at the end of the individual chapters. However, upon agreement with your contact at Springer you may list your references in a single seperate chapter at the end of your book. Deactivate the class option \texttt{sectrefs} and the \texttt{thebibliography} environment will be put out as a chapter of its own.\\\indent
References may be \textit{cited} in the text either by number (preferred) or by author/year.\footnote{Make sure that all references from the list are cited in the text. Those not cited should be moved to a separate \textit{Further Reading} section or chapter.} The reference list should ideally be \textit{sorted} in alphabetical order -- even if reference numbers are used for the their citation in the text. If there are several works by the same author, the following order should be used: 
\begin{enumerate}
\item all works by the author alone, ordered chronologically by year of publication
\item all works by the author with a coauthor, ordered alphabetically by coauthor
\item all works by the author with several coauthors, ordered chronologically by year of publication.
\end{enumerate}
The \textit{styling} of references\footnote{Always use the standard abbreviation of a journal's name according to the ISSN \textit{List of Title Word Abbreviations}, see \url{http://www.issn.org/en/node/344}} depends on the subject of your book:
\begin{itemize}
\item The \textit{two} recommended styles for references in books on \textit{mathematical, physical, statistical and computer sciences} are depicted in ~\cite{science-contrib, science-online, science-mono, science-journal, science-DOI} and ~\cite{phys-online, phys-mono, phys-journal, phys-DOI, phys-contrib}.
\item Examples of the most commonly used reference style in books on \textit{Psychology, Social Sciences} are~\cite{psysoc-mono, psysoc-online,psysoc-journal, psysoc-contrib, psysoc-DOI}.
\item Examples for references in books on \textit{Humanities, Linguistics, Philosophy} are~\cite{humlinphil-journal, humlinphil-contrib, humlinphil-mono, humlinphil-online, humlinphil-DOI}.
\item Examples of the basic Springer style used in publications on a wide range of subjects such as \textit{Computer Science, Economics, Engineering, Geosciences, Life Sciences, Medicine, Biomedicine} are ~\cite{basic-contrib, basic-online, basic-journal, basic-DOI, basic-mono}. 
\end{itemize}
}

\begin{thebibliography}{99.}%
% and use \bibitem to create references.
%
% Use the following syntax and markup for your references if 
% the subject of your book is from the field 
% "Mathematics, Physics, Statistics, Computer Science"
%
% Contribution 
\bibitem{science-contrib} Broy, M.: Software engineering --- from auxiliary to key technologies. In: Broy, M., Dener, E. (eds.) Software Pioneers, pp. 10-13. Springer, Heidelberg (2002)
%
% Online Document
\bibitem{science-online} Dod, J.: Effective substances. In: The Dictionary of Substances and Their Effects. Royal Society of Chemistry (1999) Available via DIALOG. \\
\url{http://www.rsc.org/dose/title of subordinate document. Cited 15 Jan 1999}
%
% Monograph
\bibitem{science-mono} Geddes, K.O., Czapor, S.R., Labahn, G.: Algorithms for Computer Algebra. Kluwer, Boston (1992) 
%
% Journal article
\bibitem{science-journal} Hamburger, C.: Quasimonotonicity, regularity and duality for nonlinear systems of partial differential equations. Ann. Mat. Pura. Appl. \textbf{169}, 321--354 (1995)
%
% Journal article by DOI
\bibitem{science-DOI} Slifka, M.K., Whitton, J.L.: Clinical implications of dysregulated cytokine production. J. Mol. Med. (2000) doi: 10.1007/s001090000086 
%
\bigskip

% Use the following (APS) syntax and markup for your references if 
% the subject of your book is from the field 
% "Mathematics, Physics, Statistics, Computer Science"
%
% Online Document
\bibitem{phys-online} J. Dod, in \textit{The Dictionary of Substances and Their Effects}, Royal Society of Chemistry. (Available via DIALOG, 1999), 
\url{http://www.rsc.org/dose/title of subordinate document. Cited 15 Jan 1999}
%
% Monograph
\bibitem{phys-mono} H. Ibach, H. L\"uth, \textit{Solid-State Physics}, 2nd edn. (Springer, New York, 1996), pp. 45-56 
%
% Journal article
\bibitem{phys-journal} S. Preuss, A. Demchuk Jr., M. Stuke, Appl. Phys. A \textbf{61}
%
% Journal article by DOI
\bibitem{phys-DOI} M.K. Slifka, J.L. Whitton, J. Mol. Med., doi: 10.1007/s001090000086
%
% Contribution 
\bibitem{phys-contrib} S.E. Smith, in \textit{Neuromuscular Junction}, ed. by E. Zaimis. Handbook of Experimental Pharmacology, vol 42 (Springer, Heidelberg, 1976), p. 593
%
\bigskip
%
% Use the following syntax and markup for your references if 
% the subject of your book is from the field 
% "Psychology, Social Sciences"
%
%
% Monograph
\bibitem{psysoc-mono} Calfee, R.~C., \& Valencia, R.~R. (1991). \textit{APA guide to preparing manuscripts for journal publication.} Washington, DC: American Psychological Association.
%
% Online Document
\bibitem{psysoc-online} Dod, J. (1999). Effective substances. In: The dictionary of substances and their effects. Royal Society of Chemistry. Available via DIALOG. \\
\url{http://www.rsc.org/dose/Effective substances.} Cited 15 Jan 1999.
%
% Journal article
\bibitem{psysoc-journal} Harris, M., Karper, E., Stacks, G., Hoffman, D., DeNiro, R., Cruz, P., et al. (2001). Writing labs and the Hollywood connection. \textit{J Film} Writing, 44(3), 213--245.
%
% Contribution 
\bibitem{psysoc-contrib} O'Neil, J.~M., \& Egan, J. (1992). Men's and women's gender role journeys: Metaphor for healing, transition, and transformation. In B.~R. Wainrig (Ed.), \textit{Gender issues across the life cycle} (pp. 107--123). New York: Springer.
%
% Journal article by DOI
\bibitem{psysoc-DOI}Kreger, M., Brindis, C.D., Manuel, D.M., Sassoubre, L. (2007). Lessons learned in systems change initiatives: benchmarks and indicators. \textit{American Journal of Community Psychology}, doi: 10.1007/s10464-007-9108-14.
%
%
% Use the following syntax and markup for your references if 
% the subject of your book is from the field 
% "Humanities, Linguistics, Philosophy"
%
\bigskip
%
% Journal article
\bibitem{humlinphil-journal} Alber John, Daniel C. O'Connell, and Sabine Kowal. 2002. Personal perspective in TV interviews. \textit{Pragmatics} 12:257--271
%
% Contribution 
\bibitem{humlinphil-contrib} Cameron, Deborah. 1997. Theoretical debates in feminist linguistics: Questions of sex and gender. In \textit{Gender and discourse}, ed. Ruth Wodak, 99--119. London: Sage Publications.
%
% Monograph
\bibitem{humlinphil-mono} Cameron, Deborah. 1985. \textit{Feminism and linguistic theory.} New York: St. Martin's Press.
%
% Online Document
\bibitem{humlinphil-online} Dod, Jake. 1999. Effective substances. In: The dictionary of substances and their effects. Royal Society of Chemistry. Available via DIALOG. \\
http://www.rsc.org/dose/title of subordinate document. Cited 15 Jan 1999
%
% Journal article by DOI
\bibitem{humlinphil-DOI} Suleiman, Camelia, Daniel C. O�Connell, and Sabine Kowal. 2002. `If you and I, if we, in this later day, lose that sacred fire...�': Perspective in political interviews. \textit{Journal of Psycholinguistic Research}. doi: 10.1023/A:1015592129296.
%
%
%
\bigskip
%
%
% Use the following syntax and markup for your references if 
% the subject of your book is from the field 
% "Computer Science, Economics, Engineering, Geosciences, Life Sciences"
%
%
% Contribution 
\bibitem{basic-contrib} Brown B, Aaron M (2001) The politics of nature. In: Smith J (ed) The rise of modern genomics, 3rd edn. Wiley, New York 
%
% Online Document
\bibitem{basic-online} Dod J (1999) Effective Substances. In: The dictionary of substances and their effects. Royal Society of Chemistry. Available via DIALOG. \\
\url{http://www.rsc.org/dose/title of subordinate document. Cited 15 Jan 1999}
%
% Journal article by DOI
\bibitem{basic-DOI} Slifka MK, Whitton JL (2000) Clinical implications of dysregulated cytokine production. J Mol Med, doi: 10.1007/s001090000086
%
% Journal article
\bibitem{basic-journal} Smith J, Jones M Jr, Houghton L et al (1999) Future of health insurance. N Engl J Med 965:325--329
%
% Monograph
\bibitem{basic-mono} South J, Blass B (2001) The future of modern genomics. Blackwell, London 
%
\end{thebibliography}

