%%%%%%%%%%%%%%%%%%%%% chapter.tex %%%%%%%%%%%%%%%%%%%%%%%%%%%%%%%%%
%
% sample chapter
%
% Use this file as a template for your own input.
%
%%%%%%%%%%%%%%%%%%%%%%%% Springer-Verlag %%%%%%%%%%%%%%%%%%%%%%%%%%
%\motto{Use the template \emph{chapter.tex} to style the various elements of your chapter content.}



\chapter{Symbols starting with N}
\label{cha:func-ref-N-functions-starting-with-N}

\section*{\texttt{+Need}}
\label{sec:func-ref-N-+Need}


Prefix class for mandatory \texttt{+relation}s. Note that this does not enforce
any requirements by itself, it only returns an error message if the
\texttt{mis>} message is explicitly called, e.g. by GUI functions. See also
\texttt{Database}.


\begin{wideverbatim}
(rel nr (+Need +Key +Number))  # Item number is mandatory
\end{wideverbatim}

 
\section*{\texttt{+Number}}
\label{sec:func-ref-N-+Number}


Class for numeric relations, a subclass of \texttt{+relation}. Accepts an
optional argument for the fixpoint scale (currently not used). See also
\texttt{Database}.


\begin{wideverbatim}
(rel pr (+Number) 2)  # Price, with two decimal places
\end{wideverbatim}

 
\section*{\texttt{(n== `any ..) -> flg}}
\label{sec:func-ref-N-(n== `any ..) -> flg}


Returns \texttt{T} when not all \texttt{any} arguments are the same (pointer
equality). \texttt{(n== `any ..)} is equivalent to \texttt{(not (== `any ..))}. See
also \texttt{==} and \emph{Comparing}.


\begin{wideverbatim}
: (n== 'a 'a)
-> NIL
: (n== (1) (1))
-> T
\end{wideverbatim}

 
\section*{\texttt{(n0 'any) -> flg}}
\label{sec:func-ref-N-(n0 'any) -> flg}


Returns \texttt{T} when \texttt{any} is not a number with value zero.
See also \texttt{=0}, \texttt{lt0}, \texttt{le0}, \texttt{ge0} and
\texttt{gt0}.


\begin{wideverbatim}
: (n0 (- 6 3 2 1))
-> NIL
: (n0 'a)
-> T
\end{wideverbatim}

 
\section*{\texttt{(nT 'any) -> flg}}
\label{sec:func-ref-N-(nT 'any) -> flg}


Returns \texttt{T} when \texttt{any} is not the symbol \texttt{T}. See also
\emph{=T}.

\begin{wideverbatim}
: (nT 0)
-> T
: (nT "T")
-> T
: (nT T)
-> NIL
\end{wideverbatim}

 
\section*{\texttt{(name 'sym ['sym2]) -> sym}}
\label{sec:func-ref-N-(name 'sym ['sym2]) -> sym}


Returns, if \texttt{sym2} is not given, a new transient symbol with the name of
\texttt{sym}. Otherwise \texttt{sym} must be a transient symbol, and its name is
changed to that of \texttt{sym2} (note that this may give inconsistencies if
the symbol is still referred to from other namespaces). See also \texttt{str},
\texttt{sym}, \texttt{symbols}, \texttt{zap} and \texttt{intern}.


\begin{wideverbatim}
: (name 'abc)
-> "abc"
: (name "abc")
-> "abc"
: (name '{abc})
-> "abc"
: (name (new))
-> NIL
: (de foo (Lst) (car Lst))  # 'foo' calls 'car'
-> foo
: (intern (name (zap 'car) "xxx"))  # Globally change the name of 'car'
-> xxx
: (xxx (1 2 3))
-> 1
: (pp 'foo)
(de foo (Lst)
   (xxx Lst) )                      # Name changed
-> foo
: (foo (1 2 3))                     # 'foo' still works
-> 1
: (car (1 2 3))                     # Reader returns a new 'car' symbol
!? (car (1 2 3))
car -- Undefined
?
\end{wideverbatim}

 
\section*{\texttt{(nand 'any ..) -> flg}}
\label{sec:func-ref-N-(nand 'any ..) -> flg}


Logical NAND. The expressions \texttt{any} are evaluated from left to right. If
\texttt{NIL} is encountered, \texttt{T} is returned immediately. Else \texttt{NIL} is
returned. \texttt{(nand ..)} is equivalent to \texttt{(not (and ..))}.


\begin{wideverbatim}
: (nand (lt0 7) (read))
-> T
: (nand (lt0 -7) (read))
abc
-> NIL
: (nand (lt0 -7) (read))
NIL
-> T
\end{wideverbatim}

 
\section*{\texttt{(native 'cnt1|sym1 'cnt2|sym2 'sym|lst 'any ..) -> any}}
\label{sec:func-ref-N-(native 'cnt1|sym1 'cnt2|sym2 'sym|lst 'any ..) -> any}


(64-bit version only) Calls a native C function. The first argument
should specify a shared object library, either \texttt{''@''} (the current main
program), \texttt{sym1} (a library path name), or \texttt{cnt1} (a library handle
obtained by a previous call). The second argument should be a symbol
name \texttt{sym2}, or a function pointer \texttt{cnt2} obtained by a previous call).
Practically, the first two arguments will be always passed as transient
symbols, which will get the library handle and function pointer assigned
as values to be cached and used in subsequent calls. The third \texttt{sym|lst}
argument is a return value specification, while all following arguments
are the arguments to the native function.

The return value specification may either be one of the atoms


\begin{wideverbatim}
 NIL   void
 B     byte     # Byte (unsigned 8 bit)
 C     char     # Character (UTF-8, 1-3 bytes)
 I     int      # Integer (signed 32 bit)
 N     long     # Long or pointer (signed 64 bit)
 S     string   # String (UTF-8)
-1.0   float    # Scaled fixpoint number
+1.0   double   # Scaled fixpoint number
\end{wideverbatim}

or nested lists of these atoms with size specifications to denote arrays
and structures, e.g.


\begin{wideverbatim}
(N . 4)        # long[4];           -> (1 2 3 4)
(N (C . 4))    # {long; char[4];}   -> (1234 ("a" "b" "c" NIL))
(N (B . 7))    # {long; byte[7];}   -> (1234 (1 2 3 4 5 6 7))
\end{wideverbatim}

Arguments can be

\begin{itemize}
\item integers (up to 64-bit) or pointers, passed as numbers
\item fixpoint numbers, passed as cons pairs consisting of a the value and
   the scale. If the scale is positive, the number is passed as a
   \texttt{double}, otherwise as a \texttt{float}.
\item strings, passed as symbols, or
\item structures, passed as lists with
\begin{itemize}
\item a variable in the CAR (to recieve the returned structure data,
      ignored when the CAR is \texttt{NIL})
\item a cons pair for the size and value specification in the CADR (see
      above), and
\item an optional sequence of initialization items in the CDDR, where
      each may be
\begin{itemize}
\item a positive number, stored as an unsigned byte value
\item a negative number, whose absolute value is stored as an
         unsigned integer
\item a pair \texttt{(num . cnt)} where `\texttt{num}' is stored in a field of
         `\texttt{cnt}' bytes
\item a pair \texttt{(sym . cnt)} where `\texttt{sym}' is stored as a
         null-terminated string in a field of `\texttt{cnt}' bytes
\end{itemize}
If the last CDR of the initialization sequence is a number, it is
      used as a fill-byte value for the remaining space in the
      structure.
\end{itemize}
\end{itemize}

\texttt{native} takes care of allocating memory for strings, arrays or
structures, and frees that memory when done.

The number of fixpoint arguments is limited to six. For NaN or negative
infinity \texttt{NIL}, and for positive infinity \texttt{T} is returned.


\begin{wideverbatim}
: (native "@" "getenv" 'S "TERM")  # Same as (sys "TERM")
-> "xterm"

: (native "@" "printf" 'I "abc%d%s^J" (+ 3 4) (pack "X" "Y" "Z"))
abc7XYZ
-> 8

: (native "@" "printf" 'I "This is %.3f^J" (123456 . 1000))
This is 123.456
-> 16

: (use Tim
   (native "@" "time" NIL '(Tim (8 B . 8)))  # time_t 8   # Get time_t structure
   (native "@" "localtime" '(I . 9) (cons NIL (8) Tim)) ) # Read local time
-> (32 18 13 31 11 109 4 364 0)  # 13:18:32, Dec. 31st, 2009
\end{wideverbatim}

The C function may in turn call a function


\begin{wideverbatim}
long lisp(char*, long, long, long, long, long);
\end{wideverbatim}

which accepts a symbol name as the first argument, and up to 5 numbers.
\texttt{lisp()} calls that symbol with the five numbers, and expects a numeric
return value. ``Numbers'' in this context are 64-bit scalars, and may not
only represent integers, but also pointers or other encoded data. See
also \texttt{errno} and \texttt{lisp}.

 
\section*{\texttt{(need 'cnt ['lst ['any]]) -> lst}}
\label{sec:func-ref-N-(need 'cnt ['lst ['any]]) -> lst}


\texttt{(need 'cnt ['num|sym]) -> lst}

Produces a list of at least \texttt{cnt} elements. When called without optional
arguments, a list of \texttt{cnt} \texttt{NIL}'s is returned. When \texttt{lst} is given, it
is extended to the left (if \texttt{cnt} is positive) or (destructively) to the
right (if \texttt{cnt} is negative) with \texttt{any} elements. In the second form, a
list of \texttt{cnt} atomic values is returned. See also \texttt{range}.


\begin{wideverbatim}
: (need 5)
-> (NIL NIL NIL NIL NIL)  # Allocate 5 cells
: (need 5 '(a b c))
-> (NIL NIL a b c)
: (need -5 '(a b c))
-> (a b c NIL NIL)
: (need 5 '(a b c) " ")  # String alignment
-> (" " " " a b c)
: (need 7 0)
-> (0 0 0 0 0 0 0)
\end{wideverbatim}

 
\section*{\texttt{(new ['flg|num] ['typ ['any ..]]) -> obj}}
\label{sec:func-ref-N-(new ['flg|num] ['typ ['any ..]]) -> obj}


Creates and returns a new object. If \texttt{flg} is given and non-\texttt{NIL}, the
new object will be an external symbol (created in database file 1 if
\texttt{T}, or in the corresponding database file if \texttt{num} is given). \texttt{typ}
(typically a list of classes) is assigned to the \texttt{VAL}, and the initial
\texttt{T} message is sent with the arguments \texttt{any} to the new object. If no
\texttt{T} message is defined for the classes in \texttt{typ} or their superclasses,
the \texttt{any} arguments should evaluate to alternating keys and values for
the initialization of the new object. See also \texttt{box}, \texttt{object}, \texttt{class},
\texttt{type}, \texttt{isa}, \texttt{send} and \texttt{Database}.


\begin{wideverbatim}
: (new)
-> $134426427
: (new T '(+Address))
-> {1A;3}
\end{wideverbatim}

 
\section*{\texttt{(new! 'typ ['any ..]) -> obj}}
\label{sec:func-ref-N-(new! 'typ ['any ..]) -> obj}


\emph{Transaction} wrapper function for \texttt{new}.
\texttt{(new! '(+Cls) 'key 'val ...)} is equivalent to
\texttt{(dbSync) (new (db: +Cls) '(+Cls) 'key 'val ...) (commit 'upd)}. See
also \texttt{set!}, \texttt{put!} and \texttt{inc!}.


\begin{wideverbatim}
: (new! '(+Item)  # Create a new item
   'nr 2                      # Item number
   'nm "Spare Part"           # Description
   'sup (db 'nr '+CuSu 2)     # Supplier
   'inv 100                   # Inventory
   pr 12.50 )                 # Price
\end{wideverbatim}

 
\section*{\texttt{(next) -> any}}
\label{sec:func-ref-N-(next) -> any}


Can only be used inside functions with a variable number of arguments
(with \texttt{@}). Returns the next argument from the internal list. See also
\texttt{args}, \texttt{arg}, \texttt{rest}, and \texttt{pass}.


\begin{wideverbatim}
: (de foo @ (println (next)))          # Print next argument
-> foo
: (foo)
NIL
-> NIL
: (foo 123)
123
-> 123
\end{wideverbatim}

 
\section*{\texttt{(nil . prg) -> NIL}}
\label{sec:func-ref-N-(nil . prg) -> NIL}


Executes \texttt{prg}, and returns \texttt{NIL}. See also \texttt{t}, \texttt{prog}, \texttt{prog1} and
\texttt{prog2}.


\begin{wideverbatim}
: (nil (println 'OK))
OK
-> NIL
\end{wideverbatim}

 
\section*{\texttt{nil/1}}
\label{sec:func-ref-N-nil/1}


\emph{Pilog} predicate expects an argument variable, and
succeeds if that variable is bound to \texttt{NIL}. See also \texttt{not/1}.


\begin{wideverbatim}
: (? @X NIL (nil @X))
 @X=NIL
-> NIL
\end{wideverbatim}

 
\section*{\texttt{(noLint 'sym)}}
\label{sec:func-ref-N-(noLint 'sym)}


\texttt{(noLint 'sym|(sym . cls) 'sym2)}

Excludes the check for a function definition of \texttt{sym} (in the first
form), or for variable binding and usage of \texttt{sym2} in the function
definition, file contents or method body of \texttt{sym} (second form), during
calls to \texttt{lint}. See also \texttt{lintAll}.


\begin{wideverbatim}
: (de foo ()
   (bar FreeVariable) )
-> foo
: (lint 'foo)
-> ((def bar) (bnd FreeVariable))
: (noLint 'bar)
-> bar
: (noLint 'foo 'FreeVariable)
-> (foo . FreeVariable)
: (lint 'foo)
-> NIL
\end{wideverbatim}

 
\section*{\texttt{(nond ('any1 . prg1) ('any2 . prg2) ..) -> any}}
\label{sec:func-ref-N-(nond ('any1 . prg1) ('any2 . prg2) ..) -> any}


Negated (``non-cond'') multi-way conditional: If any of the \texttt{anyN}
conditions evaluates to \texttt{NIL}, \texttt{prgN} is executed and the result
returned. Otherwise (all conditions evaluate to non-\texttt{NIL}), \texttt{NIL} is
returned. See also \texttt{cond}, \texttt{ifn} and \texttt{unless}.


\begin{wideverbatim}
: (nond
   ((= 3 3) (println 1))
   ((= 3 4) (println 2))
   (NIL (println 3)) )
2
-> 2
\end{wideverbatim}

 
\section*{\texttt{(nor 'any ..) -> flg}}
\label{sec:func-ref-N-(nor 'any ..) -> flg}


Logical NOR. The expressions \texttt{any} are evaluated from left to right. If
a non-\texttt{NIL} value is encountered, \texttt{NIL} is returned immediately. Else
\texttt{T} is returned. \texttt{(nor ..)} is equivalent to \texttt{(not (or ..))}.


\begin{wideverbatim}
: (nor (lt0 7) (= 3 4))
-> T
\end{wideverbatim}

 
\section*{\texttt{(not 'any) -> flg}}
\label{sec:func-ref-N-(not 'any) -> flg}


Logical negation. Returns \texttt{T} if \texttt{any} evaluates to \texttt{NIL}.


\begin{wideverbatim}
: (not (== 'a 'a))
-> NIL
: (not (get 'a 'a))
-> T
\end{wideverbatim}

 
\section*{\texttt{not/1}}
\label{sec:func-ref-N-not/1}


\emph{Pilog} predicate that succeeds if and only if the
goal cannot be proven. See also \texttt{nil/1}, \texttt{true/0} and \texttt{fail/0}.


\begin{wideverbatim}
: (? (equal 3 4))
-> NIL
: (? (not (equal 3 4)))
-> T
\end{wideverbatim}

 
\section*{\texttt{(nth 'lst 'cnt ..) -> lst}}
\label{sec:func-ref-N-(nth 'lst 'cnt ..) -> lst}


Returns the tail of \texttt{lst} starting from the \texttt{cnt}'th element of \texttt{lst}.
Successive \texttt{cnt} arguments operate on the results in the same way.
\texttt{(nth 'lst 2)} is equivalent to \texttt{(cdr 'lst)}. See also \texttt{get}.


\begin{wideverbatim}
: (nth '(a b c d) 2)
-> (b c d)
: (nth '(a (b c) d) 2 2)
-> (c)
: (cdadr '(a (b c) d))
-> (c)
\end{wideverbatim}

 
\section*{\texttt{(num? 'any) -> num | NIL}}
\label{sec:func-ref-N-(num? 'any) -> num | NIL}


Returns \texttt{any} when the argument \texttt{any} is a number, otherwise \texttt{NIL}.


\begin{wideverbatim}
: (num? 123)
-> 123
: (num? (1 2 3))
-> NIL
\end{wideverbatim}



% %%%%%%%%%%%%%%%%%%%%%%%% referenc.tex %%%%%%%%%%%%%%%%%%%%%%%%%%%%%%
% sample references
% %
% Use this file as a template for your own input.
%
%%%%%%%%%%%%%%%%%%%%%%%% Springer-Verlag %%%%%%%%%%%%%%%%%%%%%%%%%%
%
% BibTeX users please use
% \bibliographystyle{}
% \bibliography{}
%
\biblstarthook{In view of the parallel print and (chapter-wise) online publication of your book at \url{www.springerlink.com} it has been decided that -- as a genreral rule --  references should be sorted chapter-wise and placed at the end of the individual chapters. However, upon agreement with your contact at Springer you may list your references in a single seperate chapter at the end of your book. Deactivate the class option \texttt{sectrefs} and the \texttt{thebibliography} environment will be put out as a chapter of its own.\\\indent
References may be \textit{cited} in the text either by number (preferred) or by author/year.\footnote{Make sure that all references from the list are cited in the text. Those not cited should be moved to a separate \textit{Further Reading} section or chapter.} The reference list should ideally be \textit{sorted} in alphabetical order -- even if reference numbers are used for the their citation in the text. If there are several works by the same author, the following order should be used: 
\begin{enumerate}
\item all works by the author alone, ordered chronologically by year of publication
\item all works by the author with a coauthor, ordered alphabetically by coauthor
\item all works by the author with several coauthors, ordered chronologically by year of publication.
\end{enumerate}
The \textit{styling} of references\footnote{Always use the standard abbreviation of a journal's name according to the ISSN \textit{List of Title Word Abbreviations}, see \url{http://www.issn.org/en/node/344}} depends on the subject of your book:
\begin{itemize}
\item The \textit{two} recommended styles for references in books on \textit{mathematical, physical, statistical and computer sciences} are depicted in ~\cite{science-contrib, science-online, science-mono, science-journal, science-DOI} and ~\cite{phys-online, phys-mono, phys-journal, phys-DOI, phys-contrib}.
\item Examples of the most commonly used reference style in books on \textit{Psychology, Social Sciences} are~\cite{psysoc-mono, psysoc-online,psysoc-journal, psysoc-contrib, psysoc-DOI}.
\item Examples for references in books on \textit{Humanities, Linguistics, Philosophy} are~\cite{humlinphil-journal, humlinphil-contrib, humlinphil-mono, humlinphil-online, humlinphil-DOI}.
\item Examples of the basic Springer style used in publications on a wide range of subjects such as \textit{Computer Science, Economics, Engineering, Geosciences, Life Sciences, Medicine, Biomedicine} are ~\cite{basic-contrib, basic-online, basic-journal, basic-DOI, basic-mono}. 
\end{itemize}
}

\begin{thebibliography}{99.}%
% and use \bibitem to create references.
%
% Use the following syntax and markup for your references if 
% the subject of your book is from the field 
% "Mathematics, Physics, Statistics, Computer Science"
%
% Contribution 
\bibitem{science-contrib} Broy, M.: Software engineering --- from auxiliary to key technologies. In: Broy, M., Dener, E. (eds.) Software Pioneers, pp. 10-13. Springer, Heidelberg (2002)
%
% Online Document
\bibitem{science-online} Dod, J.: Effective substances. In: The Dictionary of Substances and Their Effects. Royal Society of Chemistry (1999) Available via DIALOG. \\
\url{http://www.rsc.org/dose/title of subordinate document. Cited 15 Jan 1999}
%
% Monograph
\bibitem{science-mono} Geddes, K.O., Czapor, S.R., Labahn, G.: Algorithms for Computer Algebra. Kluwer, Boston (1992) 
%
% Journal article
\bibitem{science-journal} Hamburger, C.: Quasimonotonicity, regularity and duality for nonlinear systems of partial differential equations. Ann. Mat. Pura. Appl. \textbf{169}, 321--354 (1995)
%
% Journal article by DOI
\bibitem{science-DOI} Slifka, M.K., Whitton, J.L.: Clinical implications of dysregulated cytokine production. J. Mol. Med. (2000) doi: 10.1007/s001090000086 
%
\bigskip

% Use the following (APS) syntax and markup for your references if 
% the subject of your book is from the field 
% "Mathematics, Physics, Statistics, Computer Science"
%
% Online Document
\bibitem{phys-online} J. Dod, in \textit{The Dictionary of Substances and Their Effects}, Royal Society of Chemistry. (Available via DIALOG, 1999), 
\url{http://www.rsc.org/dose/title of subordinate document. Cited 15 Jan 1999}
%
% Monograph
\bibitem{phys-mono} H. Ibach, H. L\"uth, \textit{Solid-State Physics}, 2nd edn. (Springer, New York, 1996), pp. 45-56 
%
% Journal article
\bibitem{phys-journal} S. Preuss, A. Demchuk Jr., M. Stuke, Appl. Phys. A \textbf{61}
%
% Journal article by DOI
\bibitem{phys-DOI} M.K. Slifka, J.L. Whitton, J. Mol. Med., doi: 10.1007/s001090000086
%
% Contribution 
\bibitem{phys-contrib} S.E. Smith, in \textit{Neuromuscular Junction}, ed. by E. Zaimis. Handbook of Experimental Pharmacology, vol 42 (Springer, Heidelberg, 1976), p. 593
%
\bigskip
%
% Use the following syntax and markup for your references if 
% the subject of your book is from the field 
% "Psychology, Social Sciences"
%
%
% Monograph
\bibitem{psysoc-mono} Calfee, R.~C., \& Valencia, R.~R. (1991). \textit{APA guide to preparing manuscripts for journal publication.} Washington, DC: American Psychological Association.
%
% Online Document
\bibitem{psysoc-online} Dod, J. (1999). Effective substances. In: The dictionary of substances and their effects. Royal Society of Chemistry. Available via DIALOG. \\
\url{http://www.rsc.org/dose/Effective substances.} Cited 15 Jan 1999.
%
% Journal article
\bibitem{psysoc-journal} Harris, M., Karper, E., Stacks, G., Hoffman, D., DeNiro, R., Cruz, P., et al. (2001). Writing labs and the Hollywood connection. \textit{J Film} Writing, 44(3), 213--245.
%
% Contribution 
\bibitem{psysoc-contrib} O'Neil, J.~M., \& Egan, J. (1992). Men's and women's gender role journeys: Metaphor for healing, transition, and transformation. In B.~R. Wainrig (Ed.), \textit{Gender issues across the life cycle} (pp. 107--123). New York: Springer.
%
% Journal article by DOI
\bibitem{psysoc-DOI}Kreger, M., Brindis, C.D., Manuel, D.M., Sassoubre, L. (2007). Lessons learned in systems change initiatives: benchmarks and indicators. \textit{American Journal of Community Psychology}, doi: 10.1007/s10464-007-9108-14.
%
%
% Use the following syntax and markup for your references if 
% the subject of your book is from the field 
% "Humanities, Linguistics, Philosophy"
%
\bigskip
%
% Journal article
\bibitem{humlinphil-journal} Alber John, Daniel C. O'Connell, and Sabine Kowal. 2002. Personal perspective in TV interviews. \textit{Pragmatics} 12:257--271
%
% Contribution 
\bibitem{humlinphil-contrib} Cameron, Deborah. 1997. Theoretical debates in feminist linguistics: Questions of sex and gender. In \textit{Gender and discourse}, ed. Ruth Wodak, 99--119. London: Sage Publications.
%
% Monograph
\bibitem{humlinphil-mono} Cameron, Deborah. 1985. \textit{Feminism and linguistic theory.} New York: St. Martin's Press.
%
% Online Document
\bibitem{humlinphil-online} Dod, Jake. 1999. Effective substances. In: The dictionary of substances and their effects. Royal Society of Chemistry. Available via DIALOG. \\
http://www.rsc.org/dose/title of subordinate document. Cited 15 Jan 1999
%
% Journal article by DOI
\bibitem{humlinphil-DOI} Suleiman, Camelia, Daniel C. O�Connell, and Sabine Kowal. 2002. `If you and I, if we, in this later day, lose that sacred fire...�': Perspective in political interviews. \textit{Journal of Psycholinguistic Research}. doi: 10.1023/A:1015592129296.
%
%
%
\bigskip
%
%
% Use the following syntax and markup for your references if 
% the subject of your book is from the field 
% "Computer Science, Economics, Engineering, Geosciences, Life Sciences"
%
%
% Contribution 
\bibitem{basic-contrib} Brown B, Aaron M (2001) The politics of nature. In: Smith J (ed) The rise of modern genomics, 3rd edn. Wiley, New York 
%
% Online Document
\bibitem{basic-online} Dod J (1999) Effective Substances. In: The dictionary of substances and their effects. Royal Society of Chemistry. Available via DIALOG. \\
\url{http://www.rsc.org/dose/title of subordinate document. Cited 15 Jan 1999}
%
% Journal article by DOI
\bibitem{basic-DOI} Slifka MK, Whitton JL (2000) Clinical implications of dysregulated cytokine production. J Mol Med, doi: 10.1007/s001090000086
%
% Journal article
\bibitem{basic-journal} Smith J, Jones M Jr, Houghton L et al (1999) Future of health insurance. N Engl J Med 965:325--329
%
% Monograph
\bibitem{basic-mono} South J, Blass B (2001) The future of modern genomics. Blackwell, London 
%
\end{thebibliography}

