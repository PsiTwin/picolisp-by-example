%%%%%%%%%%%%%%%%%%%%% chapter.tex %%%%%%%%%%%%%%%%%%%%%%%%%%%%%%%%%
%
% sample chapter
%
% Use this file as a template for your own input.
%
%%%%%%%%%%%%%%%%%%%%%%%% Springer-Verlag %%%%%%%%%%%%%%%%%%%%%%%%%%
%\motto{Use the template \emph{chapter.tex} to style the various elements of your chapter content.}

\chapter{Rosetta Code Tasks starting with Q}

\section*{Queue/Definition}

\textbf{Data Structure}\\ This illustrates a data structure, a means of
storing data within a program.

You may see other such structures in the
\emph{Data Structures} category.

% % animated graphic, no way to print
% \begin{figure}[H]
% \centering
% \includegraphics[scale=.6]{graphics/Fifo.gif}
% \end{figure}

\textbf{Task}

Implement a FIFO queue. Elements are added at one side and popped from
the other in the order of insertion.

Operations:

\begin{itemize}
\item
  push (aka \emph{enqueue}) - add element
\item
  pop (aka \emph{dequeue}) - pop first element
\item
  empty - return truth value when empty
\end{itemize}

Errors:

\begin{itemize}
\item
  handle the error of trying to pop from an empty queue (behavior
  depends on the language and platform)
\end{itemize}

See \emph{FIFO (usage)} for the built-in FIFO or
queue of your language or standard library.


\begin{wideverbatim}

The built-in function 'fifo' maintains a queue in a circular list, with direct
access to the first and the last cell

(off Queue)                # Clear Queue
(fifo 'Queue 1)            # Store number '1'
(fifo 'Queue 'abc)         # an internal symbol 'abc'
(fifo 'Queue "abc")        # a transient symbol "abc"
(fifo 'Queue '(a b c))     # and a list (a b c)
Queue                      # Show the queue

Output:

->((a b c) 1 abc "abc" .)

\end{wideverbatim}

\pagebreak{}
\section*{Queue/Usage}


\textbf{Data Structure}\\ This illustrates a data structure, a means of
storing data within a program.

You may see other such structures in the \emph{Data Structures}
category.

% % animated graphic, no way to print
% \begin{figure}[H]
% \centering
% \includegraphics[scale=.6]{graphics/Fifo.gif}
% \end{figure}

\textbf{Task}

Create a queue data structure and demonstrate its operations. (For
implementations of queues, see the \emph{FIFO} task.)

Operations:

\begin{itemize}
\item
  push (aka \emph{enqueue}) - add element
\item
  pop (aka \emph{dequeue}) - pop first element
\item
  empty - return truth value when empty
\end{itemize}


\begin{wideverbatim}

Using the implementation from [[FIFO]]:
(println (fifo 'Queue))    # Retrieve the number '1'
(println (fifo 'Queue))    # Retrieve an internal symbol 'abc'
(println (fifo 'Queue))    # Retrieve a transient symbol "abc"
(println (fifo 'Queue))    # and a list (abc)
(println (fifo 'Queue))    # Queue is empty -> NIL

Output:

1
abc
"abc"
(a b c)
NIL

\end{wideverbatim}

\pagebreak{}
\section*{Quine}


A \href{http://en.wikipedia.org/wiki/Quine\_(computing)}{Quine} is a
self-referential program that can, without any external access, output
its own source. It is named after the
\href{http://en.wikipedia.org/wiki/Willard\_Van\_Orman\_Quine}{philosopher
and logician} who studied self-reference and quoting in natural
language, as for example in the paradox ``\,`Yields falsehood when
preceded by its quotation' yields falsehood when preceded by its
quotation.''

``Source'' has one of two meanings. It can refer to the text-based
program source. For languages in which program source is represented as
a data structure, ``source'' may refer to the data structure: quines in
these languages fall into two categories: programs which print a textual
representation of themselves, or expressions which evaluate to a data
structure which is equivalent to that expression.

The usual way to code a Quine works similarly to this paradox: The
program consists of two identical parts, once as plain code and once
\emph{quoted} in some way (for example, as a character string, or a
literal data structure). The plain code then accesses the quoted code
and prints it out twice, once unquoted and once with the proper
quotation marks added. Often, the plain code and the quoted code have to
be nested.

Write a program that outputs its own source code in this way. If the
language allows it, you may add a variant that accesses the code
directly. You are not allowed to read any external files with the source
code. The program should also contain some sort of self-reference, so
constant expressions which return their own value which some top-level
interpreter will print out. Empty programs producing no output are not
allowed.

There are several difficulties that one runs into when writing a quine,
mostly dealing with quoting:

\begin{itemize}
\item
  Part of the code usually needs to be stored as a string or structural
  literal in the language, which needs to be quoted somehow. However,
  including quotation marks in the string literal itself would be
  troublesome because it requires them to be escaped, which then
  necessitates the escaping character (e.g. a backslash) in the string,
  which itself usually needs to be escaped, and so on.

  \begin{itemize}
  \item
    Some languages have a function for getting the ``source code
    representation'' of a string (i.e. adds quotation marks, etc.); in
    these languages, this can be used to circumvent the quoting problem.
  \item
    Another solution is to construct the quote character from its
    \emph{character code}, without having to
    write the quote character itself. Then the character is inserted
    into the string at the appropriate places. The ASCII code for
    double-quote is 34, and for single-quote is 39.
  \end{itemize}
\item
  Newlines in the program may have to be reproduced as newlines in the
  string, which usually requires some kind of escape sequence (e.g.
  ``\textbackslash{}n''). This causes the same problem as above, where
  the escaping character needs to itself be escaped, etc.

  \begin{itemize}
  \item
    If the language has a way of getting the ``source code
    representation'', it usually handles the escaping of characters, so
    this is not a problem.
  \item
    Some languages allow you to have a string literal that spans
    multiple lines, which embeds the newlines into the string without
    escaping.
  \item
    Write the entire program on one line, for free-form languages (as
    you can see for some of the solutions here, they run off the edge of
    the screen), thus removing the need for newlines. However, this may
    be unacceptable as some languages require a newline at the end of
    the file; and otherwise it is still generally good style to have a
    newline at the end of a file. (The task is not clear on whether a
    newline is required at the end of the file.) Some languages have a
    print statement that appends a newline; which solves the
    newline-at-the-end issue; but others do not.
  \end{itemize}
\end{itemize}

See the nostalgia note under Fortran.


\begin{wideverbatim}

Using 'quote' (= 'lambda' in PicoLisp)

('((X) (list (lit X) (lit X))) '((X) (list (lit X) (lit X))))

Output:

-> ('((X) (list (lit X) (lit X))) '((X) (list (lit X) (lit X))))

Using 'let'

(let X '(list 'let 'X (lit X) X) (list 'let 'X (lit X) X))

Output:

-> (let X '(list 'let 'X (lit X) X) (list 'let 'X (lit X) X))

\end{wideverbatim}



% %%%%%%%%%%%%%%%%%%%%%%%% referenc.tex %%%%%%%%%%%%%%%%%%%%%%%%%%%%%%
% sample references
% %
% Use this file as a template for your own input.
%
%%%%%%%%%%%%%%%%%%%%%%%% Springer-Verlag %%%%%%%%%%%%%%%%%%%%%%%%%%
%
% BibTeX users please use
% \bibliographystyle{}
% \bibliography{}
%
\biblstarthook{In view of the parallel print and (chapter-wise) online publication of your book at \url{www.springerlink.com} it has been decided that -- as a genreral rule --  references should be sorted chapter-wise and placed at the end of the individual chapters. However, upon agreement with your contact at Springer you may list your references in a single seperate chapter at the end of your book. Deactivate the class option \texttt{sectrefs} and the \texttt{thebibliography} environment will be put out as a chapter of its own.\\\indent
References may be \textit{cited} in the text either by number (preferred) or by author/year.\footnote{Make sure that all references from the list are cited in the text. Those not cited should be moved to a separate \textit{Further Reading} section or chapter.} The reference list should ideally be \textit{sorted} in alphabetical order -- even if reference numbers are used for the their citation in the text. If there are several works by the same author, the following order should be used: 
\begin{enumerate}
\item all works by the author alone, ordered chronologically by year of publication
\item all works by the author with a coauthor, ordered alphabetically by coauthor
\item all works by the author with several coauthors, ordered chronologically by year of publication.
\end{enumerate}
The \textit{styling} of references\footnote{Always use the standard abbreviation of a journal's name according to the ISSN \textit{List of Title Word Abbreviations}, see \url{http://www.issn.org/en/node/344}} depends on the subject of your book:
\begin{itemize}
\item The \textit{two} recommended styles for references in books on \textit{mathematical, physical, statistical and computer sciences} are depicted in ~\cite{science-contrib, science-online, science-mono, science-journal, science-DOI} and ~\cite{phys-online, phys-mono, phys-journal, phys-DOI, phys-contrib}.
\item Examples of the most commonly used reference style in books on \textit{Psychology, Social Sciences} are~\cite{psysoc-mono, psysoc-online,psysoc-journal, psysoc-contrib, psysoc-DOI}.
\item Examples for references in books on \textit{Humanities, Linguistics, Philosophy} are~\cite{humlinphil-journal, humlinphil-contrib, humlinphil-mono, humlinphil-online, humlinphil-DOI}.
\item Examples of the basic Springer style used in publications on a wide range of subjects such as \textit{Computer Science, Economics, Engineering, Geosciences, Life Sciences, Medicine, Biomedicine} are ~\cite{basic-contrib, basic-online, basic-journal, basic-DOI, basic-mono}. 
\end{itemize}
}

\begin{thebibliography}{99.}%
% and use \bibitem to create references.
%
% Use the following syntax and markup for your references if 
% the subject of your book is from the field 
% "Mathematics, Physics, Statistics, Computer Science"
%
% Contribution 
\bibitem{science-contrib} Broy, M.: Software engineering --- from auxiliary to key technologies. In: Broy, M., Dener, E. (eds.) Software Pioneers, pp. 10-13. Springer, Heidelberg (2002)
%
% Online Document
\bibitem{science-online} Dod, J.: Effective substances. In: The Dictionary of Substances and Their Effects. Royal Society of Chemistry (1999) Available via DIALOG. \\
\url{http://www.rsc.org/dose/title of subordinate document. Cited 15 Jan 1999}
%
% Monograph
\bibitem{science-mono} Geddes, K.O., Czapor, S.R., Labahn, G.: Algorithms for Computer Algebra. Kluwer, Boston (1992) 
%
% Journal article
\bibitem{science-journal} Hamburger, C.: Quasimonotonicity, regularity and duality for nonlinear systems of partial differential equations. Ann. Mat. Pura. Appl. \textbf{169}, 321--354 (1995)
%
% Journal article by DOI
\bibitem{science-DOI} Slifka, M.K., Whitton, J.L.: Clinical implications of dysregulated cytokine production. J. Mol. Med. (2000) doi: 10.1007/s001090000086 
%
\bigskip

% Use the following (APS) syntax and markup for your references if 
% the subject of your book is from the field 
% "Mathematics, Physics, Statistics, Computer Science"
%
% Online Document
\bibitem{phys-online} J. Dod, in \textit{The Dictionary of Substances and Their Effects}, Royal Society of Chemistry. (Available via DIALOG, 1999), 
\url{http://www.rsc.org/dose/title of subordinate document. Cited 15 Jan 1999}
%
% Monograph
\bibitem{phys-mono} H. Ibach, H. L\"uth, \textit{Solid-State Physics}, 2nd edn. (Springer, New York, 1996), pp. 45-56 
%
% Journal article
\bibitem{phys-journal} S. Preuss, A. Demchuk Jr., M. Stuke, Appl. Phys. A \textbf{61}
%
% Journal article by DOI
\bibitem{phys-DOI} M.K. Slifka, J.L. Whitton, J. Mol. Med., doi: 10.1007/s001090000086
%
% Contribution 
\bibitem{phys-contrib} S.E. Smith, in \textit{Neuromuscular Junction}, ed. by E. Zaimis. Handbook of Experimental Pharmacology, vol 42 (Springer, Heidelberg, 1976), p. 593
%
\bigskip
%
% Use the following syntax and markup for your references if 
% the subject of your book is from the field 
% "Psychology, Social Sciences"
%
%
% Monograph
\bibitem{psysoc-mono} Calfee, R.~C., \& Valencia, R.~R. (1991). \textit{APA guide to preparing manuscripts for journal publication.} Washington, DC: American Psychological Association.
%
% Online Document
\bibitem{psysoc-online} Dod, J. (1999). Effective substances. In: The dictionary of substances and their effects. Royal Society of Chemistry. Available via DIALOG. \\
\url{http://www.rsc.org/dose/Effective substances.} Cited 15 Jan 1999.
%
% Journal article
\bibitem{psysoc-journal} Harris, M., Karper, E., Stacks, G., Hoffman, D., DeNiro, R., Cruz, P., et al. (2001). Writing labs and the Hollywood connection. \textit{J Film} Writing, 44(3), 213--245.
%
% Contribution 
\bibitem{psysoc-contrib} O'Neil, J.~M., \& Egan, J. (1992). Men's and women's gender role journeys: Metaphor for healing, transition, and transformation. In B.~R. Wainrig (Ed.), \textit{Gender issues across the life cycle} (pp. 107--123). New York: Springer.
%
% Journal article by DOI
\bibitem{psysoc-DOI}Kreger, M., Brindis, C.D., Manuel, D.M., Sassoubre, L. (2007). Lessons learned in systems change initiatives: benchmarks and indicators. \textit{American Journal of Community Psychology}, doi: 10.1007/s10464-007-9108-14.
%
%
% Use the following syntax and markup for your references if 
% the subject of your book is from the field 
% "Humanities, Linguistics, Philosophy"
%
\bigskip
%
% Journal article
\bibitem{humlinphil-journal} Alber John, Daniel C. O'Connell, and Sabine Kowal. 2002. Personal perspective in TV interviews. \textit{Pragmatics} 12:257--271
%
% Contribution 
\bibitem{humlinphil-contrib} Cameron, Deborah. 1997. Theoretical debates in feminist linguistics: Questions of sex and gender. In \textit{Gender and discourse}, ed. Ruth Wodak, 99--119. London: Sage Publications.
%
% Monograph
\bibitem{humlinphil-mono} Cameron, Deborah. 1985. \textit{Feminism and linguistic theory.} New York: St. Martin's Press.
%
% Online Document
\bibitem{humlinphil-online} Dod, Jake. 1999. Effective substances. In: The dictionary of substances and their effects. Royal Society of Chemistry. Available via DIALOG. \\
http://www.rsc.org/dose/title of subordinate document. Cited 15 Jan 1999
%
% Journal article by DOI
\bibitem{humlinphil-DOI} Suleiman, Camelia, Daniel C. O�Connell, and Sabine Kowal. 2002. `If you and I, if we, in this later day, lose that sacred fire...�': Perspective in political interviews. \textit{Journal of Psycholinguistic Research}. doi: 10.1023/A:1015592129296.
%
%
%
\bigskip
%
%
% Use the following syntax and markup for your references if 
% the subject of your book is from the field 
% "Computer Science, Economics, Engineering, Geosciences, Life Sciences"
%
%
% Contribution 
\bibitem{basic-contrib} Brown B, Aaron M (2001) The politics of nature. In: Smith J (ed) The rise of modern genomics, 3rd edn. Wiley, New York 
%
% Online Document
\bibitem{basic-online} Dod J (1999) Effective Substances. In: The dictionary of substances and their effects. Royal Society of Chemistry. Available via DIALOG. \\
\url{http://www.rsc.org/dose/title of subordinate document. Cited 15 Jan 1999}
%
% Journal article by DOI
\bibitem{basic-DOI} Slifka MK, Whitton JL (2000) Clinical implications of dysregulated cytokine production. J Mol Med, doi: 10.1007/s001090000086
%
% Journal article
\bibitem{basic-journal} Smith J, Jones M Jr, Houghton L et al (1999) Future of health insurance. N Engl J Med 965:325--329
%
% Monograph
\bibitem{basic-mono} South J, Blass B (2001) The future of modern genomics. Blackwell, London 
%
\end{thebibliography}

