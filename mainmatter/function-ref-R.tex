%%%%%%%%%%%%%%%%%%%%% chapter.tex %%%%%%%%%%%%%%%%%%%%%%%%%%%%%%%%%
%
% sample chapter
%
% Use this file as a template for your own input.
%
%%%%%%%%%%%%%%%%%%%%%%%% Springer-Verlag %%%%%%%%%%%%%%%%%%%%%%%%%%
%\motto{Use the template \emph{chapter.tex} to style the various elements of your chapter content.}



\chapter{Symbols starting with R}
\label{cha:func-ref-R-functions-starting-with-R}

 
\section*{\texttt{*Run}}
\label{sec:func-ref-R-*Run}


This global variable can hold a list of \texttt{prg} expressions which are used
during \texttt{key}, \texttt{sync}, \texttt{wait} and \texttt{listen}. The first element of each
expression must either be a positive number (thus denoting a file
descriptor to wait for) or a negative number (denoting a timeout value
in milliseconds (in that case another number must follow to hold the
remaining time)). A \texttt{select} system call is performed with these values,
and the corresponding \texttt{prg} body is executed when input data are
available or when a timeout occurred. See also \texttt{task}.


\begin{wideverbatim}
: (de *Run (-2000 0 (println '2sec)))     # Install 2-sec-timer
-> *Run
: 2sec                                    # Prints "2sec" every 2 seconds
2sec
2sec
                                          # (Ctrl-D) Exit
$
\end{wideverbatim}

 
\section*{\texttt{+Ref}}
\label{sec:func-ref-R-+Ref}


Prefix class for maintaining non-unique indexes to \texttt{+relation}s, a subclass of \texttt{+index}. Accepts an optional argument for a \texttt{+Hook}
attribute. See also \texttt{Database}.


\begin{wideverbatim}
(rel tel (+Fold +Ref +String))  # Phone number with folded, non-unique index
\end{wideverbatim}

 
\section*{\texttt{+Ref2}}
\label{sec:func-ref-R-+Ref2}


Prefix class for maintaining a secondary (``backing'') index to
\texttt{+relation}s. Can only be used as a prefix class to
\texttt{+Key} or \texttt{+Ref}. It maintains an index in the current
(sub)class, in addition to that in one of the superclasses, to allow
(sub)class-specific queries. See also \texttt{Database}.


\begin{wideverbatim}
(class +Ord +Entity)             # Order class
(rel nr (+Need +Key +Number))    # Order number
...
(class +EuOrd +Ord)              # EU-specific order subclass
(rel nr (+Ref2 +Key +Number))    # Order number with backing index
\end{wideverbatim}

 
\section*{\texttt{+relation}}
\label{sec:func-ref-R-+relation}


Abstract base class of all database releations. Relation objects are
usually defined with \texttt{rel}. The class hierarchy includes the classes
\texttt{+Any}, \texttt{+Bag}, \texttt{+Bool}, \texttt{+Number}, \texttt{+Date}, \texttt{+Time}, \texttt{+Symbol},
\texttt{+String}, \texttt{+Link}, \texttt{+Joint} and \texttt{+Blob}, and the prefix classes
\texttt{+Hook}, \texttt{+index}, \texttt{+Key}, \texttt{+Ref}, \texttt{+Ref2}, \texttt{+Idx}, \texttt{+Sn}, \texttt{+Fold},
\texttt{+Aux}, \texttt{+UB}, \texttt{+Dep}, \texttt{+List}, \texttt{+Need}, \texttt{+Mis} and \texttt{+Alt}. See also
\texttt{Database} and \texttt{+Entity}.

Messages to relation objects include


\begin{wideverbatim}
mis> (Val Obj)       # Return error if mismatching type or value
has> (Val X)         # Check if the value is present
put> (Obj Old New)   # Put new value
rel> (Obj Old New)   # Maintain relational strutures
lose> (Obj Val)      # Delete relational structures
keep> (Obj Val)      # Restore deleted relational structures
zap> (Obj Val)       # Clean up relational structures
\end{wideverbatim}

 
\section*{\texttt{(rand ['cnt1 'cnt2] | ['T]) -> cnt | flg}}
\label{sec:func-ref-R-(rand ['cnt1 'cnt2] | ['T]) -> cnt | flg}


Returns a pseudo random number in the range cnt1 .. cnt2 (or
--2147483648 .. +2147483647 if no arguments are given). If the argument
is \texttt{T}, a boolean value \texttt{flg} is returned. See also \texttt{seed}.


\begin{wideverbatim}
: (rand 3 9)
-> 3
: (rand 3 9)
-> 7
\end{wideverbatim}

 
\section*{\texttt{(range 'num1 'num2 ['num3]) -> lst}}
\label{sec:func-ref-R-(range 'num1 'num2 ['num3]) -> lst}


Produces a list of numbers in the range \texttt{num1} through \texttt{num2}. When
\texttt{num3} is non-\texttt{NIL}), it is used to increment \texttt{num1} (if it is smaller
than \texttt{num2}) or to decrement \texttt{num1} (if it is greater than \texttt{num2}). See
also \texttt{need}.


\begin{wideverbatim}
: (range 1 6)
-> (1 2 3 4 5 6)
: (range 6 1)
-> (6 5 4 3 2 1)
: (range -3 3)
-> (-3 -2 -1 0 1 2 3)
: (range 3 -3 2)
-> (3 1 -1 -3)
\end{wideverbatim}

 
\section*{\texttt{range/3}}
\label{sec:func-ref-R-range/3}


\emph{Pilog} predicate that succeeds if the first argument
is in the range of the result of applying the \texttt{get} algorithm to the
following arguments. Typically used as filter predicate in \texttt{select/3}
database queries. See also \texttt{Comparing}, \texttt{isa/2}, \texttt{same/3}, \texttt{bool/3},
\texttt{head/3}, \texttt{fold/3}, \texttt{part/3} and \texttt{tolr/3}.


\begin{wideverbatim}
: (?
   @Nr (1 . 5)  # Numbers between 1 and 5
   @Nm "part"
   (select (@Item)
      ((nr +Item @Nr) (nm +Item @Nm))
      (range @Nr @Item nr)
      (part @Nm @Item nm) ) )
      @Nr=(1 . 5) @Nm="part" @Item={3-1}
      @Nr=(1 . 5) @Nm="part" @Item={3-2}
-> NIL
\end{wideverbatim}

 
\section*{\texttt{(rank 'any 'lst ['flg]) -> lst}}
\label{sec:func-ref-R-(rank 'any 'lst ['flg]) -> lst}


Searches a ranking list. \texttt{lst} should be sorted. Returns the element
from \texttt{lst} with a maximal CAR less or equal to \texttt{any} (if \texttt{flg} is
\texttt{NIL}), or with a minimal CAR greater or equal to \texttt{any} (if \texttt{flg} is
non-\texttt{NIL}), or \texttt{NIL} if no match is found. See also \texttt{assoc} and
\emph{Comparing}.


\begin{wideverbatim}
: (rank 0 '((1 . a) (100 . b) (1000 . c)))
-> NIL
: (rank 50 '((1 . a) (100 . b) (1000 . c)))
-> (1 . a)
: (rank 100 '((1 . a) (100 . b) (1000 . c)))
-> (100 . b)
: (rank 300 '((1 . a) (100 . b) (1000 . c)))
-> (100 . b)
: (rank 9999 '((1 . a) (100 . b) (1000 . c)))
-> (1000 . c)
: (rank 50 '((1000 . a) (100 . b) (1 . c)) T)
-> (100 . b)
\end{wideverbatim}

 
\section*{\texttt{(raw ['flg]) -> flg}}
\label{sec:func-ref-R-(raw ['flg]) -> flg}


Console mode control function. When called without arguments, it returns
the current console mode (\texttt{NIL} for ``cooked mode''). Otherwise, the
console is set to the new state. See also \texttt{key}.


\begin{wideverbatim}
$ pil
: (raw)
-> NIL
$ pil +
: (raw)
-> T
\end{wideverbatim}

 
\section*{\texttt{(rc 'sym 'any1 ['any2]) -> any}}
\label{sec:func-ref-R-(rc 'sym 'any1 ['any2]) -> any}


Fetches a value from a resource file \texttt{sym}, or stores a value \texttt{any2} in
that file, using a key \texttt{any1}. All values are stored in a list in the
file, using \texttt{assoc}. During the whole operation, the file is exclusively
locked with \texttt{ctl}.


\begin{wideverbatim}
: (info "a.rc")               # File exists?
-> NIL                        # No
: (rc "a.rc" 'a 1)            # Store 1 for 'a'
-> 1
: (rc "a.rc" 'b (2 3 4))      # Store (2 3 4) for 'b'
-> (2 3 4)
: (rc "a.rc" 'c 'b)           # Store 'b' for 'c'
-> b
: (info "a.rc")               # Check file
-> (28 733124 . 61673)
: (in "a.rc" (echo))          # Display it
((c . b) (b 2 3 4) (a . 1))
-> T
: (rc "a.rc" 'c)              # Fetch value for 'c'
-> b
: (rc "a.rc" @)               # Fetch value for 'b'
-> (2 3 4)
\end{wideverbatim}

 
\section*{\texttt{(rd ['sym]) -> any}}
\label{sec:func-ref-R-(rd ['sym]) -> any}


\texttt{(rd 'cnt) -> num | NIL}

Binary read: Reads one item from the current input channel in encoded
binary format. When called with a \texttt{cnt} argument (second form), that
number of raw bytes (in big endian format if \texttt{cnt} is positive,
otherwise little endian) is read as a single number. Upon end of file,
if the \texttt{sym} argument is given, it is returned, otherwise \texttt{NIL}. See
also \texttt{pr}, \texttt{tell}, \texttt{hear} and \texttt{wr}.


\begin{wideverbatim}
: (out "x" (pr 'abc "EOF" 123 "def"))
-> "def"
: (in "x" (rd))
-> abc
: (in "x"
   (make
      (use X
         (until (== "EOF" (setq X (rd "EOF")))  # '==' detects end of file
            (link X) ) ) ) )
-> (abc "EOF" 123 "def")  # as opposed to reading a symbol "EOF"

: (in "/dev/urandom" (rd 20))
-> 396737673456823753584720194864200246115286686486
\end{wideverbatim}

 
\section*{\texttt{(read ['sym1 ['sym2]]) -> any}}
\label{sec:func-ref-R-(read ['sym1 ['sym2]]) -> any}


Reads one item from the current input channel. \texttt{NIL} is returned upon
end of file. When called without arguments, an arbitrary Lisp expression
is read. Otherwise, a token (a number, or an internal or transient
symbol) is read. In that case, \texttt{sym1} specifies which set of characters
to accept for continuous symbol names (in addition to the standard
alphanumerical characters), and \texttt{sym2} an optional comment character.
See also \texttt{any}, \texttt{str}, \texttt{skip} and \texttt{eof}.


\begin{wideverbatim}
: (list (read) (read) (read))    # Read three things from console
123                              # a number
abcd                             # a symbol
(def                             # and a list
ghi
jkl
)
-> (123 abcd (def ghi jkl))
: (make (while (read "_" "#") (link @)))
abc = def_ghi("xyz"+-123) # Comment
NIL
-> (abc "=" def_ghi "(" "xyz" "+" "-" 123 ")")
\end{wideverbatim}

 
\section*{\texttt{(recur fun) -> any}}
\label{sec:func-ref-R-(recur fun) -> any}


\texttt{(recurse ..) -> any}

Implements anonymous recursion, by defining the function \texttt{recurse} on
the fly. During the execution of \texttt{fun}, the symbol \texttt{recurse} is bound to
the function definition \texttt{fun}. See also \texttt{let} and \texttt{lambda}.


\begin{wideverbatim}
: (de fibonacci (N)
   (when (lt0 N)
      (quit "Bad fibonacci" N) )
   (recur (N)
      (if (> 2 N)
         1
         (+
            (recurse (dec N))
            (recurse (- N 2)) ) ) ) )
-> fibonacci
: (fibonacci 22)
-> 28657
: (fibonacci -7)
-7 -- Bad fibonacci
\end{wideverbatim}

 
\section*{\texttt{(redef sym . fun) -> sym}}
\label{sec:func-ref-R-(redef sym . fun) -> sym}


Redefines \texttt{sym} in terms of itself. The current definition is saved in a
new symbol, which is substituted for each occurrence of \texttt{sym} in \texttt{fun},
and which is also returned. See also \texttt{de}, \texttt{undef}, \texttt{daemon} and
\texttt{patch}.


\begin{wideverbatim}
: (de hello () (prinl "Hello world!"))
-> hello
: (pp 'hello)
(de hello NIL
   (prinl "Hello world!") )
-> hello

: (redef hello (A B)
   (println 'Before A)
   (prog1 (hello) (println 'After B)) )
-> "hello"
: (pp 'hello)
(de hello (A B)
   (println 'Before A)
   (prog1 ("hello") (println 'After B)) )
-> hello
: (hello 1 2)
Before 1
Hello world!
After 2
-> "Hello world!"

: (redef * @
   (msg (rest))
   (pass *) )
-> "*"
: (* 1 2 3)
(1 2 3)
-> 6

: (redef + @
   (pass (ifn (num? (next)) pack +) (arg)) )
-> "+"
: (+ 1 2 3)
-> 6
: (+ "a" 'b '(c d e))
-> "abcde"
\end{wideverbatim}

 
\section*{\texttt{(rel var lst [any ..]) -> any}}
\label{sec:func-ref-R-(rel var lst [any ..]) -> any}


Defines a relation for \texttt{var} in the current class \texttt{*Class}, using \texttt{lst}
as the list of classes for that relation, and possibly additional
arguments \texttt{any} for its initialization. See also
\emph{Database}, \emph{class},
\emph{extend}, \emph{dm} and
\emph{var}.


\begin{wideverbatim}
(class +Person +Entity)
(rel nm  (+List +Ref +String))            # Names
(rel tel (+Ref +String))                  # Telephone
(rel adr (+Joint) prs (+Address))         # Address

(class +Address +Entity)
(rel cit (+Need +Hook +Link) (+City))     # City
(rel str (+List +Ref +String) cit)        # Street
(rel prs (+List +Joint) adr (+Person))    # Inhabitants

(class +City +Entity)
(rel nm  (+List +Ref +String))            # Zip / Names
\end{wideverbatim}

 
\section*{\texttt{(release 'sym) -> NIL}}
\label{sec:func-ref-R-(release 'sym) -> NIL}


Releases the mutex represented by the file `sym'. This is the reverse
operation of \texttt{acquire}.


\begin{wideverbatim}
: (release "sema1")
-> NIL
\end{wideverbatim}

 
\section*{\texttt{remote/2}}
\label{sec:func-ref-R-remote/2}


\emph{Pilog} predicate for remote database queries. It
takes a list and an arbitrary number of clauses. The list should contain
a Pilog variable for the result in the CAR, and a list of resources in
the CDR. The clauses will be evaluated on remote machines according to
these resources. Each resource must be a cons pair of two functions, an
``out'' function in the CAR, and an ``in'' function in the CDR. See also
\texttt{*Ext}, \texttt{select/3} and \texttt{db/3}.


\begin{wideverbatim}
(setq *Ext           # Set up external offsets
   (mapcar
      '((@Host @Ext)
         (cons @Ext
            (curry (@Host @Ext (Sock)) (Obj)
               (when (or Sock (setq Sock (connect @Host 4040)))
                  (ext @Ext
                     (out Sock (pr (cons 'qsym Obj)))
                     (prog1 (in Sock (rd))
                        (unless @
                           (close Sock)
                           (off Sock) ) ) ) ) ) ) )
      '("localhost")
      '(20) ) )

(de rsrc ()  # Simple resource handler, ignoring errors or EOFs
   (extract
      '((@Ext Host)
         (let? @Sock (connect Host 4040)
            (cons
               (curry (@Ext @Sock) (X)  # out
                  (ext @Ext (out @Sock (pr X))) )
               (curry (@Ext @Sock) ()  # in
                  (ext @Ext (in @Sock (rd))) ) ) ) )
      '(20)
      '("localhost") ) )

: (?
   @Nr (1 . 3)
   @Sup 2
   @Rsrc (rsrc)
   (remote (@Item . @Rsrc)
      (db nr +Item @Nr @Item)
      (val @Sup @Item sup nr) )
   (show @Item) )
{L-2} (+Item)
   pr 1250
   inv 100
   sup {K-2}
   nm Spare Part
   nr 2
 @Nr=(1 . 3) @Sup=2 @Rsrc=((((X) (ext 20 (out 16 (pr X)))) 
      NIL (ext 20 (in 16 (rd))))) @Item={L-2}
-> NIL
\end{wideverbatim}

 
\section*{\texttt{(remove 'cnt 'lst) -> lst}}
\label{sec:func-ref-R-(remove 'cnt 'lst) -> lst}


Removes the element at position \texttt{cnt} from \texttt{lst}. This is a
non-destructive operation. See also \texttt{insert}, \texttt{place}, \texttt{append},
\texttt{delete} and \texttt{replace}.


\begin{wideverbatim}
: (remove 3 '(a b c d e))
-> (a b d e)
: (remove 1 '(a b c d e))
-> (b c d e)
: (remove 9 '(a b c d e))
-> (a b c d e)
\end{wideverbatim}

 
\section*{\texttt{(repeat) -> lst}}
\label{sec:func-ref-R-(repeat) -> lst}


Makes the current \emph{Pilog} definition ``tail recursive'',
by closing the previously defined clauses in the T property to a
circular list. See also \texttt{be}.


\begin{wideverbatim}
(be a (1))     # Define three facts
(be a (2))
(be a (3))
(repeat)       # Unlimited supply

: (? (a @N))
 @N=1
 @N=2
 @N=3
 @N=1
 @N=2
 @N=3.         # Stop
-> NIL
\end{wideverbatim}

 
\section*{\texttt{repeat/0}}
\label{sec:func-ref-R-repeat/0}


\emph{Pilog} predicate that always succeeds, also on
backtracking. See also \texttt{repeat} and \texttt{true/0}.


\begin{wideverbatim}
: (be int (@N)       # Generate unlimited supply of integers
   (@ zero *N)
   (repeat)          # Repeat from here
   (@N inc '*N) )
-> int

:  (? (int @X))
 @X=1
 @X=2
 @X=3
 @X=4.               # Stop
-> NIL
\end{wideverbatim}

 
\section*{\texttt{(replace 'lst 'any1 'any2 ..) -> lst}}
\label{sec:func-ref-R-(replace 'lst 'any1 'any2 ..) -> lst}


Replaces in \texttt{lst} all occurrences of \texttt{any1} with \texttt{any2}. For optional
additional argument pairs, this process is repeated. This is a
non-destructive operation. See also \texttt{append}, \texttt{delete}, \texttt{insert},
\texttt{remove} and \texttt{place}.


\begin{wideverbatim}
: (replace '(a b b a) 'a 'A)
-> (A b b A)
: (replace '(a b b a) 'b 'B)
-> (a B B a)
: (replace '(a b b a) 'a 'B 'b 'A)
-> (B A A B)
\end{wideverbatim}

 
\section*{\texttt{(request 'typ 'var ['hook] 'val ..) -> obj}}
\label{sec:func-ref-R-(request 'typ 'var ['hook] 'val ..) -> obj}


Returns a database object. If a matching object cannot be found (using
\texttt{db}), a new object of the given type is created (using \texttt{new}). See also
\texttt{obj}.


\begin{wideverbatim}
: (request '(+Item) 'nr 2)
-> {3-2}
\end{wideverbatim}

 
\section*{\texttt{(rest) -> lst}}
\label{sec:func-ref-R-(rest) -> lst}


Can only be used inside functions with a variable number of arguments
(with \texttt{@}). Returns the list of all remaining arguments from the
internal list. See also \texttt{args}, \texttt{next}, \texttt{arg} and \texttt{pass}.


\begin{wideverbatim}
: (de foo @ (println (rest)))
-> foo
: (foo 1 2 3)
(1 2 3)
-> (1 2 3)
\end{wideverbatim}


\section*{\texttt{(retract) -> lst}}

Removes a \emph{Pilog} fact or rule. See also \texttt{be},
\texttt{clause}, \texttt{asserta} and \texttt{assertz}.

\begin{wideverbatim}
: (be a (1))
-> a
: (be a (2))
-> a
: (be a (3))
-> a

: (retract '(a (2)))
-> (((1)) ((3)))

:  (? (a @N))
 @N=1
 @N=3
-> NIL
\end{wideverbatim}

 
\section*{\texttt{retract/1}}
\label{sec:func-ref-R-(retract) -> lst}


\emph{Pilog} predicate that removes a fact or rule. See
also \texttt{retract}, \texttt{asserta/1} and \texttt{assertz/1}.


\begin{wideverbatim}
: (be a (1))
-> a
: (be a (2))
-> a
: (be a (3))
-> a

: (? (retract (a 2)))
-> T
: (rules 'a)
1 (be a (1))
2 (be a (3))
-> a
\end{wideverbatim}

 
\section*{\texttt{(reverse 'lst) -> lst}}
\label{sec:func-ref-R-(reverse 'lst) -> lst}


Returns a reversed copy of \texttt{lst}. See also \texttt{flip}.


\begin{wideverbatim}
: (reverse (1 2 3 4))
-> (4 3 2 1)
\end{wideverbatim}

 
\section*{\texttt{(rewind) -> flg}}
\label{sec:func-ref-R-(rewind) -> flg}


Sets the file position indicator for the current output stream to the
beginning of the file, and truncates the file length to zero. Returns
\texttt{T} when successful. See also \texttt{flush}.


\begin{wideverbatim}
: (out "a" (prinl "Hello world"))
-> "Hello world"
: (in "a" (echo))
Hello world
-> T
: (info "a")
-> (12 733216 . 53888)
: (out "a" (rewind))
-> T
: (info "a")
-> (0 733216 . 53922)
\end{wideverbatim}

 
\section*{\texttt{(rollback) -> T}}
\label{sec:func-ref-R-(rollback) -> T}


Cancels a transaction, by discarding all modifications of external
symbols. See also \texttt{commit}.


\begin{wideverbatim}
: (pool "db")
-> T
# .. Modify external objects ..
: (rollback)            # Rollback
-> T
\end{wideverbatim}

 
\section*{\texttt{(root 'tree) -> (num . sym)}}
\label{sec:func-ref-R-(root 'tree) -> (num . sym)}


Returns the root of a database index tree, with the number of entries in
\texttt{num}, and the base node in \texttt{sym}. See also \texttt{tree}.


\begin{wideverbatim}
: (root (tree 'nr '+Item))
-> (7 . {7-1})
\end{wideverbatim}

 
\section*{\texttt{(rot 'lst ['cnt]) -> lst}}
\label{sec:func-ref-R-(rot 'lst ['cnt]) -> lst}


Rotate: The contents of the cells of \texttt{lst} are (destructively) shifted
right, and the value from the last cell is stored in the first cell.
Without the optional \texttt{cnt} argument, the whole list is rotated,
otherwise only the first \texttt{cnt} elements. See also \texttt{flip} .


\begin{wideverbatim}
: (rot (1 2 3 4))             # Rotate all four elements
-> (4 1 2 3)
: (rot (1 2 3 4 5 6) 3)       # Rotate only the first three elements
-> (3 1 2 4 5 6)
\end{wideverbatim}

 
\section*{\texttt{(round 'num1 'num2) -> sym}}
\label{sec:func-ref-R-(round 'num1 'num2) -> sym}


Formats a number \texttt{num1} with \texttt{num2} decimal places, according to the
current scale \texttt{*Scl}. \texttt{num2} defaults to 3. See also
\emph{Numbers} and \texttt{format}.


\begin{wideverbatim}
: (scl 4)               # Set scale to 4
-> 4
: (round 123456)        # Format with three decimal places
-> "12.346"
: (round 123456 2)      # Format with two decimal places
-> "12.35"
: (format 123456 *Scl)  # Format with full precision
-> "12.3456"
\end{wideverbatim}

 
\section*{\texttt{(rules 'sym ..) -> sym}}
\label{sec:func-ref-R-(rules 'sym ..) -> sym}


Prints all rules defined for the \texttt{sym} arguments. See also
\emph{Pilog} and \texttt{be}.


\begin{wideverbatim}
: (rules 'member 'append)
1 (be member (@X (@X . @)))
2 (be member (@X (@ . @Y)) (member @X @Y))
1 (be append (NIL @X @X))
2 (be append ((@A . @X) @Y (@A . @Z)) (append @X @Y @Z))
-> append
\end{wideverbatim}

 
\section*{\texttt{(run 'any ['cnt ['lst]]) -> any}}
\label{sec:func-ref-R-(run 'any ['cnt ['lst]]) -> any}


If \texttt{any} is an atom, \texttt{run} behaves like \texttt{eval}. Otherwise \texttt{any} is a
list, which is evaluated in sequence. The last result is returned. If a
binding environment offset \texttt{cnt} is given, that evaluation takes place
in the corresponding environment, and an optional \texttt{lst} of excluded
symbols can be supplied. See also \texttt{up}.


\begin{wideverbatim}
: (run '((println (+ 1 2 3)) (println 'OK)))
6
OK
-> OK
\end{wideverbatim}


