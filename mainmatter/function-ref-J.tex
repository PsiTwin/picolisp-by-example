%%%%%%%%%%%%%%%%%%%%% chapter.tex %%%%%%%%%%%%%%%%%%%%%%%%%%%%%%%%%
%
% sample chapter
%
% Use this file as a template for your own input.
%
%%%%%%%%%%%%%%%%%%%%%%%% Springer-Verlag %%%%%%%%%%%%%%%%%%%%%%%%%%
%\motto{Use the template \emph{chapter.tex} to style the various elements of your chapter content.}



\chapter{Symbols starting with J}
\label{cha:func-ref-J-functions-starting-with-J}
 
\section*{\texttt{+Joint}}
\label{sec:func-ref-J-+Joint}


Class for bidirectional object relations, a subclass of \texttt{+Link}. Expects
a (symbolic) attribute, and list of classes as \texttt{type} of the referred
database object (of class \texttt{+Entity}). A \texttt{+Joint} corresponds to two
\texttt{+Link}s, where the attribute argument is the relation of the back-link in the referred object. See also \texttt{Database}.


\begin{wideverbatim}
(class +Ord +Entity)                   # Order class
(rel pos (+List +Joint) ord (+Pos))    # List of positions in that order
...
(class +Pos +Entity)    # Position class
(rel ord (+Joint)       # Back-link to the parent order
\end{wideverbatim}

 
\section*{\texttt{(job 'lst . prg) -> any}}
\label{sec:func-ref-J-(job 'lst . prg) -> any}


Executes a job within its own environment (as specified by symbol-value
pairs in \texttt{lst}). The current values of all symbols are saved, the
symbols are bound to the values in \texttt{lst}, \texttt{prg} is executed, then the
(possibly modified) symbol values are (destructively) stored in the
environment list, and the symbols are restored to their original values.
The return value is the result of \texttt{prg}. Typically used in \texttt{curried}
functions and \texttt{*Run} tasks. See also \texttt{env}, \texttt{bind}, \texttt{let}, \texttt{use} and
\texttt{state}.


\begin{wideverbatim}
: (de tst ()
   (job '((A . 0) (B . 0))
      (println (inc 'A) (inc 'B 2)) ) )
-> tst
: (tst)
1 2
-> 2
: (tst)
2 4
-> 4
: (tst)
3 6
-> 6
: (pp 'tst)
(de tst NIL
   (job '((A . 3) (B . 6))
      (println (inc 'A) (inc 'B 2)) ) )
-> tst
\end{wideverbatim}

 
\section*{\texttt{(journal 'any ..) -> T}}
\label{sec:func-ref-J-(journal 'any ..) -> T}


Reads journal data from the files with the names \texttt{any}, and writes all
changes to the database. See also \texttt{pool}.


\begin{wideverbatim}
: (journal "db.log")
-> T
\end{wideverbatim}


