%%%%%%%%%%%%%%%%%%%%% chapter.tex %%%%%%%%%%%%%%%%%%%%%%%%%%%%%%%%%
%
% sample chapter
%
% Use this file as a template for your own input.
%
%%%%%%%%%%%%%%%%%%%%%%%% Springer-Verlag %%%%%%%%%%%%%%%%%%%%%%%%%%
%\motto{Use the template \emph{chapter.tex} to style the various elements of your chapter content.}



\chapter{Symbols starting with H}
\label{cha:func-ref-H-functions-starting-with-H}

 
\section*{\texttt{*Hup}}
\label{sec:func-ref-H-*Hup}


Global variable holding a (possibly empty) \texttt{prg} body, which will be
executed when a SIGHUP signal is sent to the current process. See also
\texttt{alarm}, \texttt{sigio} and \texttt{*Sig[12]}.


\begin{wideverbatim}
: (de *Hup (msg 'SIGHUP))
-> *Hup
\end{wideverbatim}

 
\section*{\texttt{+Hook}}
\label{sec:func-ref-H-+Hook}


Prefix class for \texttt{+relation}s, typically \texttt{+Link} or
\texttt{+Joint}. In essence, this maintains an local database in the
referred object. See also \texttt{Database}.


\begin{wideverbatim}
(rel sup (+Hook +Link) (+Sup))   # Supplier
(rel nr (+Key +Number) sup)      # Item number, unique per supplier
(rel dsc (+Ref +String) sup)     # Item description, indexed per supplier
\end{wideverbatim}

 
\section*{\texttt{(hash 'any) -> cnt}}
\label{sec:func-ref-H-(hash 'any) -> cnt}


Generates a 16-bit number (1--65536) from \texttt{any}, suitable as a hash
value for various purposes, like randomly balanced \texttt{idx} structures. See
also \texttt{cache} and \texttt{seed}.


\begin{wideverbatim}
: (hash 0)
-> 1
: (hash 1)
-> 55682
: (hash "abc")
-> 45454
\end{wideverbatim}

 
\section*{\texttt{(hax 'num) -> sym}}
\label{sec:func-ref-H-(hax 'num) -> sym}


\texttt{(hax 'sym) -> num}

Converts a number \texttt{num} to a string in hexadecimal/alpha
notation, or a hexadecimal/alpha formatted string to a number. The
digits are represented with `\texttt{@}' (zero) and the letters
`\texttt{A}' - `\texttt{O}' (from ``alpha'' to ``omega''). This format
is used internally for the names of \texttt{external symbols} in the
64-bit version. See also \texttt{fmt64}, \texttt{hex}, \texttt{bin}
and \texttt{oct}.


\begin{wideverbatim}
: (hax 7)
-> "G"
: (hax 16)
-> "A@"
: (hax 255)
-> "OO"
: (hax "A")
-> 1
\end{wideverbatim}

 
\section*{\texttt{(hd 'sym ['cnt]) -> NIL}}
\label{sec:func-ref-H-(hd 'sym ['cnt]) -> NIL}


Displays a hexadecimal dump of the file given by \texttt{sym}, limited to \texttt{cnt}
lines. See also \texttt{proc}.


\begin{wideverbatim}
:  (hd "lib.l" 4)
00000000  23 20 32 33 64 65 63 30 39 61 62 75 0A 23 20 28  # 23dec09abu.# (
00000010  63 29 20 53 6F 66 74 77 61 72 65 20 4C 61 62 2E  c) Software Lab.
00000020  20 41 6C 65 78 61 6E 64 65 72 20 42 75 72 67 65   Alexander Burge
00000030  72 0A 0A 28 64 65 20 74 61 73 6B 20 28 4B 65 79  r..(de task (Key
-> NIL
\end{wideverbatim}

 
\section*{\texttt{(head 'cnt|lst 'lst) -> lst}}
\label{sec:func-ref-H-(head 'cnt|lst 'lst) -> lst}


Returns a new list made of the first \texttt{cnt} elements of \texttt{lst}. If \texttt{cnt}
is negative, it is added to the length of \texttt{lst}. If the first argument
is a \texttt{lst}, \texttt{head} is a predicate function returning that argument list
if it is \texttt{equal} to the head of the second argument, and \texttt{NIL}
otherwise. See also \texttt{tail}.


\begin{wideverbatim}
: (head 3 '(a b c d e f))
-> (a b c)
: (head 0 '(a b c d e f))
-> NIL
: (head 10 '(a b c d e f))
-> (a b c d e f)
: (head -2 '(a b c d e f))
-> (a b c d)
: (head '(a b c) '(a b c d e f))
-> (a b c)
\end{wideverbatim}

 
\section*{\texttt{head/3}}
\label{sec:func-ref-H-head/3}


\emph{Pilog} predicate that succeeds if the first (string)
argument is a prefix of the string representation of the result of
applying the \texttt{get} algorithm to the following arguments. Typically used
as filter predicate in \texttt{select/3} database queries. See also \texttt{pre?},
\texttt{isa/2}, \texttt{same/3}, \texttt{bool/3}, \texttt{range/3}, \texttt{fold/3}, \texttt{part/3} and \texttt{tolr/3}.


\begin{wideverbatim}
: (?
   @Nm "Muller"
   @Tel "37"
   (select (@CuSu)
      ((nm +CuSu @Nm) (tel +CuSu @Tel))
      (tolr @Nm @CuSu nm)
      (head @Tel @CuSu tel) )
   (val @Name @CuSu nm)
   (val @Phone @CuSu tel) )
 @Nm="Muller" @Tel="37" @CuSu={2-3} @Name="Miller" @Phone="37 4773 82534"
-> NIL
\end{wideverbatim}

 
\section*{\texttt{(heap 'flg) -> cnt}}
\label{sec:func-ref-H-(heap 'flg) -> cnt}


Returns the total size of the cell heap space in megabytes. If \texttt{flg} is
non-\texttt{NIL}, the size of the currently free space is returned. See also
\texttt{stack} and \texttt{gc}.


\begin{wideverbatim}
: (gc 4)
-> 4
: (heap)
-> 5
: (heap T)
-> 4
\end{wideverbatim}

 
\section*{\texttt{(hear 'cnt) -> cnt}}
\label{sec:func-ref-H-(hear 'cnt) -> cnt}


Uses the file descriptor \texttt{cnt} as an asynchronous command input channel.
Any executable list received via this channel will be executed in the
background. As this mechanism is also used for inter-family
communication (see \texttt{tell}), \texttt{hear} is usually only called explicitly by
a top level parent process.


\begin{wideverbatim}
: (call 'mkfifo "fifo/cmd")
-> T
: (hear (open "fifo/cmd"))
-> 3
\end{wideverbatim}

 
\section*{\texttt{(here ['sym]) -> sym}}
\label{sec:func-ref-H-(here ['sym]) -> sym}


Echoes the current input stream until \texttt{sym} is encountered, or until end
of file. See also \texttt{echo}.


\begin{wideverbatim}
$ cat hello.l
(html 0 "Hello" "lib.css" NIL
   (<h2> NIL "Hello")
   (here) )
<p>Hello!</p>
<p>This is a test.</p>

$ pil @lib/http.l @lib/xhtml.l hello.l
HTTP/1.0 200 OK
Server: PicoLisp
Date: Sun, 03 Jun 2007 11:41:27 GMT
Cache-Control: max-age=0
Cache-Control: no-cache
Content-Type: text/html; charset=utf-8

<!DOCTYPE html PUBLIC "-//W3C//DTD XHTML 1.0 Strict//EN" "http://www.w3.org/TR/xhtml1/DTD/xhtml1-strict.dtd">
<html xmlns="http://www.w3.org/1999/xhtml" xml:lang="en" lang="en">
<head>
<title>Hello</title>
<link rel="stylesheet" href="http://:/lib.css" type="text/css"/>
</head>
<body><h2>Hello</h2>
<p>Hello!</p>
<p>This is a test.</p>
</body>
</html>
\end{wideverbatim}

 
\section*{\texttt{(hex 'num ['num]) -> sym}}
\label{sec:func-ref-H-(hex 'num ['num]) -> sym}


\texttt{(hex 'sym) -> num}

Converts a number \texttt{num} to a hexadecimal string, or a hexadecimal string
\texttt{sym} to a number. In the first case, if the second argument is given,
the result is separated by spaces into groups of such many digits. See
also \texttt{bin}, \texttt{oct}, \texttt{fmt64}, \texttt{hax} and \texttt{format}.


\begin{wideverbatim}
: (hex 273)
-> "111"
: (hex "111")
-> 273
: (hex 1234567 4)
-> "12 D687"
\end{wideverbatim}

 
\section*{\texttt{(host 'any) -> sym}}
\label{sec:func-ref-H-(host 'any) -> sym}


Returns the hostname corresponding to the given IP address. See also
\texttt{*Adr}.


\begin{wideverbatim}
: (host "80.190.158.9")
-> "www.leo.org"
\end{wideverbatim}




% %%%%%%%%%%%%%%%%%%%%%%%% referenc.tex %%%%%%%%%%%%%%%%%%%%%%%%%%%%%%
% sample references
% %
% Use this file as a template for your own input.
%
%%%%%%%%%%%%%%%%%%%%%%%% Springer-Verlag %%%%%%%%%%%%%%%%%%%%%%%%%%
%
% BibTeX users please use
% \bibliographystyle{}
% \bibliography{}
%
\biblstarthook{In view of the parallel print and (chapter-wise) online publication of your book at \url{www.springerlink.com} it has been decided that -- as a genreral rule --  references should be sorted chapter-wise and placed at the end of the individual chapters. However, upon agreement with your contact at Springer you may list your references in a single seperate chapter at the end of your book. Deactivate the class option \texttt{sectrefs} and the \texttt{thebibliography} environment will be put out as a chapter of its own.\\\indent
References may be \textit{cited} in the text either by number (preferred) or by author/year.\footnote{Make sure that all references from the list are cited in the text. Those not cited should be moved to a separate \textit{Further Reading} section or chapter.} The reference list should ideally be \textit{sorted} in alphabetical order -- even if reference numbers are used for the their citation in the text. If there are several works by the same author, the following order should be used: 
\begin{enumerate}
\item all works by the author alone, ordered chronologically by year of publication
\item all works by the author with a coauthor, ordered alphabetically by coauthor
\item all works by the author with several coauthors, ordered chronologically by year of publication.
\end{enumerate}
The \textit{styling} of references\footnote{Always use the standard abbreviation of a journal's name according to the ISSN \textit{List of Title Word Abbreviations}, see \url{http://www.issn.org/en/node/344}} depends on the subject of your book:
\begin{itemize}
\item The \textit{two} recommended styles for references in books on \textit{mathematical, physical, statistical and computer sciences} are depicted in ~\cite{science-contrib, science-online, science-mono, science-journal, science-DOI} and ~\cite{phys-online, phys-mono, phys-journal, phys-DOI, phys-contrib}.
\item Examples of the most commonly used reference style in books on \textit{Psychology, Social Sciences} are~\cite{psysoc-mono, psysoc-online,psysoc-journal, psysoc-contrib, psysoc-DOI}.
\item Examples for references in books on \textit{Humanities, Linguistics, Philosophy} are~\cite{humlinphil-journal, humlinphil-contrib, humlinphil-mono, humlinphil-online, humlinphil-DOI}.
\item Examples of the basic Springer style used in publications on a wide range of subjects such as \textit{Computer Science, Economics, Engineering, Geosciences, Life Sciences, Medicine, Biomedicine} are ~\cite{basic-contrib, basic-online, basic-journal, basic-DOI, basic-mono}. 
\end{itemize}
}

\begin{thebibliography}{99.}%
% and use \bibitem to create references.
%
% Use the following syntax and markup for your references if 
% the subject of your book is from the field 
% "Mathematics, Physics, Statistics, Computer Science"
%
% Contribution 
\bibitem{science-contrib} Broy, M.: Software engineering --- from auxiliary to key technologies. In: Broy, M., Dener, E. (eds.) Software Pioneers, pp. 10-13. Springer, Heidelberg (2002)
%
% Online Document
\bibitem{science-online} Dod, J.: Effective substances. In: The Dictionary of Substances and Their Effects. Royal Society of Chemistry (1999) Available via DIALOG. \\
\url{http://www.rsc.org/dose/title of subordinate document. Cited 15 Jan 1999}
%
% Monograph
\bibitem{science-mono} Geddes, K.O., Czapor, S.R., Labahn, G.: Algorithms for Computer Algebra. Kluwer, Boston (1992) 
%
% Journal article
\bibitem{science-journal} Hamburger, C.: Quasimonotonicity, regularity and duality for nonlinear systems of partial differential equations. Ann. Mat. Pura. Appl. \textbf{169}, 321--354 (1995)
%
% Journal article by DOI
\bibitem{science-DOI} Slifka, M.K., Whitton, J.L.: Clinical implications of dysregulated cytokine production. J. Mol. Med. (2000) doi: 10.1007/s001090000086 
%
\bigskip

% Use the following (APS) syntax and markup for your references if 
% the subject of your book is from the field 
% "Mathematics, Physics, Statistics, Computer Science"
%
% Online Document
\bibitem{phys-online} J. Dod, in \textit{The Dictionary of Substances and Their Effects}, Royal Society of Chemistry. (Available via DIALOG, 1999), 
\url{http://www.rsc.org/dose/title of subordinate document. Cited 15 Jan 1999}
%
% Monograph
\bibitem{phys-mono} H. Ibach, H. L\"uth, \textit{Solid-State Physics}, 2nd edn. (Springer, New York, 1996), pp. 45-56 
%
% Journal article
\bibitem{phys-journal} S. Preuss, A. Demchuk Jr., M. Stuke, Appl. Phys. A \textbf{61}
%
% Journal article by DOI
\bibitem{phys-DOI} M.K. Slifka, J.L. Whitton, J. Mol. Med., doi: 10.1007/s001090000086
%
% Contribution 
\bibitem{phys-contrib} S.E. Smith, in \textit{Neuromuscular Junction}, ed. by E. Zaimis. Handbook of Experimental Pharmacology, vol 42 (Springer, Heidelberg, 1976), p. 593
%
\bigskip
%
% Use the following syntax and markup for your references if 
% the subject of your book is from the field 
% "Psychology, Social Sciences"
%
%
% Monograph
\bibitem{psysoc-mono} Calfee, R.~C., \& Valencia, R.~R. (1991). \textit{APA guide to preparing manuscripts for journal publication.} Washington, DC: American Psychological Association.
%
% Online Document
\bibitem{psysoc-online} Dod, J. (1999). Effective substances. In: The dictionary of substances and their effects. Royal Society of Chemistry. Available via DIALOG. \\
\url{http://www.rsc.org/dose/Effective substances.} Cited 15 Jan 1999.
%
% Journal article
\bibitem{psysoc-journal} Harris, M., Karper, E., Stacks, G., Hoffman, D., DeNiro, R., Cruz, P., et al. (2001). Writing labs and the Hollywood connection. \textit{J Film} Writing, 44(3), 213--245.
%
% Contribution 
\bibitem{psysoc-contrib} O'Neil, J.~M., \& Egan, J. (1992). Men's and women's gender role journeys: Metaphor for healing, transition, and transformation. In B.~R. Wainrig (Ed.), \textit{Gender issues across the life cycle} (pp. 107--123). New York: Springer.
%
% Journal article by DOI
\bibitem{psysoc-DOI}Kreger, M., Brindis, C.D., Manuel, D.M., Sassoubre, L. (2007). Lessons learned in systems change initiatives: benchmarks and indicators. \textit{American Journal of Community Psychology}, doi: 10.1007/s10464-007-9108-14.
%
%
% Use the following syntax and markup for your references if 
% the subject of your book is from the field 
% "Humanities, Linguistics, Philosophy"
%
\bigskip
%
% Journal article
\bibitem{humlinphil-journal} Alber John, Daniel C. O'Connell, and Sabine Kowal. 2002. Personal perspective in TV interviews. \textit{Pragmatics} 12:257--271
%
% Contribution 
\bibitem{humlinphil-contrib} Cameron, Deborah. 1997. Theoretical debates in feminist linguistics: Questions of sex and gender. In \textit{Gender and discourse}, ed. Ruth Wodak, 99--119. London: Sage Publications.
%
% Monograph
\bibitem{humlinphil-mono} Cameron, Deborah. 1985. \textit{Feminism and linguistic theory.} New York: St. Martin's Press.
%
% Online Document
\bibitem{humlinphil-online} Dod, Jake. 1999. Effective substances. In: The dictionary of substances and their effects. Royal Society of Chemistry. Available via DIALOG. \\
http://www.rsc.org/dose/title of subordinate document. Cited 15 Jan 1999
%
% Journal article by DOI
\bibitem{humlinphil-DOI} Suleiman, Camelia, Daniel C. O�Connell, and Sabine Kowal. 2002. `If you and I, if we, in this later day, lose that sacred fire...�': Perspective in political interviews. \textit{Journal of Psycholinguistic Research}. doi: 10.1023/A:1015592129296.
%
%
%
\bigskip
%
%
% Use the following syntax and markup for your references if 
% the subject of your book is from the field 
% "Computer Science, Economics, Engineering, Geosciences, Life Sciences"
%
%
% Contribution 
\bibitem{basic-contrib} Brown B, Aaron M (2001) The politics of nature. In: Smith J (ed) The rise of modern genomics, 3rd edn. Wiley, New York 
%
% Online Document
\bibitem{basic-online} Dod J (1999) Effective Substances. In: The dictionary of substances and their effects. Royal Society of Chemistry. Available via DIALOG. \\
\url{http://www.rsc.org/dose/title of subordinate document. Cited 15 Jan 1999}
%
% Journal article by DOI
\bibitem{basic-DOI} Slifka MK, Whitton JL (2000) Clinical implications of dysregulated cytokine production. J Mol Med, doi: 10.1007/s001090000086
%
% Journal article
\bibitem{basic-journal} Smith J, Jones M Jr, Houghton L et al (1999) Future of health insurance. N Engl J Med 965:325--329
%
% Monograph
\bibitem{basic-mono} South J, Blass B (2001) The future of modern genomics. Blackwell, London 
%
\end{thebibliography}

