%%%%%%%%%%%%%%%%%%%%%part.tex%%%%%%%%%%%%%%%%%%%%%%%%%%%%%%%%%%
% 
% sample part title
%
% Use this file as a template for your own input.
%
%%%%%%%%%%%%%%%%%%%%%%%% Springer %%%%%%%%%%%%%%%%%%%%%%%%%%

\begin{partbacktext}
\part{Ninety-Nine Lisp Problems}
\noindent Based on a Prolog problem list by werner.hett@hti.bfh.ch. The original
is at

\href{https://prof.ti.bfh.ch/hew1/informatik3/prolog/p-99}{https://prof.ti.bfh.ch/hew1/informatik3/prolog/p-99}.

Work in progress! Until now, only about half of the problems are solved.
Another possibility, of course, would be translating the Prolog
solutions to Pilog ;-)
\end{partbacktext}