%%%%%%%%%%%%%%%%%%%%%part.tex%%%%%%%%%%%%%%%%%%%%%%%%%%%%%%%%%%
% 
% sample part title
%
% Use this file as a template for your own input.
%
%%%%%%%%%%%%%%%%%%%%%%%% Springer %%%%%%%%%%%%%%%%%%%%%%%%%%

\begin{partbacktext}
\part{Rosetta Code}
\noindent Rosetta Code
(\href{http://rosettacode.org/wiki/Rosetta_Code}{rosettacode.org})
is a programming chrestomathy site. The idea is to present solutions
to the same task in as many different languages as possible, to
demonstrate how languages are similar and different, and to aid a
person with a grounding in one approach to a problem in learning
another. Rosetta Code currently\footnote{accessed online 21-08-2012}
has 600 tasks, 97 draft tasks, and is aware of 471 languages.
\end{partbacktext}