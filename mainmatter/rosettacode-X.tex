%%%%%%%%%%%%%%%%%%%%% chapter.tex %%%%%%%%%%%%%%%%%%%%%%%%%%%%%%%%%
%
% sample chapter
%
% Use this file as a template for your own input.
%
%%%%%%%%%%%%%%%%%%%%%%%% Springer-Verlag %%%%%%%%%%%%%%%%%%%%%%%%%%
%\motto{Use the template \emph{chapter.tex} to style the various elements of your chapter content.}

\chapter{Rosetta Code Tasks starting with X}

\section*{XML/DOM serialization}

Create a simple DOM and having it serialize to:

\begin{verbatim}
 <?xml version="1.0" ?>
 <root>
     <element>
         Some text here
     </element>
 </root>
\end{verbatim}


\begin{wideverbatim}

(load "@lib/xm.l")

(xml? T)
(xml '(root NIL (element NIL "Some text here")))

Output:

<?xml version="1.0" encoding="utf-8"?>
<root>
   <element>Some text here</element>
</root>

\end{wideverbatim}

\pagebreak{}
\section*{XML/Input}

Given the following XML fragment, extract the list of \emph{student
  names} using whatever means desired. If the only viable method is to
use XPath, refer the reader to the task \emph{XML and XPath}.

\begin{wideverbatim}
<Students>
  <Student Name="April" Gender="F" DateOfBirth="1989-01-02" />
  <Student Name="Bob" Gender="M"  DateOfBirth="1990-03-04" />
  <Student Name="Chad" Gender="M"  DateOfBirth="1991-05-06" />
  <Student Name="Dave" Gender="M"  DateOfBirth="1992-07-08">
    <Pet Type="dog" Name="Rover" />
  </Student>
  <Student DateOfBirth="1993-09-10" Gender="F" Name="&#x00C9;mily" />
</Students>
\end{wideverbatim}

Expected Output

\begin{verbatim}
April
Bob
Chad
Dave
Émily
\end{verbatim}


\begin{wideverbatim}

(load "@lib/xm.l")

(mapcar
   '((L) (attr L 'Name))
   (body (in "file.xml" (xml))) )

Output:

-> ("April" "Bob" "Chad" "Dave" "Émily")

\end{wideverbatim}

\pagebreak{}
\section*{XML/Output}

Create a function that takes a list of character names and a list of
corresponding remarks and returns an XML document of
\texttt{\textless{}Character\textgreater{}} elements each with a name
attributes and each enclosing its remarks. All
\texttt{\textless{}Character\textgreater{}} elements are to be enclosed
in turn, in an outer \texttt{\textless{}CharacterRemarks\textgreater{}}
element.

As an example, calling the function with the three names of:

\begin{verbatim}
April
Tam O'Shanter
Emily
\end{verbatim}

And three remarks of:

\begin{verbatim}
Bubbly: I'm > Tam and <= Emily
Burns: "When chapman billies leave the street ..."
Short & shrift
\end{verbatim}

Should produce the XML (but not necessarily with the indentation):

\begin{wideverbatim}

<CharacterRemarks>
    <Character name="April">
      Bubbly: I'm &gt; Tam and &lt;= Emily
    </Character>
    <Character name="Tam O'Shanter">
      Burns:"When chapman billies leave the street ..."
    </Character>
    <Character name="Emily">
      Short &amp; shrift
    </Character>
</CharacterRemarks>

\end{wideverbatim}

The document may include an \texttt{\textless{}?xml?\textgreater{}}
declaration and document type declaration, but these are optional. If
attempting this task by direct string manipulation, the implementation
\emph{must} include code to perform entity substitution for the
characters that have entities defined in the XML 1.0 specification.

Note: the example is chosen to show correct escaping of XML strings.
Note too that although the task is written to take two lists of
corresponding data, a single mapping/hash/dictionary of names to remarks
is also acceptable.

\pagebreak{}

\textbf{Note to editors:} Program output with escaped characters will be
viewed as the character on the page so you need to `escape-the-escapes'
to make the RC entry display what would be shown in a plain text viewer
(See \emph{this}).
Alternately, output can be placed in \textless{}lang
xml\textgreater{}\textless{}/lang\textgreater{} tags without any special
treatment.



\begin{wideverbatim}

(load "@lib/xm.l")

(de characterRemarks (Names Remarks)
   (xml
      (cons
         'CharacterRemarks
         NIL
         (mapcar
            '((Name Remark)
               (list 'Character (list (cons 'name Name)) Remark) )
            Names
            Remarks ) ) ) )

(characterRemarks
   '("April" "Tam O'Shanter" "Emily")
   (quote
      "I'm > Tam and <= Emily"
      "Burns: \"When chapman billies leave the street ..."
      "Short \& shrift" ) )

Output:

<CharacterRemarks>
   <Character name="April">I'm > Tam and \&#60;= Emily</Character>
   <Character name="Tam O'Shanter">Burns: \&#34;
   When chapman billies leave the street ...</Character>
   <Character name="Emily">Short \&#38; shrift</Character>
</CharacterRemarks>

\end{wideverbatim}

\pagebreak{}
\section*{XML/XPath}

Perform the following three XPath queries on the XML Document below:

\begin{itemize}
\item
  Retrieve the first ``item'' element
\item
  Perform an action on each ``price'' element (print it out)
\item
  Get an array of all the ``name'' elements
\end{itemize}

XML Document:

\begin{wideverbatim}
<inventory title="OmniCorp Store #45x10^3">
  <section name="health">
    <item upc="123456789" stock="12">
      <name>Invisibility Cream</name>
      <price>14.50</price>
      <description>Makes you invisible</description>
    </item>
    <item upc="445322344" stock="18">
      <name>Levitation Salve</name>
      <price>23.99</price>
      <description>Levitate yourself for up to 3 hours per application
      </description>
    </item>
  </section>
  <section name="food">
    <item upc="485672034" stock="653">
      <name>Blork and Freen Instameal</name>
      <price>4.95</price>
      <description>A tasty meal in a tablet; just add water</description>
    </item>
    <item upc="132957764" stock="44">
      <name>Grob winglets</name>
      <price>3.56</price>
      <description>Tender winglets of Grob. Just add water</description>
    </item>
  </section>
</inventory>
\end{wideverbatim}

\begin{wideverbatim}

(load "@lib/xm.l")

(let Sections (body (in "file.xml" (xml)))
   (pretty (car (body (car Sections))))
   (prinl)
   (for S Sections
      (for L (body S)
         (prinl (car (body L 'price))) ) )
   (make
      (for S Sections
         (for L (body S)
            (link (car (body L 'name))) ) ) ) )

Output:

(item
   ((upc . "123456789") (stock . "12"))
   (name NIL "Invisibility Cream")
   (price NIL "14.50")
   (description NIL "Makes you invisible") )
14.50
23.99
4.95
3.56
-> ("Invisibility Cream" "Levitation Salve"
    "Blork and Freen Instameal" "Grob winglets")

\end{wideverbatim}

\pagebreak{}
\section*{Xiaolin Wu's line algorithm}

Implement the
\href{http://en.wikipedia.org/wiki/Xiaolin\_Wu\%27s\_line\_algorithm}{Xiaolin
  Wu's line algorithm} as described in Wikipedia. This algorithm draw
antialiased lines. See \emph{Bresenham's line algorithm} for
\emph{aliased} lines.


\begin{wideverbatim}

(scl 2)

(de plot (Img X Y C)
   (set (nth Img (*/ Y 1.0) (*/ X 1.0)) (- 100 C)) )

(de ipart (X)
   (* 1.0 (/ X 1.0)) )

(de iround (X)
   (ipart (+ X 0.5)) )

(de fpart (X)
   (\% X 1.0) )

(de rfpart (X)
   (- 1.0 (fpart X)) )


\end{wideverbatim}

\begin{wideverbatim}


(de xiaolin (Img X1 Y1 X2 Y2)
   (let (DX (- X2 X1)  DY (- Y2 Y1))
      (use (Grad Xend Yend Xgap Xpxl1 Ypxl1 Xpxl2 Ypxl2 Intery)
         (when (> (abs DY) (abs DX))
            (xchg 'X1 'Y1  'X2 'Y2) )
         (when (> X1 X2)
            (xchg 'X1 'X2  'Y1 'Y2) )
         (setq
            Grad (*/ DY 1.0 DX)
            Xend (iround X1)
            Yend (+ Y1 (*/ Grad (- Xend X1) 1.0))
            Xgap (rfpart (+ X1 0.5))
            Xpxl1 Xend
            Ypxl1 (ipart Yend) )
         (plot Img Xpxl1 Ypxl1 (*/ (rfpart Yend) Xgap 1.0))
         (plot Img Xpxl1 (+ 1.0 Ypxl1) (*/ (fpart Yend) Xgap 1.0))
         (setq
            Intery (+ Yend Grad)
            Xend (iround X2)
            Yend (+ Y2 (*/ Grad (- Xend X2) 1.0))
            Xgap (fpart (+ X2 0.5))
            Xpxl2 Xend
            Ypxl2 (ipart Yend) )
         (plot Img Xpxl2 Ypxl2 (*/ (rfpart Yend) Xgap 1.0))
         (plot Img Xpxl2 (+ 1.0 Ypxl2) (*/ (fpart Yend) Xgap 1.0))
         (for (X (+ Xpxl1 1.0)  (>= (- Xpxl2 1.0) X)  (+ X 1.0))
            (plot Img X (ipart Intery) (rfpart Intery))
            (plot Img X (+ 1.0 (ipart Intery)) (fpart Intery))
            (inc 'Intery Grad) ) ) ) )

(let Img (make (do 90 (link (need 120 99))))       # Create image 120 x 90
   (xiaolin Img 10.0 10.0 110.0 80.0)              # Draw lines
   (xiaolin Img 10.0 10.0 110.0 45.0)
   (xiaolin Img 10.0 80.0 110.0 45.0)
   (xiaolin Img 10.0 80.0 110.0 10.0)
   (out "img.pgm"                                  # Write to bitmap file
      (prinl "P2")
      (prinl 120 " " 90)
      (prinl 100)
      (for Y Img (apply printsp Y)) ) )

\end{wideverbatim}



% %%%%%%%%%%%%%%%%%%%%%%%% referenc.tex %%%%%%%%%%%%%%%%%%%%%%%%%%%%%%
% sample references
% %
% Use this file as a template for your own input.
%
%%%%%%%%%%%%%%%%%%%%%%%% Springer-Verlag %%%%%%%%%%%%%%%%%%%%%%%%%%
%
% BibTeX users please use
% \bibliographystyle{}
% \bibliography{}
%
\biblstarthook{In view of the parallel print and (chapter-wise) online publication of your book at \url{www.springerlink.com} it has been decided that -- as a genreral rule --  references should be sorted chapter-wise and placed at the end of the individual chapters. However, upon agreement with your contact at Springer you may list your references in a single seperate chapter at the end of your book. Deactivate the class option \texttt{sectrefs} and the \texttt{thebibliography} environment will be put out as a chapter of its own.\\\indent
References may be \textit{cited} in the text either by number (preferred) or by author/year.\footnote{Make sure that all references from the list are cited in the text. Those not cited should be moved to a separate \textit{Further Reading} section or chapter.} The reference list should ideally be \textit{sorted} in alphabetical order -- even if reference numbers are used for the their citation in the text. If there are several works by the same author, the following order should be used: 
\begin{enumerate}
\item all works by the author alone, ordered chronologically by year of publication
\item all works by the author with a coauthor, ordered alphabetically by coauthor
\item all works by the author with several coauthors, ordered chronologically by year of publication.
\end{enumerate}
The \textit{styling} of references\footnote{Always use the standard abbreviation of a journal's name according to the ISSN \textit{List of Title Word Abbreviations}, see \url{http://www.issn.org/en/node/344}} depends on the subject of your book:
\begin{itemize}
\item The \textit{two} recommended styles for references in books on \textit{mathematical, physical, statistical and computer sciences} are depicted in ~\cite{science-contrib, science-online, science-mono, science-journal, science-DOI} and ~\cite{phys-online, phys-mono, phys-journal, phys-DOI, phys-contrib}.
\item Examples of the most commonly used reference style in books on \textit{Psychology, Social Sciences} are~\cite{psysoc-mono, psysoc-online,psysoc-journal, psysoc-contrib, psysoc-DOI}.
\item Examples for references in books on \textit{Humanities, Linguistics, Philosophy} are~\cite{humlinphil-journal, humlinphil-contrib, humlinphil-mono, humlinphil-online, humlinphil-DOI}.
\item Examples of the basic Springer style used in publications on a wide range of subjects such as \textit{Computer Science, Economics, Engineering, Geosciences, Life Sciences, Medicine, Biomedicine} are ~\cite{basic-contrib, basic-online, basic-journal, basic-DOI, basic-mono}. 
\end{itemize}
}

\begin{thebibliography}{99.}%
% and use \bibitem to create references.
%
% Use the following syntax and markup for your references if 
% the subject of your book is from the field 
% "Mathematics, Physics, Statistics, Computer Science"
%
% Contribution 
\bibitem{science-contrib} Broy, M.: Software engineering --- from auxiliary to key technologies. In: Broy, M., Dener, E. (eds.) Software Pioneers, pp. 10-13. Springer, Heidelberg (2002)
%
% Online Document
\bibitem{science-online} Dod, J.: Effective substances. In: The Dictionary of Substances and Their Effects. Royal Society of Chemistry (1999) Available via DIALOG. \\
\url{http://www.rsc.org/dose/title of subordinate document. Cited 15 Jan 1999}
%
% Monograph
\bibitem{science-mono} Geddes, K.O., Czapor, S.R., Labahn, G.: Algorithms for Computer Algebra. Kluwer, Boston (1992) 
%
% Journal article
\bibitem{science-journal} Hamburger, C.: Quasimonotonicity, regularity and duality for nonlinear systems of partial differential equations. Ann. Mat. Pura. Appl. \textbf{169}, 321--354 (1995)
%
% Journal article by DOI
\bibitem{science-DOI} Slifka, M.K., Whitton, J.L.: Clinical implications of dysregulated cytokine production. J. Mol. Med. (2000) doi: 10.1007/s001090000086 
%
\bigskip

% Use the following (APS) syntax and markup for your references if 
% the subject of your book is from the field 
% "Mathematics, Physics, Statistics, Computer Science"
%
% Online Document
\bibitem{phys-online} J. Dod, in \textit{The Dictionary of Substances and Their Effects}, Royal Society of Chemistry. (Available via DIALOG, 1999), 
\url{http://www.rsc.org/dose/title of subordinate document. Cited 15 Jan 1999}
%
% Monograph
\bibitem{phys-mono} H. Ibach, H. L\"uth, \textit{Solid-State Physics}, 2nd edn. (Springer, New York, 1996), pp. 45-56 
%
% Journal article
\bibitem{phys-journal} S. Preuss, A. Demchuk Jr., M. Stuke, Appl. Phys. A \textbf{61}
%
% Journal article by DOI
\bibitem{phys-DOI} M.K. Slifka, J.L. Whitton, J. Mol. Med., doi: 10.1007/s001090000086
%
% Contribution 
\bibitem{phys-contrib} S.E. Smith, in \textit{Neuromuscular Junction}, ed. by E. Zaimis. Handbook of Experimental Pharmacology, vol 42 (Springer, Heidelberg, 1976), p. 593
%
\bigskip
%
% Use the following syntax and markup for your references if 
% the subject of your book is from the field 
% "Psychology, Social Sciences"
%
%
% Monograph
\bibitem{psysoc-mono} Calfee, R.~C., \& Valencia, R.~R. (1991). \textit{APA guide to preparing manuscripts for journal publication.} Washington, DC: American Psychological Association.
%
% Online Document
\bibitem{psysoc-online} Dod, J. (1999). Effective substances. In: The dictionary of substances and their effects. Royal Society of Chemistry. Available via DIALOG. \\
\url{http://www.rsc.org/dose/Effective substances.} Cited 15 Jan 1999.
%
% Journal article
\bibitem{psysoc-journal} Harris, M., Karper, E., Stacks, G., Hoffman, D., DeNiro, R., Cruz, P., et al. (2001). Writing labs and the Hollywood connection. \textit{J Film} Writing, 44(3), 213--245.
%
% Contribution 
\bibitem{psysoc-contrib} O'Neil, J.~M., \& Egan, J. (1992). Men's and women's gender role journeys: Metaphor for healing, transition, and transformation. In B.~R. Wainrig (Ed.), \textit{Gender issues across the life cycle} (pp. 107--123). New York: Springer.
%
% Journal article by DOI
\bibitem{psysoc-DOI}Kreger, M., Brindis, C.D., Manuel, D.M., Sassoubre, L. (2007). Lessons learned in systems change initiatives: benchmarks and indicators. \textit{American Journal of Community Psychology}, doi: 10.1007/s10464-007-9108-14.
%
%
% Use the following syntax and markup for your references if 
% the subject of your book is from the field 
% "Humanities, Linguistics, Philosophy"
%
\bigskip
%
% Journal article
\bibitem{humlinphil-journal} Alber John, Daniel C. O'Connell, and Sabine Kowal. 2002. Personal perspective in TV interviews. \textit{Pragmatics} 12:257--271
%
% Contribution 
\bibitem{humlinphil-contrib} Cameron, Deborah. 1997. Theoretical debates in feminist linguistics: Questions of sex and gender. In \textit{Gender and discourse}, ed. Ruth Wodak, 99--119. London: Sage Publications.
%
% Monograph
\bibitem{humlinphil-mono} Cameron, Deborah. 1985. \textit{Feminism and linguistic theory.} New York: St. Martin's Press.
%
% Online Document
\bibitem{humlinphil-online} Dod, Jake. 1999. Effective substances. In: The dictionary of substances and their effects. Royal Society of Chemistry. Available via DIALOG. \\
http://www.rsc.org/dose/title of subordinate document. Cited 15 Jan 1999
%
% Journal article by DOI
\bibitem{humlinphil-DOI} Suleiman, Camelia, Daniel C. O�Connell, and Sabine Kowal. 2002. `If you and I, if we, in this later day, lose that sacred fire...�': Perspective in political interviews. \textit{Journal of Psycholinguistic Research}. doi: 10.1023/A:1015592129296.
%
%
%
\bigskip
%
%
% Use the following syntax and markup for your references if 
% the subject of your book is from the field 
% "Computer Science, Economics, Engineering, Geosciences, Life Sciences"
%
%
% Contribution 
\bibitem{basic-contrib} Brown B, Aaron M (2001) The politics of nature. In: Smith J (ed) The rise of modern genomics, 3rd edn. Wiley, New York 
%
% Online Document
\bibitem{basic-online} Dod J (1999) Effective Substances. In: The dictionary of substances and their effects. Royal Society of Chemistry. Available via DIALOG. \\
\url{http://www.rsc.org/dose/title of subordinate document. Cited 15 Jan 1999}
%
% Journal article by DOI
\bibitem{basic-DOI} Slifka MK, Whitton JL (2000) Clinical implications of dysregulated cytokine production. J Mol Med, doi: 10.1007/s001090000086
%
% Journal article
\bibitem{basic-journal} Smith J, Jones M Jr, Houghton L et al (1999) Future of health insurance. N Engl J Med 965:325--329
%
% Monograph
\bibitem{basic-mono} South J, Blass B (2001) The future of modern genomics. Blackwell, London 
%
\end{thebibliography}

