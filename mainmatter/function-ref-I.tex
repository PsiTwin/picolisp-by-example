%%%%%%%%%%%%%%%%%%%%% chapter.tex %%%%%%%%%%%%%%%%%%%%%%%%%%%%%%%%%
%
% sample chapter
%
% Use this file as a template for your own input.
%
%%%%%%%%%%%%%%%%%%%%%%%% Springer-Verlag %%%%%%%%%%%%%%%%%%%%%%%%%%
%\motto{Use the template \emph{chapter.tex} to style the various elements of your chapter content.}


\chapter{Symbols starting with I}
\label{cha:func-ref-I-functions-starting-with-I}

 
\section*{\texttt{+Idx}}
\label{sec:func-ref-I-+Idx}

Prefix class for maintaining non-unique full-text indexes to \texttt{+String}
relations, a subclass of \texttt{+Ref}. Accepts optional arguments for the
minimally indexed substring length (defaults to 3), and a \texttt{+Hook}
attribute. Often used in combination with the \texttt{+Sn} soundex index, or
the \texttt{+Fold} index prefix classes. See also \texttt{Database}.


\begin{wideverbatim}
(rel nm (+Sn +Idx +String))  # Name
\end{wideverbatim}

 
\section*{\texttt{+index}}
\label{sec:func-ref-I-+index}


Abstract base class of all database B-Tree index relations (prefix
classes for \texttt{+relation}s). The class hierarchy includes
\texttt{+Key}, \texttt{+Ref} and \texttt{+Idx}. See also
\texttt{Database}.


\begin{wideverbatim}
(isa '+index Rel)  # Check for an index relation
\end{wideverbatim}

 
\section*{\texttt{(id 'num ['num]) -> sym}}
\label{sec:func-ref-I-(id 'num ['num]) -> sym}


\texttt{(id 'sym [NIL]) -> num}

\texttt{(id 'sym T) -> (num . num)}

Converts one or two numbers to an external symbol, or an external symbol
to a number or a pair of numbers.


\begin{wideverbatim}
: (id 7)
-> {7}
: (id 1 2)
-> {2}
: (id '{1-2})
-> 2
: (id '{1-2} T)
-> (1 . 2)
\end{wideverbatim}

 
\section*{\texttt{(idx 'var 'any 'flg) -> lst (idx 'var 'any) -> lst (idx 'var) -> lst}}
\label{sec:func-ref-I-(idx 'var 'any 'flg) -> lst (idx 'var 'any) -> lst (idx 'var) -> lst}


Maintains an index tree in \texttt{var}, and checks for the existence of \texttt{any}.
If \texttt{any} is contained in \texttt{var}, the corresponding subtree is returned,
otherwise \texttt{NIL}. In the first form, \texttt{any} is destructively inserted into
the tree if \texttt{flg} is non-\texttt{NIL} (and \texttt{any} was not already there), or
deleted from the tree if \texttt{flg} is \texttt{NIL}. The second form only checks for
existence, but does not change the index tree. In the third form (when
called with a single \texttt{var} argument) the contents of the tree are
returned as a sorted list. If all elements are inserted in sorted order,
the tree degenerates into a linear list. See also \texttt{lup}, \texttt{hash},
\texttt{depth}, \texttt{sort}, \texttt{balance} and \texttt{member}.


\begin{wideverbatim}
: (idx 'X 'd T)                              # Insert data
-> NIL
: (idx 'X 2 T)
-> NIL
: (idx 'X '(a b c) T)
-> NIL
: (idx 'X 17 T)
-> NIL
: (idx 'X 'A T)
-> NIL
: (idx 'X 'd T)
-> (d (2 NIL 17 NIL A) (a b c))              # 'd' already existed
: (idx 'X T T)
-> NIL
: X                                          # View the index tree
-> (d (2 NIL 17 NIL A) (a b c) NIL T)
: (idx 'X 'A)                                # Check for 'A'
-> (A)
: (idx 'X 'B)                                # Check for 'B'
-> NIL
: (idx 'X)
-> (2 17 A d (a b c) T)                      # Get list
: (idx 'X 17 NIL)                            # Delete '17'
-> (17 NIL A)
: X
-> (d (2 NIL A) (a b c) NIL T)               # View it again
: (idx 'X)
-> (2 A d (a b c) T)                         # '17' is deleted
\end{wideverbatim}

 
\section*{\texttt{(if 'any1 'any2 . prg) -> any}}
\label{sec:func-ref-I-(if 'any1 'any2 . prg) -> any}


Conditional execution: If the condition \texttt{any1} evaluates to non-\texttt{NIL},
\texttt{any2} is evaluated and returned. Otherwise, \texttt{prg} is executed and the
result returned. See also \texttt{cond}, \texttt{when} and \texttt{if2}.


\begin{wideverbatim}
: (if (> 4 3) (println 'OK) (println 'Bad))
OK
-> OK
: (if (> 3 4) (println 'OK) (println 'Bad))
Bad
-> Bad
\end{wideverbatim}

 
\section*{\texttt{(if2 'any1 'any2 'any3 'any4 'any5 . prg) -> any}}
\label{sec:func-ref-I-(if2 'any1 'any2 'any3 'any4 'any5 . prg) -> any}


Four-way conditional execution for two conditions: If both conditions
\texttt{any1} and \texttt{any2} evaluate to non-\texttt{NIL}, \texttt{any3} is evaluated and
returned. Otherwise, \texttt{any4} or \texttt{any5} is evaluated and returned if
\texttt{any1} or \texttt{any2} evaluate to non-\texttt{NIL}, respectively. If none of the
conditions evaluate to non-\texttt{NIL}, \texttt{prg} is executed and the result
returned. See also \texttt{if} and \texttt{cond}.


\begin{wideverbatim}
: (if2 T T 'both 'first 'second 'none)
-> both
: (if2 T NIL 'both 'first 'second 'none)
-> first
: (if2 NIL T 'both 'first 'second 'none)
-> second
: (if2 NIL NIL 'both 'first 'second 'none)
-> none
\end{wideverbatim}

 
\section*{\texttt{(ifn 'any1 'any2 . prg) -> any}}
\label{sec:func-ref-I-(ifn 'any1 'any2 . prg) -> any}


Conditional execution (``If not''): If the condition \texttt{any1} evaluates to
\texttt{NIL}, \texttt{any2} is evaluated and returned. Otherwise, \texttt{prg} is executed
and the result returned.


\begin{wideverbatim}
: (ifn (= 3 4) (println 'OK) (println 'Bad))
OK
-> OK
\end{wideverbatim}

 
\section*{\texttt{(import lst) -> NIL}}
\label{sec:func-ref-I-(import lst) -> NIL}


Wrapper function for \texttt{intern}. Typically used to import symbols from
other namespaces, as created by \texttt{symbols}. \texttt{lst} should be a list of
symbols. An import conflict error is issued when a symbol with the same
name already exists in the current namespace. See also \texttt{pico} and
\texttt{local}.


\begin{wideverbatim}
: (import libA~foo libB~bar)
-> NIL
\end{wideverbatim}

 
\section*{\texttt{(in 'any . prg) -> any}}
\label{sec:func-ref-I-(in 'any . prg) -> any}


Opens \texttt{any} as input channel during the execution of \texttt{prg}. The current
input channel will be saved and restored appropriately. If the argument
is \texttt{NIL}, standard input is used. If the argument is a symbol, it is
used as a file name (opened for reading \emph{and} writing if the first
character is ``\texttt{+}''). If it is a positive number, it is used as the
descriptor of an open file. If it is a negative number, the saved input
channel such many levels above the current one is used. Otherwise (if it
is a list), it is taken as a command with arguments, and a pipe is
opened for input. See also \texttt{ipid}, \texttt{call}, \texttt{load}, \texttt{file}, \texttt{out}, \texttt{err},
\texttt{poll}, \texttt{pipe} and \texttt{ctl}.


\begin{wideverbatim}
: (in "a" (list (read) (read) (read)))  # Read three items from file "a"
-> (123 (a b c) def)
\end{wideverbatim}

 
\section*{\texttt{(inc 'num) -> num (inc 'var ['num]) -> num}}
\label{sec:func-ref-I-(inc 'num) -> num (inc 'var ['num]) -> num}


The first form returns the value of \texttt{num} incremented by 1. The second
form increments the \texttt{VAL} of \texttt{var} by 1, or by \texttt{num}. If the first
argument is \texttt{NIL}, it is returned immediately. \texttt{(inc 'num)} is
equivalent to \texttt{(+ 'num 1)} and \texttt{(inc 'var)} is equivalent to
\texttt{(set 'var (+ var 1))}. See also \texttt{dec} and \texttt{+}.


\begin{wideverbatim}
: (inc 7)
-> 8
: (inc -1)
-> 0
: (zero N)
-> 0
: (inc 'N)
-> 1
: (inc 'N 7)
-> 8
: N
-> 8

: (setq L (1 2 3 4))
-> (1 2 3 4)
: (inc (cdr L))
-> 3
: L
-> (1 3 3 4)
\end{wideverbatim}

 
\section*{\texttt{(inc! 'obj 'sym ['num]) -> num}}
\label{sec:func-ref-I-(inc! 'obj 'sym ['num]) -> num}


\emph{Transaction} wrapper function for \texttt{inc}. \texttt{num}
defaults to 1. Note that for incrementing a property value of an entity
typically the \texttt{inc!>} message is used. See also \texttt{new!}, \texttt{set!} and
\texttt{put!}.


\begin{wideverbatim}
(inc! Obj 'cnt 0)  # Incrementing a property of a non-entity object
\end{wideverbatim}

 
\section*{\texttt{(index 'any 'lst) -> cnt | NIL}}
\label{sec:func-ref-I-(index 'any 'lst) -> cnt | NIL}


Returns the \texttt{cnt} position of \texttt{any} in \texttt{lst}, or \texttt{NIL} if it is not
found. See also \texttt{offset}.


\begin{wideverbatim}
: (index 'c '(a b c d e f))
-> 3
: (index '(5 6) '((1 2) (3 4) (5 6) (7 8)))
-> 3
\end{wideverbatim}

 
\section*{\texttt{(info 'any) -> (cnt|T dat . tim)}}
\label{sec:func-ref-I-(info 'any) -> (cnt|T dat . tim)}


Returns information about a file with the name \texttt{any}: The current size
\texttt{cnt} in bytes, and the modification date and time (UTC). For
directories, \texttt{T} is returned instead of the a size. See also \texttt{dir},
\texttt{date}, \texttt{time} and \texttt{lines}.


\begin{wideverbatim}
$ ls -l x.l
-rw-r--r--   1 abu      users         208 Jun 17 08:58 x.l
$ pil +
: (info "x.l")
-> (208 730594 . 32315)
: (stamp 730594 32315)
-> "2000-06-17 08:58:35"
\end{wideverbatim}

 
\section*{\texttt{(init 'tree ['any1] ['any2]) -> lst}}
\label{sec:func-ref-I-(init 'tree ['any1] ['any2]) -> lst}


Initializes a structure for stepping iteratively through a database
tree. \texttt{any1} and \texttt{any2} may specify a range of keys. If \texttt{any2} is
greater than \texttt{any1}, the traversal will be in opposite direction. See
also \texttt{tree}, \texttt{step}, \texttt{iter} and \texttt{scan}.


\begin{wideverbatim}
: (init (tree 'nr '+Item) 3 5)
-> (((3 . 5) ((3 NIL . {3-3}) (4 NIL . {3-4}) (5 NIL . {3-5}) (6 NIL . {3-6}) (7 NIL . {3-8}))))
\end{wideverbatim}

 
\section*{\texttt{(insert 'cnt 'lst 'any) -> lst}}
\label{sec:func-ref-I-(insert 'cnt 'lst 'any) -> lst}


Inserts \texttt{any} into \texttt{lst} at position \texttt{cnt}. This is a non-destructive
operation. See also \texttt{remove}, \texttt{place}, \texttt{append}, \texttt{delete} and \texttt{replace}.


\begin{wideverbatim}
: (insert 3 '(a b c d e) 777)
-> (a b 777 c d e)
: (insert 1 '(a b c d e) 777)
-> (777 a b c d e)
: (insert 9 '(a b c d e) 777)
-> (a b c d e 777)
\end{wideverbatim}

 
\section*{\texttt{(intern 'sym) -> sym}}
\label{sec:func-ref-I-(intern 'sym) -> sym}


Creates or finds an internal symbol. If a symbol with the name \texttt{sym} is
already intern, it is returned. Otherwise, \texttt{sym} is interned and
returned. See also \texttt{symbols}, \texttt{zap}, \texttt{extern} and \texttt{====}.


\begin{wideverbatim}
: (intern "abc")
-> abc
: (intern 'car)
-> car
: ((intern (pack "c" "a" "r")) (1 2 3))
-> 1
\end{wideverbatim}

 
\section*{\texttt{(ipid) -> pid | NIL}}
\label{sec:func-ref-I-(ipid) -> pid | NIL}


Returns the corresponding process ID when the current input channel is
reading from a pipe, otherwise \texttt{NIL}. See also \texttt{opid}, \texttt{in}, \texttt{pipe} and
\texttt{load}.


\begin{wideverbatim}
: (in '(ls "-l") (println (line T)) (kill (ipid)))
"total 7364"
-> T
\end{wideverbatim}

 
\section*{\texttt{(isa 'cls|typ 'obj) -> obj | NIL}}
\label{sec:func-ref-I-(isa 'cls|typ 'obj) -> obj | NIL}


Returns \texttt{obj} when it is an object that inherits from \texttt{cls} or \texttt{type}.
See also \texttt{OO Concepts}, \texttt{class}, \texttt{type}, \texttt{new} and \texttt{object}.


\begin{wideverbatim}
: (isa '+Address Obj)
-> {1-17}
: (isa '(+Male +Person) Obj)
-> NIL
\end{wideverbatim}

 
\section*{\texttt{isa/2}}
\label{sec:func-ref-I-isa/2}


\emph{Pilog} predicate that succeeds if the second argument
is of the type or class given by the first argument, according to the
\texttt{isa} function. Typically used in \texttt{db/3} or \texttt{select/3} database queries.
See also \texttt{same/3}, \texttt{bool/3}, \texttt{range/3}, \texttt{head/3}, \texttt{fold/3}, \texttt{part/3} and
\texttt{tolr/3}.


\begin{wideverbatim}
: (? (db nm +Person @Prs) (isa +Woman @Prs) (val @Nm @Prs nm))
 @Prs={2-Y} @Nm="Alexandra of Denmark"
 @Prs={2-1I} @Nm="Alice Maud Mary"
 @Prs={2-F} @Nm="Anne"
 @Prs={2-j} @Nm="Augusta Victoria".   # Stop
\end{wideverbatim}

 
\section*{\texttt{(iter 'tree ['fun] ['any1] ['any2] ['flg])}}
\label{sec:func-ref-I-(iter 'tree ['fun] ['any1] ['any2] ['flg])}


Iterates through a database tree by applying \texttt{fun} to all values. \texttt{fun}
defaults to \texttt{println}. \texttt{any1} and \texttt{any2} may specify a range of keys. If
\texttt{any2} is greater than \texttt{any1}, the traversal will be in opposite
direction. Note that the keys need not to be atomic, depending on the
application's index structure. If \texttt{flg} is non-\texttt{NIL}, partial keys are
skipped. See also \texttt{tree}, \texttt{scan}, \texttt{init} and \texttt{step}.


\begin{wideverbatim}
: (iter (tree 'nr '+Item))
{3-1}
{3-2}
{3-3}
{3-4}
{3-5}
{3-6}
{3-8}
-> {7-1}
: (iter (tree 'nr '+Item) '((This) (println (: nm))))
"Main Part"
"Spare Part"
"Auxiliary Construction"
"Enhancement Additive"
"Metal Fittings"
"Gadget Appliance"
"Testartikel"
-> {7-1}
\end{wideverbatim}


% %%%%%%%%%%%%%%%%%%%%%%%% referenc.tex %%%%%%%%%%%%%%%%%%%%%%%%%%%%%%
% sample references
% %
% Use this file as a template for your own input.
%
%%%%%%%%%%%%%%%%%%%%%%%% Springer-Verlag %%%%%%%%%%%%%%%%%%%%%%%%%%
%
% BibTeX users please use
% \bibliographystyle{}
% \bibliography{}
%
\biblstarthook{In view of the parallel print and (chapter-wise) online publication of your book at \url{www.springerlink.com} it has been decided that -- as a genreral rule --  references should be sorted chapter-wise and placed at the end of the individual chapters. However, upon agreement with your contact at Springer you may list your references in a single seperate chapter at the end of your book. Deactivate the class option \texttt{sectrefs} and the \texttt{thebibliography} environment will be put out as a chapter of its own.\\\indent
References may be \textit{cited} in the text either by number (preferred) or by author/year.\footnote{Make sure that all references from the list are cited in the text. Those not cited should be moved to a separate \textit{Further Reading} section or chapter.} The reference list should ideally be \textit{sorted} in alphabetical order -- even if reference numbers are used for the their citation in the text. If there are several works by the same author, the following order should be used: 
\begin{enumerate}
\item all works by the author alone, ordered chronologically by year of publication
\item all works by the author with a coauthor, ordered alphabetically by coauthor
\item all works by the author with several coauthors, ordered chronologically by year of publication.
\end{enumerate}
The \textit{styling} of references\footnote{Always use the standard abbreviation of a journal's name according to the ISSN \textit{List of Title Word Abbreviations}, see \url{http://www.issn.org/en/node/344}} depends on the subject of your book:
\begin{itemize}
\item The \textit{two} recommended styles for references in books on \textit{mathematical, physical, statistical and computer sciences} are depicted in ~\cite{science-contrib, science-online, science-mono, science-journal, science-DOI} and ~\cite{phys-online, phys-mono, phys-journal, phys-DOI, phys-contrib}.
\item Examples of the most commonly used reference style in books on \textit{Psychology, Social Sciences} are~\cite{psysoc-mono, psysoc-online,psysoc-journal, psysoc-contrib, psysoc-DOI}.
\item Examples for references in books on \textit{Humanities, Linguistics, Philosophy} are~\cite{humlinphil-journal, humlinphil-contrib, humlinphil-mono, humlinphil-online, humlinphil-DOI}.
\item Examples of the basic Springer style used in publications on a wide range of subjects such as \textit{Computer Science, Economics, Engineering, Geosciences, Life Sciences, Medicine, Biomedicine} are ~\cite{basic-contrib, basic-online, basic-journal, basic-DOI, basic-mono}. 
\end{itemize}
}

\begin{thebibliography}{99.}%
% and use \bibitem to create references.
%
% Use the following syntax and markup for your references if 
% the subject of your book is from the field 
% "Mathematics, Physics, Statistics, Computer Science"
%
% Contribution 
\bibitem{science-contrib} Broy, M.: Software engineering --- from auxiliary to key technologies. In: Broy, M., Dener, E. (eds.) Software Pioneers, pp. 10-13. Springer, Heidelberg (2002)
%
% Online Document
\bibitem{science-online} Dod, J.: Effective substances. In: The Dictionary of Substances and Their Effects. Royal Society of Chemistry (1999) Available via DIALOG. \\
\url{http://www.rsc.org/dose/title of subordinate document. Cited 15 Jan 1999}
%
% Monograph
\bibitem{science-mono} Geddes, K.O., Czapor, S.R., Labahn, G.: Algorithms for Computer Algebra. Kluwer, Boston (1992) 
%
% Journal article
\bibitem{science-journal} Hamburger, C.: Quasimonotonicity, regularity and duality for nonlinear systems of partial differential equations. Ann. Mat. Pura. Appl. \textbf{169}, 321--354 (1995)
%
% Journal article by DOI
\bibitem{science-DOI} Slifka, M.K., Whitton, J.L.: Clinical implications of dysregulated cytokine production. J. Mol. Med. (2000) doi: 10.1007/s001090000086 
%
\bigskip

% Use the following (APS) syntax and markup for your references if 
% the subject of your book is from the field 
% "Mathematics, Physics, Statistics, Computer Science"
%
% Online Document
\bibitem{phys-online} J. Dod, in \textit{The Dictionary of Substances and Their Effects}, Royal Society of Chemistry. (Available via DIALOG, 1999), 
\url{http://www.rsc.org/dose/title of subordinate document. Cited 15 Jan 1999}
%
% Monograph
\bibitem{phys-mono} H. Ibach, H. L\"uth, \textit{Solid-State Physics}, 2nd edn. (Springer, New York, 1996), pp. 45-56 
%
% Journal article
\bibitem{phys-journal} S. Preuss, A. Demchuk Jr., M. Stuke, Appl. Phys. A \textbf{61}
%
% Journal article by DOI
\bibitem{phys-DOI} M.K. Slifka, J.L. Whitton, J. Mol. Med., doi: 10.1007/s001090000086
%
% Contribution 
\bibitem{phys-contrib} S.E. Smith, in \textit{Neuromuscular Junction}, ed. by E. Zaimis. Handbook of Experimental Pharmacology, vol 42 (Springer, Heidelberg, 1976), p. 593
%
\bigskip
%
% Use the following syntax and markup for your references if 
% the subject of your book is from the field 
% "Psychology, Social Sciences"
%
%
% Monograph
\bibitem{psysoc-mono} Calfee, R.~C., \& Valencia, R.~R. (1991). \textit{APA guide to preparing manuscripts for journal publication.} Washington, DC: American Psychological Association.
%
% Online Document
\bibitem{psysoc-online} Dod, J. (1999). Effective substances. In: The dictionary of substances and their effects. Royal Society of Chemistry. Available via DIALOG. \\
\url{http://www.rsc.org/dose/Effective substances.} Cited 15 Jan 1999.
%
% Journal article
\bibitem{psysoc-journal} Harris, M., Karper, E., Stacks, G., Hoffman, D., DeNiro, R., Cruz, P., et al. (2001). Writing labs and the Hollywood connection. \textit{J Film} Writing, 44(3), 213--245.
%
% Contribution 
\bibitem{psysoc-contrib} O'Neil, J.~M., \& Egan, J. (1992). Men's and women's gender role journeys: Metaphor for healing, transition, and transformation. In B.~R. Wainrig (Ed.), \textit{Gender issues across the life cycle} (pp. 107--123). New York: Springer.
%
% Journal article by DOI
\bibitem{psysoc-DOI}Kreger, M., Brindis, C.D., Manuel, D.M., Sassoubre, L. (2007). Lessons learned in systems change initiatives: benchmarks and indicators. \textit{American Journal of Community Psychology}, doi: 10.1007/s10464-007-9108-14.
%
%
% Use the following syntax and markup for your references if 
% the subject of your book is from the field 
% "Humanities, Linguistics, Philosophy"
%
\bigskip
%
% Journal article
\bibitem{humlinphil-journal} Alber John, Daniel C. O'Connell, and Sabine Kowal. 2002. Personal perspective in TV interviews. \textit{Pragmatics} 12:257--271
%
% Contribution 
\bibitem{humlinphil-contrib} Cameron, Deborah. 1997. Theoretical debates in feminist linguistics: Questions of sex and gender. In \textit{Gender and discourse}, ed. Ruth Wodak, 99--119. London: Sage Publications.
%
% Monograph
\bibitem{humlinphil-mono} Cameron, Deborah. 1985. \textit{Feminism and linguistic theory.} New York: St. Martin's Press.
%
% Online Document
\bibitem{humlinphil-online} Dod, Jake. 1999. Effective substances. In: The dictionary of substances and their effects. Royal Society of Chemistry. Available via DIALOG. \\
http://www.rsc.org/dose/title of subordinate document. Cited 15 Jan 1999
%
% Journal article by DOI
\bibitem{humlinphil-DOI} Suleiman, Camelia, Daniel C. O�Connell, and Sabine Kowal. 2002. `If you and I, if we, in this later day, lose that sacred fire...�': Perspective in political interviews. \textit{Journal of Psycholinguistic Research}. doi: 10.1023/A:1015592129296.
%
%
%
\bigskip
%
%
% Use the following syntax and markup for your references if 
% the subject of your book is from the field 
% "Computer Science, Economics, Engineering, Geosciences, Life Sciences"
%
%
% Contribution 
\bibitem{basic-contrib} Brown B, Aaron M (2001) The politics of nature. In: Smith J (ed) The rise of modern genomics, 3rd edn. Wiley, New York 
%
% Online Document
\bibitem{basic-online} Dod J (1999) Effective Substances. In: The dictionary of substances and their effects. Royal Society of Chemistry. Available via DIALOG. \\
\url{http://www.rsc.org/dose/title of subordinate document. Cited 15 Jan 1999}
%
% Journal article by DOI
\bibitem{basic-DOI} Slifka MK, Whitton JL (2000) Clinical implications of dysregulated cytokine production. J Mol Med, doi: 10.1007/s001090000086
%
% Journal article
\bibitem{basic-journal} Smith J, Jones M Jr, Houghton L et al (1999) Future of health insurance. N Engl J Med 965:325--329
%
% Monograph
\bibitem{basic-mono} South J, Blass B (2001) The future of modern genomics. Blackwell, London 
%
\end{thebibliography}

