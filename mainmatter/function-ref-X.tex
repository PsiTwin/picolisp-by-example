%%%%%%%%%%%%%%%%%%%%% chapter.tex %%%%%%%%%%%%%%%%%%%%%%%%%%%%%%%%%
%
% sample chapter
%
% Use this file as a template for your own input.
%
%%%%%%%%%%%%%%%%%%%%%%%% Springer-Verlag %%%%%%%%%%%%%%%%%%%%%%%%%%
%\motto{Use the template \emph{chapter.tex} to style the various elements of your chapter content.}



\chapter{Symbols starting with X}
\label{cha:func-ref-X-functions-starting-with-X}
 
\section*{\texttt{(xchg 'var 'var ..) -> any}}
\label{sec:func-ref-X-(xchg 'var 'var ..) -> any}


Exchange the values of successive \texttt{var} argument pairs.


\begin{wideverbatim}
: (setq  A 1  B 2  C '(a b c))
-> (a b c)
: (xchg  'A C  'B (cdr C))
-> 2
: A
-> a
: B
-> b
: C
-> (1 2 c)
\end{wideverbatim}

 
\section*{\texttt{(xor 'any 'any) -> flg}}
\label{sec:func-ref-X-(xor 'any 'any) -> flg}


Returns T if exactly one of the arguments evaluates to non-\texttt{NIL}.


\begin{wideverbatim}
: (xor T NIL)
-> T
: (xor T T)
-> NIL
\end{wideverbatim}

 
\section*{\texttt{(x| 'num ..) -> num}}
\label{sec:func-ref-X-(x| 'num ..) -> num}


Returns the bitwise \texttt{XOR} of all \texttt{num} arguments. When one of the
arguments evaluates to \texttt{NIL}, it is returned immediately. See also \texttt{\&},
\texttt{|} and \texttt{bit?}.


\begin{wideverbatim}
: (x| 2 7)
-> 5
: (x| 2 7 1)
-> 4
\end{wideverbatim}


