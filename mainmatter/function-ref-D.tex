%%%%%%%%%%%%%%%%%%%%% chapter.tex %%%%%%%%%%%%%%%%%%%%%%%%%%%%%%%%%
%
% sample chapter
%
% Use this file as a template for your own input.
%
%%%%%%%%%%%%%%%%%%%%%%%% Springer-Verlag %%%%%%%%%%%%%%%%%%%%%%%%%%
%\motto{Use the template \emph{chapter.tex} to style the various elements of your chapter content.}



\chapter{Symbols starting with D}
\label{sec:func-ref-D-symbols-starting-with-D}

 
\section*{\texttt{*DB}}
\label{sec:func-ref-D-*DB}

A global constant holding the external symbol \texttt{\{1\}}, the \texttt{database}
root. All transient symbols in a database can be reached from that root.
Except during debugging, any explicit literal access to symbols in the
database should be avoided, because otherwise a memory leak might occur
(The garbage collector temporarily sets \texttt{*DB} to \texttt{NIL} and restores its
value after collection, thus disposing of all external symbols not
currently used in the program).


\begin{wideverbatim}
: (show *DB)
{1} NIL
   +City {P}
   +Person {3}
-> {1}
: (show '{P})
{P} NIL
   nm (566 . {AhDx})
-> {P}
: (show '{3})
{3} NIL
   tel (681376 . {Agyl})
   nm (1461322 . {2gu7})
-> {3}
\end{wideverbatim}

 
\section*{\texttt{*Dbg}}
\label{sec:func-ref-D-*Dbg}


A boolean variable indicating ``debug mode''. It can be conveniently
switched on with a trailing \texttt{+} command line argument (see
\emph{Invocation}). When non-\texttt{NIL}, the \texttt{\$} (tracing) and
\texttt{!} (breakpoint) functions are enabled, and the current line number and
file name will be stored in symbol properties by \texttt{de}, \texttt{def} and \texttt{dm}.
See also \texttt{debug}, \texttt{trace} and \texttt{lint}.


\begin{wideverbatim}
: (de foo (A B) (* A B))
-> foo
: (trace 'foo)
-> foo
: (foo 3 4)
 foo : 3 4
 foo = 12
-> 12
: (let *Dbg NIL (foo 3 4))
-> 12
\end{wideverbatim}

 
\section*{\texttt{*Dbs}}
\label{sec:func-ref-D-*Dbs}


A global variable holding a list of numbers (block size scale factors,
as needed by \texttt{pool}). It is typically set by \texttt{dbs} and \texttt{dbs+}.


\begin{wideverbatim}
: *Dbs
-> (1 2 1 0 2 3 3 3)
\end{wideverbatim}

 
\section*{\texttt{+Date}}
\label{sec:func-ref-D-+Date}


Class for calender dates (as calculated by \texttt{date}), a subclass of
\texttt{+Number}. See also \texttt{Database}.


\begin{wideverbatim}
(rel dat (+Ref +Date))  # Indexed date
\end{wideverbatim}

 
\section*{\texttt{+Dep}}
\label{sec:func-ref-D-+Dep}


Prefix class for maintaining depenencies between \texttt{+relation}s. Expects a
list of (symbolic) attributes that depend on this relation. Whenever
this relations is cleared (receives a value of \texttt{NIL}), the dependent
relations will also be cleared, triggering all required side-effects.
See also \texttt{Database}.

In the following example, the index entry for the item pointing to the
position (and, therefore, to the order) is cleared in case the order is
deleted, or this position is deleted from the order:


\begin{wideverbatim}
(class +Pos +Entity)                # Position class
(rel ord (+Dep +Joint)              # Order of that position
   (itm)                               # 'itm' specifies the dependency
   pos (+Ord) )                        # Arguments to '+Joint'
(rel itm (+Ref +Link) NIL (+Item))  # Item depends on the order
\end{wideverbatim}

 
\section*{\texttt{(d) -> T}}
\label{sec:func-ref-D-(d) -> T}


Inserts \texttt{!} breakpoints into all subexpressions of the current
breakpoint. Typically used when single-stepping a function or method
with \texttt{debug}. See also \texttt{u} and \texttt{unbug}.


\begin{wideverbatim}
! (d)         # Debug subexpression(s) at breakpoint
-> T
\end{wideverbatim}

 
\section*{\texttt{(daemon 'sym . prg) -> fun}}
\label{sec:func-ref-D-(daemon 'sym . prg) -> fun}


\texttt{(daemon '(sym . cls) . prg) -> fun}

\texttt{(daemon '(sym sym2 [. cls]) . prg) -> fun}

Inserts \texttt{prg} in the beginning of the function (first form),
the method body of \texttt{sym} in \texttt{cls} (second form) or in
the class obtained by \texttt{get}ing \texttt{sym2} from
\texttt{*Class} (or \texttt{cls} if given) (third form). Built-in
functions (C-function pointer) are automatically converted to Lisp
expressions. See also \texttt{expr}, \texttt{patch} and
\texttt{redef}.


\begin{wideverbatim}
: (de hello () (prinl "Hello world!"))
-> hello

: (daemon 'hello (prinl "# This is the hello world program"))
-> (NIL (prinl "# This is the hello world program") (prinl "Hello world!"))
: (hello)
# This is the hello world program
Hello world!
-> "Hello world!"

: (daemon '* (msg 'Multiplying))
-> (@ (msg 'Multiplying) (pass $134532148))
: *
-> (@ (msg 'Multiplying) (pass $134532148))
: (* 1 2 3)
Multiplying
-> 6
\end{wideverbatim}

 
\section*{\texttt{(dat\$ 'dat ['sym]) -> sym}}
\label{sec:func-ref-D-dat}

Formats a \emph{date} \texttt{dat} in ISO format, with an optional
delimiter character \texttt{sym}. See also \texttt{\$dat},
\texttt{tim\$}, \texttt{datStr} and \texttt{datSym}.


\begin{wideverbatim}
: (dat$ (date))
-> "20070601"
: (dat$ (date) "-")
-> "2007-06-01"
\end{wideverbatim}

 
\section*{\texttt{(datStr 'dat ['flg]) -> sym}}
\label{sec:func-ref-D-(datStr 'dat ['flg]) -> sym}


Formats a \texttt{date} according to the current \texttt{locale}. If \texttt{flg} is
non-\texttt{NIL}, the year will be formatted modulo 100. See also \texttt{dat\$},
\texttt{datSym}, \texttt{strDat}, \texttt{expDat}, \texttt{expTel} and \texttt{day}.


\begin{wideverbatim}
: (datStr (date))
-> "2007-06-01"
: (locale "DE" "de")
-> NIL
: (datStr (date))
-> "01.06.2007"
: (datStr (date) T)
-> "01.06.07"
\end{wideverbatim}

 
\section*{\texttt{(datSym 'dat) -> sym}}
\label{sec:func-ref-D-(datSym 'dat) -> sym}


Formats a \texttt{date} \texttt{dat} in in symbolic format (DDmmmYY). See also \texttt{dat\$}
and \texttt{datStr}.


\begin{wideverbatim}
: (datSym (date))
-> "01jun07"
\end{wideverbatim}

 
\section*{\texttt{(date ['T]) -> dat}}
\label{sec:func-ref-D-(date ['T]) -> dat}


\texttt{(date 'dat) -> (y m d)}

\texttt{(date 'y 'm 'd) -> dat | NIL}

\texttt{(date '(y m d)) -> dat | NIL}

Calculates a (gregorian) calendar date. It is represented as a day
number, starting first of March of the year 0 AD. When called without
arguments, the current date is returned. When called with a \texttt{T}
argument, the current Coordinated Universal Time (UTC) is returned. When
called with a single number \texttt{dat}, it is taken as a date and a list with
the corresponding year, month and day is returned. When called with
three numbers (or a list of three numbers) for the year, month and day,
the corresponding date is returned (or \texttt{NIL} if they do not represent a
legal date). See also \texttt{time}, \texttt{stamp}, \texttt{\$dat}, \texttt{dat\$}, \texttt{datSym},
\texttt{datStr}, \texttt{strDat}, \texttt{expDat}, \texttt{day}, \texttt{week} and \texttt{ultimo}.


\begin{wideverbatim}
: (date)                         # Today
-> 730589
: (date 2000 6 12)               # 12-06-2000
-> 730589
: (date 2000 22 5)               # Illegal date
-> NIL
: (date (date))                  # Today's year, month and day
-> (2000 6 12)
: (- (date) (date 2000 1 1))     # Number of days since first of January
-> 163
\end{wideverbatim}

 
\section*{\texttt{(day 'dat ['lst]) -> sym}}
\label{sec:func-ref-D-(day 'dat ['lst]) -> sym}


Returns the name of the day for a given \texttt{date} \texttt{dat}, in the language of
the current \texttt{locale}. If \texttt{lst} is given, it should be a list of
alternative weekday names. See also \texttt{week}, \texttt{datStr} and \texttt{strDat}.


\begin{wideverbatim}
: (day (date))
-> "Friday"
: (locale "DE" "de")
-> NIL
: (day (date))
-> "Freitag"
: (day (date) '("Mo" "Tu" "We" "Th" "Fr" "Sa" "Su"))
-> "Fr"
\end{wideverbatim}

 
\section*{\texttt{(db 'var 'cls ['hook] 'any ['var 'any ..]) -> sym | NIL}}
\label{sec:func-ref-D-(db 'var 'cls ['hook] 'any ['var 'any ..]) -> sym | NIL}


Returns a database object of class \texttt{cls}, where the values for the \texttt{var}
arguments correspond to the \texttt{any} arguments. If a matching object cannot
be found, \texttt{NIL} is returned. \texttt{var}, \texttt{cls} and \texttt{hook} should specify a
\texttt{tree} for \texttt{cls} or one of its superclasses. See also \texttt{aux}, \texttt{collect},
\texttt{request}, \texttt{fetch}, \texttt{init} and \texttt{step}.


\begin{wideverbatim}
: (db 'nr '+Item 1)
-> {3-1}
: (db 'nm '+Item "Main Part")
-> {3-1}
\end{wideverbatim}

 
\section*{\texttt{db/3}}
\label{sec:func-ref-D-db/3}


\texttt{db/4}

\texttt{db/5}

\emph{Pilog} database predicate that returns objects
matching the given key/value (and optional hook) relation. The relation
should be of type \texttt{+index}. For the key pattern applies:

\begin{itemize}
\item a symbol (string) returns all entries which start with that string
\item other atoms (numbers, external symbols) match as they are
\item cons pairs constitute a range, returning objects
\begin{itemize}
\item in increasing order if the CDR is greater than the CAR
\item in decreasing order otherwise
\end{itemize}
\item other lists are matched for \texttt{+Aux} key combinations
\end{itemize}

The optional hook can be supplied as the third argument. See also
\texttt{select/3} and \texttt{remote/2}.


\begin{wideverbatim}
: (? (db nr +Item @Item))              # No value given
 @Item={3-1}
 @Item={3-2}
 @Item={3-3}
 @Item={3-4}
 @Item={3-5}
 @Item={3-6}
-> NIL

: (? (db nr +Item 2 @Item))            # Get item no. 2
 @Item={3-2}
-> NIL

: (? (db nm +Item Spare @Item) (show @Item))  # Search for "Spare.."
{3-2} (+Item)
   pr 1250
   inv 100
   sup {2-2}
   nm "Spare Part"
   nr 2
 @Item={3-2}
-> NIL
\end{wideverbatim}

 
\section*{\texttt{(db: cls ..) -> num}}
\label{sec:func-ref-D-(db: cls ..) -> num}


Returns the database file number for objects of the type given by the
\texttt{cls} argument(s). Needed, for example, for the creation of \texttt{new}
objects. See also \texttt{dbs}.


\begin{wideverbatim}
: (db: +Item)
-> 3
\end{wideverbatim}

 
\section*{\texttt{(dbSync) -> flg}}
\label{sec:func-ref-D-(dbSync) -> flg}


Starts a database transaction, by trying to obtain a \texttt{lock} on the
database root object \texttt{*DB}, and then calling \texttt{sync} to synchronize with
possible changes from other processes. When all desired modifications to
external symbols are done, \texttt{(commit 'upd)} should be called. See also
\texttt{Database}.


\begin{wideverbatim}
(let? Obj (rd)             # Get object?
   (dbSync)                # Yes: Start transaction
   (put> Obj 'nm (rd))     # Update
   (put> Obj 'nr (rd))
   (put> Obj 'val (rd))
   (commit 'upd) )         # Close transaction
\end{wideverbatim}

 
\section*{\texttt{(dbck ['cnt] 'flg) -> any}}
\label{sec:func-ref-D-(dbck ['cnt] 'flg) -> any}


Performs a low-level integrity check of the current (or \texttt{cnt}'th)
database file, and returns \texttt{NIL} (or the number of blocks and symbols if
\texttt{flg} is non-\texttt{NIL}) if everything seems correct. Otherwise, a string
indicating an error is returned. As a side effect, possibly unused
blocks (as there might be when a \texttt{rollback} is done before \texttt{commit}ing newly allocated (\texttt{new}) external symbols) are appended to the free list.


\begin{wideverbatim}
: (pool "db")
-> T
: (dbck)
-> NIL
\end{wideverbatim}

 
\section*{\texttt{(dbs . lst)}}
\label{sec:func-ref-D-(dbs . lst)}


Initializes the global variable \texttt{*Dbs}. Each element in \texttt{lst} has a
number in its CAR (the block size scale factor of a database file, to be
stored in \texttt{*Dbs}). The CDR elements are either classes (so that objects
of that class are later stored in the corresponding file), or lists with
a class in the CARs and a list of relations in the CDRs (so that index
trees for these relations go into that file). See also \texttt{dbs+} and
\texttt{pool}.


\begin{wideverbatim}
(dbs
   (1 +Role +User +Sal)                         # (1 . 128)
   (2 +CuSu)                                    # (2 . 256)
   (1 +Item +Ord)                               # (3 . 128)
   (0 +Pos)                                     # (4 . 64)
   (2 (+Role nm) (+User nm) (+Sal nm))          # (5 . 256)
   (4 (+CuSu nr plz tel mob))                   # (6 . 1024)
   (4 (+CuSu nm))                               # (7 . 1024)
   (4 (+CuSu ort))                              # (8 . 1024)
   (4 (+Item nr sup pr))                        # (9 . 1024)
   (4 (+Item nm))                               # (10 . 1024)
   (4 (+Ord nr dat cus))                        # (11 . 1024)
   (4 (+Pos itm)) )                             # (12 . 1024)

: *Dbs
-> (1 2 1 0 2 4 4 4 4 4 4 4)
: (get '+Item 'Dbf)
-> (3 . 128)
: (get '+Item 'nr 'dbf)
-> (9 . 1024)
\end{wideverbatim}

 
\section*{\texttt{(dbs+ 'num . lst)}}
\label{sec:func-ref-D-(dbs+ 'num . lst)}


Extends the list of database sizes stored in \texttt{*Dbs}. \texttt{num} is the
initial offset into the list. See also \texttt{dbs}.


\begin{wideverbatim}
(dbs+ 9
   (1 +NewCls)                                  # (9 . 128)
   (3 (+NewCls nr nm)) )                        # (10 . 512)
\end{wideverbatim}

 
\section*{\texttt{(de sym . any) -> sym}}
\label{sec:func-ref-D-(de sym . any) -> sym}


Assigns a definition to the \texttt{sym} argument, by setting its \texttt{VAL} to the
\texttt{any} argument. If the symbol has already another value, a ``redefined''
message is issued. When the value of the global variable
\emph{*Dbg} is non-\texttt{NIL}, the current line number and file
name (if any) are stored in the \texttt{*Dbg} property of \texttt{sym}. \texttt{de} is the
standard way to define a function. See also \texttt{def}, \texttt{dm} and \texttt{undef}.


\begin{wideverbatim}
: (de foo (X Y) (* X (+ X Y)))  # Define a function
-> foo
: (foo 3 4)
-> 21

: (de *Var . 123)  # Define a variable value
: *Var
-> 123
\end{wideverbatim}

 
\section*{\texttt{(debug 'sym) -> T}}
\label{sec:func-ref-D-(debug 'sym) -> T}


\texttt{(debug 'sym 'cls) -> T}

\texttt{(debug '(sym . cls)) -> T}

Inserts a \texttt{!} breakpoint function call at the beginning and all
top-level expressions of the function or method body of \texttt{sym}, to allow
a stepwise execution. Typing \texttt{(d)} at a breakpoint will also debug the
current subexpression, and \texttt{(e)} will evaluate the current
subexpression. The current subexpression is stored in the global
variable \texttt{\textasciicircum{}}. See also \texttt{unbug}, \texttt{*Dbg}, \texttt{trace} and \texttt{lint}.


\begin{wideverbatim}
: (de tst (N)                    # Define tst
   (println (+ 3 N)) )
-> tst
: (debug 'tst)                   # Set breakpoints
-> T
: (pp 'tst)
(de tst (N)
   (! println (+ 3 N)) )         # Breakpoint '!'
-> tst
: (tst 7)                        # Execute
(println (+ 3 N))                # Stopped at beginning of 'tst'
! (d)                            # Debug subexpression
-> T
!                                # Continue
(+ 3 N)                          # Stopped in subexpression
! N                              # Inspect variable 'N'
-> 7
!                                # Continue
10                               # Output of print statement
-> 10                            # Done
: (unbug 'tst)
-> T
: (pp 'tst)                      # Restore to original
(de tst (N)
   (println (+ 3 N)) )
-> tst
\end{wideverbatim}

 
\section*{\texttt{(dec 'num) -> num}}
\label{sec:func-ref-D-(dec 'num) -> num}


\texttt{(dec 'var ['num]) -> num}

The first form returns the value of \texttt{num} decremented by 1. The second
form decrements the \texttt{VAL} of \texttt{var} by 1, or by \texttt{num}. If the first
argument is \texttt{NIL}, it is returned immediately. \texttt{(dec 'num)} is
equivalent to \texttt{(- 'num 1)} and \texttt{(dec 'var)} is equivalent to
\texttt{(set 'var (- var 1))}. See also \texttt{inc} and \texttt{-}.


\begin{wideverbatim}
: (dec -1)
-> -2
: (dec 7)
-> 6
: (setq N 7)
-> 7
: (dec 'N)
-> 6
: (dec 'N 3)
-> 3
\end{wideverbatim}

 
\section*{\texttt{(def 'sym 'any) -> sym}}
\label{sec:func-ref-D-(def 'sym 'any) -> sym}


\texttt{(def 'sym1 'sym2 'any) -> sym1}

The first form assigns a definition to the first \texttt{sym} argument, by
setting its \texttt{VAL}'s to \texttt{any}. The second form defines a property value
\texttt{any} for the first argument's \texttt{sym2} key. If any of these values
existed and was changed in the process, a ``redefined'' message is issued.
When the value of the global variable \emph{*Dbg} is
non-\texttt{NIL}, the current line number and file name (if any) are stored in
the \texttt{*Dbg} property of \texttt{sym}. See also \texttt{de} and \texttt{dm}.


\begin{wideverbatim}
: (def 'b '((X Y) (* X (+ X Y))))
-> b
: (def 'b 999)
# b redefined
-> b
\end{wideverbatim}

 
\section*{\texttt{(default var 'any ..) -> any}}
\label{sec:func-ref-D-(default var 'any ..) -> any}


Stores new values \texttt{any} in the \texttt{var} arguments only if their current
values are \texttt{NIL}. Otherwise, their values are left unchanged. In any
case, the last \texttt{var}'s value is returned. \texttt{default} is used typically in
functions to initialize optional arguments.


\begin{wideverbatim}
: (de foo (A B)               # Function with two optional arguments
   (default  A 1  B 2)        # The default values are 1 and 2
   (list A B) )
-> foo
: (foo 333 444)               # Called with two arguments
-> (333 444)
: (foo 333)                   # Called with one arguments
-> (333 2)
: (foo)                       # Called without arguments
-> (1 2)
\end{wideverbatim}

 
\section*{\texttt{(del 'any 'var) -> lst}}
\label{sec:func-ref-D-(del 'any 'var) -> lst}


Deletes \texttt{any} from the list in the value of \texttt{var}, and returns the
remaining list. \texttt{(del 'any 'var)} is equivalent to
\texttt{(set 'var (delete 'any var))}. See also \texttt{delete}, \texttt{cut} and \texttt{pop}.


\begin{wideverbatim}
: (setq S '((a b c) (d e f)))
-> ((a b c) (d e f))
: (del '(d e f) 'S)
-> ((a b c))
: (del 'b S)
-> (a c)
\end{wideverbatim}

 
\section*{\texttt{(delete 'any 'lst) -> lst}}
\label{sec:func-ref-D-(delete 'any 'lst) -> lst}


Deletes \texttt{any} from \texttt{lst}. If \texttt{any} is contained more than once in \texttt{lst},
only the first occurrence is deleted. See also \texttt{delq}, \texttt{remove} and
\texttt{insert}.


\begin{wideverbatim}
: (delete 2 (1 2 3))
-> (1 3)
: (delete (3 4) '((1 2) (3 4) (5 6) (3 4)))
-> ((1 2) (5 6) (3 4))
\end{wideverbatim}

 
\section*{\texttt{delete/3}}
\label{sec:func-ref-D-delete/3}


\emph{Pilog} predicate that succeeds if deleting the first
argument from the list in the second argument is equal to the third
argument. See also \texttt{delete} and \texttt{member/2}.


\begin{wideverbatim}
: (? (delete b (a b c) @X))
 @X=(a c)
-> NIL
\end{wideverbatim}

 
\section*{\texttt{(delq 'any 'lst) -> lst}}
\label{sec:func-ref-D-(delq 'any 'lst) -> lst}


Deletes \texttt{any} from \texttt{lst}. If \texttt{any} is contained more than once in \texttt{lst},
only the first occurrence is deleted. \texttt{==} is used for comparison
(pointer equality). See also \texttt{delete}, \texttt{asoq}, \texttt{memq}, \texttt{mmeq} and
\emph{Comparing}.


\begin{wideverbatim}
: (delq 'b '(a b c))
-> (a c)
: (delq (2) (1 (2) 3))
-> (1 (2) 3)
\end{wideverbatim}

 
\section*{\texttt{(dep 'cls) -> cls}}
\label{sec:func-ref-D-(dep 'cls) -> cls}


Displays the ``dependencies'' of \texttt{cls}, i.e. the tree of superclasses and
the tree of subclasses. See also \texttt{OO Concepts}, \texttt{class} and \texttt{can}.


\begin{wideverbatim}
: (dep '+Number)           # Dependencies of '+Number'
   +relation               # Single superclass is '+relation'
+Number
   +Date                   # Subclasses are '+Date' and '+Time'
   +Time
-> +Number
\end{wideverbatim}

 
\section*{\texttt{(depth 'lst) -> (cnt1 . cnt2)}}
\label{sec:func-ref-D-(depth 'lst) -> (cnt1 . cnt2)}


Returns the maximal (\texttt{cnt1}) and the average (\texttt{cnt2}) ``depth'' of a tree
structure as maintained by \texttt{idx}. See also \texttt{length} and \texttt{size}.


\begin{wideverbatim}
: (off X)                                    # Clear variable
-> NIL
: (for N (1 2 3 4 5 6 7) (idx 'X N T))       # Build a degenerated tree
-> NIL
: X
-> (1 NIL 2 NIL 3 NIL 4 NIL 5 NIL 6 NIL 7)   # Only right branches
: (depth X)
-> (7 . 4)                                   # Depth is 7, average 4
\end{wideverbatim}

 
\section*{\texttt{(diff 'lst 'lst) -> lst}}
\label{sec:func-ref-D-(diff 'lst 'lst) -> lst}


Returns the difference of the \texttt{lst} arguments. See also \texttt{sect}.


\begin{wideverbatim}
: (diff (1 2 3 4 5) (2 4))
-> (1 3 5)
: (diff (1 2 3) (1 2 3))
-> NIL
\end{wideverbatim}

 
\section*{\texttt{different/2}}
\label{sec:func-ref-D-different/2}


\emph{Pilog} predicate that succeeds if the two arguments
are different. See also \texttt{equal/2}.


\begin{wideverbatim}
: (? (different 3 4))
-> T
\end{wideverbatim}

 
\section*{\texttt{(dir ['any] ['flg]) -> lst}}
\label{sec:func-ref-D-(dir ['any] ['flg]) -> lst}


Returns a list of all filenames in the directory \texttt{any}. Names
starting with a dot `\texttt{.}' are ignored, unless \texttt{flg} is
non-\texttt{NIL}. See also \texttt{cd} and \texttt{info}.


\begin{wideverbatim}
: (filter '((F) (tail '(. c) (chop F))) (dir "src/"))
-> ("main.c" "subr.c" "gc.c" "io.c" "big.c" "sym.c" "tab.c" "flow.c" ..
\end{wideverbatim}

 
\section*{\texttt{(dirname 'any) -> sym}}
\label{sec:func-ref-D-(dirname 'any) -> sym}


Returns the directory part of a path name \texttt{any}. See also \texttt{basename} and
\texttt{path}.


\begin{wideverbatim}
: (dirname "a/b/c/d")
-> "a/b/c/"
\end{wideverbatim}

 
\section*{\texttt{(dm sym . fun|cls2) -> sym}}
\label{sec:func-ref-D-(dm sym . fun|cls2) -> sym}


\texttt{(dm (sym . cls) . fun|cls2) -> sym}

\texttt{(dm (sym sym2 [. cls]) . fun|cls2) -> sym}

Defines a method for the message \texttt{sym} in the current class,
implicitly given by the value of the global variable \texttt{*Class},
or - in the second form - for the explicitly given class \texttt{cls}.
In the third form, the class object is obtained by \texttt{get}ing
\texttt{sym2} from \texttt{*Class} (or \texttt{cls} if given). If the
method for that class existed and was changed in the process, a
``redefined'' message is issued. If - instead of a method \texttt{fun}
\begin{itemize}
\item a symbol specifying another class \texttt{cls2} is given, the method from
\end{itemize}
that class is used (explicit inheritance). When the value of the global
variable \emph{*Dbg} is non-\texttt{NIL}, the current line number
and file name (if any) are stored in the \texttt{*Dbg} property of \texttt{sym}. See
also \texttt{OO Concepts}, \texttt{de}, \texttt{undef}, \emph{class},
\emph{rel}, \emph{var},
\emph{method}, \emph{send} and
\emph{try}.


\begin{wideverbatim}
: (dm start> ()
   (super)
   (mapc 'start> (: fields))
   (mapc 'start> (: arrays)) )

: (dm foo> . +OtherClass)  # Explicitly inherit 'foo>' from '+OtherClass'
\end{wideverbatim}

 
\section*{\texttt{(do 'flg|num ['any | (NIL 'any . prg) | (T 'any . prg) ..]) -> any}}
\label{sec:func-ref-D-(do 'flg|num ['any | (NIL 'any . prg) | (T 'any . prg) ..]) -> any}


Counted loop with multiple conditional exits: The body is executed at
most \texttt{num} times (or never (if the first argument is \texttt{NIL}), or an
infinite number of times (if the first argument is \texttt{T})). If a clause
has \texttt{NIL} or \texttt{T} as its CAR, the clause's second element is evaluated as
a condition and - if the result is \texttt{NIL} or non-\texttt{NIL}, respectively -
the \texttt{prg} is executed and the result returned. Otherwise (if count drops
to zero), the result of the last expression is returned. See also \texttt{loop}
and \texttt{for}.


\begin{wideverbatim}
: (do 4 (printsp 'OK))
OK OK OK OK -> OK
: (do 4 (printsp 'OK) (T (= 3 3) (printsp 'done)))
OK done -> done
\end{wideverbatim}

 
\section*{\texttt{(doc ['sym1] ['sym2])}}
\label{sec:func-ref-D-(doc ['sym1] ['sym2])}


Opens a browser, and tries to display the reference documentation for
\texttt{sym1}. \texttt{sym2} may be the name of a browser. If not given, the value of
the environment variable \texttt{BROWSER}, or the \texttt{w3m} browser is tried. If
\texttt{sym1} is \texttt{NIL}, the \emph{PicoLisp Reference} manual is opened.
See also \emph{Function Reference} and \texttt{vi}.


\begin{wideverbatim}
: (doc '+)  # Function reference
-> T
: (doc '+relation)  # Class reference
-> T
: (doc)  # Reference manual
-> T
:  (doc 'vi "firefox")  # Use alternative browser
-> T
\end{wideverbatim}




% %%%%%%%%%%%%%%%%%%%%%%%% referenc.tex %%%%%%%%%%%%%%%%%%%%%%%%%%%%%%
% sample references
% %
% Use this file as a template for your own input.
%
%%%%%%%%%%%%%%%%%%%%%%%% Springer-Verlag %%%%%%%%%%%%%%%%%%%%%%%%%%
%
% BibTeX users please use
% \bibliographystyle{}
% \bibliography{}
%
\biblstarthook{In view of the parallel print and (chapter-wise) online publication of your book at \url{www.springerlink.com} it has been decided that -- as a genreral rule --  references should be sorted chapter-wise and placed at the end of the individual chapters. However, upon agreement with your contact at Springer you may list your references in a single seperate chapter at the end of your book. Deactivate the class option \texttt{sectrefs} and the \texttt{thebibliography} environment will be put out as a chapter of its own.\\\indent
References may be \textit{cited} in the text either by number (preferred) or by author/year.\footnote{Make sure that all references from the list are cited in the text. Those not cited should be moved to a separate \textit{Further Reading} section or chapter.} The reference list should ideally be \textit{sorted} in alphabetical order -- even if reference numbers are used for the their citation in the text. If there are several works by the same author, the following order should be used: 
\begin{enumerate}
\item all works by the author alone, ordered chronologically by year of publication
\item all works by the author with a coauthor, ordered alphabetically by coauthor
\item all works by the author with several coauthors, ordered chronologically by year of publication.
\end{enumerate}
The \textit{styling} of references\footnote{Always use the standard abbreviation of a journal's name according to the ISSN \textit{List of Title Word Abbreviations}, see \url{http://www.issn.org/en/node/344}} depends on the subject of your book:
\begin{itemize}
\item The \textit{two} recommended styles for references in books on \textit{mathematical, physical, statistical and computer sciences} are depicted in ~\cite{science-contrib, science-online, science-mono, science-journal, science-DOI} and ~\cite{phys-online, phys-mono, phys-journal, phys-DOI, phys-contrib}.
\item Examples of the most commonly used reference style in books on \textit{Psychology, Social Sciences} are~\cite{psysoc-mono, psysoc-online,psysoc-journal, psysoc-contrib, psysoc-DOI}.
\item Examples for references in books on \textit{Humanities, Linguistics, Philosophy} are~\cite{humlinphil-journal, humlinphil-contrib, humlinphil-mono, humlinphil-online, humlinphil-DOI}.
\item Examples of the basic Springer style used in publications on a wide range of subjects such as \textit{Computer Science, Economics, Engineering, Geosciences, Life Sciences, Medicine, Biomedicine} are ~\cite{basic-contrib, basic-online, basic-journal, basic-DOI, basic-mono}. 
\end{itemize}
}

\begin{thebibliography}{99.}%
% and use \bibitem to create references.
%
% Use the following syntax and markup for your references if 
% the subject of your book is from the field 
% "Mathematics, Physics, Statistics, Computer Science"
%
% Contribution 
\bibitem{science-contrib} Broy, M.: Software engineering --- from auxiliary to key technologies. In: Broy, M., Dener, E. (eds.) Software Pioneers, pp. 10-13. Springer, Heidelberg (2002)
%
% Online Document
\bibitem{science-online} Dod, J.: Effective substances. In: The Dictionary of Substances and Their Effects. Royal Society of Chemistry (1999) Available via DIALOG. \\
\url{http://www.rsc.org/dose/title of subordinate document. Cited 15 Jan 1999}
%
% Monograph
\bibitem{science-mono} Geddes, K.O., Czapor, S.R., Labahn, G.: Algorithms for Computer Algebra. Kluwer, Boston (1992) 
%
% Journal article
\bibitem{science-journal} Hamburger, C.: Quasimonotonicity, regularity and duality for nonlinear systems of partial differential equations. Ann. Mat. Pura. Appl. \textbf{169}, 321--354 (1995)
%
% Journal article by DOI
\bibitem{science-DOI} Slifka, M.K., Whitton, J.L.: Clinical implications of dysregulated cytokine production. J. Mol. Med. (2000) doi: 10.1007/s001090000086 
%
\bigskip

% Use the following (APS) syntax and markup for your references if 
% the subject of your book is from the field 
% "Mathematics, Physics, Statistics, Computer Science"
%
% Online Document
\bibitem{phys-online} J. Dod, in \textit{The Dictionary of Substances and Their Effects}, Royal Society of Chemistry. (Available via DIALOG, 1999), 
\url{http://www.rsc.org/dose/title of subordinate document. Cited 15 Jan 1999}
%
% Monograph
\bibitem{phys-mono} H. Ibach, H. L\"uth, \textit{Solid-State Physics}, 2nd edn. (Springer, New York, 1996), pp. 45-56 
%
% Journal article
\bibitem{phys-journal} S. Preuss, A. Demchuk Jr., M. Stuke, Appl. Phys. A \textbf{61}
%
% Journal article by DOI
\bibitem{phys-DOI} M.K. Slifka, J.L. Whitton, J. Mol. Med., doi: 10.1007/s001090000086
%
% Contribution 
\bibitem{phys-contrib} S.E. Smith, in \textit{Neuromuscular Junction}, ed. by E. Zaimis. Handbook of Experimental Pharmacology, vol 42 (Springer, Heidelberg, 1976), p. 593
%
\bigskip
%
% Use the following syntax and markup for your references if 
% the subject of your book is from the field 
% "Psychology, Social Sciences"
%
%
% Monograph
\bibitem{psysoc-mono} Calfee, R.~C., \& Valencia, R.~R. (1991). \textit{APA guide to preparing manuscripts for journal publication.} Washington, DC: American Psychological Association.
%
% Online Document
\bibitem{psysoc-online} Dod, J. (1999). Effective substances. In: The dictionary of substances and their effects. Royal Society of Chemistry. Available via DIALOG. \\
\url{http://www.rsc.org/dose/Effective substances.} Cited 15 Jan 1999.
%
% Journal article
\bibitem{psysoc-journal} Harris, M., Karper, E., Stacks, G., Hoffman, D., DeNiro, R., Cruz, P., et al. (2001). Writing labs and the Hollywood connection. \textit{J Film} Writing, 44(3), 213--245.
%
% Contribution 
\bibitem{psysoc-contrib} O'Neil, J.~M., \& Egan, J. (1992). Men's and women's gender role journeys: Metaphor for healing, transition, and transformation. In B.~R. Wainrig (Ed.), \textit{Gender issues across the life cycle} (pp. 107--123). New York: Springer.
%
% Journal article by DOI
\bibitem{psysoc-DOI}Kreger, M., Brindis, C.D., Manuel, D.M., Sassoubre, L. (2007). Lessons learned in systems change initiatives: benchmarks and indicators. \textit{American Journal of Community Psychology}, doi: 10.1007/s10464-007-9108-14.
%
%
% Use the following syntax and markup for your references if 
% the subject of your book is from the field 
% "Humanities, Linguistics, Philosophy"
%
\bigskip
%
% Journal article
\bibitem{humlinphil-journal} Alber John, Daniel C. O'Connell, and Sabine Kowal. 2002. Personal perspective in TV interviews. \textit{Pragmatics} 12:257--271
%
% Contribution 
\bibitem{humlinphil-contrib} Cameron, Deborah. 1997. Theoretical debates in feminist linguistics: Questions of sex and gender. In \textit{Gender and discourse}, ed. Ruth Wodak, 99--119. London: Sage Publications.
%
% Monograph
\bibitem{humlinphil-mono} Cameron, Deborah. 1985. \textit{Feminism and linguistic theory.} New York: St. Martin's Press.
%
% Online Document
\bibitem{humlinphil-online} Dod, Jake. 1999. Effective substances. In: The dictionary of substances and their effects. Royal Society of Chemistry. Available via DIALOG. \\
http://www.rsc.org/dose/title of subordinate document. Cited 15 Jan 1999
%
% Journal article by DOI
\bibitem{humlinphil-DOI} Suleiman, Camelia, Daniel C. O�Connell, and Sabine Kowal. 2002. `If you and I, if we, in this later day, lose that sacred fire...�': Perspective in political interviews. \textit{Journal of Psycholinguistic Research}. doi: 10.1023/A:1015592129296.
%
%
%
\bigskip
%
%
% Use the following syntax and markup for your references if 
% the subject of your book is from the field 
% "Computer Science, Economics, Engineering, Geosciences, Life Sciences"
%
%
% Contribution 
\bibitem{basic-contrib} Brown B, Aaron M (2001) The politics of nature. In: Smith J (ed) The rise of modern genomics, 3rd edn. Wiley, New York 
%
% Online Document
\bibitem{basic-online} Dod J (1999) Effective Substances. In: The dictionary of substances and their effects. Royal Society of Chemistry. Available via DIALOG. \\
\url{http://www.rsc.org/dose/title of subordinate document. Cited 15 Jan 1999}
%
% Journal article by DOI
\bibitem{basic-DOI} Slifka MK, Whitton JL (2000) Clinical implications of dysregulated cytokine production. J Mol Med, doi: 10.1007/s001090000086
%
% Journal article
\bibitem{basic-journal} Smith J, Jones M Jr, Houghton L et al (1999) Future of health insurance. N Engl J Med 965:325--329
%
% Monograph
\bibitem{basic-mono} South J, Blass B (2001) The future of modern genomics. Blackwell, London 
%
\end{thebibliography}

   