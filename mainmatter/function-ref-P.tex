%%%%%%%%%%%%%%%%%%%%% chapter.tex %%%%%%%%%%%%%%%%%%%%%%%%%%%%%%%%%
%
% sample chapter
%
% Use this file as a template for your own input.
%
%%%%%%%%%%%%%%%%%%%%%%%% Springer-Verlag %%%%%%%%%%%%%%%%%%%%%%%%%%
%\motto{Use the template \emph{chapter.tex} to style the various elements of your chapter content.}



\chapter{Symbols starting with P}
\label{cha:func-ref-P-functions-starting-with-P}
 
\section*{\texttt{*PPid}}
\label{sec:func-ref-P-*PPid}


A global constant holding the process-id of the parent picolisp process,
or \texttt{NIL} if the current process is a top level process.


\begin{wideverbatim}
: (println *PPid *Pid)
NIL 5286

: (unless (fork) (println *PPid *Pid) (bye))
5286 5522
\end{wideverbatim}

 
\section*{\texttt{*Pid}}
\label{sec:func-ref-P-*Pid}


A global constant holding the current process-id.


\begin{wideverbatim}
: *Pid
-> 6386
: (call "ps")  # Show processes
  PID TTY          TIME CMD
 .... ...      ........ .....
 6386 pts/1    00:00:00 pil   # <- current process
 6388 pts/1    00:00:00 ps
-> T
\end{wideverbatim}

 
\section*{\texttt{*Prompt}}
\label{sec:func-ref-P-*Prompt}


Global variable holding a (possibly empty) \texttt{prg} body, which is executed
\begin{itemize}
\item and the result \texttt{prin}ted - every time before a prompt is output to the
\end{itemize}
console in the ``read-eval-print-loop'' (REPL).


\begin{wideverbatim}
: (de *Prompt (pack "[" (stamp) "]"))
# *Prompt redefined
-> *Prompt
[2011-10-11 16:50:05]: (+ 1 2 3)
-> 6
[2011-10-11 16:50:11]:
\end{wideverbatim}

 
\section*{\texttt{(pack 'any ..) -> sym}}
\label{sec:func-ref-P-(pack 'any ..) -> sym}


Returns a transient symbol whose name is concatenated from all arguments
\texttt{any}. A \texttt{NIL} arguments contributes nothing to the result string, a
number is converted to a digit string, a symbol supplies the characters
of its name, and for a list its elements are taken. See also \texttt{text} and
\texttt{glue}.


\begin{wideverbatim}
: (pack 'car " is " 1 '(" symbol " name))
-> "car is 1 symbol name"
\end{wideverbatim}

 
\section*{\texttt{(pad 'cnt 'any) -> sym}}
\label{sec:func-ref-P-(pad 'cnt 'any) -> sym}


Returns a transient symbol with \texttt{any} \texttt{pack}ed with
leading '0' characters, up to a field width of \texttt{cnt}. See also
\texttt{format} and \texttt{align}.


\begin{wideverbatim}
: (pad 5 1)
-> "00001"
: (pad 5 123456789)
-> "123456789"
\end{wideverbatim}

 
\section*{\texttt{(pair 'any) -> any}}
\label{sec:func-ref-P-(pair 'any) -> any}


Returns \texttt{any} when the argument a cons pair cell. See also \texttt{atom}.


\begin{wideverbatim}
: (pair NIL)
-> NIL
: (pair (1 . 2))
-> (1 . 2)
: (pair (1 2 3))
-> (1 2 3)
\end{wideverbatim}

 
\section*{\texttt{part/3}}
\label{sec:func-ref-P-part/3}


\emph{Pilog} predicate that succeeds if the first argument, after
\texttt{fold}ing it to a canonical form, is a /substring/ of the
folded string representation of the result of applying the
\texttt{get} algorithm to the following arguments. Typically used as
filter predicate in \texttt{select/3} database queries. See also
\texttt{sub?}, \texttt{isa/2}, \texttt{same/3}, \texttt{bool/3},
\texttt{range/3}, \texttt{head/3}, \texttt{fold/3} and
\texttt{tolr/3}.


\begin{wideverbatim}
: (?
   @Nr (1 . 5)
   @Nm "part"
   (select (@Item)
      ((nr +Item @Nr) (nm +Item @Nm))
      (range @Nr @Item nr)
      (part @Nm @Item nm) ) )
      @Nr=(1 . 5) @Nm="part" @Item={3-1}
      @Nr=(1 . 5) @Nm="part" @Item={3-2}
-> NIL
\end{wideverbatim}

 
\section*{\texttt{(pass 'fun ['any ..]) -> any}}
\label{sec:func-ref-P-(pass 'fun ['any ..]) -> any}


Passes to \texttt{fun} all arguments \texttt{any}, and all remaining variable
arguments (\texttt{@}) as they would be returned by \texttt{rest}. \texttt{(pass 'fun 'any)}
is equivalent to \texttt{(apply 'fun (rest) 'any)}. See also \texttt{apply}.


\begin{wideverbatim}
: (de bar (A B . @)
   (println 'bar A B (rest)) )
-> bar
: (de foo (A B . @)
   (println 'foo A B)
   (pass bar 1)
   (pass bar 2) )
-> foo
: (foo 'a 'b 'c 'd 'e 'f)
foo a b
bar 1 c (d e f)
bar 2 c (d e f)
-> (d e f)
\end{wideverbatim}

 
\section*{\texttt{(pat? 'any) -> pat | NIL}}
\label{sec:func-ref-P-(pat? 'any) -> pat | NIL}


Returns \texttt{any} when the argument \texttt{any} is a symbol whose name starts with
an at-mark ``\texttt{@}'', otherwise \texttt{NIL}.


\begin{wideverbatim}
: (pat? '@)
-> @
: (pat? "@Abc")
-> "@Abc"
: (pat? "ABC")
-> NIL
: (pat? 123)
-> NIL
\end{wideverbatim}

 
\section*{\texttt{(patch 'lst 'any . prg) -> any}}
\label{sec:func-ref-P-(patch 'lst 'any . prg) -> any}


Destructively replaces all sub-expressions of \texttt{lst}, that \texttt{match} the
pattern \texttt{any}, by the result of the execution of \texttt{prg}. See also
\texttt{daemon} and \texttt{redef}.


\begin{wideverbatim}
: (pp 'hello)
(de hello NIL
   (prinl "Hello world!") )
-> hello

: (patch hello 'prinl 'println)
-> NIL
: (pp 'hello)
(de hello NIL
   (println "Hello world!") )
-> hello

: (patch hello '(prinl @S) (fill '(println "We said: " . @S)))
-> NIL
: (hello)
We said: Hello world!
-> "Hello world!"
\end{wideverbatim}

 
\section*{\texttt{(path 'any) -> sym}}
\label{sec:func-ref-P-(path 'any) -> sym}


Substitutes any leading ``\texttt{@}'' character in the \texttt{any}
argument with the PicoLisp Home Directory, as it was remembered during
interpreter startup. Optionally, the name may be preceded by a
``\texttt{+}'' character (as used by \texttt{in} and \texttt{out}).
This mechanism is used internally by all I/O functions. See also
\emph{Invocation}, \texttt{basename} and \texttt{dirname}.


\begin{wideverbatim}
$ /usr/bin/picolisp /usr/lib/picolisp/lib.l
: (path "a/b/c")
-> "a/b/c"
: (path "@a/b/c")
-> "/usr/lib/picolisp/a/b/c"
: (path "+@a/b/c")
-> "+/usr/lib/picolisp/a/b/c"
\end{wideverbatim}

 
\section*{\texttt{(peek) -> sym}}
\label{sec:func-ref-P-(peek) -> sym}


Single character look-ahead: Returns the same character as the next call
to \texttt{char} would return. See also \texttt{skip}.


\begin{wideverbatim}
$ cat a
# Comment
abcd
$ pil +
: (in "a" (list (peek) (char)))
-> ("#" "#")
\end{wideverbatim}

 
\section*{\texttt{permute/2}}
\label{sec:func-ref-P-permute/2}


\emph{Pilog} predicate that succeeds if the second argument
is a permutation of the list in the second argument. See also
\texttt{append/3}.


\begin{wideverbatim}
: (? (permute (a b c) @X))
 @X=(a b c)
 @X=(a c b)
 @X=(b a c)
 @X=(b c a)
 @X=(c a b)
 @X=(c b a)
-> NIL
\end{wideverbatim}

 
\section*{\texttt{(pick 'fun 'lst ..) -> any}}
\label{sec:func-ref-P-(pick 'fun 'lst ..) -> any}


Applies \texttt{fun} to successive elements of \texttt{lst} until non-\texttt{NIL} is
returned. Returns that value, or \texttt{NIL} if \texttt{fun} did not return non-\texttt{NIL}
for any element of \texttt{lst}. When additional \texttt{lst} arguments are given,
their elements are also passed to \texttt{fun}. \texttt{(pick 'fun 'lst)} is
equivalent to \texttt{(fun (find 'fun 'lst))}. See also \texttt{seek}, \texttt{find} and
\texttt{extract}.


\begin{wideverbatim}
: (setq A NIL  B 1  C NIL  D 2  E NIL  F 3)
-> 3
: (find val '(A B C D E))
-> B
: (pick val '(A B C D E))
-> 1
\end{wideverbatim}

 
\section*{\texttt{pico}}
\label{sec:func-ref-P-pico}


(64-bit version only) A global constant holding the initial (default)
namespace of internal symbols. Its value is a cons pair of two `\texttt{idx}'
trees, one for symbols with short names and one for symbols with long
names (more than 7 bytes in the name). See also \texttt{symbols}, \texttt{import} and
\texttt{intern}.


\begin{wideverbatim}
: (symbols)
-> pico
: (cdr pico)
-> (rollback (*NoTrace (ledSearch (expandTab (********)) *CtryCode ...
\end{wideverbatim}

 
\section*{\texttt{(pil ['any ..]) -> sym}}
\label{sec:func-ref-P-(pil ['any ..]) -> sym}


Returns the path name to the \texttt{pack}ed \texttt{any} arguments in the directory
``.pil/'' in the user's home directory. See also \texttt{tmp}.


\begin{wideverbatim}
: (pil "history")  # Path to the line editor's history file
-> "/home/app/.pil/history"
\end{wideverbatim}

 
\section*{\texttt{(pilog 'lst . prg) -> any}}
\label{sec:func-ref-P-(pilog 'lst . prg) -> any}


Evaluates a \emph{Pilog} query, and executes \texttt{prg} for each
result set with all Pilog variables bound to their matching values. See
also \texttt{solve}, \texttt{?}, \texttt{goal} and \texttt{prove}.


\begin{wideverbatim}
: (pilog '((append @X @Y (a b c))) (println @X '- @Y))
NIL - (a b c)
(a) - (b c)
(a b) - (c)
(a b c) - NIL
-> NIL
\end{wideverbatim}

 
\section*{\texttt{(pipe exe) -> cnt}}
\label{sec:func-ref-P-(pipe exe) -> cnt}


\texttt{(pipe exe . prg) -> any}

Executes \texttt{exe} in a \texttt{fork}'ed child process (which terminates
thereafter). In the first form, \texttt{pipe} just returns a file descriptor to
read from the standard output of that process. In the second form, it
opens the standard output of that process as input channel during the
execution of \texttt{prg}. The current input channel will be saved and restored
appropriately. See also \texttt{later}, \texttt{ipid}, \texttt{in} and \texttt{out}.


\begin{wideverbatim}
: (pipe                                # equivalent to 'any'
   (prinl "(a b # Comment^Jc d)")         # (child process)
   (read) )                               # (parent process)
-> (a b c d)
: (pipe                                # pipe through an external program
   (out '(tr "[a-z]" "[A-Z]")             # (child process)
      (prinl "abc def ghi") )
   (line T) )                             # (parent process)
-> "ABC DEF GHI"
\end{wideverbatim}

 
\section*{\texttt{(place 'cnt 'lst 'any) -> lst}}
\label{sec:func-ref-P-(place 'cnt 'lst 'any) -> lst}


Places \texttt{any} into \texttt{lst} at position \texttt{cnt}. This is a non-destructive
operation. See also \texttt{insert}, \texttt{remove}, \texttt{append}, \texttt{delete} and
\texttt{replace}.


\begin{wideverbatim}
: (place 3 '(a b c d e) 777)
-> (a b 777 d e)
: (place 1 '(a b c d e) 777)
-> (777 b c d e)
: (place 9 '(a b c d e) 777)
-> (a b c d e 777)
\end{wideverbatim}

 
\section*{\texttt{(poll 'cnt) -> cnt | NIL}}
\label{sec:func-ref-P-(poll 'cnt) -> cnt | NIL}


Checks for the availability of data for reading on the file descriptor
\texttt{cnt}. See also \texttt{open}, \texttt{in} and \texttt{close}.


\begin{wideverbatim}
: (and (poll *Fd) (in @ (read)))  # Prevent blocking
\end{wideverbatim}

 
\section*{\texttt{(pool ['sym1 ['lst] ['sym2] ['sym3]]) -> T}}
\label{sec:func-ref-P-(pool ['sym1 ['lst] ['sym2] ['sym3]]) -> T}


Opens the file \texttt{sym1} as a database file in read/write mode. If the file
does not exist, it is created. A currently open database is closed.
\texttt{lst} is a list of block size scale factors (i.e. numbers), defaulting
to (2) (for a single file with a 256 byte block size). If \texttt{lst} is
given, an individual database file is opened for each item. If \texttt{sym2} is
non-\texttt{NIL}, it is opened in append-mode as an asynchronous replication
journal. If \texttt{sym3} is non-\texttt{NIL}, it is opened for reading and appending,
to be used as a synchronous transaction log during \texttt{commit}s. See also \texttt{dbs}, \texttt{*Dbs} and \texttt{journal}.


\begin{wideverbatim}
: (pool "/dev/hda2")
-> T

: *Dbs
-> (1 2 2 4)
: (pool "dbFile" *Dbs)
-> T
:
abu:~/pico  ls -l dbFile*
-rw-r--r-- 1 abu abu 256 2007-06-11 07:57 dbFile1
-rw-r--r-- 1 abu abu  13 2007-06-11 07:57 dbFile2
-rw-r--r-- 1 abu abu  13 2007-06-11 07:57 dbFile3
-rw-r--r-- 1 abu abu  13 2007-06-11 07:57 dbFile4
\end{wideverbatim}

 
\section*{\texttt{(pop 'var) -> any}}
\label{sec:func-ref-P-(pop 'var) -> any}


Pops the first element (CAR) from the stack in \texttt{var}. See also \texttt{push},
\texttt{queue}, \texttt{cut}, \texttt{del} and \texttt{fifo}.


\begin{wideverbatim}
: (setq S '((a b c) (1 2 3)))
-> ((a b c) (1 2 3))
: (pop S)
-> a
: (pop (cdr S))
-> 1
: (pop 'S)
-> (b c)
: S
-> ((2 3))
\end{wideverbatim}

 
\section*{\texttt{(port ['T] 'cnt|(cnt . cnt) ['var]) -> cnt}}
\label{sec:func-ref-P-(port ['T] 'cnt|(cnt . cnt) ['var]) -> cnt}


Opens a TCP-Port \texttt{cnt} (or a UDP-Port if the first argument is \texttt{T}), and
returns a socket descriptor suitable as an argument for \texttt{listen} or
\texttt{accept} (or \texttt{udp}, respectively). If \texttt{cnt} is zero, some free port
number is allocated. If a pair of \texttt{cnt}s is given instead, it should be a range of numbers which are tried in turn. When \texttt{var} is given, it is
bound to the port number.


\begin{wideverbatim}
: (port 0 'A)                       # Allocate free port
-> 4
: A
-> 1034                             # Got 1034
: (port (4000 . 4008) 'A)           # Try one of these ports
-> 5
: A
-> 4002
\end{wideverbatim}

 
\section*{\texttt{(pp 'sym) -> sym}}
\label{sec:func-ref-P-(pp 'sym) -> sym}


\texttt{(pp 'sym 'cls) -> sym}

\texttt{(pp '(sym . cls)) -> sym}

Pretty-prints the function or method definition of \texttt{sym}. The output
format would regenerate that same definition when read and executed. See
also \texttt{pretty}, \texttt{debug} and \texttt{vi}.


\begin{wideverbatim}
: (pp 'tab)
(de tab (Lst . @)
   (for N Lst
      (let V (next)
         (and (gt0 N) (space (- N (length V))))
         (prin V)
         (and
            (lt0 N)
            (space (- 0 N (length V))) ) ) )
   (prinl) )
-> tab

: (pp 'has> '+Entity)
(dm has> (Var Val)
   (or
      (nor Val (get This Var))
      (has> (meta This Var) Val (get This Var)) ) )
-> has>

: (more (can 'has>) pp)
(dm (has> . +relation) (Val X)
   (and (= Val X) X) )

(dm (has> . +Fold) (Val X)
   (extra
      Val
      (if (= Val (fold Val)) (fold X) X) ) )

(dm (has> . +Entity) (Var Val)
   (or
      (nor Val (get This Var))
      (has> (meta This Var) Val (get This Var)) ) )

(dm (has> . +List) (Val X)
   (and
      Val
      (or
         (extra Val X)
         (find '((X) (extra Val X)) X) ) ) )

(dm (has> . +Bag) (Val X)
   (and
      Val
      (or (super Val X) (car (member Val X))) ) )
\end{wideverbatim}

 
\section*{\texttt{(pr 'any ..) -> any}}
\label{sec:func-ref-P-(pr 'any ..) -> any}


Binary print: Prints all \texttt{any} arguments to the current output channel
in encoded binary format. See also \texttt{rd}, \texttt{tell}, \texttt{hear} and \texttt{wr}.


\begin{wideverbatim}
: (out "x" (pr 7 "abc" (1 2 3) 'a))  # Print to "x"
-> a
: (hd "x")
00000000  04 0E 0E 61 62 63 01 04 02 04 04 04 06 03 05 61  ...abc.........a
-> NIL
\end{wideverbatim}

 
\section*{\texttt{(prEval 'prg ['cnt]) -> any}}
\label{sec:func-ref-P-(prEval 'prg ['cnt]) -> any}


Executes \texttt{prg}, similar to \texttt{run}, by evaluating all expressions in \texttt{prg}
(within the binding environment given by \texttt{cnt-1}). As a side effect, all
atomic expressions will be printed with \texttt{prinl}. See also \texttt{eval}.


\begin{wideverbatim}
: (let Prg 567
   (prEval
      '("abc" (prinl (+ 1 2 3)) Prg 987) ) )
abc
6
567
987
-> 987
\end{wideverbatim}

 
\section*{\texttt{(pre? 'any1 'any2) -> any2 | NIL}}
\label{sec:func-ref-P-(pre? 'any1 'any2) -> any2 | NIL}


Returns \texttt{any2} when the string representation of \texttt{any1} is a prefix of
the string representation of \texttt{any2}. See also \texttt{sub?}.


\begin{wideverbatim}
: (pre? "abc" "abcdef")
-> "abcdef"
: (pre? "def" "abcdef")
-> NIL
: (pre? (+ 3 4) "7fach")
-> "7fach"
: (pre? NIL "abcdef")
-> "abcdef"
\end{wideverbatim}

 
\section*{\texttt{(pretty 'any 'cnt)}}
\label{sec:func-ref-P-(pretty 'any 'cnt)}


Pretty-prints \texttt{any}. If \texttt{any} is an atom, or a list with a \texttt{size} not
greater than 12, it is \texttt{print}ed as is. Otherwise, only the opening
parenthesis and the CAR of the list is printed, all other elementes are
pretty-printed recursively indented by three spaces, followed by a space
and the corresponding closing parenthesis. The initial indentation level
\texttt{cnt} defaults to zero. See also \texttt{pp}.


\begin{wideverbatim}
: (pretty '(a (b c d) (e (f (g) (h) (i)) (j (k) (l) (m))) (n o p) q))
(a
   (b c d)
   (e
      (f (g) (h) (i))
      (j (k) (l) (m)) )
   (n o p)
   q )-> ")"
\end{wideverbatim}

 
\section*{\texttt{(prin 'any ..) -> any}}
\label{sec:func-ref-P-(prin 'any ..) -> any}


Prints the string representation of all \texttt{any} arguments to the current
output channel. No space or newline is printed between individual items,
or after the last item. For lists, all elements are \texttt{prin}'ted
recursively. See also \texttt{prinl}.


\begin{wideverbatim}
: (prin 'abc 123 '(a 1 b 2))
abc123a1b2-> (a 1 b 2)
\end{wideverbatim}

 
\section*{\texttt{(prinl 'any ..) -> any}}
\label{sec:func-ref-P-(prinl 'any ..) -> any}


Prints the string representation of all \texttt{any} arguments to the current
output channel, followed by a newline. No space or newline is printed
between individual items. For lists, all elements are \texttt{prin}'ted
recursively. See also \texttt{prin}.


\begin{wideverbatim}
: (prinl 'abc 123 '(a 1 b 2))
abc123a1b2
-> (a 1 b 2)
\end{wideverbatim}

 
\section*{\texttt{(print 'any ..) -> any}}
\label{sec:func-ref-P-(print 'any ..) -> any}


Prints all \texttt{any} arguments to the current output channel. If there is
more than one argument, a space is printed between successive arguments.
No space or newline is printed after the last item. See also \texttt{println},
\texttt{printsp}, \texttt{sym} and \texttt{str}


\begin{wideverbatim}
: (print 123)
123-> 123
: (print 1 2 3)
1 2 3-> 3
: (print '(a b c) 'def)
(a b c) def-> def
\end{wideverbatim}

 
\section*{\texttt{(println 'any ..) -> any}}
\label{sec:func-ref-P-(println 'any ..) -> any}


Prints all \texttt{any} arguments to the current output channel, followed by a
newline. If there is more than one argument, a space is printed between
successive arguments. See also \texttt{print}, \texttt{printsp}.


\begin{wideverbatim}
: (println '(a b c) 'def)
(a b c) def
-> def
\end{wideverbatim}

 
\section*{\texttt{(printsp 'any ..) -> any}}
\label{sec:func-ref-P-(printsp 'any ..) -> any}


Prints all \texttt{any} arguments to the current output channel, followed by a
space. If there is more than one argument, a space is printed between
successive arguments. See also \texttt{print}, \texttt{println}.


\begin{wideverbatim}
: (printsp '(a b c) 'def)
(a b c) def -> def
\end{wideverbatim}

 
\section*{\texttt{(prior 'lst1 'lst2) -> lst | NIL}}
\label{sec:func-ref-P-(prior 'lst1 'lst2) -> lst | NIL}


Returns the cell in \texttt{lst2} which immediately precedes the cell \texttt{lst1},
or \texttt{NIL} if \texttt{lst1} is not found in \texttt{lst2} or is the very first cell.
\texttt{==} is used for comparison (pointer equality). See also \texttt{offset} and
\texttt{memq}.


\begin{wideverbatim}
: (setq L (1 2 3 4 5 6))
-> (1 2 3 4 5 6)
: (setq X (cdddr L))
-> (4 5 6)
: (prior X L)
-> (3 4 5 6)
\end{wideverbatim}

 
\section*{\texttt{(proc 'sym ..) -> T}}
\label{sec:func-ref-P-(proc 'sym ..) -> T}


Shows a list of processes with command names given by the \texttt{sym}
arguments, using the system \texttt{ps} utility. See also \texttt{hd}.


\begin{wideverbatim}
: (proc 'pil)
  PID    PPID  STARTED  SIZE %CPU WCHAN  
  16993  3267 12:38:21  1516  0.5  -
  CMD
  /usr/bin/picolisp /usr/lib/picolisp/lib.l /usr/bin/pil +

  PID    PPID  STARTED  SIZE %CPU WCHAN  
  15731  1834 12:36:35  2544  0.1  -
  CMD
  /usr/bin/picolisp /usr/lib/picolisp/lib.l /usr/bin/pil app/main.l -main -go +

  PID    PPID  STARTED  SIZE %CPU WCHAN  
  15823 15731 12:36:44  2548  0.0  -
  CMD
  /usr/bin/picolisp /usr/lib/picolisp/lib.l /usr/bin/pil app/main.l -main -go +
-> T
\end{wideverbatim}

 
\section*{\texttt{(prog . prg) -> any}}
\label{sec:func-ref-P-(prog . prg) -> any}


Executes \texttt{prg}, and returns the result of the last expression. See also
\texttt{nil}, \texttt{t}, \texttt{prog1} and \texttt{prog2}.


\begin{wideverbatim}
: (prog (print 1) (print 2) (print 3))
123-> 3
\end{wideverbatim}

 
\section*{\texttt{(prog1 'any1 . prg) -> any1}}
\label{sec:func-ref-P-(prog1 'any1 . prg) -> any1}


Executes all arguments, and returns the result of the first expression
\texttt{any1}. See also \texttt{nil}, \texttt{t}, \texttt{prog} and \texttt{prog2}.


\begin{wideverbatim}
: (prog1 (print 1) (print 2) (print 3))
123-> 1
\end{wideverbatim}

 
\section*{\texttt{(prog2 'any1 'any2 . prg) -> any2}}
\label{sec:func-ref-P-(prog2 'any1 'any2 . prg) -> any2}


Executes all arguments, and returns the result of the second expression
\texttt{any2}. See also \texttt{nil}, \texttt{t}, \texttt{prog} and \texttt{prog1}.


\begin{wideverbatim}
: (prog2 (print 1) (print 2) (print 3))
123-> 2
\end{wideverbatim}

 
\section*{\texttt{(prop 'sym1|lst ['sym2|cnt ..] 'sym) -> var}}
\label{sec:func-ref-P-(prop 'sym1|lst ['sym2|cnt ..] 'sym) -> var}


Fetches a property for a property key \texttt{sym} from a symbol. That symbol
is \texttt{sym1} (if no other arguments are given), or a symbol found by
applying the \texttt{get} algorithm to \texttt{sym1|lst} and the following arguments.
The property (the cell, not just its value) is returned, suitable for
direct (destructive) manipulations with functions expecting a \texttt{var}
argument. See also \texttt{::}.


\begin{wideverbatim}
: (put 'X 'cnt 0)
-> 0
: (prop 'X 'cnt)
-> (0 . cnt)
: (inc (prop 'X 'cnt))        # Directly manipulate the property value
-> 1
: (get 'X 'cnt)
-> 1
\end{wideverbatim}

 
\section*{\texttt{(protect . prg) -> any}}
\label{sec:func-ref-P-(protect . prg) -> any}


Executes \texttt{prg}, and returns the result of the last expression. If a
signal is received during that time, its handling will be delayed until
the execution of \texttt{prg} is completed. See also \texttt{alarm},
\emph{*Hup}, \emph{*Sig\footnote{DEFINITION NOT FOUND: 12 } and \texttt{kill}.


\begin{wideverbatim}
: (protect (journal "db1.log" "db2.log"))
-> T
\end{wideverbatim}

 
\section*{\texttt{(prove 'lst ['lst]) -> lst}}
\label{sec:func-ref-P-(prove 'lst ['lst]) -> lst}


The \emph{Pilog} interpreter. Tries to prove the query list
in the first argument, and returns an association list of symbol-value
pairs, or \texttt{NIL} if not successful. The query list is modified as a side
effect, allowing subsequent calls to \texttt{prove} for further results. The
optional second argument may contain a list of symbols; in that case the
successful matches of rules defined for these symbols will be traced.
See also \texttt{goal}, \texttt{->} and \texttt{unify}.


\begin{wideverbatim}
: (prove (goal '((equal 3 3))))
-> T
: (prove (goal '((equal 3 @X))))
-> ((@X . 3))
: (prove (goal '((equal 3 4))))
-> NIL
\end{wideverbatim}

 
\section*{\texttt{(prune ['flg])}}
\label{sec:func-ref-P-(prune ['flg])}


Optimizes memory usage by pruning in-memory leaf nodes of database
trees. Typically called repeatedly during heavy data imports. If \texttt{flg}
is non-\texttt{NIL}, further pruning will be disabled. See also \texttt{lieu}.


\begin{wideverbatim}
(in File1
   (while (someData)
      (new T '(+Cls1) ..)
      (at (0 . 10000) (commit) (prune)) ) )
(in File2
   (while (moreData)
      (new T '(+Cls2) ..)
      (at (0 . 10000) (commit) (prune)) ) )
(commit)
(prune T)
\end{wideverbatim}

 
\section*{\texttt{(push 'var 'any ..) -> any}}
\label{sec:func-ref-P-(push 'var 'any ..) -> any}


Implements a stack using a list in \texttt{var}. The \texttt{any} arguments are
cons'ed in front of the value list. See also \texttt{push1}, \texttt{pop}, \texttt{queue} and
\texttt{fifo}.


\begin{wideverbatim}
: (push 'S 3)              # Use the VAL of 'S' as a stack
-> 3
: S
-> (3)
: (push 'S 2)
-> 2
: (push 'S 1)
-> 1
: S
-> (1 2 3)
: (push S 999)             # Now use the CAR of the list in 'S'
-> 999
: (push S 888 777)
-> 777
: S
-> ((777 888 999 . 1) 2 3)
\end{wideverbatim}

 
\section*{\texttt{(push1 'var 'any ..) -> any}}
\label{sec:func-ref-P-(push1 'var 'any ..) -> any}


Maintains a unique list in \texttt{var}. Each \texttt{any} argument is cons'ed in
front of the value list only if it is not already a \texttt{member} of that
list. See also \texttt{push}, \texttt{pop} and \texttt{queue}.


\begin{wideverbatim}
: (push1 'S 1 2 3)
-> 3
: S
-> (3 2 1)
: (push1 'S 2 4)
-> 4
: S
-> (4 3 2 1)
\end{wideverbatim}

 
\section*{\texttt{(put 'sym1|lst ['sym2|cnt ..] 'sym|0 'any) -> any}}
\label{sec:func-ref-P-(put 'sym1|lst ['sym2|cnt ..] 'sym|0 'any) -> any}


Stores a new value \texttt{any} for a property key \texttt{sym} (or in the value cell
for zero) in a symbol. That symbol is \texttt{sym1} (if no other arguments are
given), or a symbol found by applying the \texttt{get} algorithm to \texttt{sym1|lst}
and the following arguments. See also \texttt{=:}.


\begin{wideverbatim}
: (put 'X 'a 1)
-> 1
: (get 'X 'a)
-> 1
: (prop 'X 'a)
-> (1 . a)

: (setq L '(A B C))
-> (A B C)
: (setq B 'D)
-> D
: (put L 2 0 'p 5)  # Store '5' under the 'p' propery of the value of 'B'
-> 5
: (getl 'D)
-> ((5 . p))
\end{wideverbatim}

 
\section*{\texttt{(put! 'obj 'sym 'any) -> any}}
\label{sec:func-ref-P-(put! 'obj 'sym 'any) -> any}


\emph{Transaction} wrapper function for \texttt{put}. Note that
for setting property values of entities typically the \texttt{put!>} message is
used. See also \texttt{new!}, \texttt{set!} and \texttt{inc!}.


\begin{wideverbatim}
(put! Obj 'cnt 0)  # Setting a property of a non-entity object
\end{wideverbatim}

 
\section*{\texttt{(putl 'sym1|lst1 ['sym2|cnt ..] 'lst) -> lst}}
\label{sec:func-ref-P-(putl 'sym1|lst1 ['sym2|cnt ..] 'lst) -> lst}


Stores a complete new property list \texttt{lst} in a symbol. That symbol is
\texttt{sym1} (if no other arguments are given), or a symbol found by applying
the \texttt{get} algorithm to \texttt{sym1|lst1} and the following arguments. All
previously defined properties for that symbol are lost. See also \texttt{getl}
and \texttt{maps}.


\begin{wideverbatim}
: (putl 'X '((123 . a) flg ("Hello" . b)))
-> ((123 . a) flg ("Hello" . b))
: (get 'X 'a)
-> 123
: (get 'X 'b)
-> "Hello"
: (get 'X 'flg)
-> T
\end{wideverbatim}

 
\section*{\texttt{(pwd) -> sym}}
\label{sec:func-ref-P-(pwd) -> sym}


Returns the path to the current working directory. See also \texttt{dir} and
\texttt{cd}.


\begin{wideverbatim}
: (pwd)
-> "/home/app/"
\end{wideverbatim}


