%%%%%%%%%%%%%%%%%%%%% chapter.tex %%%%%%%%%%%%%%%%%%%%%%%%%%%%%%%%%
%
% sample chapter
%
% Use this file as a template for your own input.
%
%%%%%%%%%%%%%%%%%%%%%%%% Springer-Verlag %%%%%%%%%%%%%%%%%%%%%%%%%%
%\motto{Use the template \emph{chapter.tex} to style the various elements of your chapter content.}

\chapter{Rosetta Code Tasks starting with U}

\section*{URL decoding}


This task (the reverse of \emph{URL encoding}) is to provide a
function or mechanism to convert a url-encoded string into its
original unencoded form.

\textbf{Example}

The encoded string "\texttt{http\%3A\%2F\%2Ffoo\%20bar\%2F}" should be
reverted to the unencoded form "\texttt{http://foo bar/}".


\begin{wideverbatim}

: (ht:Pack (chop "http\%3A\%2F\%2Ffoo\%20bar\%2F"))
-> "http://foo bar/"

\end{wideverbatim}

\pagebreak{}
\section*{URL encoding}

The task is to provide a function or mechanism to convert a provided
string into URL encoding representation.

In URL encoding, special characters, control characters and extended
characters are converted into a percent symbol followed by a two digit
hexadecimal code, So a space character encodes into~\%20 within the
string.

For the purposes of this task, every character except 0-9, A-Z and a-z
requires conversion, so the following characters all require conversion
by default:

\begin{itemize}
\item
  ASCII control codes (Character ranges 00-1F hex (0-31 decimal) and 7F
  (127 decimal).
\item
  ASCII symbols (Character ranges 32-47 decimal (20-2F hex))
\item
  ASCII symbols (Character ranges 58-64 decimal (3A-40 hex))
\item
  ASCII symbols (Character ranges 91-96 decimal (5B-60 hex))
\item
  ASCII symbols (Character ranges 123-126 decimal (7B-7E hex))
\item
  Extended characters with character codes of 128 decimal (80 hex) and
  above.
\end{itemize}

\textbf{Example}

The string "\texttt{http://foo bar/}" would be encoded as \\
"\texttt{http\%3A\%2F\%2Ffoo\%20bar\%2F}".

\textbf{Variations}

\begin{itemize}
\item
  Lowercase escapes are legal, as in \\
  "\texttt{http\%3a\%2f\%2ffoo\%20bar\%2f}".
\item
  Some standards give different rules:
  \href{//tools.ietf.org/html/rfc3986}{RFC 3986}, \emph{Uniform Resource
  Identifier (URI): Generic Syntax}, section 2.3, says that
  ``-.\_\textasciitilde{}'' should not be encoded. HTML 5, section
  \href{http://www.whatwg.org/specs/web-apps/current-work/multipage/association-of-controls-and-forms.html\#url-encoded-form-data}{4.10.22.5
  URL-encoded form data}, says to preserve ``-.\_*'', and to encode
  space " " to ``+''. The options below provide for utilization of an
  exception string, enabling preservation (non encoding) of particular
  characters to meet specific standards.
\end{itemize}

\textbf{Options}

It is permissible to use an exception string (containing a set of
symbols that do not need to be converted). However, this is an optional
feature and is not a requirement of this task.

\textbf{See also}

\emph{URL decoding}


\begin{wideverbatim}

(de urlEncodeTooMuch (Str)
   (pack
      (mapcar
         '((C)
            (if (or (>= "9" C "0") (>= "Z" (uppc C) "A"))
               C
               (list '\% (hex (char C))) ) )
         (chop Str) ) ) )

Test:

: (urlEncodeTooMuch "http://foo bar/")
-> "http\%3A\%2F\%2Ffoo\%20bar\%2F"

\end{wideverbatim}

\pagebreak{}
\section*{Unbias a random generator}

Given a weighted one bit generator of random numbers where the
probability of a one occuring, \emph{P}\textsubscript{1}, is not the
same as \emph{P}\textsubscript{0}, the probability of a zero occuring,
the probability of the occurrence of a one followed by a zero is
\emph{P}\textsubscript{1} × \emph{P}\textsubscript{0}. This is the same
as the probability of a zero followed by a one:
\emph{P}\textsubscript{0} × \emph{P}\textsubscript{1}.

\textbf{Task Details}

\begin{itemize}
\item Use your language's random number generator to create a
  \emph{function | method | subroutine | \ldots{}} \textbf{randN} that
  returns a one or a zero, but with one occurring, on average, 1 out
  of N times, where N is an integer from the range 3 to 6 inclusive.
\item
  Create a function \textbf{unbiased} that uses only randN as its source
  of randomness to become an unbiased generator of random ones and
  zeroes.
\item
  For N over its range, generate and show counts of the outputs of randN
  and unbiased(randN).
\end{itemize}

The actual unbiasing should be done by generating two numbers at a time
from randN and only returning a 1 or 0 if they are different. As long as
you always return the first number or always return the second number,
the probabilities discussed above should take over the biased
probability of randN.



\begin{wideverbatim}

(de randN (N)
   (if (= 1 (rand 1 N)) 1 0) )

(de unbiased (N)
   (use (A B)
      (while
         (=
            (setq A (randN N))
            (setq B (randN N)) ) )
      A ) )

(for N (range 3 6)
   (tab (2 1 7 2 7 2)
      N ":"
      (format
         (let S 0 (do 10000 (inc 'S (randN N))))
         2 )
      "\%"
      (format
         (let S 0 (do 10000 (inc 'S (unbiased N))))
         2 )
      "\%" ) )

Output:

 3:  33.21 \%  50.48 \%
 4:  25.06 \%  49.79 \%
 5:  20.04 \%  49.75 \%
 6:  16.32 \%  49.02 \%

\end{wideverbatim}

\pagebreak{}
\section*{Undefined values}


For languages which have an explicit notion of an undefined value,
identify and exercise those language's mechanisms for identifying and
manipulating a variable's value's status as being undefined

\begin{wideverbatim}

An internal symbol is initialized to NIL. Depending on the context,
this is interpreted as "undefined". When called as a function, an
error is issued:

: (myfoo 3 4)
!? (myfoo 3 4)
myfoo -- Undefined
?

The function 'default' can be used to initialize a variable if and
only if its current value is NIL:

: MyVar
-> NIL

: (default MyVar 7)
-> 7

: MyVar
-> 7

: (default MyVar 8)
-> 7

: MyVar
-> 7

\end{wideverbatim}

% \pagebreak{}
% \section*{Undefined values/Check if a variable is defined}

% \begin{wideverbatim}

% New internal symbols are initialized with the value NIL. NIL is also the value
% for "false", so there is never really an "undefined value".
% '[http://software-lab.de/doc/refN.html#not not]' is the predicate to check for
% NIL, but many other (typically flow control) functions can be used.

% (if (not MyNewVariable)
%    (handle value-is-NIL) )

% or

% (unless MyNewVariable
%    (handle value-is-NIL) )

% \end{wideverbatim}

\pagebreak{}
\section*{Unicode strings}

As the world gets smaller each day, internationalization becomes more
and more important. For handling multiple languages,
\emph{Unicode} is your best friend. It is a very capable
tool, but also quite complex compared to older single- and double-byte
character encodings. How well prepared is your programming language for
Unicode? Discuss and demonstrate its unicode awareness and capabilities.
Some suggested topics:

\begin{itemize}
\item
  How easy is it to present Unicode strings in source code? Can Unicode
  literals be written directly, or be part of identifiers/keywords/etc?
\item
  How well can the language communicate with the rest of the world? Is
  it good at input/output with Unicode?
\item
  Is it convenient to manipulate Unicode strings in the language?
\item
  How broad/deep does the language support Unicode? What encodings (e.g.
  UTF-8, UTF-16, etc) can be used? Normalization?
\end{itemize}

\textbf{Note} This task is a bit unusual in that it encourages general
discussion rather than clever coding.

See also:

\begin{itemize}
\item \emph{Unicode variable names}
\item \emph{Terminal control/Display an extended character}
\end{itemize}



\begin{wideverbatim}

PicoLisp can directly handle _only_ Unicode (UTF-8) strings. So the problem is
rather how to handle non-Unicode strings: They must be pre- or post-processed by
external tools, typically with pipes during I/O. For example, to read a line
from a file in 8859 encoding:

   (in '(iconv "-f" "ISO-8859-15" "file.txt") (line))

\end{wideverbatim}

\pagebreak{}
\section*{Unicode variable names}

\begin{enumerate}
\item
  Describe, and give a pointer to documentation on your languages use of
  characters \emph{beyond} those of the ASCII character set in the
  naming of variables.
\item
  Show how to:
\end{enumerate}

\begin{itemize}
\item
  Set a variable with a name including the `Δ', (delta
  \emph{character}), to 1
\item
  Increment it
\item
  Print its value.
\end{itemize}

\begin{description}
\item[Cf.]
\end{description}

\begin{itemize}
\item
  \emph{Case-sensitivity of
  identifiers}
\end{itemize}

\begin{wideverbatim}

Variables are usually
[http://software-lab.de/doc/ref.html#internal-io Internal Symbols],
and their names may contain any UTF-8 character except null-bytes.
White space, and 11 special characters (see the reference) must be
escaped with a backslash.
[http://software-lab.de/doc/ref.html#transient-io Transient Symbols]
are often used as variables too, they follow the syntax of strings
in other languages.

: (setq Δ 1)
-> 1
: Δ
-> 1
: (inc 'Δ)
-> 2
: Δ
-> 2

\end{wideverbatim}

\pagebreak{}
\section*{Update a configuration file}

We have a configuration file as follows:

\begin{wideverbatim}
# This is a configuration file in standard configuration file format
#
# Lines begininning with a hash or a semicolon are ignored by the application
# program. Blank lines are also ignored by the application program.

# The first word on each non comment line is the configuration option.
# Remaining words or numbers on the line are configuration parameter
# data fields.

# Note that configuration option names are not case sensitive. However,
# configuration parameter data is case sensitive and the lettercase must
# be preserved.

# This is a favourite fruit
FAVOURITEFRUIT banana

# This is a boolean that should be set
NEEDSPEELING

# This boolean is commented out
; SEEDSREMOVED

# How many bananas we have
NUMBEROFBANANAS 48
\end{wideverbatim}

The task is to manipulate the configuration file as follows:

\begin{itemize}
\item
  Disable the needspeeling option (using a semicolon prefix)
\item
  Enable the seedsremoved option by removing the semicolon and any
  leading whitespace
\item
  Change the numberofbananas parameter to 1024
\item
  Enable (or create if it does not exist in the file) a parameter for
  numberofstrawberries with a value of 62000
\end{itemize}

Note that configuration option names are not case sensitive. This means
that changes should be effected, regardless of the case.

Options should always be disabled by prefixing them with a semicolon.

Lines beginning with hash symbols should not be manipulated and left
unchanged in the revised file.

If a configuration option does not exist within the file (in either
enabled or disabled form), it should be added during this update.
Duplicate configuration option names in the file should be removed,
leaving just the first entry.

For the purpose of this task, the revised file should contain
appropriate entries, whether enabled or not for
needspeeling,seedsremoved,numberofbananas and numberofstrawberries.)

The update should rewrite configuration option names in capital letters.
However lines beginning with hashes and any parameter data must not be
altered (eg the banana for favourite fruit must not become capitalized).
The update process should also replace double semicolon prefixes with
just a single semicolon (unless it is uncommenting the option, in which
case it should remove all leading semicolons).

Any lines beginning with a semicolon or groups of semicolons, but no
following option should be removed, as should any leading or trailing
whitespace on the lines. Whitespace between the option and paramters
should consist only of a single space, and any non ascii extended
characters, tabs characters, or control codes (other than end of line
markers), should also be removed.

\textbf{See also:}

\begin{itemize}
\item
  \emph{Read a configuration file}
\end{itemize}


\begin{wideverbatim}

(let Data  # Read all data
   (in "config"
      (make
         (until (eof)
            (link (trim (split (line) " "))) ) ) )
   (setq Data  # Fix comments
      (mapcar
         '((L)
            (while (head '(";" ";") (car L))
               (pop L) )
            (if (= '(";") (car L))
               L
               (cons NIL L) ) )
         Data ) )
   (let (Need NIL  Seed NIL  NBan NIL  NStr NIL  Favo NIL)
      (map
         '((L)
            (let D (mapcar uppc (cadar L))
               (cond
                  ((= '`(chop "NEEDSPEELING") D)
                     (if Need
                        (set L)
                        (on Need)
                        (unless (caar L)
                           (set (car L) '(";")) ) ) )
                  ((= '`(chop "SEEDSREMOVED") D)
                     (if Seed
                        (set L)
                        (on Seed)
                        (when (caar L)
                           (set (car L)) ) ) )
                  ((= '`(chop "NUMBEROFBANANAS") D)
                     (if NBan
                        (set L)
                        (on NBan)
                        (set (cddar L) 1024) ) )
                  ((= '`(chop "NUMBEROFSTRAWBERRIES") D)
                     (if NStr
                        (set L)
                        (on NStr) ) )
                  ((= '`(chop "FAVOURITEFRUIT") D)
                     (if Favo
                        (set L)
                        (on Favo) ) ) ) ) )
         Data )

\end{wideverbatim}

\begin{wideverbatim}

      (unless Need
         (conc Data (cons (list NIL "NEEDSPEELING"))) )
      (unless Seed
         (conc Data (cons (list NIL "SEEDSREMOVED"))) )
      (unless NBan
         (conc Data (cons (list NIL "NUMBEROFBANANAS" 1024))) )
      (unless NStr
         (conc Data (cons (list NIL "NUMBEROFSTRAWBERRIES" 62000))) ) )
   (out "config"
      (for L Data
         (prinl (glue " " (if (car L) L (cdr L)))) ) ) )

\end{wideverbatim}

\pagebreak{}
\section*{User input/Graphical}

In this task, the goal is to input a string and the integer 75000,
from \emph{graphical user interface}.

See also: \emph{User input/Text}

\begin{wideverbatim}

(and
   (call 'sh "-c"
      (pack
         "dialog \
            --inputbox 'Input a string' 8 60 \
            --inputbox 'Input a number' 8 20 \
            2>"
         (tmp "dlg") ) )
   (split (in (tmp "dlg") (line)) "^I")
   (cons (pack (car @)) (format (cadr @))) )

Output:

-> ("Hello world" . 12345)

\end{wideverbatim}

\pagebreak{}
\section*{User input/Text}

\textbf{User input/Text} is part of \emph{Short Circuit}'s
\textbf{\emph{Console Program Basics}} selection.

In this task, the goal is to input a string and the integer 75000, from
the text console.

See also: \emph{User input/Graphical}

\begin{wideverbatim}

(in NIL  # Guarantee reading from standard input
   (let (Str (read)  Num (read))
      (prinl "The string is: \"" Str "\"")
      (prinl "The number is: " Num) ) )

\end{wideverbatim}



% %%%%%%%%%%%%%%%%%%%%%%%% referenc.tex %%%%%%%%%%%%%%%%%%%%%%%%%%%%%%
% sample references
% %
% Use this file as a template for your own input.
%
%%%%%%%%%%%%%%%%%%%%%%%% Springer-Verlag %%%%%%%%%%%%%%%%%%%%%%%%%%
%
% BibTeX users please use
% \bibliographystyle{}
% \bibliography{}
%
\biblstarthook{In view of the parallel print and (chapter-wise) online publication of your book at \url{www.springerlink.com} it has been decided that -- as a genreral rule --  references should be sorted chapter-wise and placed at the end of the individual chapters. However, upon agreement with your contact at Springer you may list your references in a single seperate chapter at the end of your book. Deactivate the class option \texttt{sectrefs} and the \texttt{thebibliography} environment will be put out as a chapter of its own.\\\indent
References may be \textit{cited} in the text either by number (preferred) or by author/year.\footnote{Make sure that all references from the list are cited in the text. Those not cited should be moved to a separate \textit{Further Reading} section or chapter.} The reference list should ideally be \textit{sorted} in alphabetical order -- even if reference numbers are used for the their citation in the text. If there are several works by the same author, the following order should be used: 
\begin{enumerate}
\item all works by the author alone, ordered chronologically by year of publication
\item all works by the author with a coauthor, ordered alphabetically by coauthor
\item all works by the author with several coauthors, ordered chronologically by year of publication.
\end{enumerate}
The \textit{styling} of references\footnote{Always use the standard abbreviation of a journal's name according to the ISSN \textit{List of Title Word Abbreviations}, see \url{http://www.issn.org/en/node/344}} depends on the subject of your book:
\begin{itemize}
\item The \textit{two} recommended styles for references in books on \textit{mathematical, physical, statistical and computer sciences} are depicted in ~\cite{science-contrib, science-online, science-mono, science-journal, science-DOI} and ~\cite{phys-online, phys-mono, phys-journal, phys-DOI, phys-contrib}.
\item Examples of the most commonly used reference style in books on \textit{Psychology, Social Sciences} are~\cite{psysoc-mono, psysoc-online,psysoc-journal, psysoc-contrib, psysoc-DOI}.
\item Examples for references in books on \textit{Humanities, Linguistics, Philosophy} are~\cite{humlinphil-journal, humlinphil-contrib, humlinphil-mono, humlinphil-online, humlinphil-DOI}.
\item Examples of the basic Springer style used in publications on a wide range of subjects such as \textit{Computer Science, Economics, Engineering, Geosciences, Life Sciences, Medicine, Biomedicine} are ~\cite{basic-contrib, basic-online, basic-journal, basic-DOI, basic-mono}. 
\end{itemize}
}

\begin{thebibliography}{99.}%
% and use \bibitem to create references.
%
% Use the following syntax and markup for your references if 
% the subject of your book is from the field 
% "Mathematics, Physics, Statistics, Computer Science"
%
% Contribution 
\bibitem{science-contrib} Broy, M.: Software engineering --- from auxiliary to key technologies. In: Broy, M., Dener, E. (eds.) Software Pioneers, pp. 10-13. Springer, Heidelberg (2002)
%
% Online Document
\bibitem{science-online} Dod, J.: Effective substances. In: The Dictionary of Substances and Their Effects. Royal Society of Chemistry (1999) Available via DIALOG. \\
\url{http://www.rsc.org/dose/title of subordinate document. Cited 15 Jan 1999}
%
% Monograph
\bibitem{science-mono} Geddes, K.O., Czapor, S.R., Labahn, G.: Algorithms for Computer Algebra. Kluwer, Boston (1992) 
%
% Journal article
\bibitem{science-journal} Hamburger, C.: Quasimonotonicity, regularity and duality for nonlinear systems of partial differential equations. Ann. Mat. Pura. Appl. \textbf{169}, 321--354 (1995)
%
% Journal article by DOI
\bibitem{science-DOI} Slifka, M.K., Whitton, J.L.: Clinical implications of dysregulated cytokine production. J. Mol. Med. (2000) doi: 10.1007/s001090000086 
%
\bigskip

% Use the following (APS) syntax and markup for your references if 
% the subject of your book is from the field 
% "Mathematics, Physics, Statistics, Computer Science"
%
% Online Document
\bibitem{phys-online} J. Dod, in \textit{The Dictionary of Substances and Their Effects}, Royal Society of Chemistry. (Available via DIALOG, 1999), 
\url{http://www.rsc.org/dose/title of subordinate document. Cited 15 Jan 1999}
%
% Monograph
\bibitem{phys-mono} H. Ibach, H. L\"uth, \textit{Solid-State Physics}, 2nd edn. (Springer, New York, 1996), pp. 45-56 
%
% Journal article
\bibitem{phys-journal} S. Preuss, A. Demchuk Jr., M. Stuke, Appl. Phys. A \textbf{61}
%
% Journal article by DOI
\bibitem{phys-DOI} M.K. Slifka, J.L. Whitton, J. Mol. Med., doi: 10.1007/s001090000086
%
% Contribution 
\bibitem{phys-contrib} S.E. Smith, in \textit{Neuromuscular Junction}, ed. by E. Zaimis. Handbook of Experimental Pharmacology, vol 42 (Springer, Heidelberg, 1976), p. 593
%
\bigskip
%
% Use the following syntax and markup for your references if 
% the subject of your book is from the field 
% "Psychology, Social Sciences"
%
%
% Monograph
\bibitem{psysoc-mono} Calfee, R.~C., \& Valencia, R.~R. (1991). \textit{APA guide to preparing manuscripts for journal publication.} Washington, DC: American Psychological Association.
%
% Online Document
\bibitem{psysoc-online} Dod, J. (1999). Effective substances. In: The dictionary of substances and their effects. Royal Society of Chemistry. Available via DIALOG. \\
\url{http://www.rsc.org/dose/Effective substances.} Cited 15 Jan 1999.
%
% Journal article
\bibitem{psysoc-journal} Harris, M., Karper, E., Stacks, G., Hoffman, D., DeNiro, R., Cruz, P., et al. (2001). Writing labs and the Hollywood connection. \textit{J Film} Writing, 44(3), 213--245.
%
% Contribution 
\bibitem{psysoc-contrib} O'Neil, J.~M., \& Egan, J. (1992). Men's and women's gender role journeys: Metaphor for healing, transition, and transformation. In B.~R. Wainrig (Ed.), \textit{Gender issues across the life cycle} (pp. 107--123). New York: Springer.
%
% Journal article by DOI
\bibitem{psysoc-DOI}Kreger, M., Brindis, C.D., Manuel, D.M., Sassoubre, L. (2007). Lessons learned in systems change initiatives: benchmarks and indicators. \textit{American Journal of Community Psychology}, doi: 10.1007/s10464-007-9108-14.
%
%
% Use the following syntax and markup for your references if 
% the subject of your book is from the field 
% "Humanities, Linguistics, Philosophy"
%
\bigskip
%
% Journal article
\bibitem{humlinphil-journal} Alber John, Daniel C. O'Connell, and Sabine Kowal. 2002. Personal perspective in TV interviews. \textit{Pragmatics} 12:257--271
%
% Contribution 
\bibitem{humlinphil-contrib} Cameron, Deborah. 1997. Theoretical debates in feminist linguistics: Questions of sex and gender. In \textit{Gender and discourse}, ed. Ruth Wodak, 99--119. London: Sage Publications.
%
% Monograph
\bibitem{humlinphil-mono} Cameron, Deborah. 1985. \textit{Feminism and linguistic theory.} New York: St. Martin's Press.
%
% Online Document
\bibitem{humlinphil-online} Dod, Jake. 1999. Effective substances. In: The dictionary of substances and their effects. Royal Society of Chemistry. Available via DIALOG. \\
http://www.rsc.org/dose/title of subordinate document. Cited 15 Jan 1999
%
% Journal article by DOI
\bibitem{humlinphil-DOI} Suleiman, Camelia, Daniel C. O�Connell, and Sabine Kowal. 2002. `If you and I, if we, in this later day, lose that sacred fire...�': Perspective in political interviews. \textit{Journal of Psycholinguistic Research}. doi: 10.1023/A:1015592129296.
%
%
%
\bigskip
%
%
% Use the following syntax and markup for your references if 
% the subject of your book is from the field 
% "Computer Science, Economics, Engineering, Geosciences, Life Sciences"
%
%
% Contribution 
\bibitem{basic-contrib} Brown B, Aaron M (2001) The politics of nature. In: Smith J (ed) The rise of modern genomics, 3rd edn. Wiley, New York 
%
% Online Document
\bibitem{basic-online} Dod J (1999) Effective Substances. In: The dictionary of substances and their effects. Royal Society of Chemistry. Available via DIALOG. \\
\url{http://www.rsc.org/dose/title of subordinate document. Cited 15 Jan 1999}
%
% Journal article by DOI
\bibitem{basic-DOI} Slifka MK, Whitton JL (2000) Clinical implications of dysregulated cytokine production. J Mol Med, doi: 10.1007/s001090000086
%
% Journal article
\bibitem{basic-journal} Smith J, Jones M Jr, Houghton L et al (1999) Future of health insurance. N Engl J Med 965:325--329
%
% Monograph
\bibitem{basic-mono} South J, Blass B (2001) The future of modern genomics. Blackwell, London 
%
\end{thebibliography}

