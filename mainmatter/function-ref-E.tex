%%%%%%%%%%%%%%%%%%%%% chapter.tex %%%%%%%%%%%%%%%%%%%%%%%%%%%%%%%%%
%
% sample chapter
%
% Use this file as a template for your own input.
%
%%%%%%%%%%%%%%%%%%%%%%%% Springer-Verlag %%%%%%%%%%%%%%%%%%%%%%%%%%
%\motto{Use the template \emph{chapter.tex} to style the various elements of your chapter content.}



\chapter{Symbols starting with E}
\label{cha:func-ref-E-functions-starting-with-E}

 
\section*{\texttt{*Err}}
\label{sec:func-ref-E-*Err}


A global variable holding a (possibly empty) \texttt{prg} body, which will be
executed during error processing. See also \texttt{Error Handling}, \texttt{*Msg} and
\texttt{\textasciicircum{}}.


\begin{wideverbatim}
: (de *Err (prinl "Fatal error!"))
-> ((prinl "Fatal error!"))
: (/ 3 0)
!? (/ 3 0)
Div/0
Fatal error!
$
\end{wideverbatim}

 
\section*{\texttt{*Ext}}
\label{sec:func-ref-E-*Ext}


A global variable holding a sorted list of cons pairs. The CAR of each
pair specifies an external symbol offset (suitable for \texttt{ext}), and the
CDR should be a function taking a single external symbol as an argument.
This function should return a list, with the value for that symbol in
its CAR, and the property list (in the format used by \texttt{getl} and \texttt{putl})
in its CDR. The symbol will be set to this value and property list upon
access. Typically this function will access the corresponding symbol in
a remote database process. See also \texttt{qsym} and \texttt{external symbols}.


\begin{wideverbatim}
### On the local machine ###
: (setq *Ext  # Define extension functions
   (mapcar
      '((@Host @Ext)
         (cons @Ext
            (curry (@Host @Ext (Sock)) (Obj)
               (when (or Sock (setq Sock (connect @Host 4040)))
                  (ext @Ext
                     (out Sock (pr (cons 'qsym Obj)))
                     (prog1 (in Sock (rd))
                        (unless @
                           (close Sock)
                           (off Sock) ) ) ) ) ) ) )
      '("10.10.12.1" "10.10.12.2" "10.10.12.3" "10.10.12.4")
      (20 40 60 80) ) )

### On the remote machines ###
(de go ()
   ...
   (task (port 4040)                      # Set up background query server
      (let? Sock (accept @)               # Accept a connection
         (unless (fork)                   # In child process
            (in Sock
               (while (rd)                # Handle requests
                  (sync)
                  (out Sock
                     (pr (eval @)) ) ) )
            (bye) )                       # Exit child process
         (close Sock) ) )
   (forked)                               # Close task in children
   ...
\end{wideverbatim}

 
\section*{\texttt{+Entity}}
\label{sec:func-ref-E-+Entity}


Base class of all database objects. See also \texttt{+relation} and \texttt{Database}.

Messages to entity objects include


\begin{wideverbatim}
zap> ()              # Clean up relational structures, for removal from the DB
url> (Tab)           # Call the GUI on that object (in optional Tab)
upd> (X Old)         # Callback method when object is created/modified/deleted
has> (Var Val)       # Check if value is present
put> (Var Val)       # Put a new value
put!> (Var Val)      # Put a new value, single transaction
del> (Var Val)       # Delete value (also partial)
del!> (Var Val)      # Delete value (also partial), single transaction
inc> (Var Val)       # Increment numeric value
inc!> (Var Val)      # Increment numeric value, single transaction
dec> (Var Val)       # Decrement numeric value
dec!> (Var Val)      # Decrement numeric value, single transaction
mis> (Var Val)       # Return error message if value or type mismatch
lose1> (Var)         # Delete relational structures for a single attribute
lose> (Lst)          # Delete relational structures (excluding 'Lst')
lose!> ()            # Delete relational structures, single transaction
keep1> (Var)         # Restore relational structures for single attribute
keep> (Lst)          # Restore relational structures (excluding 'Lst')
keep?> (Lst)         # Test for restauration (excluding 'Lst')
keep!> ()            # Restore relational structures, single transaction
set> (Val)           # Set the value (type, i.e. class list)
set!> (Val)          # Set the value, single transaction
clone> ()            # Object copy
clone!> ()           # Object copy, single transaction
\end{wideverbatim}

 
\section*{\texttt{(e . prg) -> any}}
\label{sec:func-ref-E-(e . prg) -> any}


Used in a breakpoint. Evaluates \texttt{prg} in the execution environment, or
the currently executed expression if \texttt{prg} is not given. See also
\texttt{debug}, \texttt{!}, \texttt{\textasciicircum{}} and \texttt{*Dbg}.


\begin{wideverbatim}
: (! + 3 4)
(+ 3 4)
! (e)
-> 7
\end{wideverbatim}

 
\section*{\texttt{(echo ['cnt ['cnt]] | ['sym ..]) -> sym}}
\label{sec:func-ref-E-(echo ['cnt ['cnt]] | ['sym ..]) -> sym}


Reads the current input channel, and writes to the current output
channel. If \texttt{cnt} is given, only that many bytes are actually echoed. In
case of two \texttt{cnt} arguments, the first one specifies the number of bytes
to skip in the input stream. Otherwise, if one or more \texttt{sym} arguments
are given, the echo process stops as soon as one of the symbol's names
is encountered in the input stream. In this case the name will be read
and returned, but not written. Returns non-\texttt{NIL} if the operation was
successfully completed. See also \texttt{from}.


\begin{wideverbatim}
: (in "x.l" (echo))  # Display file on console
 ..

: (out "x2.l" (in "x.l" (echo)))  # Copy file "x.l" to "x2.l"
\end{wideverbatim}

 
\section*{\texttt{(edit 'sym ..) -> NIL}}
\label{sec:func-ref-E-(edit 'sym ..) -> NIL}


Edits the value and property list of the argument symbol(s) by calling
the \texttt{vim} editor on a temporary file with these data. When closing the
editor, the modified data are read and stored into the symbol(s). During
the edit session, individual symbols are separated by the pattern
\texttt{(********)}. These separators should not be modified. When moving the
cursor to the beginning of a symbol (no matter if internal, transient or
external), and hitting `\texttt{K}', that symbol is added to the currently
edited symbols. Hitting `\texttt{Q}' will go back one step and return to the
previously edited list of symbols.

\texttt{edit} is especially useful for browsing through the database (with
`\texttt{K}' and `\texttt{Q}'), inspecting external symbols, but care must be taken
when modifying any data as then the \emph{entity/relation}
mechanisms are circumvented, and \texttt{commit} has to be called manually if
the changes should be persistent.

Another typical use case is inserting or removing \texttt{!} breakpoints at
arbitrary code locations, or doing other temporary changes to the code
for debugging purposes.

See also \texttt{update}, \texttt{show} and \texttt{vi}.


\begin{wideverbatim}
: (edit (db 'nr '+Item 1))  # Edit a database symbol
### 'vim' shows this ###
{3-1} (+Item)
   nr 1
   inv 100
   pr 29900
   sup {2-1}  # (+CuSu)
   nm "Main Part"

(********)
### Hitting 'K' on the '{' of '{2-1} ###
{2-1} (+CuSu)
   nr 1
   plz "3425"
   mob "37 176 86303"
   tel "37 4967 6846-0"
   fax "37 4967 68462"
   nm "Active Parts Inc."
   nm2 "East Division"
   ort "Freetown"
   str "Wildcat Lane"
   em "info@api.tld"

(********)

{3-1} (+Item)
   nr 1
   inv 100
   pr 29900
   sup {2-1}  # (+CuSu)
   nm "Main Part"

(********)
### Entering ':q' in vim ###
-> NIL
\end{wideverbatim}

 
\section*{\texttt{(env ['lst] | ['sym 'val] ..) -> lst}}
\label{sec:func-ref-E-(env ['lst] | ['sym 'val] ..) -> lst}


Return a list of symbol-value pairs of all dynamically bound symbols
if called without arguments, or of the symbols or symbol-value pairs
in \texttt{lst}, or the explicitly given \texttt{sym}-\texttt{val}
arguments. See also \texttt{bind} and \texttt{job}.


\begin{wideverbatim}
: (env)
-> NIL
: (let (A 1 B 2) (env))
-> ((A . 1) (B . 2))
: (let (A 1 B 2) (env '(A B)))
-> ((B . 2) (A . 1))
: (let (A 1 B 2) (env 'X 7 '(A B (C . 3)) 'Y 8))
-> ((Y . 8) (C . 3) (B . 2) (A . 1) (X . 7))
\end{wideverbatim}

 
\section*{\texttt{(eof ['flg]) -> flg}}
\label{sec:func-ref-E-(eof ['flg]) -> flg}


Returns the end-of-file status of the current input channel. If \texttt{flg} is
non-\texttt{NIL}, the channel's status is forced to end-of-file, so that the
next call to \texttt{eof} will return \texttt{T}, and calls to \texttt{char}, \texttt{peek}, \texttt{line},
\texttt{from}, \texttt{till}, \texttt{read} or \texttt{skip} will return \texttt{NIL}. Note that \texttt{eof}
cannot be used with the binary \texttt{rd} function. See also \texttt{eol}.


\begin{wideverbatim}
: (in "file" (until (eof) (println (line T))))
...
\end{wideverbatim}

 
\section*{\texttt{(eol) -> flg}}
\label{sec:func-ref-E-(eol) -> flg}


Returns the end-of-line status of the current input channel. See also
\texttt{eof}.


\begin{wideverbatim}
: (make (until (prog (link (read)) (eol))))  # Read line into a list
a b c (d e f) 123
-> (a b c (d e f) 123)
\end{wideverbatim}

 
\section*{\texttt{equal/2}}
\label{sec:func-ref-E-equal/2}


\emph{Pilog} predicate that succeeds if the two
arguments are equal. See also \texttt{=}, \texttt{different/2} and
\texttt{member/2}.


\begin{wideverbatim}
: (? (equal 3 4))
-> NIL
: (? (equal @N 7))
 @N=7
-> NIL
\end{wideverbatim}

 
\section*{\texttt{(err 'sym . prg) -> any}}
\label{sec:func-ref-E-(err 'sym . prg) -> any}


Redirects the standard error stream to \texttt{sym} during the execution of
\texttt{prg}. The current standard error stream will be saved and restored
appropriately. If the argument is \texttt{NIL}, the current output stream will
be used. Otherwise, \texttt{sym} is taken as a file name (opened in ``append''
mode if the first character is ``+''), where standard error is to be
written to. See also \texttt{in}, \texttt{out} and \texttt{ctl}.


\begin{wideverbatim}
: (err "/dev/null"             # Suppress error messages
   (call 'ls 'noSuchFile) )
-> NIL
\end{wideverbatim}

 
\section*{\texttt{(errno) -> cnt}}
\label{sec:func-ref-E-(errno) -> cnt}


(64-bit version only) Returns the value of the standard I/O `errno'
variable. See also \texttt{native}.


\begin{wideverbatim}
: (in "foo")                           # Produce an error
!? (in "foo")
"foo" -- Open error: No such file or directory
? (errno)
-> 2                                   # Returned 'ENOENT'
\end{wideverbatim}

 
\section*{\texttt{(eval 'any ['cnt ['lst]]) -> any}}
\label{sec:func-ref-E-(eval 'any ['cnt ['lst]]) -> any}


Evaluates \texttt{any}. Note that because of the standard argument evaluation,
\texttt{any} is actually evaluated twice. If a binding environment offset \texttt{cnt}
is given, the second evaluation takes place in the corresponding
environment, and an optional \texttt{lst} of excluded symbols can be supplied.
See also \texttt{run} and \texttt{up}.


\begin{wideverbatim}
: (eval (list '+ 1 2 3))
-> 6
: (setq X 'Y  Y 7)
-> 7
: X
-> Y
: Y
-> 7
: (eval X)
-> 7
\end{wideverbatim}

 
\section*{\texttt{(expDat 'sym) -> dat}}
\label{sec:func-ref-E-(expDat 'sym) -> dat}


Expands a \texttt{date} string according to the current \texttt{locale} (delimiter,
and order of year, month and day). Accepts abbreviated input, without
delimiter and with only the day, or the day and month, or the day, month
and year of current century. See also \texttt{datStr}, \texttt{day}, \texttt{expTel}.


\begin{wideverbatim}
: (date)
-> 733133
: (date (date))
-> (2007 5 31)
: (expDat "31")
-> 733133
: (expDat "315")
-> 733133
: (expDat "3105")
-> 733133
: (expDat "31057")
-> 733133
: (expDat "310507")
-> 733133
: (expDat "2007-05-31")
-> 733133
: (expDat "7-5-31")
-> 733133

: (locale "DE" "de")
-> NIL
: (expDat "31.5")
-> 733133
: (expDat "31.5.7")
-> 733133
\end{wideverbatim}

 
\section*{\texttt{(expTel 'sym) -> sym}}
\label{sec:func-ref-E-(expTel 'sym) -> sym}


Expands a telephone number string. Multiple spaces or hyphens are
coalesced. A leading \texttt{+} or \texttt{00} is removed, a leading \texttt{0} is replaced
with the current country code. Otherwise, \texttt{NIL} is returned. See also
\texttt{telStr}, \texttt{expDat} and \texttt{locale}.


\begin{wideverbatim}
: (expTel "+49 1234 5678-0")
-> "49 1234 5678-0"
: (expTel "0049 1234 5678-0")
-> "49 1234 5678-0"
: (expTel "01234 5678-0")
-> NIL
: (locale "DE" "de")
-> NIL
: (expTel "01234 5678-0")
-> "49 1234 5678-0"
\end{wideverbatim}

 
\section*{\texttt{(expr 'sym) -> fun}}
\label{sec:func-ref-E-(expr 'sym) -> fun}


Converts a C-function (``subr'') to a Lisp-function. Useful only for
normal functions (i.e. functions that evaluate all arguments). See also
\texttt{subr}.


\begin{wideverbatim}
: car
-> 67313448
: (expr 'car)
-> (@ (pass $385260187))
: (car (1 2 3))
-> 1
\end{wideverbatim}

 
\section*{\texttt{(ext 'cnt . prg) -> any}}
\label{sec:func-ref-E-(ext 'cnt . prg) -> any}


During the execution of \texttt{prg}, all \texttt{external symbols} processed by \texttt{rd},
\texttt{pr} or \texttt{udp} are modified by an offset \texttt{cnt} suitable for mapping via
the \texttt{*Ext} mechanism. All external symbol's file numbers are decremented
by \texttt{cnt} during output, and incremented by \texttt{cnt} during input.


\begin{wideverbatim}
: (out 'a (ext 5 (pr '({6-2} ({8-9} . a) ({7-7} . b)))))
-> ({6-2} ({8-9} . a) ({7-7} . b))

: (in 'a (rd))
-> ({2} ({3-9} . a) ({2-7} . b))

: (in 'a (ext 5 (rd)))
-> ({6-2} ({8-9} . a) ({7-7} . b))
\end{wideverbatim}

 
\section*{\texttt{(ext? 'any) -> sym | NIL}}
\label{sec:func-ref-E-(ext? 'any) -> sym | NIL}


Returns the argument \texttt{any} when it is an existing external symbol,
otherwise \texttt{NIL}. See also \texttt{sym?}, \texttt{box?}, \texttt{str?}, \texttt{extern} and \texttt{lieu}.


\begin{wideverbatim}
: (ext? *DB)
-> {1}
: (ext? 'abc)
-> NIL
: (ext? "abc")
-> NIL
: (ext? 123)
-> NIL
\end{wideverbatim}

 
\section*{\texttt{(extend cls) -> cls}}
\label{sec:func-ref-E-(extend cls) -> cls}


Extends the class \texttt{cls}, by storing it in the global variable \texttt{*Class}.
As a consequence, all following method, relation and class variable
definitions are applied to that class. See also \texttt{OO Concepts}, \texttt{class},
\texttt{dm}, \texttt{var}, \texttt{rel}, \texttt{type} and \texttt{isa}.

 
\section*{\texttt{(extern 'sym) -> sym | NIL}}
\label{sec:func-ref-E-(extern 'sym) -> sym | NIL}

Creates or finds an external symbol. If a symbol with the name \texttt{sym} is
already extern, it is returned. Otherwise, a new external symbol is
returned. \texttt{NIL} is returned if \texttt{sym} does not exist in the database. See
also \texttt{intern} and \texttt{ext?}.

\begin{wideverbatim}
    : (extern "A1b")
    -> {A1b}
    : (extern "{A1b}")
    -> {A1b}
\end{wideverbatim}

 
\section*{\texttt{(extra ['any ..]) -> any}}
\label{sec:func-ref-E-(extra ['any ..]) -> any}


Can only be used inside methods. Sends the current message to the
current object \texttt{This}, this time starting the search for a method at the
remaining branches of the inheritance tree of the class where the
current method was found. See also \texttt{OO Concepts}, \texttt{super}, \texttt{method},
\texttt{meth}, \texttt{send} and \texttt{try}.


\begin{wideverbatim}
(dm key> (C)            # 'key>' method of the '+Uppc' class
   (uppc (extra C)) )   # Convert 'key>' of extra classes to upper case
\end{wideverbatim}

 
\section*{\texttt{(extract 'fun 'lst ..) -> lst}}
\label{sec:func-ref-E-(extract 'fun 'lst ..) -> lst}

Applies \texttt{fun} to each element of \texttt{lst}. When additional \texttt{lst} arguments
are given, their elements are also passed to \texttt{fun}. Returns a list of
all non-\texttt{NIL} values returned by \texttt{fun}. \texttt{(extract 'fun 'lst)} is
equivalent to \texttt{(mapcar 'fun (filter 'fun 'lst))} or, for non-NIL
results, to \texttt{(mapcan '((X) (and (fun X) (cons @))) 'lst)}. See also
\texttt{filter}, \texttt{find}, \texttt{pick} and \texttt{mapcan}.


\begin{wideverbatim}
: (setq A NIL  B 1  C NIL  D 2  E NIL  F 3)
-> 3
: (filter val '(A B C D E F))
-> (B D F)
: (extract val '(A B C D E F))
-> (1 2 3)
\end{wideverbatim}




% %%%%%%%%%%%%%%%%%%%%%%%% referenc.tex %%%%%%%%%%%%%%%%%%%%%%%%%%%%%%
% sample references
% %
% Use this file as a template for your own input.
%
%%%%%%%%%%%%%%%%%%%%%%%% Springer-Verlag %%%%%%%%%%%%%%%%%%%%%%%%%%
%
% BibTeX users please use
% \bibliographystyle{}
% \bibliography{}
%
\biblstarthook{In view of the parallel print and (chapter-wise) online publication of your book at \url{www.springerlink.com} it has been decided that -- as a genreral rule --  references should be sorted chapter-wise and placed at the end of the individual chapters. However, upon agreement with your contact at Springer you may list your references in a single seperate chapter at the end of your book. Deactivate the class option \texttt{sectrefs} and the \texttt{thebibliography} environment will be put out as a chapter of its own.\\\indent
References may be \textit{cited} in the text either by number (preferred) or by author/year.\footnote{Make sure that all references from the list are cited in the text. Those not cited should be moved to a separate \textit{Further Reading} section or chapter.} The reference list should ideally be \textit{sorted} in alphabetical order -- even if reference numbers are used for the their citation in the text. If there are several works by the same author, the following order should be used: 
\begin{enumerate}
\item all works by the author alone, ordered chronologically by year of publication
\item all works by the author with a coauthor, ordered alphabetically by coauthor
\item all works by the author with several coauthors, ordered chronologically by year of publication.
\end{enumerate}
The \textit{styling} of references\footnote{Always use the standard abbreviation of a journal's name according to the ISSN \textit{List of Title Word Abbreviations}, see \url{http://www.issn.org/en/node/344}} depends on the subject of your book:
\begin{itemize}
\item The \textit{two} recommended styles for references in books on \textit{mathematical, physical, statistical and computer sciences} are depicted in ~\cite{science-contrib, science-online, science-mono, science-journal, science-DOI} and ~\cite{phys-online, phys-mono, phys-journal, phys-DOI, phys-contrib}.
\item Examples of the most commonly used reference style in books on \textit{Psychology, Social Sciences} are~\cite{psysoc-mono, psysoc-online,psysoc-journal, psysoc-contrib, psysoc-DOI}.
\item Examples for references in books on \textit{Humanities, Linguistics, Philosophy} are~\cite{humlinphil-journal, humlinphil-contrib, humlinphil-mono, humlinphil-online, humlinphil-DOI}.
\item Examples of the basic Springer style used in publications on a wide range of subjects such as \textit{Computer Science, Economics, Engineering, Geosciences, Life Sciences, Medicine, Biomedicine} are ~\cite{basic-contrib, basic-online, basic-journal, basic-DOI, basic-mono}. 
\end{itemize}
}

\begin{thebibliography}{99.}%
% and use \bibitem to create references.
%
% Use the following syntax and markup for your references if 
% the subject of your book is from the field 
% "Mathematics, Physics, Statistics, Computer Science"
%
% Contribution 
\bibitem{science-contrib} Broy, M.: Software engineering --- from auxiliary to key technologies. In: Broy, M., Dener, E. (eds.) Software Pioneers, pp. 10-13. Springer, Heidelberg (2002)
%
% Online Document
\bibitem{science-online} Dod, J.: Effective substances. In: The Dictionary of Substances and Their Effects. Royal Society of Chemistry (1999) Available via DIALOG. \\
\url{http://www.rsc.org/dose/title of subordinate document. Cited 15 Jan 1999}
%
% Monograph
\bibitem{science-mono} Geddes, K.O., Czapor, S.R., Labahn, G.: Algorithms for Computer Algebra. Kluwer, Boston (1992) 
%
% Journal article
\bibitem{science-journal} Hamburger, C.: Quasimonotonicity, regularity and duality for nonlinear systems of partial differential equations. Ann. Mat. Pura. Appl. \textbf{169}, 321--354 (1995)
%
% Journal article by DOI
\bibitem{science-DOI} Slifka, M.K., Whitton, J.L.: Clinical implications of dysregulated cytokine production. J. Mol. Med. (2000) doi: 10.1007/s001090000086 
%
\bigskip

% Use the following (APS) syntax and markup for your references if 
% the subject of your book is from the field 
% "Mathematics, Physics, Statistics, Computer Science"
%
% Online Document
\bibitem{phys-online} J. Dod, in \textit{The Dictionary of Substances and Their Effects}, Royal Society of Chemistry. (Available via DIALOG, 1999), 
\url{http://www.rsc.org/dose/title of subordinate document. Cited 15 Jan 1999}
%
% Monograph
\bibitem{phys-mono} H. Ibach, H. L\"uth, \textit{Solid-State Physics}, 2nd edn. (Springer, New York, 1996), pp. 45-56 
%
% Journal article
\bibitem{phys-journal} S. Preuss, A. Demchuk Jr., M. Stuke, Appl. Phys. A \textbf{61}
%
% Journal article by DOI
\bibitem{phys-DOI} M.K. Slifka, J.L. Whitton, J. Mol. Med., doi: 10.1007/s001090000086
%
% Contribution 
\bibitem{phys-contrib} S.E. Smith, in \textit{Neuromuscular Junction}, ed. by E. Zaimis. Handbook of Experimental Pharmacology, vol 42 (Springer, Heidelberg, 1976), p. 593
%
\bigskip
%
% Use the following syntax and markup for your references if 
% the subject of your book is from the field 
% "Psychology, Social Sciences"
%
%
% Monograph
\bibitem{psysoc-mono} Calfee, R.~C., \& Valencia, R.~R. (1991). \textit{APA guide to preparing manuscripts for journal publication.} Washington, DC: American Psychological Association.
%
% Online Document
\bibitem{psysoc-online} Dod, J. (1999). Effective substances. In: The dictionary of substances and their effects. Royal Society of Chemistry. Available via DIALOG. \\
\url{http://www.rsc.org/dose/Effective substances.} Cited 15 Jan 1999.
%
% Journal article
\bibitem{psysoc-journal} Harris, M., Karper, E., Stacks, G., Hoffman, D., DeNiro, R., Cruz, P., et al. (2001). Writing labs and the Hollywood connection. \textit{J Film} Writing, 44(3), 213--245.
%
% Contribution 
\bibitem{psysoc-contrib} O'Neil, J.~M., \& Egan, J. (1992). Men's and women's gender role journeys: Metaphor for healing, transition, and transformation. In B.~R. Wainrig (Ed.), \textit{Gender issues across the life cycle} (pp. 107--123). New York: Springer.
%
% Journal article by DOI
\bibitem{psysoc-DOI}Kreger, M., Brindis, C.D., Manuel, D.M., Sassoubre, L. (2007). Lessons learned in systems change initiatives: benchmarks and indicators. \textit{American Journal of Community Psychology}, doi: 10.1007/s10464-007-9108-14.
%
%
% Use the following syntax and markup for your references if 
% the subject of your book is from the field 
% "Humanities, Linguistics, Philosophy"
%
\bigskip
%
% Journal article
\bibitem{humlinphil-journal} Alber John, Daniel C. O'Connell, and Sabine Kowal. 2002. Personal perspective in TV interviews. \textit{Pragmatics} 12:257--271
%
% Contribution 
\bibitem{humlinphil-contrib} Cameron, Deborah. 1997. Theoretical debates in feminist linguistics: Questions of sex and gender. In \textit{Gender and discourse}, ed. Ruth Wodak, 99--119. London: Sage Publications.
%
% Monograph
\bibitem{humlinphil-mono} Cameron, Deborah. 1985. \textit{Feminism and linguistic theory.} New York: St. Martin's Press.
%
% Online Document
\bibitem{humlinphil-online} Dod, Jake. 1999. Effective substances. In: The dictionary of substances and their effects. Royal Society of Chemistry. Available via DIALOG. \\
http://www.rsc.org/dose/title of subordinate document. Cited 15 Jan 1999
%
% Journal article by DOI
\bibitem{humlinphil-DOI} Suleiman, Camelia, Daniel C. O�Connell, and Sabine Kowal. 2002. `If you and I, if we, in this later day, lose that sacred fire...�': Perspective in political interviews. \textit{Journal of Psycholinguistic Research}. doi: 10.1023/A:1015592129296.
%
%
%
\bigskip
%
%
% Use the following syntax and markup for your references if 
% the subject of your book is from the field 
% "Computer Science, Economics, Engineering, Geosciences, Life Sciences"
%
%
% Contribution 
\bibitem{basic-contrib} Brown B, Aaron M (2001) The politics of nature. In: Smith J (ed) The rise of modern genomics, 3rd edn. Wiley, New York 
%
% Online Document
\bibitem{basic-online} Dod J (1999) Effective Substances. In: The dictionary of substances and their effects. Royal Society of Chemistry. Available via DIALOG. \\
\url{http://www.rsc.org/dose/title of subordinate document. Cited 15 Jan 1999}
%
% Journal article by DOI
\bibitem{basic-DOI} Slifka MK, Whitton JL (2000) Clinical implications of dysregulated cytokine production. J Mol Med, doi: 10.1007/s001090000086
%
% Journal article
\bibitem{basic-journal} Smith J, Jones M Jr, Houghton L et al (1999) Future of health insurance. N Engl J Med 965:325--329
%
% Monograph
\bibitem{basic-mono} South J, Blass B (2001) The future of modern genomics. Blackwell, London 
%
\end{thebibliography}

