%%%%%%%%%%%%%%%%%%%%% chapter.tex %%%%%%%%%%%%%%%%%%%%%%%%%%%%%%%%%
%
% sample chapter
%
% Use this file as a template for your own input.
%
%%%%%%%%%%%%%%%%%%%%%%%% Springer-Verlag %%%%%%%%%%%%%%%%%%%%%%%%%%
%\motto{Use the template \emph{chapter.tex} to style the various elements of your chapter content.}



\chapter{Symbols starting with F}
\label{sec:func-ref-F-functions-starting-with-F}


 
\section*{\texttt{*Fork}}
\label{sec:func-ref-F-*Fork}


A global variable holding a (possibly empty) \texttt{prg} body, to be executed
after a call to \texttt{fork} in the child process.


\begin{wideverbatim}
: (push '*Fork '(off *Tmp))   # Clear '*Tmp' in child process
-> (off *Tmp)
\end{wideverbatim}

 
\section*{\texttt{+Fold}}
\label{sec:func-ref-F-+Fold}


Prefix class for maintaining \texttt{fold}ed indexes to
\texttt{+String} relations. Typically used in combination with the
\texttt{+Ref} or \texttt{+Idx} prefix classes. See also
\texttt{Database}.


\begin{wideverbatim}
(rel nm (+Fold +Idx +String))   # Item Description
...
(rel tel (+Fold +Ref +String))  # Phone number
\end{wideverbatim}

 
\section*{\texttt{(fail) -> lst}}
\label{sec:func-ref-F-(fail) -> lst}


Constructs an empty \emph{Pilog} query, i.e. a query that
will aways fail. See also \texttt{goal}.


\begin{wideverbatim}
(dm clr> ()                # Clear query chart in search dialogs
   (query> This (fail)) )
\end{wideverbatim}

 
\section*{\texttt{fail/0}}
\label{sec:func-ref-F-fail/0}


\emph{Pilog} predicate that always fails. See also
\texttt{true/0}.


\begin{wideverbatim}
:  (? (fail))
-> NIL
\end{wideverbatim}

 
\section*{\texttt{(fetch 'tree 'any) -> any}}
\label{sec:func-ref-F-(fetch 'tree 'any) -> any}


Fetches a value for the key \texttt{any} from a database tree. See also \texttt{tree}
and \texttt{store}.


\begin{wideverbatim}
: (fetch (tree 'nr '+Item) 2)
-> {3-2}
\end{wideverbatim}

 
\section*{\texttt{(fifo 'var ['any ..]) -> any}}
\label{sec:func-ref-F-(fifo 'var ['any ..]) -> any}


Implements a first-in-first-out structure using a circular list. When
called with \texttt{any} arguments, they will be concatenated to end of the
structure. Otherwise, the first element is removed from the structure
and returned. See also \texttt{queue}, \texttt{push}, \texttt{pop}, \texttt{rot} and \texttt{circ}.


\begin{wideverbatim}
: (fifo 'X 1)
-> 1
: (fifo 'X 2 3)
-> 3
: X
-> (3 1 2 .)
: (fifo 'X)
-> 1
: (fifo 'X)
-> 2
: X
-> (3 .)
\end{wideverbatim}

 
\section*{\texttt{(file) -> (sym1 sym2 . num) | NIL}}
\label{sec:func-ref-F-(file) -> (sym1 sym2 . num) | NIL}


Returns for the current input channel the path name \texttt{sym1}, the file
name \texttt{sym2}, and the current line number \texttt{num}. If the current input
channel is not a file, \texttt{NIL} is returned. See also \texttt{info}, \texttt{in} and
\texttt{load}.


\begin{wideverbatim}
: (load (pack (car (file)) "localFile.l"))  # Load a file in same directory
\end{wideverbatim}

 
\section*{\texttt{(fill 'any ['sym|lst]) -> any}}
\label{sec:func-ref-F-(fill 'any ['sym|lst]) -> any}


Fills a pattern \texttt{any}, by substituting \texttt{sym}, or all symbols in \texttt{lst},
or - if no second argument is given - each pattern symbol in \texttt{any} (see
\texttt{pat?}), with its current value. \texttt{@} itself is not considered a pattern
symbol here. In any case, expressions following the symbol \texttt{\textasciicircum{}} should
evaluate to lists which are then (destructively) spliced into the
result. See also \texttt{match}.


\begin{wideverbatim}
: (setq  @X 1234  @Y (1 2 3 4))
-> (1 2 3 4)
: (fill '@X)
-> 1234
: (fill '(a b (c @X) ((@Y . d) e)))
-> (a b (c 1234) (((1 2 3 4) . d) e))
: (let X 2 (fill (1 X 3) 'X))
-> (1 2 3)

: (fill (1 ^ (list 'a 'b 'c) 9))
-> (1 a b c 9)

: (match '(This is @X) '(This is a pen))
-> T
: (fill '(Got ^ @X))
-> (Got a pen)
\end{wideverbatim}

 
\section*{\texttt{(filter 'fun 'lst ..) -> lst}}
\label{sec:func-ref-F-(filter 'fun 'lst ..) -> lst}


Applies \texttt{fun} to each element of \texttt{lst}. When additional \texttt{lst} arguments
are given, their elements are also passed to \texttt{fun}. Returns a list of
all elements of \texttt{lst} where \texttt{fun} returned non-\texttt{NIL}. See also \texttt{fish},
\texttt{find}, \texttt{pick} and \texttt{extract}.


\begin{wideverbatim}
: (filter num? (1 A 2 (B) 3 CDE))
-> (1 2 3)
\end{wideverbatim}

 
\section*{\texttt{(fin 'any) -> num|sym}}
\label{sec:func-ref-F-(fin 'any) -> num|sym}


Returns \texttt{any} if it is an atom, otherwise the CDR of its last cell. See
also \texttt{last} and \texttt{tail}.


\begin{wideverbatim}
: (fin 'a)
-> a
: (fin '(a . b))
-> b
: (fin '(a b . c))
-> c
: (fin '(a b c))
-> NIL
\end{wideverbatim}

 
\section*{\texttt{(finally exe . prg) -> any}}
\label{sec:func-ref-F-(finally exe . prg) -> any}


\texttt{prg} is executed, then \texttt{exe} is evaluated, and the result of \texttt{prg} is
returned. \texttt{exe} will also be evaluated if \texttt{prg} does not terminate
normally due to a runtime error or a call to \texttt{throw}. See also \texttt{bye},
\texttt{catch}, \texttt{quit} and \texttt{Error Handling}.


\begin{wideverbatim}
: (finally (prinl "Done!")
   (println 123)
   (quit)
   (println 456) )
123
Done!
: (catch 'A
   (finally (prinl "Done!")
      (println 1)
      (throw 'A 123)
      (println 2) ) )
1
Done!
-> 123
\end{wideverbatim}

 
\section*{\texttt{(find 'fun 'lst ..) -> any}}
\label{sec:func-ref-F-(find 'fun 'lst ..) -> any}


Applies \texttt{fun} to successive elements of \texttt{lst} until non-\texttt{NIL} is
returned. Returns that element, or \texttt{NIL} if \texttt{fun} did not return
non-\texttt{NIL} for any element of \texttt{lst}. When additional \texttt{lst} arguments are
given, their elements are also passed to \texttt{fun}. See also \texttt{seek}, \texttt{pick}
and \texttt{filter}.


\begin{wideverbatim}
: (find pair (1 A 2 (B) 3 CDE))
-> (B)
: (find '((A B) (> A B)) (1 2 3 4 5 6) (6 5 4 3 2 1))
-> 4
: (find > (1 2 3 4 5 6) (6 5 4 3 2 1))  # shorter
-> 4
\end{wideverbatim}

 
\section*{\texttt{(fish 'fun 'any) -> lst}}
\label{sec:func-ref-F-(fish 'fun 'any) -> lst}


Applies \texttt{fun} to each element - and recursively to all sublists - of
\texttt{any}. Returns a list of all items where \texttt{fun} returned non-\texttt{NIL}. See
also \texttt{filter}.


\begin{wideverbatim}
: (fish gt0 '(a -2 (1 b (-3 c 2)) 3 d -1))
-> (1 2 3)
: (fish sym? '(a -2 (1 b (-3 c 2)) 3 d -1))
-> (a b c d)
\end{wideverbatim}

 
\section*{\texttt{(flg? 'any) -> flg}}
\label{sec:func-ref-F-(flg? 'any) -> flg}


Returns \texttt{T} when the argument \texttt{any} is either \texttt{NIL} or \texttt{T}. See also
\texttt{bool}. \texttt{(flg? X)} is equivalent to \texttt{(or (not X) (=T X))}.


\begin{wideverbatim}
: (flg? (= 3 3))
-> T
: (flg? (= 3 4))
-> T
: (flg? (+ 3 4))
-> NIL
\end{wideverbatim}

 
\section*{\texttt{(flip 'lst ['cnt]) -> lst}}
\label{sec:func-ref-F-(flip 'lst ['cnt]) -> lst}


Returns \texttt{lst} (destructively) reversed. Without the optional \texttt{cnt}
argument, the whole list is flipped, otherwise only the first \texttt{cnt}
elements. See also \texttt{reverse} and \texttt{rot}.


\begin{wideverbatim}
: (flip (1 2 3 4))         # Flip all  four elements
-> (4 3 2 1)
: (flip (1 2 3 4 5 6) 3)   # Flip only the first three elements
-> (3 2 1 4 5 6)
\end{wideverbatim}

 
\section*{\texttt{(flush) -> flg}}
\label{sec:func-ref-F-(flush) -> flg}


Flushes the current output stream by writing all buffered data. A call
to \texttt{flush} for standard output is done automatically before a call to
\texttt{key}. Returns \texttt{T} when successful. See also \texttt{rewind}.


\begin{wideverbatim}
: (flush)
-> T
\end{wideverbatim}

 
\section*{\texttt{(fmt64 'num) -> sym}}
\label{sec:func-ref-F-(fmt64 'num) -> sym}


\texttt{(fmt64 'sym) -> num}

Converts a number \texttt{num} to a string in base--64 notation, or a base--64
formatted string to a number. The digits are represented with the
characters \texttt{0} - \texttt{9}, \texttt{:}, \texttt{;}, \texttt{A} - \texttt{Z} and \texttt{a} - \texttt{z}. This format is
used internally for the names of \texttt{external symbols} in the 32-bit
version. See also \texttt{hax}, \texttt{hex}, \texttt{bin} and \texttt{oct}.


\begin{wideverbatim}
: (fmt64 9)
-> "9"
: (fmt64 10)
-> ":"
: (fmt64 11)
-> ";"
: (fmt64 12)
-> "A"
: (fmt64 "100")
-> 4096
\end{wideverbatim}

 
\section*{\texttt{(fold 'any ['cnt]) -> sym}}
\label{sec:func-ref-F-(fold 'any ['cnt]) -> sym}


Folding to a canonical form: If \texttt{any} is not a symbol, \texttt{NIL} is
returned. Otherwise, a new transient symbol with all digits and all
letters of \texttt{any}, converted to lower case, is returned. If the \texttt{cnt}
argument is given, the result is truncated to that length (or not
truncated if \texttt{cnt} is zero). Otherwise \texttt{cnt} defaults to 24. See also
\texttt{lowc}.


\begin{wideverbatim}
: (fold " 1A 2-b/3")
-> "1a2b3"
: (fold " 1A 2-B/3" 3)
-> "1a2"
\end{wideverbatim}

 
\section*{\texttt{fold/3}}
\label{sec:func-ref-F-fold/3}


\emph{Pilog} predicate that succeeds if the first
argument, after \texttt{fold}ing it to a canonical form, is a /prefix/
of the folded string representation of the result of applying the
\texttt{get} algorithm to the following arguments. Typically used as
filter predicate in \texttt{select/3} database queries. See also
\texttt{pre?}, \texttt{isa/2}, \texttt{same/3}, \texttt{bool/3},
\texttt{range/3}, \texttt{head/3}, \texttt{part/3} and
\texttt{tolr/3}.


\begin{wideverbatim}
: (?
   @Nr (1 . 5)
   @Nm "main"
   (select (@Item)
      ((nr +Item @Nr) (nm +Item @Nm))
      (range @Nr @Item nr)
      (fold @Nm @Item nm) ) )
 @Nr=(1 . 5) @Nm="main" @Item={3-1}
-> NIL
\end{wideverbatim}

 
\section*{\texttt{(for sym 'num ['any | (NIL 'any . prg) | (T 'any . prg) ..]) -> any}}
\label{sec:func-ref-F-(for sym 'num ['any | (NIL 'any . prg) | (T 'any . prg) ..]) -> any}


\texttt{(for sym|(sym2 . sym) 'lst ['any | (NIL 'any . prg) | (T 'any . prg) ..]) -> any}

\texttt{(for (sym|(sym2 . sym) 'any1 'any2 [. prg]) ['any | (NIL 'any . prg) | (T 'any . prg) ..]) -> any}

Conditional loop with local variable(s) and multiple conditional exits:
In the first form, the value of \texttt{sym} is saved, \texttt{sym} is bound to \texttt{1},
and the body is executed with increasing values up to (and including)
\texttt{num}. In the second form, the value of \texttt{sym} is saved, \texttt{sym} is
subsequently bound to the elements of \texttt{lst}, and the body is executed
each time. In the third form, the value of \texttt{sym} is saved, and \texttt{sym} is
bound to \texttt{any1}. If \texttt{sym2} is given, it is treated as a counter
variable, first bound to 1 and then incremented for each execution of
the body. While the condition \texttt{any2} evaluates to non-\texttt{NIL}, the body is
repeatedly executed and, if \texttt{prg} is given, \texttt{sym} is re-bound to the
result of its evaluation. If a clause has \texttt{NIL} or \texttt{T} as its CAR, the
clause's second element is evaluated as a condition and - if the result
is \texttt{NIL} or non-\texttt{NIL}, respectively - the \texttt{prg} is executed and the
result returned. If the body is never executed, \texttt{NIL} is returned. See
also \texttt{do} and \texttt{loop}.


\begin{wideverbatim}
: (for (N 1 (>= 8 N) (inc N)) (printsp N))
1 2 3 4 5 6 7 8 -> 8
: (for (L (1 2 3 4 5 6 7 8) L) (printsp (pop 'L)))
1 2 3 4 5 6 7 8 -> 8
: (for X (1 a 2 b) (printsp X))
1 a 2 b -> b
: (for ((I . L) '(a b c d e f) L (cddr L)) (println I L))
1 (a b c d e f)
2 (c d e f)
3 (e f)
-> (e f)
: (for (I . X) '(a b c d e f) (println I X))
1 a
2 b
3 c
4 d
5 e
6 f
-> f
\end{wideverbatim}

 
\section*{\texttt{(fork) -> pid | NIL}}
\label{sec:func-ref-F-(fork) -> pid | NIL}


Forks a child process. Returns \texttt{NIL} in the child, and the child's
process ID \texttt{pid} in the parent. In the child, the \texttt{VAL} of the global
variable \texttt{*Fork} (should be a \texttt{prg}) is executed. See also \texttt{pipe} and
\texttt{tell}.


\begin{wideverbatim}
: (unless (fork) (do 5 (println 'OK) (wait 1000)) (bye))
-> NIL
OK                                              # Child's output
: OK
OK
OK
OK
\end{wideverbatim}

 
\section*{\texttt{(forked)}}
\label{sec:func-ref-F-(forked)}


Installs maintenance code in \texttt{*Fork} to close server sockets and clean
up \texttt{*Run} code in child processes. Should only be called immediately
after \texttt{task}.


\begin{wideverbatim}
: (task -60000 60000 (msg 'OK))     # Install timer task
-> (-60000 60000 (msg 'OK))
: (forked)                          # No timer in child processes
-> (task -60000)
: *Run
-> ((-60000 56432 (msg 'OK)))
: *Fork
-> ((task -60000) (del '(saveHistory) '*Bye))
\end{wideverbatim}

 
\section*{\texttt{(format 'num ['cnt ['sym1 ['sym2]]]) -> sym}}
\label{sec:func-ref-F-(format 'num ['cnt ['sym1 ['sym2]]]) -> sym}


\texttt{(format 'sym|lst ['cnt ['sym1 ['sym2]]]) -> num}

Converts a number \texttt{num} to a string, or a string \texttt{sym|lst} to a number.
In both cases, optionally a precision \texttt{cnt}, a decimal-separator \texttt{sym1}
and a thousands-separator \texttt{sym2} can be supplied. Returns \texttt{NIL} if the
conversion is unsuccessful. See also \emph{Numbers} and
\texttt{round}.


\begin{wideverbatim}
: (format 123456789)                   # Integer conversion
-> "123456789"
: (format 123456789 2)                 # Fixed point
-> "1234567.89"
: (format 123456789 2 ",")             # Comma as decimal-separator
-> "1234567,89"
: (format 123456789 2 "," ".")         # and period as thousands-separator
-> "1.234.567,89"

: (format "123456789")                 # String to number
-> 123456789
: (format (1 "23" (4 5 6)))
-> 123456
: (format "1234567.89" 4)              # scaled to four digits
-> 12345678900
: (format "1.234.567,89")              # separators not recognized
-> NIL
: (format "1234567,89" 4 ",")
-> 12345678900
: (format "1.234.567,89" 4 ",")        # thousands-separator not recognized
-> NIL
: (format "1.234.567,89" 4 "," ".")
-> 12345678900
\end{wideverbatim}

 
\section*{\texttt{(free 'cnt) -> (sym . lst)}}
\label{sec:func-ref-F-(free 'cnt) -> (sym . lst)}


Returns, for the \texttt{cnt}'th database file, the next available symbol \texttt{sym}
(i.e. the first symbol greater than any symbol in the database), and the
list \texttt{lst} of free symbols. See also \texttt{seq}, \texttt{zap} and \texttt{dbck}.


\begin{wideverbatim}
: (pool "x")      # A new database
-> T
: (new T)         # Create a new symbol
-> {2}
: (new T)         # Create another symbol
-> {3}
: (commit)        # Commit changes
-> T
: (zap '{2})      # Delete the first symbol
-> {2}
: (free 1)        # Show free list
-> ({4})          # {3} was the last symbol allocated
: (commit)        # Commit the deletion of {2}
-> T
: (free 1)        # Now {2} is in the free list
-> ({4} {2})
\end{wideverbatim}

 
\section*{\texttt{(from 'any ..) -> sym}}
\label{sec:func-ref-F-(from 'any ..) -> sym}


Skips the current input channel until one of the strings \texttt{any} is found,
and starts subsequent reading from that point. The found \texttt{any} argument,
or \texttt{NIL} (if none is found) is returned. See also \texttt{till} and \texttt{echo}.


\begin{wideverbatim}
: (and (from "val='") (till "'" T))
test val='abc'
-> "abc"
\end{wideverbatim}

 
\section*{\texttt{(full 'any) -> bool}}
\label{sec:func-ref-F-(full 'any) -> bool}


Returns \texttt{NIL} if \texttt{any} is a non-empty list with at least one \texttt{NIL}
element, otherwise \texttt{T}. \texttt{(full X)} is equivalent to
\texttt{(not (memq NIL X))}.


\begin{wideverbatim}
: (full (1 2 3))
-> T
: (full (1 NIL 3))
-> NIL
: (full 123)
-> T
\end{wideverbatim}

 
\section*{\texttt{(fun? 'any) -> any}}
\label{sec:func-ref-F-(fun? 'any) -> any}


Returns \texttt{NIL} when the argument \texttt{any} is neither a number suitable for a
code-pointer, nor a list suitable for a lambda expression (function).
Otherwise a number is returned for a code-pointer, \texttt{T} for a function
without arguments, and a single formal parameter or a list of formal
parameters for a function. See also \texttt{getd}.


\begin{wideverbatim}
: (fun? 1000000000)              # Might be a code pointer
-> 1000000000
: (fun? 100000000000000)         # Too big for a code pointer
-> NIL
: (fun? 1000000001)              # Cannot be a code pointer (odd)
-> NIL
: (fun? '((A B) (* A B)))        # Lambda expression
-> (A B)
: (fun? '((A B) (* A B) . C))    # Not a lambda expression
-> NIL
: (fun? '(1 2 3 4))              # Not a lambda expression
-> NIL
: (fun? '((A 2 B) (* A B)))      # Not a lambda expression
-> NIL
\end{wideverbatim}



% %%%%%%%%%%%%%%%%%%%%%%%% referenc.tex %%%%%%%%%%%%%%%%%%%%%%%%%%%%%%
% sample references
% %
% Use this file as a template for your own input.
%
%%%%%%%%%%%%%%%%%%%%%%%% Springer-Verlag %%%%%%%%%%%%%%%%%%%%%%%%%%
%
% BibTeX users please use
% \bibliographystyle{}
% \bibliography{}
%
\biblstarthook{In view of the parallel print and (chapter-wise) online publication of your book at \url{www.springerlink.com} it has been decided that -- as a genreral rule --  references should be sorted chapter-wise and placed at the end of the individual chapters. However, upon agreement with your contact at Springer you may list your references in a single seperate chapter at the end of your book. Deactivate the class option \texttt{sectrefs} and the \texttt{thebibliography} environment will be put out as a chapter of its own.\\\indent
References may be \textit{cited} in the text either by number (preferred) or by author/year.\footnote{Make sure that all references from the list are cited in the text. Those not cited should be moved to a separate \textit{Further Reading} section or chapter.} The reference list should ideally be \textit{sorted} in alphabetical order -- even if reference numbers are used for the their citation in the text. If there are several works by the same author, the following order should be used: 
\begin{enumerate}
\item all works by the author alone, ordered chronologically by year of publication
\item all works by the author with a coauthor, ordered alphabetically by coauthor
\item all works by the author with several coauthors, ordered chronologically by year of publication.
\end{enumerate}
The \textit{styling} of references\footnote{Always use the standard abbreviation of a journal's name according to the ISSN \textit{List of Title Word Abbreviations}, see \url{http://www.issn.org/en/node/344}} depends on the subject of your book:
\begin{itemize}
\item The \textit{two} recommended styles for references in books on \textit{mathematical, physical, statistical and computer sciences} are depicted in ~\cite{science-contrib, science-online, science-mono, science-journal, science-DOI} and ~\cite{phys-online, phys-mono, phys-journal, phys-DOI, phys-contrib}.
\item Examples of the most commonly used reference style in books on \textit{Psychology, Social Sciences} are~\cite{psysoc-mono, psysoc-online,psysoc-journal, psysoc-contrib, psysoc-DOI}.
\item Examples for references in books on \textit{Humanities, Linguistics, Philosophy} are~\cite{humlinphil-journal, humlinphil-contrib, humlinphil-mono, humlinphil-online, humlinphil-DOI}.
\item Examples of the basic Springer style used in publications on a wide range of subjects such as \textit{Computer Science, Economics, Engineering, Geosciences, Life Sciences, Medicine, Biomedicine} are ~\cite{basic-contrib, basic-online, basic-journal, basic-DOI, basic-mono}. 
\end{itemize}
}

\begin{thebibliography}{99.}%
% and use \bibitem to create references.
%
% Use the following syntax and markup for your references if 
% the subject of your book is from the field 
% "Mathematics, Physics, Statistics, Computer Science"
%
% Contribution 
\bibitem{science-contrib} Broy, M.: Software engineering --- from auxiliary to key technologies. In: Broy, M., Dener, E. (eds.) Software Pioneers, pp. 10-13. Springer, Heidelberg (2002)
%
% Online Document
\bibitem{science-online} Dod, J.: Effective substances. In: The Dictionary of Substances and Their Effects. Royal Society of Chemistry (1999) Available via DIALOG. \\
\url{http://www.rsc.org/dose/title of subordinate document. Cited 15 Jan 1999}
%
% Monograph
\bibitem{science-mono} Geddes, K.O., Czapor, S.R., Labahn, G.: Algorithms for Computer Algebra. Kluwer, Boston (1992) 
%
% Journal article
\bibitem{science-journal} Hamburger, C.: Quasimonotonicity, regularity and duality for nonlinear systems of partial differential equations. Ann. Mat. Pura. Appl. \textbf{169}, 321--354 (1995)
%
% Journal article by DOI
\bibitem{science-DOI} Slifka, M.K., Whitton, J.L.: Clinical implications of dysregulated cytokine production. J. Mol. Med. (2000) doi: 10.1007/s001090000086 
%
\bigskip

% Use the following (APS) syntax and markup for your references if 
% the subject of your book is from the field 
% "Mathematics, Physics, Statistics, Computer Science"
%
% Online Document
\bibitem{phys-online} J. Dod, in \textit{The Dictionary of Substances and Their Effects}, Royal Society of Chemistry. (Available via DIALOG, 1999), 
\url{http://www.rsc.org/dose/title of subordinate document. Cited 15 Jan 1999}
%
% Monograph
\bibitem{phys-mono} H. Ibach, H. L\"uth, \textit{Solid-State Physics}, 2nd edn. (Springer, New York, 1996), pp. 45-56 
%
% Journal article
\bibitem{phys-journal} S. Preuss, A. Demchuk Jr., M. Stuke, Appl. Phys. A \textbf{61}
%
% Journal article by DOI
\bibitem{phys-DOI} M.K. Slifka, J.L. Whitton, J. Mol. Med., doi: 10.1007/s001090000086
%
% Contribution 
\bibitem{phys-contrib} S.E. Smith, in \textit{Neuromuscular Junction}, ed. by E. Zaimis. Handbook of Experimental Pharmacology, vol 42 (Springer, Heidelberg, 1976), p. 593
%
\bigskip
%
% Use the following syntax and markup for your references if 
% the subject of your book is from the field 
% "Psychology, Social Sciences"
%
%
% Monograph
\bibitem{psysoc-mono} Calfee, R.~C., \& Valencia, R.~R. (1991). \textit{APA guide to preparing manuscripts for journal publication.} Washington, DC: American Psychological Association.
%
% Online Document
\bibitem{psysoc-online} Dod, J. (1999). Effective substances. In: The dictionary of substances and their effects. Royal Society of Chemistry. Available via DIALOG. \\
\url{http://www.rsc.org/dose/Effective substances.} Cited 15 Jan 1999.
%
% Journal article
\bibitem{psysoc-journal} Harris, M., Karper, E., Stacks, G., Hoffman, D., DeNiro, R., Cruz, P., et al. (2001). Writing labs and the Hollywood connection. \textit{J Film} Writing, 44(3), 213--245.
%
% Contribution 
\bibitem{psysoc-contrib} O'Neil, J.~M., \& Egan, J. (1992). Men's and women's gender role journeys: Metaphor for healing, transition, and transformation. In B.~R. Wainrig (Ed.), \textit{Gender issues across the life cycle} (pp. 107--123). New York: Springer.
%
% Journal article by DOI
\bibitem{psysoc-DOI}Kreger, M., Brindis, C.D., Manuel, D.M., Sassoubre, L. (2007). Lessons learned in systems change initiatives: benchmarks and indicators. \textit{American Journal of Community Psychology}, doi: 10.1007/s10464-007-9108-14.
%
%
% Use the following syntax and markup for your references if 
% the subject of your book is from the field 
% "Humanities, Linguistics, Philosophy"
%
\bigskip
%
% Journal article
\bibitem{humlinphil-journal} Alber John, Daniel C. O'Connell, and Sabine Kowal. 2002. Personal perspective in TV interviews. \textit{Pragmatics} 12:257--271
%
% Contribution 
\bibitem{humlinphil-contrib} Cameron, Deborah. 1997. Theoretical debates in feminist linguistics: Questions of sex and gender. In \textit{Gender and discourse}, ed. Ruth Wodak, 99--119. London: Sage Publications.
%
% Monograph
\bibitem{humlinphil-mono} Cameron, Deborah. 1985. \textit{Feminism and linguistic theory.} New York: St. Martin's Press.
%
% Online Document
\bibitem{humlinphil-online} Dod, Jake. 1999. Effective substances. In: The dictionary of substances and their effects. Royal Society of Chemistry. Available via DIALOG. \\
http://www.rsc.org/dose/title of subordinate document. Cited 15 Jan 1999
%
% Journal article by DOI
\bibitem{humlinphil-DOI} Suleiman, Camelia, Daniel C. O�Connell, and Sabine Kowal. 2002. `If you and I, if we, in this later day, lose that sacred fire...�': Perspective in political interviews. \textit{Journal of Psycholinguistic Research}. doi: 10.1023/A:1015592129296.
%
%
%
\bigskip
%
%
% Use the following syntax and markup for your references if 
% the subject of your book is from the field 
% "Computer Science, Economics, Engineering, Geosciences, Life Sciences"
%
%
% Contribution 
\bibitem{basic-contrib} Brown B, Aaron M (2001) The politics of nature. In: Smith J (ed) The rise of modern genomics, 3rd edn. Wiley, New York 
%
% Online Document
\bibitem{basic-online} Dod J (1999) Effective Substances. In: The dictionary of substances and their effects. Royal Society of Chemistry. Available via DIALOG. \\
\url{http://www.rsc.org/dose/title of subordinate document. Cited 15 Jan 1999}
%
% Journal article by DOI
\bibitem{basic-DOI} Slifka MK, Whitton JL (2000) Clinical implications of dysregulated cytokine production. J Mol Med, doi: 10.1007/s001090000086
%
% Journal article
\bibitem{basic-journal} Smith J, Jones M Jr, Houghton L et al (1999) Future of health insurance. N Engl J Med 965:325--329
%
% Monograph
\bibitem{basic-mono} South J, Blass B (2001) The future of modern genomics. Blackwell, London 
%
\end{thebibliography}

