%%%%%%%%%%%%%%%%%%%%% chapter.tex %%%%%%%%%%%%%%%%%%%%%%%%%%%%%%%%%
%
% sample chapter
%
% Use this file as a template for your own input.
%
%%%%%%%%%%%%%%%%%%%%%%%% Springer-Verlag %%%%%%%%%%%%%%%%%%%%%%%%%%
%\motto{Use the template \emph{chapter.tex} to style the various elements of your chapter content.}



\chapter{Symbols starting with K}
\labe{cha:func-ref-K-functions-starting-with-K}


 
\section*{\texttt{+Key}}
\label{sec:func-ref-K-+Key}


Prefix class for maintaining unique indexes to \texttt{+relation}s, a
subclass of \texttt{+index}. Accepts an optional argument for a
\texttt{+Hook} attribute. See also \texttt{Database}.


\begin{wideverbatim}
(rel nr (+Need +Key +Number))  # Mandatory, unique Customer/Supplier number
\end{wideverbatim}

 
\section*{\texttt{(key ['cnt]) -> sym}}
\label{sec:func-ref-K-(key ['cnt]) -> sym}


Returns the next character from standard input as a single-character
transient symbol. The console is set to raw mode. While waiting for a
key press, a \texttt{select} system call is executed for all file descriptors
and timers in the \texttt{VAL} of the global variable \texttt{*Run}. If \texttt{cnt} is
non-\texttt{NIL}, that amount of milliseconds is waited maximally, and \texttt{NIL} is
returned upon timeout. See also \texttt{raw} and \texttt{wait}.


\begin{wideverbatim}
: (key)           # Wait for a key
-> "a"            # 'a' pressed
\end{wideverbatim}

 
\section*{\texttt{(kill 'pid ['cnt]) -> flg}}
\label{sec:func-ref-K-(kill 'pid ['cnt]) -> flg}


Sends a signal with the signal number \texttt{cnt} (or SIGTERM if \texttt{cnt} is not
given) to the process with the ID \texttt{pid}. Returns \texttt{T} if successful.


\begin{wideverbatim}
: (kill *Pid 20)                                # Stop current process

[2]+  Stopped               pil +               # Unix shell
$ fg                                            # Job control: Foreground
pil +
-> T                                            # 'kill' was successful
\end{wideverbatim}


