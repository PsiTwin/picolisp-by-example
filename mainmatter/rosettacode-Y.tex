%%%%%%%%%%%%%%%%%%%%% chapter.tex %%%%%%%%%%%%%%%%%%%%%%%%%%%%%%%%%
%
% sample chapter
%
% Use this file as a template for your own input.
%
%%%%%%%%%%%%%%%%%%%%%%%% Springer-Verlag %%%%%%%%%%%%%%%%%%%%%%%%%%
%\motto{Use the template \emph{chapter.tex} to style the various elements of your chapter content.}

\chapter{Rosetta Code Tasks starting with Y}

\section*{Y combinator}

In strict
\href{http://en.wikipedia.org/wiki/Functional\_programming}{functional
  programming} and the
\href{http://en.wikipedia.org/wiki/lambda\_calculus}{lambda calculus},
functions (lambda expressions) don't have state and are only allowed
to refer to arguments of enclosing functions. This rules out the usual
definition of a recursive function wherein a function is associated
with the state of a variable and this variable's state is used in the
body of the function.

The \href{http://mvanier.livejournal.com/2897.html}{Y combinator} is
itself a stateless function that, when applied to another stateless
function, returns a recursive version of the function. The Y combinator
is the simplest of the class of such functions, called
\href{http://en.wikipedia.org/wiki/Fixed-point\_combinator}{fixed-point
combinators}.

The task is to define the stateless Y combinator and use it to compute
\href{http://en.wikipedia.org/wiki/Factorial}{factorials} and
\href{http://en.wikipedia.org/wiki/Fibonacci\_number}{Fibonacci numbers}
from other stateless functions or lambda expressions.

\begin{description}
\item[Cf]
\end{description}

\begin{itemize}
\item
  \href{http://vimeo.com/45140590}{Jim Weirich: Adventures in Functional
  Programming}
\end{itemize}



\begin{wideverbatim}

(de Y (F)
   (let X (curry (F) (Y) (F (curry (Y) @ (pass (Y Y)))))
      (X X) ) )

# Factorial
(de fact (F)
   (curry (F) (N)
      (if (=0 N)
         1
         (* N (F (dec N))) ) ) )

: ((Y fact) 6)
-> 720

# Fibonacci
(de fibo (F)
   (curry (F) (N)
      (if (> 2 N)
         1
         (+ (F (dec N)) (F (- N 2))) ) ) )

: ((Y fibo) 22)
-> 28657

# Ackermann
(de ack (F)
   (curry (F) (X Y)
      (cond
         ((=0 X) (inc Y))
         ((=0 Y) (F (dec X) 1))
         (T (F (dec X) (F X (dec Y)))) ) ) )

: ((Y ack) 3 4)
-> 125

\end{wideverbatim}

\pagebreak{}
\section*{Yahoo! Search}

Create a class for searching Yahoo! results. It must implement a
\textbf{Next Page} method, and read URL, Title and Content from results.

\begin{wideverbatim}

(load "@lib/http.l")

(de yahoo (Query Page)
   (default Page 1)
   (client "search.yahoo.com" 80
      (pack
         "search?p=" (ht:Fmt Query)
         "\&b=" (inc (* 10 (dec Page))) )
      (make
         (while (from "<a class=\"yschttl spt\" href=\"")
            (link
               (make
                  (link (till "\"" T))       # Url
                  (from "<b>")
                  (link (till "<" T))        # Title
                  (from "class=\"abstr\"")
                  (from ">")
                  (link                      # Content
                     (pack
                        (make
                           (loop
                              (link (till "<" T))
                              (T (eof))
                              (T (= "</div" (till ">" T)))
                              (char) ) ) ) ) ) ) ) ) ) )

Output:

: (more (yahoo "test"))
("http://www.test.com/" "Test" "Offers practice online tests for many ...
("http://www.test.com/aboutus.htm" "Test" "Test.com has a successful ...
("http://en.wikipedia.org/wiki/Test" "Test" "YUI Test is a testing ...
("http://en.wikipedia.org/wiki/F-test" "test " "test n. A procedure for ...
...

\end{wideverbatim}

\pagebreak{}
\section*{Yin and yang}

Create a function that given a variable representing size, generates a
\href{http://en.wikipedia.org/wiki/File:Yin\_and\_Yang.svg}{Yin and
yang} also known as a
\href{http://en.wikipedia.org/wiki/Taijitu}{Taijitu} symbol scaled to
that size.

Generate and display the symbol generated for two different (small)
sizes.


\begin{wideverbatim}

(de circle (X Y C R)
   (>=
      (* R R)
      (+
         (* (setq X (/ X 2)) X)
         (* (dec 'Y C) Y) ) ) )

(de yinYang (R)
   (for Y (range (- R) R)
      (for X (range (- 0 R R) (+ R R))
         (prin
            (cond
               ((circle X Y (- (/ R 2)) (/ R 6))
                  "#" )
               ((circle X Y (/ R 2) (/ R 6))
                  "." )
               ((circle X Y (- (/ R 2)) (/ R 2))
                  "." )
               ((circle X Y (/ R 2) (/ R 2))
                  "#" )
               ((circle X Y 0 R)
                  (if (lt0 X) "." "#") )
               (T " ") ) ) )
      (prinl) ) )

\end{wideverbatim}

\begin{wideverbatim}


Test:

: (yinYang 18)
                                   ...
                         .....................##
                   .............................######
                 .................................######
             .......................................########
           ...........................................########
         ..........................###................##########
       ........................###########............############
       ........................###########............############
     ........................###############............############
   ............................###########............################
   ............................###########............################
   ................................###................################
 .....................................................##################
 ...................................................####################
 .................................................######################
 ...............................................########################
 .............................................##########################
......................................###################################
 ..........................#############################################
 ........................###############################################
 ......................#################################################
 ....................###################################################
 ..................#####################################################
   ................################...################################
   ................############...........############################
   ................############...........############################
     ............############...............########################
       ............############...........########################
       ............############...........########################
         ..........################...##########################
           ........###########################################
             ........#######################################
                 ......#################################
                   ......#############################
                         ..#####################
                                   ###

\end{wideverbatim}



% %%%%%%%%%%%%%%%%%%%%%%%% referenc.tex %%%%%%%%%%%%%%%%%%%%%%%%%%%%%%
% sample references
% %
% Use this file as a template for your own input.
%
%%%%%%%%%%%%%%%%%%%%%%%% Springer-Verlag %%%%%%%%%%%%%%%%%%%%%%%%%%
%
% BibTeX users please use
% \bibliographystyle{}
% \bibliography{}
%
\biblstarthook{In view of the parallel print and (chapter-wise) online publication of your book at \url{www.springerlink.com} it has been decided that -- as a genreral rule --  references should be sorted chapter-wise and placed at the end of the individual chapters. However, upon agreement with your contact at Springer you may list your references in a single seperate chapter at the end of your book. Deactivate the class option \texttt{sectrefs} and the \texttt{thebibliography} environment will be put out as a chapter of its own.\\\indent
References may be \textit{cited} in the text either by number (preferred) or by author/year.\footnote{Make sure that all references from the list are cited in the text. Those not cited should be moved to a separate \textit{Further Reading} section or chapter.} The reference list should ideally be \textit{sorted} in alphabetical order -- even if reference numbers are used for the their citation in the text. If there are several works by the same author, the following order should be used: 
\begin{enumerate}
\item all works by the author alone, ordered chronologically by year of publication
\item all works by the author with a coauthor, ordered alphabetically by coauthor
\item all works by the author with several coauthors, ordered chronologically by year of publication.
\end{enumerate}
The \textit{styling} of references\footnote{Always use the standard abbreviation of a journal's name according to the ISSN \textit{List of Title Word Abbreviations}, see \url{http://www.issn.org/en/node/344}} depends on the subject of your book:
\begin{itemize}
\item The \textit{two} recommended styles for references in books on \textit{mathematical, physical, statistical and computer sciences} are depicted in ~\cite{science-contrib, science-online, science-mono, science-journal, science-DOI} and ~\cite{phys-online, phys-mono, phys-journal, phys-DOI, phys-contrib}.
\item Examples of the most commonly used reference style in books on \textit{Psychology, Social Sciences} are~\cite{psysoc-mono, psysoc-online,psysoc-journal, psysoc-contrib, psysoc-DOI}.
\item Examples for references in books on \textit{Humanities, Linguistics, Philosophy} are~\cite{humlinphil-journal, humlinphil-contrib, humlinphil-mono, humlinphil-online, humlinphil-DOI}.
\item Examples of the basic Springer style used in publications on a wide range of subjects such as \textit{Computer Science, Economics, Engineering, Geosciences, Life Sciences, Medicine, Biomedicine} are ~\cite{basic-contrib, basic-online, basic-journal, basic-DOI, basic-mono}. 
\end{itemize}
}

\begin{thebibliography}{99.}%
% and use \bibitem to create references.
%
% Use the following syntax and markup for your references if 
% the subject of your book is from the field 
% "Mathematics, Physics, Statistics, Computer Science"
%
% Contribution 
\bibitem{science-contrib} Broy, M.: Software engineering --- from auxiliary to key technologies. In: Broy, M., Dener, E. (eds.) Software Pioneers, pp. 10-13. Springer, Heidelberg (2002)
%
% Online Document
\bibitem{science-online} Dod, J.: Effective substances. In: The Dictionary of Substances and Their Effects. Royal Society of Chemistry (1999) Available via DIALOG. \\
\url{http://www.rsc.org/dose/title of subordinate document. Cited 15 Jan 1999}
%
% Monograph
\bibitem{science-mono} Geddes, K.O., Czapor, S.R., Labahn, G.: Algorithms for Computer Algebra. Kluwer, Boston (1992) 
%
% Journal article
\bibitem{science-journal} Hamburger, C.: Quasimonotonicity, regularity and duality for nonlinear systems of partial differential equations. Ann. Mat. Pura. Appl. \textbf{169}, 321--354 (1995)
%
% Journal article by DOI
\bibitem{science-DOI} Slifka, M.K., Whitton, J.L.: Clinical implications of dysregulated cytokine production. J. Mol. Med. (2000) doi: 10.1007/s001090000086 
%
\bigskip

% Use the following (APS) syntax and markup for your references if 
% the subject of your book is from the field 
% "Mathematics, Physics, Statistics, Computer Science"
%
% Online Document
\bibitem{phys-online} J. Dod, in \textit{The Dictionary of Substances and Their Effects}, Royal Society of Chemistry. (Available via DIALOG, 1999), 
\url{http://www.rsc.org/dose/title of subordinate document. Cited 15 Jan 1999}
%
% Monograph
\bibitem{phys-mono} H. Ibach, H. L\"uth, \textit{Solid-State Physics}, 2nd edn. (Springer, New York, 1996), pp. 45-56 
%
% Journal article
\bibitem{phys-journal} S. Preuss, A. Demchuk Jr., M. Stuke, Appl. Phys. A \textbf{61}
%
% Journal article by DOI
\bibitem{phys-DOI} M.K. Slifka, J.L. Whitton, J. Mol. Med., doi: 10.1007/s001090000086
%
% Contribution 
\bibitem{phys-contrib} S.E. Smith, in \textit{Neuromuscular Junction}, ed. by E. Zaimis. Handbook of Experimental Pharmacology, vol 42 (Springer, Heidelberg, 1976), p. 593
%
\bigskip
%
% Use the following syntax and markup for your references if 
% the subject of your book is from the field 
% "Psychology, Social Sciences"
%
%
% Monograph
\bibitem{psysoc-mono} Calfee, R.~C., \& Valencia, R.~R. (1991). \textit{APA guide to preparing manuscripts for journal publication.} Washington, DC: American Psychological Association.
%
% Online Document
\bibitem{psysoc-online} Dod, J. (1999). Effective substances. In: The dictionary of substances and their effects. Royal Society of Chemistry. Available via DIALOG. \\
\url{http://www.rsc.org/dose/Effective substances.} Cited 15 Jan 1999.
%
% Journal article
\bibitem{psysoc-journal} Harris, M., Karper, E., Stacks, G., Hoffman, D., DeNiro, R., Cruz, P., et al. (2001). Writing labs and the Hollywood connection. \textit{J Film} Writing, 44(3), 213--245.
%
% Contribution 
\bibitem{psysoc-contrib} O'Neil, J.~M., \& Egan, J. (1992). Men's and women's gender role journeys: Metaphor for healing, transition, and transformation. In B.~R. Wainrig (Ed.), \textit{Gender issues across the life cycle} (pp. 107--123). New York: Springer.
%
% Journal article by DOI
\bibitem{psysoc-DOI}Kreger, M., Brindis, C.D., Manuel, D.M., Sassoubre, L. (2007). Lessons learned in systems change initiatives: benchmarks and indicators. \textit{American Journal of Community Psychology}, doi: 10.1007/s10464-007-9108-14.
%
%
% Use the following syntax and markup for your references if 
% the subject of your book is from the field 
% "Humanities, Linguistics, Philosophy"
%
\bigskip
%
% Journal article
\bibitem{humlinphil-journal} Alber John, Daniel C. O'Connell, and Sabine Kowal. 2002. Personal perspective in TV interviews. \textit{Pragmatics} 12:257--271
%
% Contribution 
\bibitem{humlinphil-contrib} Cameron, Deborah. 1997. Theoretical debates in feminist linguistics: Questions of sex and gender. In \textit{Gender and discourse}, ed. Ruth Wodak, 99--119. London: Sage Publications.
%
% Monograph
\bibitem{humlinphil-mono} Cameron, Deborah. 1985. \textit{Feminism and linguistic theory.} New York: St. Martin's Press.
%
% Online Document
\bibitem{humlinphil-online} Dod, Jake. 1999. Effective substances. In: The dictionary of substances and their effects. Royal Society of Chemistry. Available via DIALOG. \\
http://www.rsc.org/dose/title of subordinate document. Cited 15 Jan 1999
%
% Journal article by DOI
\bibitem{humlinphil-DOI} Suleiman, Camelia, Daniel C. O�Connell, and Sabine Kowal. 2002. `If you and I, if we, in this later day, lose that sacred fire...�': Perspective in political interviews. \textit{Journal of Psycholinguistic Research}. doi: 10.1023/A:1015592129296.
%
%
%
\bigskip
%
%
% Use the following syntax and markup for your references if 
% the subject of your book is from the field 
% "Computer Science, Economics, Engineering, Geosciences, Life Sciences"
%
%
% Contribution 
\bibitem{basic-contrib} Brown B, Aaron M (2001) The politics of nature. In: Smith J (ed) The rise of modern genomics, 3rd edn. Wiley, New York 
%
% Online Document
\bibitem{basic-online} Dod J (1999) Effective Substances. In: The dictionary of substances and their effects. Royal Society of Chemistry. Available via DIALOG. \\
\url{http://www.rsc.org/dose/title of subordinate document. Cited 15 Jan 1999}
%
% Journal article by DOI
\bibitem{basic-DOI} Slifka MK, Whitton JL (2000) Clinical implications of dysregulated cytokine production. J Mol Med, doi: 10.1007/s001090000086
%
% Journal article
\bibitem{basic-journal} Smith J, Jones M Jr, Houghton L et al (1999) Future of health insurance. N Engl J Med 965:325--329
%
% Monograph
\bibitem{basic-mono} South J, Blass B (2001) The future of modern genomics. Blackwell, London 
%
\end{thebibliography}

