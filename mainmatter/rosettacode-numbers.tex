
%%%%%%%%%%%%%%%%%%%%% chapter.tex %%%%%%%%%%%%%%%%%%%%%%%%%%%%%%%%%
%
% sample chapter
%
% Use this file as a template for your own input.
%
%%%%%%%%%%%%%%%%%%%%%%%% Springer-Verlag %%%%%%%%%%%%%%%%%%%%%%%%%%
%\motto{Use the template \emph{chapter.tex} to style the various elements of your chapter content.}

\chapter{Rosetta Code Tasks starting with Numbers}
\label{rosettacode-numbers}

\section*{100 doors}

Problem: You have 100 doors in a row that are all initially closed.
You make 100 passes by the doors. The first time through, you visit
every door and toggle the door (if the door is closed, you open it; if
it is open, you close it). The second time you only visit every 2nd
door (door \#2, \#4, \#6, \ldots{}). The third time, every 3rd door
(door \#3, \#6, \#9, \ldots{}), etc, until you only visit the 100th
door.

Question: What state are the doors in after the last pass? Which are
open, which are closed?

\textbf{Alternate:} As noted in this page's discussion page, the only
doors that remain open are whose numbers are perfect squares of
integers. Opening only those doors is an optimization that may also be
expressed.

\begin{wideverbatim}
unoptimized

(let Doors (need 100)
   (for I 100
      (for (D (nth Doors I)  D  (cdr (nth D I)))
         (set D (not (car D))) ) )
   (println Doors) )

optimized

(let Doors (need 100)
   (for I (sqrt 100)
      (set (nth Doors (* I I)) T) )
   (println Doors) )
\end{wideverbatim}

\pagebreak{}
\section*{24 game}

The 24 Game tests one's mental arithmetic.

Write a program that randomly chooses and displays four digits, each
from one to nine, with repetitions allowed. The program should prompt
for the player to enter an equation using \emph{just} those, and
\emph{all} of those four digits. The program should \emph{check} then
evaluate the expression. The goal is for the player to enter an
expression that evaluates to \textbf{24}.

\begin{itemize}
\item
  Only multiplication, division, addition, and subtraction
  operators/functions are allowed.
\item
  Division should use floating point or rational arithmetic, etc, to
  preserve remainders.
\item
  Brackets are allowed, if using an infix expression evaluator.
\item
  Forming multiple digit numbers from the supplied digits is
  \emph{disallowed}. (So an answer of 12+12 when given 1, 2, 2, and 1 is
  wrong).
\item
  The order of the digits when given does not have to be preserved.
\end{itemize}

Note:

\begin{itemize}
\item The type of expression evaluator used is not mandated. An RPN
  evaluator is equally acceptable for example.
\item
  The task is not for the program to generate the expression, or test
  whether an expression is even possible.
\end{itemize}

C.f: 24 game Player

\textbf{Reference}

\begin{enumerate}
\item
  \href{http://www.bbc.co.uk/dna/h2g2/A933121}{The 24 Game} on h2g2.
\end{enumerate}


\begin{wideverbatim}

(de checkExpression (Lst Exe)
   (make
      (when (diff Lst (fish num? Exe))
         (link "Not all numbers used" ) )
      (when (diff (fish num? Exe) Lst)
         (link "Using wrong number(s)") )
      (when (diff (fish sym? Exe) '(+ - * /))
         (link "Using illegal operator(s)") ) ) )

(loop
   (setq Numbers (make (do 4 (link (rand 1 9)))))
   (prinl
      "Please enter a Lisp expression using (, ), +, -, *, / and "
      (glue ", " Numbers) )
   (prin "Or a single dot '.' to stop: ")
   (T (= "." (setq Reply (catch '(NIL) (in NIL (read)))))
      (bye) )
   (cond
      ((str? Reply)
         (prinl "-- Input error: " Reply) )
      ((checkExpression Numbers Reply)
         (prinl "-- Illegal Expression")
         (for S @
            (space 3)
            (prinl S) ) )
      ((str? (setq Result (catch '(NIL) (eval Reply))))
         (prinl "-- Evaluation error: " @) )
      ((= 24 Result)
         (prinl "++ Congratulations! Correct result :-)") )
      (T (prinl "Sorry, this gives " Result)) )
   (prinl) )

\end{wideverbatim}

\begin{wideverbatim}

Output:

Please enter a Lisp expression using (, ), +, -, *, / and 1, 3, 3, 5
Or a single dot '.' to stop: (* (+ 3 1) (+ 5 1))
++ Congratulations! Correct result :-)

Please enter a Lisp expression using (, ), +, -, *, / and 8, 4, 7, 1
Or a single dot '.' to stop: (* 8 (\% 7 3) 9)
-- Illegal Expression
   Not all numbers used
   Using wrong number(s)
   Using illegal operator(s)

Please enter a Lisp expression using (, ), +, -, *, / and 4, 2, 2, 3
Or a single dot '.' to stop: (/ (+ 4 3) (- 2 2))
-- Evaluation error: Div/0

Please enter a Lisp expression using (, ), +, -, *, / and 8, 4, 5, 9
Or a single dot '.' to stop: .

\end{wideverbatim}

\pagebreak{}
\section*{24 game/Solve}


Write a function that given four digits subject to the rules of the
24 game, computes an expression to solve the game
if possible.

Show examples of solutions generated by the function

C.F: Arithmetic Evaluator


\begin{wideverbatim}

We use Pilog (PicoLisp Prolog) to solve this task

(be play24 (@Lst @Expr)                # Define Pilog rule
   (permute @Lst (@A @B @C @D))
   (member @Op1 (+ - * /))
   (member @Op2 (+ - * /))
   (member @Op3 (+ - * /))
   (or
      ((equal @Expr (@Op1 (@Op2 @A @B) (@Op3 @C @D))))
      ((equal @Expr (@Op1 @A (@Op2 @B (@Op3 @C @D))))) )
   (@ = 24 (catch '("Div/0") (eval (-> @Expr)))) )

(de play24 (A B C D)                   # Define PicoLisp function
   (pilog
      (quote
         @L (list A B C D)
         (play24 @L @X) )
      (println @X) ) )

(play24 5 6 7 8)                       # Call 'play24' function

Output:

(* (+ 5 7) (- 8 6))
(* 6 (+ 5 (- 7 8)))
(* 6 (- 5 (- 8 7)))
(* 6 (- 5 (/ 8 7)))
(* 6 (+ 7 (- 5 8)))
(* 6 (- 7 (- 8 5)))
(* 6 (/ 8 (- 7 5)))
(/ (* 6 8) (- 7 5))
(* (+ 7 5) (- 8 6))
(* (- 8 6) (+ 5 7))
(* (- 8 6) (+ 7 5))
(* 8 (/ 6 (- 7 5)))
(/ (* 8 6) (- 7 5))

\end{wideverbatim}

\pagebreak{}
\section*{99 Bottles of Beer}


In this puzzle, write code to print out the entire ``99 bottles of beer
on the wall'' song. For those who do not know the song, the lyrics
follow this form:

\begin{verbatim}
X bottles of beer on the wall
X bottles of beer
Take one down, pass it around
X-1 bottles of beer on the wall

X-1 bottles of beer on the wall
...
Take one down, pass it around
0 bottles of beer on the wall
\end{verbatim}

Where X and X-1 are replaced by numbers of course. Grammatical support
for ``1 bottle of beer'' is optional. As with any puzzle, try to do it
in as creative/concise/comical a way as possible (simple, obvious
solutions allowed, too).

See also:
\href{http://99-bottles-of-beer.net/}{http://99-bottles-of-beer.net/}


\begin{wideverbatim}

(de bottles (N)
   (case N
      (0 "No more beer")
      (1 "One bottle of beer")
      (T (cons N " bottles of beer")) ) )

(for (N 99 (gt0 N))
   (prinl (bottles N) " on the wall,")
   (prinl (bottles N) ".")
   (prinl "Take one down, pass it around,")
   (prinl (bottles (dec 'N)) " on the wall.")
   (prinl) )

\end{wideverbatim}




% %%%%%%%%%%%%%%%%%%%%%%%% referenc.tex %%%%%%%%%%%%%%%%%%%%%%%%%%%%%%
% sample references
% %
% Use this file as a template for your own input.
%
%%%%%%%%%%%%%%%%%%%%%%%% Springer-Verlag %%%%%%%%%%%%%%%%%%%%%%%%%%
%
% BibTeX users please use
% \bibliographystyle{}
% \bibliography{}
%
\biblstarthook{In view of the parallel print and (chapter-wise) online publication of your book at \url{www.springerlink.com} it has been decided that -- as a genreral rule --  references should be sorted chapter-wise and placed at the end of the individual chapters. However, upon agreement with your contact at Springer you may list your references in a single seperate chapter at the end of your book. Deactivate the class option \texttt{sectrefs} and the \texttt{thebibliography} environment will be put out as a chapter of its own.\\\indent
References may be \textit{cited} in the text either by number (preferred) or by author/year.\footnote{Make sure that all references from the list are cited in the text. Those not cited should be moved to a separate \textit{Further Reading} section or chapter.} The reference list should ideally be \textit{sorted} in alphabetical order -- even if reference numbers are used for the their citation in the text. If there are several works by the same author, the following order should be used: 
\begin{enumerate}
\item all works by the author alone, ordered chronologically by year of publication
\item all works by the author with a coauthor, ordered alphabetically by coauthor
\item all works by the author with several coauthors, ordered chronologically by year of publication.
\end{enumerate}
The \textit{styling} of references\footnote{Always use the standard abbreviation of a journal's name according to the ISSN \textit{List of Title Word Abbreviations}, see \url{http://www.issn.org/en/node/344}} depends on the subject of your book:
\begin{itemize}
\item The \textit{two} recommended styles for references in books on \textit{mathematical, physical, statistical and computer sciences} are depicted in ~\cite{science-contrib, science-online, science-mono, science-journal, science-DOI} and ~\cite{phys-online, phys-mono, phys-journal, phys-DOI, phys-contrib}.
\item Examples of the most commonly used reference style in books on \textit{Psychology, Social Sciences} are~\cite{psysoc-mono, psysoc-online,psysoc-journal, psysoc-contrib, psysoc-DOI}.
\item Examples for references in books on \textit{Humanities, Linguistics, Philosophy} are~\cite{humlinphil-journal, humlinphil-contrib, humlinphil-mono, humlinphil-online, humlinphil-DOI}.
\item Examples of the basic Springer style used in publications on a wide range of subjects such as \textit{Computer Science, Economics, Engineering, Geosciences, Life Sciences, Medicine, Biomedicine} are ~\cite{basic-contrib, basic-online, basic-journal, basic-DOI, basic-mono}. 
\end{itemize}
}

\begin{thebibliography}{99.}%
% and use \bibitem to create references.
%
% Use the following syntax and markup for your references if 
% the subject of your book is from the field 
% "Mathematics, Physics, Statistics, Computer Science"
%
% Contribution 
\bibitem{science-contrib} Broy, M.: Software engineering --- from auxiliary to key technologies. In: Broy, M., Dener, E. (eds.) Software Pioneers, pp. 10-13. Springer, Heidelberg (2002)
%
% Online Document
\bibitem{science-online} Dod, J.: Effective substances. In: The Dictionary of Substances and Their Effects. Royal Society of Chemistry (1999) Available via DIALOG. \\
\url{http://www.rsc.org/dose/title of subordinate document. Cited 15 Jan 1999}
%
% Monograph
\bibitem{science-mono} Geddes, K.O., Czapor, S.R., Labahn, G.: Algorithms for Computer Algebra. Kluwer, Boston (1992) 
%
% Journal article
\bibitem{science-journal} Hamburger, C.: Quasimonotonicity, regularity and duality for nonlinear systems of partial differential equations. Ann. Mat. Pura. Appl. \textbf{169}, 321--354 (1995)
%
% Journal article by DOI
\bibitem{science-DOI} Slifka, M.K., Whitton, J.L.: Clinical implications of dysregulated cytokine production. J. Mol. Med. (2000) doi: 10.1007/s001090000086 
%
\bigskip

% Use the following (APS) syntax and markup for your references if 
% the subject of your book is from the field 
% "Mathematics, Physics, Statistics, Computer Science"
%
% Online Document
\bibitem{phys-online} J. Dod, in \textit{The Dictionary of Substances and Their Effects}, Royal Society of Chemistry. (Available via DIALOG, 1999), 
\url{http://www.rsc.org/dose/title of subordinate document. Cited 15 Jan 1999}
%
% Monograph
\bibitem{phys-mono} H. Ibach, H. L\"uth, \textit{Solid-State Physics}, 2nd edn. (Springer, New York, 1996), pp. 45-56 
%
% Journal article
\bibitem{phys-journal} S. Preuss, A. Demchuk Jr., M. Stuke, Appl. Phys. A \textbf{61}
%
% Journal article by DOI
\bibitem{phys-DOI} M.K. Slifka, J.L. Whitton, J. Mol. Med., doi: 10.1007/s001090000086
%
% Contribution 
\bibitem{phys-contrib} S.E. Smith, in \textit{Neuromuscular Junction}, ed. by E. Zaimis. Handbook of Experimental Pharmacology, vol 42 (Springer, Heidelberg, 1976), p. 593
%
\bigskip
%
% Use the following syntax and markup for your references if 
% the subject of your book is from the field 
% "Psychology, Social Sciences"
%
%
% Monograph
\bibitem{psysoc-mono} Calfee, R.~C., \& Valencia, R.~R. (1991). \textit{APA guide to preparing manuscripts for journal publication.} Washington, DC: American Psychological Association.
%
% Online Document
\bibitem{psysoc-online} Dod, J. (1999). Effective substances. In: The dictionary of substances and their effects. Royal Society of Chemistry. Available via DIALOG. \\
\url{http://www.rsc.org/dose/Effective substances.} Cited 15 Jan 1999.
%
% Journal article
\bibitem{psysoc-journal} Harris, M., Karper, E., Stacks, G., Hoffman, D., DeNiro, R., Cruz, P., et al. (2001). Writing labs and the Hollywood connection. \textit{J Film} Writing, 44(3), 213--245.
%
% Contribution 
\bibitem{psysoc-contrib} O'Neil, J.~M., \& Egan, J. (1992). Men's and women's gender role journeys: Metaphor for healing, transition, and transformation. In B.~R. Wainrig (Ed.), \textit{Gender issues across the life cycle} (pp. 107--123). New York: Springer.
%
% Journal article by DOI
\bibitem{psysoc-DOI}Kreger, M., Brindis, C.D., Manuel, D.M., Sassoubre, L. (2007). Lessons learned in systems change initiatives: benchmarks and indicators. \textit{American Journal of Community Psychology}, doi: 10.1007/s10464-007-9108-14.
%
%
% Use the following syntax and markup for your references if 
% the subject of your book is from the field 
% "Humanities, Linguistics, Philosophy"
%
\bigskip
%
% Journal article
\bibitem{humlinphil-journal} Alber John, Daniel C. O'Connell, and Sabine Kowal. 2002. Personal perspective in TV interviews. \textit{Pragmatics} 12:257--271
%
% Contribution 
\bibitem{humlinphil-contrib} Cameron, Deborah. 1997. Theoretical debates in feminist linguistics: Questions of sex and gender. In \textit{Gender and discourse}, ed. Ruth Wodak, 99--119. London: Sage Publications.
%
% Monograph
\bibitem{humlinphil-mono} Cameron, Deborah. 1985. \textit{Feminism and linguistic theory.} New York: St. Martin's Press.
%
% Online Document
\bibitem{humlinphil-online} Dod, Jake. 1999. Effective substances. In: The dictionary of substances and their effects. Royal Society of Chemistry. Available via DIALOG. \\
http://www.rsc.org/dose/title of subordinate document. Cited 15 Jan 1999
%
% Journal article by DOI
\bibitem{humlinphil-DOI} Suleiman, Camelia, Daniel C. O�Connell, and Sabine Kowal. 2002. `If you and I, if we, in this later day, lose that sacred fire...�': Perspective in political interviews. \textit{Journal of Psycholinguistic Research}. doi: 10.1023/A:1015592129296.
%
%
%
\bigskip
%
%
% Use the following syntax and markup for your references if 
% the subject of your book is from the field 
% "Computer Science, Economics, Engineering, Geosciences, Life Sciences"
%
%
% Contribution 
\bibitem{basic-contrib} Brown B, Aaron M (2001) The politics of nature. In: Smith J (ed) The rise of modern genomics, 3rd edn. Wiley, New York 
%
% Online Document
\bibitem{basic-online} Dod J (1999) Effective Substances. In: The dictionary of substances and their effects. Royal Society of Chemistry. Available via DIALOG. \\
\url{http://www.rsc.org/dose/title of subordinate document. Cited 15 Jan 1999}
%
% Journal article by DOI
\bibitem{basic-DOI} Slifka MK, Whitton JL (2000) Clinical implications of dysregulated cytokine production. J Mol Med, doi: 10.1007/s001090000086
%
% Journal article
\bibitem{basic-journal} Smith J, Jones M Jr, Houghton L et al (1999) Future of health insurance. N Engl J Med 965:325--329
%
% Monograph
\bibitem{basic-mono} South J, Blass B (2001) The future of modern genomics. Blackwell, London 
%
\end{thebibliography}

