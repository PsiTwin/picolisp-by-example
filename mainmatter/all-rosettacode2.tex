# 21jul12abu
# (c) Software Lab. Alexander Burger

\chapter{Rosetta Code Tasks starting with Numbers}
\label{rosettacode-numbers}

\section*{100 doors}

\begin{verbatim}

unoptimized

(let Doors (need 100)
   (for I 100
      (for (D (nth Doors I)  D  (cdr (nth D I)))
         (set D (not (car D))) ) )
   (println Doors) )

optimized

(let Doors (need 100)
   (for I (sqrt 100)
      (set (nth Doors (* I I)) T) )
   (println Doors) )

\end{verbatim}

\section*{24 game}

\begin{verbatim}

(de checkExpression (Lst Exe)
   (make
      (when (diff Lst (fish num? Exe))
         (link "Not all numbers used" ) )
      (when (diff (fish num? Exe) Lst)
         (link "Using wrong number(s)") )
      (when (diff (fish sym? Exe) '(+ - * /))
         (link "Using illegal operator(s)") ) ) )

(loop
   (setq Numbers (make (do 4 (link (rand 1 9)))))
   (prinl
      "Please enter a Lisp expression using (, ), +, -, *, / and "
      (glue ", " Numbers) )
   (prin "Or a single dot '.' to stop: ")
   (T (= "." (setq Reply (catch '(NIL) (in NIL (read)))))
      (bye) )
   (cond
      ((str? Reply)
         (prinl "-- Input error: " Reply) )
      ((checkExpression Numbers Reply)
         (prinl "-- Illegal Expression")
         (for S @
            (space 3)
            (prinl S) ) )
      ((str? (setq Result (catch '(NIL) (eval Reply))))
         (prinl "-- Evaluation error: " @) )
      ((= 24 Result)
         (prinl "++ Congratulations! Correct result :-)") )
      (T (prinl "Sorry, this gives " Result)) )
   (prinl) )

Output:

Please enter a Lisp expression using (, ), +, -, *, / and 1, 3, 3, 5
Or a single dot '.' to stop: (* (+ 3 1) (+ 5 1))
++ Congratulations! Correct result :-)

Please enter a Lisp expression using (, ), +, -, *, / and 8, 4, 7, 1
Or a single dot '.' to stop: (* 8 (\% 7 3) 9)
-- Illegal Expression
   Not all numbers used
   Using wrong number(s)
   Using illegal operator(s)

Please enter a Lisp expression using (, ), +, -, *, / and 4, 2, 2, 3
Or a single dot '.' to stop: (/ (+ 4 3) (- 2 2))
-- Evaluation error: Div/0

Please enter a Lisp expression using (, ), +, -, *, / and 8, 4, 5, 9
Or a single dot '.' to stop: .

\end{verbatim}

\section*{24 game/Solve}

\begin{verbatim}

We use Pilog (PicoLisp Prolog) to solve this task

(be play24 (@Lst @Expr)                # Define Pilog rule
   (permute @Lst (@A @B @C @D))
   (member @Op1 (+ - * /))
   (member @Op2 (+ - * /))
   (member @Op3 (+ - * /))
   (or
      ((equal @Expr (@Op1 (@Op2 @A @B) (@Op3 @C @D))))
      ((equal @Expr (@Op1 @A (@Op2 @B (@Op3 @C @D))))) )
   (@ = 24 (catch '("Div/0") (eval (-> @Expr)))) )

(de play24 (A B C D)                   # Define PicoLisp function
   (pilog
      (quote
         @L (list A B C D)
         (play24 @L @X) )
      (println @X) ) )

(play24 5 6 7 8)                       # Call 'play24' function

Output:

(* (+ 5 7) (- 8 6))
(* 6 (+ 5 (- 7 8)))
(* 6 (- 5 (- 8 7)))
(* 6 (- 5 (/ 8 7)))
(* 6 (+ 7 (- 5 8)))
(* 6 (- 7 (- 8 5)))
(* 6 (/ 8 (- 7 5)))
(/ (* 6 8) (- 7 5))
(* (+ 7 5) (- 8 6))
(* (- 8 6) (+ 5 7))
(* (- 8 6) (+ 7 5))
(* 8 (/ 6 (- 7 5)))
(/ (* 8 6) (- 7 5))

\end{verbatim}

\section*{99 Bottles of Beer}

\begin{verbatim}

(de bottles (N)
   (case N
      (0 "No more beer")
      (1 "One bottle of beer")
      (T (cons N " bottles of beer")) ) )

(for (N 99 (gt0 N))
   (prinl (bottles N) " on the wall,")
   (prinl (bottles N) ".")
   (prinl "Take one down, pass it around,")
   (prinl (bottles (dec 'N)) " on the wall.")
   (prinl) )

\end{verbatim}


\chapter{Rosetta Code Tasks starting with A}
\label{rosettacode-numbers}

\section*{A+B}

\begin{verbatim}

(+ (read) (read))
3 4
-> 7

\end{verbatim}

\section*{Abstract type}

\begin{verbatim}

# In PicoLisp there is no formal difference between abstract and concrete
# classes, just a naming convention where abstract classes start with a
# lower case character after the '+' (the naming convention for classes).
# This tells the programmer that this class has not sufficient methods
# defined to survive on its own.

(class +abstractClass)

(dm someMethod> ()
   (foo)
   (bar) )

\end{verbatim}

\section*{Accumulator factory}

\begin{verbatim}

(de accumulator (Sum)
   (curry (Sum) (N)
      (inc 'Sum N) ) )

(def 'a (accumulator 7))
(a 1)  # Output: -> 8
(a 2)  # Output: -> 10
(a -5)  # Output: -> 5

\end{verbatim}

\section*{Ackermann function}

\begin{verbatim}

(de ack (X Y)
   (cond
      ((=0 X) (inc Y))
      ((=0 Y) (ack (dec X) 1))
      (T (ack (dec X) (ack X (dec Y)))) ) )

\end{verbatim}

\section*{Active Directory/Connect}

\begin{verbatim}

(unless (=0 (setq Ldap (native "libldap.so" "ldap_open" 'N "example.com" 389)))
   (quit "Can't open LDAP") )

(native "libldap.so" "ldap_simple_bind_s" 'I Ldap "user" "password")

\end{verbatim}

\section*{Active Directory/Search for a user}

\begin{verbatim}

(de ldapsearch (Sn)
   (in
      (list "ldapsearch" "-xH" "ldap://db.debian.org"
         "-b" "dc=debian,dc=org"
         (pack "sn=" Sn) )
      (list
         (cons 'cn (prog (from "cn: ") (line T)))
         (cons 'uid (prog (from "uid: ") (line T))) ) ) )

Test:

: (ldapsearch "Fischer")
-> ((cn . "Mika") (uid . "mf"))

\end{verbatim}

\section*{Active object}

\begin{verbatim}

(load "@lib/math.l")

(class +Active)
# inp val sum usec

(dm T ()
   (unless (assoc -100 *Run)           # Install timer task
      (task -100 100                   # Update objects every 0.1 sec
         (mapc 'update> *Actives) ) )
   (=: inp '((U) 0))                   # Set zero input function
   (=: val 0)                          # Initialize last value
   (=: sum 0)                          # Initialize sum
   (=: usec (usec))                    # and time
   (push '*Actives This) )             # Install in notification list

(dm input> (Fun)
   (=: inp Fun) )

(dm update> ()
   (let (U (usec)  V ((: inp) U))      # Get current time, calculate value
      (inc (:: sum)
         (*/
            (+ V (: val))              # (K(t[1]) + K(t[0])) *
            (- U (: usec))             # (t[1] - t[0]) /
            2.0 ) )                    # 2.0
      (=: val V)
      (=: usec U) ) )

(dm output> ()
   (format (: sum) *Scl) )             # Get result

(dm stop> ()
   (unless (del This '*Actives)        # Removing the last active object?
      (task -100) ) )                  # Yes: Uninstall timer task

(de integrate ()                       # Test it
   (let Obj (new '(+Active))           # Create an active object
      (input> Obj                      # Set input function
         '((U) (sin (*/ pi U 1.0))) )  # to sin(π * t)
      (wait 2000)                      # Wait 2 sec
      (input> Obj '((U) 0))            # Reset input function
      (wait 500)                       # Wait 0.5 sec
      (prinl "Output: " (output> Obj)) # Print return value
      (stop> Obj) ) )                  # Stop active object

\end{verbatim}

\section*{Add a variable to a class instance at runtime}

\begin{verbatim}

In general, all instance variables in PicoLisp are dynamically created at
runtime.

: (setq MyObject (new '(+MyClass)))       # Create some object
-> \$385605941
: (put MyObject 'newvar '(some value))    # Set variable
-> (some value)
: (show MyObject)                         # Show the object
\$385605941 (+MyClass)
   newvar (some value)
-> \$385605941

\end{verbatim}

\section*{Address of a variable}

\begin{verbatim}

The PicoLisp function '[http://software-lab.de/doc/refA.html#adr adr]' returns
the address of a variable. A variable may be either a symbol or a cons pair in
PicoLisp.

The returned address is a number representing an encoded pointer. For symbols,
it is a negative number, and for cons pairs a positive number. The same function
'adr' can then be used to convert that pointer back to the original object.

: (setq X 7)
-> 7

: (adr 'X)
-> -2985527269106

: (val (adr -2985527269106))
-> 7

: (set (adr -2985527269106) '(a b c))
-> (a b c)

: X
-> (a b c)

\end{verbatim}

\section*{Align columns}

\begin{verbatim}

(let Sizes NIL                         # Build a list of sizes
   (let Lines                          # and of lines
      (make
         (in "input.txt"                     # Reading input file
            (while (split (line) "$")        # delimited by '$'
               (let (L (link (mapcar pack @))  S Sizes)
                  (setq Sizes                   # Maintain sizes
                     (make
                        (while (or L S)
                           (link
                              (max
                                 (inc (length (pop 'L)))
                                 (pop 'S) ) ) ) ) ) ) ) ) )
      (for L Lines                                 # Print lines
         (prinl (apply align L (mapcar - Sizes))) )   # left aligned
      (prinl)
      (for L Lines
         (prinl (apply align L Sizes)) )              # right aligned
      (prinl)
      (for L Lines
         (prinl (apply center L Sizes)) ) ) )         # and centered

\end{verbatim}

\section*{Amb}

\begin{verbatim}

For backtracking, Pilog (PicoLisp Prolog) is the natural choice.

{{trans|Prolog}}

(be amb (@E @Lst)
   (lst @E @Lst) )

(be joins (@Left @Right)
   (@T last (chop (-> @Left)))
   (@R car (chop (-> @Right)))
   (or
      ((equal @T @R))
      ((amb @ NIL)) ) )  # Explicitly using amb fail as required

(be ambExample ((@Word1 @Word2 @Word3 @Word4))
  (amb @Word1 ("the" "that" "a"))
  (amb @Word2 ("frog" "elephant" "thing"))
  (amb @Word3 ("walked" "treaded" "grows"))
  (amb @Word4 ("slowly" "quickly"))
  (joins @Word1 @Word2)
  (joins @Word2 @Word3)
  (joins @Word3 @Word4) )

Output:

: (? (ambExample @Result))
 @Result=("that" "thing" "grows" "slowly")
-> NIL

\end{verbatim}

\section*{Anagrams}

\begin{verbatim}

A straight-forward implementation using 'group' takes 48 seconds
on a 1.7 GHz Pentium:

(flip
   (by length sort
      (by '((L) (sort (copy L))) group
         (in "unixdict.txt" (make (while (line) (link @)))) ) ) )

Using a binary tree with the 'idx' function, it takes only 0.42 seconds
on the same machine, a factor of 100 faster:

(let Words NIL
   (in "unixdict.txt"
      (while (line)
         (let (Word (pack @)  Key (pack (sort @)))
            (if (idx 'Words Key T)
               (push (car @) Word)
               (set Key (list Word)) ) ) ) )
   (flip (by length sort (mapcar val (idx 'Words)))) )

Output:

-> (("vile" "veil" "live" "levi" "evil") ("trace" "crate" "cater" "carte" "caret
") ("regal" "large" "lager" "glare" "alger") ("neal" "lena" "lean" "lane" "elan"
) ("lange" "glean" "galen" "angle" "angel") ("elba" "bela" "bale" "able" "abel")
 ("tulsa" "talus" "sault" "latus") ...

\end{verbatim}

\section*{Anagrams/Deranged anagrams}

\begin{verbatim}

(let Words NIL
   (in "unixdict.txt"
      (while (line)
         (let (Word @  Key (pack (sort (copy @))))
            (if (idx 'Words Key T)
               (push (car @) Word)
               (set Key (list Word)) ) ) ) )
   (maxi '((X) (length (car X)))
      (extract
         '((Key)
            (pick
               '((Lst)
                  (and
                     (find
                        '((L) (not (find = L Lst)))
                        (val Key) )
                     (cons (pack @) (pack Lst)) ) )
               (val Key) ) )
         (idx 'Words) ) ) )

Output:

-> ("excitation" . "intoxicate")

\end{verbatim}

\section*{Animate a pendulum}

\begin{verbatim}

A minimalist solution. The pendulum consists of the center point '+', and the
swinging xterm cursor.

(load "@lib/math.l")

(de pendulum (X Y Len)
   (let (Angle pi/2  V 0)
      (call 'clear)
      (call 'tput "cup" Y X)
      (prin '+)
      (call 'tput "cup" 1 (+ X Len))
      (until (key 25)                        # 25 ms
         (let A (*/ (sin Angle) -9.81 1.0)
            (inc 'V (*/ A 40))               # DT = 25 ms = 1/40 sec
            (inc 'Angle (*/ V 40)) )
         (call 'tput "cup"
            (+ Y (*/ Len (cos Angle) 2.2))   # Compensate for aspect ratio
            (+ X (*/ Len (sin Angle) 1.0)) ) ) ) )

Test (hit any key to stop):

(pendulum 40 1 36)

\end{verbatim}

\section*{Animation}

\begin{verbatim}

Plain text

A plain text version. The following script works in an XTerm window.

#!/usr/bin/picolisp /usr/lib/picolisp/lib.l

(prin "^[[?9h")  # Mouse reporting on

(setq Dir 1  Text (chop "Hello World! "))

(loop
   (prin (do Dir (rot Text)))
   (when (= "^[" (key 200))
      (key) (key)
      (when (= " " (key))  # Left button
         (setq Dir (if (= 1 Dir) 12 1)) )
      (key) (key) )
   (do (length Text) (prin "^H")) )

HTML/JavaScript

The standard PicoLisp GUI is HTTP based. Connect your browser to
http://localhost:8080 after starting the following script.

The scrolling text is displayed in a button. Clicking on the button
reverses the scroll direction.

#!/usr/bin/picolisp /usr/lib/picolisp/lib.l

(load "@ext.l" "@lib/http.l" "@lib/xhtml.l" "@lib/form.l")

(one *Dir)

(de start ()
   (app)
   (action
      (html 0 "Animation" "@lib.css" NIL
         (form NIL
            (gui '(+Button)
               '(pack (do *Dir (rot '`(chop "Hello World! "))))
               '(setq *Dir (if (= 1 *Dir) 12 1)) )
            (gui '(+Click +Auto +Button) 400 'This 1000 "Start") ) ) ) )

(server 8080 "!start")
(wait)

Java/Swing

This solution works on ErsatzLisp, the Java version of PicoLisp.

#!ersatz/pil

(setq
   Dir 1
   Text (chop "Hello World! ")
   Frame (java "javax.swing.JFrame" T "Animation")
   Label (java "javax.swing.JLabel" T (pack Text)) )

(java Label 'addMouseListener
   (interface "java.awt.event.MouseListener"
      'mouseClicked '((Ev) (setq Dir (if (= 1 Dir) 12 1)))
      'mouseEntered nil
      'mouseExited nil
      'mousePressed nil
      'mouseReleased nil ) )

(java Frame 'add Label)
(java Frame 'pack)
(java Frame 'setVisible T)
(loop
   (wait 200)
   (java Label 'setText (pack (do Dir (rot Text)))) )

\end{verbatim}

\section*{Anonymous recursion (!)}

\begin{verbatim}

(de fibo (N)
   (if (lt0 N)
      (quit "Illegal argument" N) )
   (recur (N)
      (if (> 2 N)
         1
         (+ (recurse (dec N)) (recurse (- N 2))) ) ) )

Explanation: The above uses the
'[http://software-lab.de/doc/refR.html#recur recur]' /
'[http://software-lab.de/doc/refR.html#recurse recurse]' function pair, which is
defined as a standard language extensions as

(de recur recurse
   (run (cdr recurse)) )

Note how 'recur' dynamically defines the function 'recurse' at runtime, by
binding the rest of the expression (i.e. the body of the 'recur' statement) to
the symbol 'recurse'.

\end{verbatim}

\section*{Apply a callback to an array}

\begin{verbatim}

: (mapc println (1 2 3 4 5))  # Print numbers
1
2
3
4
5
-> 5

: (mapcar '((N) (* N N)) (1 2 3 4 5))  # Calculate squares
-> (1 4 9 16 25)

: (mapcar ** (1 2 3 4 5) (2 .))  # Same, using a circular list
-> (1 4 9 16 25)

: (mapcar if '(T NIL T NIL) '(1 2 3 4) '(5 6 7 8))  # Conditional function
-> (1 6 3 8)

\end{verbatim}

\section*{Arbitrary-precision integers (included)}

\begin{verbatim}

(let L (chop (** 5 (** 4 (** 3 2))))
   (prinl (head 20 L) "..." (tail 20 L))
   (length L) )

Output:

62060698786608744707...92256259918212890625
-> 183231

\end{verbatim}

\section*{Arena storage pool}

\begin{verbatim}

PicoLisp allocates any kind of data from a single pool, because everything
is built out of a "cell" primitive. Most of this allocation happens
automatically, but can also be done explicitly with
'[http://software-lab.de/doc/refN.html#new new]' or
'[http://software-lab.de/doc/refB.html#box box]'. For memory-allocated
objects, there is no explicit way of freeing them. Database objects can be
freed with '[http://software-lab.de/doc/refZ.html#zap zap]'.

\end{verbatim}

\section*{Arithmetic evaluation}

\begin{verbatim}

The built-in function 'str' splits a string into a list of lexical tokens
(numbers and transient symbols). From that, a recursive descendent parser can
build an expression tree, resulting in directly executable Lisp code.

(de ast (Str)
   (let *L (str Str "")
      (aggregate) ) )

(de aggregate ()
   (let X (product)
      (while (member (car *L) '("+" "-"))
         (setq X (list (intern (pop '*L)) X (product))) )
      X ) )

(de product ()
   (let X (term)
      (while (member (car *L) '("*" "/"))
         (setq X (list (intern (pop '*L)) X (term))) )
      X ) )

(de term ()
   (let X (pop '*L)
      (cond
         ((num? X) X)
         ((= "+" X) (term))
         ((= "-" X) (list '- (term)))
         ((= "(" X) (prog1 (aggregate) (pop '*L)))) ) ) )

Output:

: (ast "1+2+3*-4/(1+2)")
-> (+ (+ 1 2) (/ (* 3 (- 4)) (+ 1 2)))

: (ast "(1+2+3)*-4/(1+2)")
-> (/ (* (+ (+ 1 2) 3) (- 4)) (+ 1 2))

\end{verbatim}

\section*{Arithmetic-geometric mean}

\begin{verbatim}

(scl 80)

(de agm (A G)
   (do 7
      (prog1 (/ (+ A G) 2)
         (setq G (sqrt (* A G))  A @) ) ) )

(round
   (agm 1.0 (*/ 1.0 1.0 (sqrt (* 2.0 1.0))))
   70 )

Output:
-> "0.8472130847939790866064991234821916364814459103269421850605793726597340"

\end{verbatim}

\section*{Arithmetic/Complex}

\begin{verbatim}

(load "@lib/math.l")

(de addComplex (A B)
   (cons
      (+ (car A) (car B))        # Real
      (+ (cdr A) (cdr B)) ) )    # Imag

(de mulComplex (A B)
   (cons
      (-
         (*/ (car A) (car B) 1.0)
         (*/ (cdr A) (cdr B) 1.0) )
      (+
         (*/ (car A) (cdr B) 1.0)
         (*/ (cdr A) (car B) 1.0) ) ) )

(de invComplex (A)
   (let Denom
      (+
         (*/ (car A) (car A) 1.0)
         (*/ (cdr A) (cdr A) 1.0) )
      (cons
         (*/ (car A) 1.0 Denom)
         (- (*/ (cdr A) 1.0 Denom)) ) ) )

(de negComplex (A)
   (cons (- (car A)) (- (cdr A))) )

(de fmtComplex (A)
   (pack
      (round (car A) (dec *Scl))
      (and (gt0 (cdr A)) "+")
      (round (cdr A) (dec *Scl))
      "i" ) )

(let (A (1.0 . 1.0)  B (cons pi 1.2))
   (prinl "A = " (fmtComplex A))
   (prinl "B = " (fmtComplex B))
   (prinl "A+B = " (fmtComplex (addComplex A B)))
   (prinl "A*B = " (fmtComplex (mulComplex A B)))
   (prinl "1/A = " (fmtComplex (invComplex A)))
   (prinl "-A = " (fmtComplex (negComplex A))) )

Output:

A = 1.00000+1.00000i
B = 3.14159+1.20000i
A+B = 4.14159+2.20000i
A*B = 1.94159+4.34159i
1/A = 0.50000-0.50000i
-A = -1.00000-1.00000i

\end{verbatim}

\section*{Arithmetic/Rational}

\begin{verbatim}

(load "@lib/frac.l")

(for (N 2  (> (** 2 19) N)  (inc N))
   (let (Sum (frac 1 N)  Lim (sqrt N))
      (for (F 2  (>= Lim F) (inc F))
         (when (=0 (\% N F))
            (setq Sum
               (f+ Sum
                  (f+ (frac 1 F) (frac 1 (/ N F))) ) ) ) )
      (when (= 1 (cdr Sum))
         (prinl
            "Perfect " N
            ", sum is " (car Sum)
            (and (= 1 (car Sum)) ": perfect") ) ) ) )

Output:

Perfect 6, sum is 1: perfect
Perfect 28, sum is 1: perfect
Perfect 120, sum is 2
Perfect 496, sum is 1: perfect
Perfect 672, sum is 2
Perfect 8128, sum is 1: perfect
Perfect 30240, sum is 3
Perfect 32760, sum is 3
Perfect 523776, sum is 2
\end{verbatim}

\section*{Arithmetic/Integer}

\begin{verbatim}

(de math (A B)
   (prinl "Add      " (+ A B))
   (prinl "Subtract " (- A B))
   (prinl "Multiply " (* A B))
   (prinl "Divide   " (/ A B))        # Trucates towards zero
   (prinl "Div/rnd  " (*/ A B))       # Rounds to next integer
   (prinl "Modulus  " (\% A B))        # Sign of the first operand
   (prinl "Power    " (** A B)) )

\end{verbatim}

\section*{Array concatenation}

\begin{verbatim}

PicoLisp has no built-in array data type. Lists are used instead.

There are destructive concatenations:

: (setq  A (1 2 3)  B '(a b c))
-> (a b c)
: (conc A B)                        # Concatenate lists in 'A' and 'B'
-> (1 2 3 a b c)
: A
-> (1 2 3 a b c)                    # Side effect: List in 'A' is modified!

and non-destructive concatenations:

: (setq  A (1 2 3)  B '(a b c))
-> (a b c)
: (append A B)                      # Append lists in 'A' and 'B'
-> (1 2 3 a b c)
: A
-> (1 2 3)
: B
-> (a b c)                          # Arguments are not modified

\end{verbatim}

\section*{Arrays}

\begin{verbatim}

PicoLisp has no built-in array data type. Lists are used instead.

(setq A '((1 2 3) (a b c) ((d e) NIL 777)))  # Create a 3x3 structure
(mapc println A)  # Show it

Output:

(1 2 3)
(a b c)
((d e) NIL 777)

Replace 'b' with 'B' in middle row:

(set (nth A 2 2) 'B)
(mapc println A)

Output:

(1 2 3)
(a B c)
((d e) NIL 777)

Insert '1' in front of the middle row:

(push (cdr A) 1)
(mapc println A)

Output:

(1 2 3)
(1 a B c)
((d e) NIL 777)

Append '9' to the middle row:

(queue (cdr A) 9)
(mapc println A)

Output:

(1 2 3)
(1 a B c 9)
((d e) NIL 777)

\end{verbatim}

\section*{Assertions}

\begin{verbatim}

The '[http://software-lab.de/doc/refA.html#assert assert]' function, in
combination with the tilde read macro, generates code only in debug mode:

...
~(assert (= N 42))  # Exists only in debug mode
...

Other possibilities are either to break into an error handler:

(let N 41
   (unless (= N 42) (quit "Incorrect N" N)) )  # 'quit' throws an error
41 -- Incorrect N
?

or to stop at a debug break point, allowing to continue with the program:

(let N 41
   (unless (= N 42) (! setq N 42)) )   # '!' is a breakpoint
(setq N 42)                            # Manually fix the value
!                                      # Hit ENTER to leave the breakpoint
-> 42

\end{verbatim}

\section*{Associative arrays/Creation}

\begin{verbatim}

Here we use symbol properties. Other possiblities could be index trees or
association lists.

(put 'A 'foo 5)
(put 'A 'bar 10)
(put 'A 'baz 15)
(put 'A 'foo 20)

: (get 'A 'bar)
-> 10

: (get 'A 'foo)
-> 20

: (show 'A)
A NIL
   foo 20
   bar 10
   baz 15

\end{verbatim}

\section*{Associative arrays/Iteration}

\begin{verbatim}

# Using properties

(put 'A 'foo 5)
(put 'A 'bar 10)
(put 'A 'baz 15)

: (getl 'A)                            # Get the whole property list
-> ((15 . baz) (10 . bar) (5 . foo))

: (mapcar cdr (getl 'A))               # Get all keys
-> (baz bar foo)

: (mapcar car (getl 'A))               # Get all values
-> (15 10 5)

# Using an index tree

(idx 'A (def "foo" 5) T)
(idx 'A (def "bar" 10) T)
(idx 'A (def "baz" 15) T)

: A                                    # Get the whole tree
-> ("foo" ("bar" NIL "baz"))

:  (idx 'A)                            # Get all keys
-> ("bar" "baz" "foo")

:  (mapcar val (idx 'A))               # Get all values
-> (10 15 5)

\end{verbatim}

\section*{Atomic updates}

\begin{verbatim}

(de *Buckets . 15)  # Number of buckets

# E/R model
(class +Bucket +Entity)
(rel key (+Key +Number))  # Key  1 .. *Buckets
(rel val (+Number))       # Value 1 .. 999


# Start with an empty DB
(call 'rm "-f" "buckets.db")  # Remove old DB (if any)
(pool "buckets.db")           # Create new DB file


# Create *Buckets buckets with values between 1 and 999
(for K *Buckets
   (new T '(+Bucket)  'key K  'val (rand 1 999)) )
(commit)


# Pick a random bucket
(de pickBucket ()
   (db 'key '+Bucket (rand 1 *Buckets)) )


# First process
(unless (fork)
   (seed *Pid)  # Ensure local random sequence
   (loop
      (let (B1 (pickBucket)  B2 (pickBucket))  # Pick two buckets 'B1' and 'B2'
         (dbSync)                              # Atomic DB operation
         (let (V1 (; B1 val)  V2 (; B2 val))   # Get current values
            (cond
               ((> V1 V2)
                  (dec> B1 'val)               # Make them closer to equal
                  (inc> B2 'val) )
               ((> V2 V1)
                  (dec> B2 'val)
                  (inc> B1 'val) ) ) )
         (commit 'upd) ) ) )                   # Close transaction

# Second process
(unless (fork)
   (seed *Pid)  # Ensure local random sequence
   (loop
      (let (B1 (pickBucket)  B2 (pickBucket))  # Pick two buckets 'B1' and 'B2'
         (unless (== B1 B2)                    # Found two different ones?
            (dbSync)                              # Atomic DB operation
            (let (V1 (; B1 val)  V2 (; B2 val))   # Get current values
               (cond
                  ((> V1 V2 0)
                     (inc> B1 'val)               # Redistribute them
                     (dec> B2 'val) )
                  ((> V2 V1 0)
                     (inc> B2 'val)
                     (dec> B1 'val) ) ) )
            (commit 'upd) ) ) ) )                 # Close transaction

# Third process
(unless (fork)
   (loop
      (dbSync)                         # Atomic DB operation
      (let Lst (collect 'key '+Bucket) # Get all buckets
         (for This Lst                 # Print current values
            (printsp (: val)) )
         (prinl                        # and total sum
            "-- Total: "
            (sum '((This) (: val)) Lst) ) )
      (rollback)
      (wait 2000) ) )                  # Sleep two seconds

(wait)

Output:

70 236 582 30 395 215 525 653 502 825 129 769 722 440 708 -- Total: 6801
0 156 566 352 198 263 0 743 0 1316 58 1180 897 0 1072 -- Total: 6801
0 0 424 101 0 0 0 682 0 1809 0 1549 961 0 1275 -- Total: 6801
0 0 0 0 0 0 0 452 0 2226 0 1838 884 0 1401 -- Total: 6801
54 55 56 55 54 55 54 102 54 2363 54 1816 666 55 1308 -- Total: 6801
198 198 197 196 198 198 197 197 196 1903 197 1438 345 197 946 -- Total: 6801
342 344 343 344 344 342 344 343 343 1278 343 992 343 343 413 -- Total: 6801
^C

\end{verbatim}

\section*{Averages/Arithmetic mean}

\begin{verbatim}

(de mean (Lst)
   (if (atom Lst)
      0
      (/ (apply + Lst) (length Lst)) ) )

Output:

: (mean (range 1 1000))
-> 500

\end{verbatim}

\section*{Averages/Median}

\begin{verbatim}

(de median (Lst)
   (let N (length Lst)
      (if (bit? 1 N)
         (get (sort Lst) (/ (inc N) 2))
         (setq Lst (nth (sort Lst) (/ N 2)))
         (/ (+ (car Lst) (cadr Lst)) 2) ) ) )

(scl 2)
(prinl (round (median (1.0 2.0 3.0))))
(prinl (round (median (1.0 2.0 3.0 4.0))))
(prinl (round (median (5.1 2.6 6.2 8.8 4.6 4.1))))
(prinl (round (median (5.1 2.6 8.8 4.6 4.1))))

Output:

2.00
2.50
4.85
4.60

\end{verbatim}

\section*{Averages/Mode}

\begin{verbatim}

(de modes (Lst)
   (let A NIL
      (for X Lst
         (accu 'A X 1) )
      (mapcar car
         (maxi cdar
            (by cdr group A) ) ) ) )

Output:

: (modes (1 3 6 6 6 6 7 7 12 12 17))
-> (6)

: (modes (1 1 2 4 4))
-> (4 1)

: (modes (chop "ABRAHAMASANTACLARA"))
-> ("A")

: (modes (1 4 A 3 2 7 1 B B 3 6 2 4 C C 5 2 5 B A 3 2 C 3 5 5 4 C 7 7))
-> (5 C 2 3)

\end{verbatim}

\section*{Averages/Pythagorean means}

\begin{verbatim}

(load "@lib/math.l")

(let (Lst (1.0 2.0 3.0 4.0 5.0 6.0 7.0 8.0 9.0 10.0)  Len (length Lst))
   (prinl "Arithmetic mean: "
      (format
         (/ (apply + Lst) Len)
         *Scl ) )
   (prinl "Geometric mean: "
      (format
         (pow (*/ (apply * Lst) (** 1.0 (dec Len))) (/ 1.0 Len))
         *Scl ) )
   (prinl "Harmonic mean: "
      (format
         (*/ (* 1.0 Len) 1.0 (sum '((N) (*/ 1.0 1.0 N)) Lst))
         *Scl ) ) )

Output:

Arithmetic mean: 5.500000
Geometric mean: 4.528729
Harmonic mean: 3.414172

\end{verbatim}

\section*{Averages/Root mean square}

\begin{verbatim}

(scl 5)

(let Lst (1.0 2.0 3.0 4.0 5.0 6.0 7.0 8.0 9.0 10.0)
   (prinl
      (format
         (sqrt
            (*/
               (sum '((N) (*/ N N 1.0)) Lst)
               1.0
               (length Lst) )
            T )
         *Scl ) ) )

Output:

6.20484

\end{verbatim}

\section*{Averages/Simple moving average}

\begin{verbatim}

(de sma (@Len)
   (curry (@Len (Data)) (N)
      (push 'Data N)
      (and (nth Data @Len) (con @))  # Truncate
      (*/ (apply + Data) (length Data)) ) )


(def 'sma3 (sma 3))
(def 'sma5 (sma 5))

(scl 2)
(for N (1.0 2.0 3.0 4.0 5.0 5.0 4.0 3.0 2.0 1.0)
   (prinl
      (format N *Scl)
      "   (sma3) "
      (format (sma3 N) *Scl)
      "   (sma5) "
      (format (sma5 N) *Scl) ) )

Output:

1.00   (sma3) 1.00   (sma5) 1.00
2.00   (sma3) 1.50   (sma5) 1.50
3.00   (sma3) 2.00   (sma5) 2.00
4.00   (sma3) 3.00   (sma5) 2.50
5.00   (sma3) 4.00   (sma5) 3.00
5.00   (sma3) 4.67   (sma5) 3.80
4.00   (sma3) 4.67   (sma5) 4.20
3.00   (sma3) 4.00   (sma5) 4.20
2.00   (sma3) 3.00   (sma5) 3.80
1.00   (sma3) 2.00   (sma5) 3.00

\end{verbatim}

\chapter{Rosetta Code Tasks starting with B}

\section*{Balanced brackets}

\begin{verbatim}

(load "@lib/simul.l")  # For 'shuffle'

(de generateBrackets (N)
   (shuffle (make (do N (link "[" "]")))) )

(de checkBrackets (S)
   (let N 0
      (for C S
         (if (= C "[")
            (inc 'N)
            (if2 (= C "]") (=0 N)
               (off N)
               (dec 'N) ) ) )
      (=0 N) ) )

(for N 10
   (prinl (if (checkBrackets (prin (generateBrackets N))) " OK" "not OK")) )

Output:

[] OK
[[]] OK
]]][[[not OK
[[[][]]] OK
[][][[[]]] OK
[]][[[][[]]]not OK
[[[]]][][][][] OK
]][][[[[]][]]][[not OK
[]][][[[][[]]][]][not OK
[[[][]]]]][][[]]][[[not OK

\end{verbatim}

\section*{Best shuffle}

\begin{verbatim}

(de bestShuffle (Str)
   (let Lst NIL
      (for C (setq Str (chop Str))
         (if (assoc C Lst)
            (con @ (cons C (cdr @)))
            (push 'Lst (cons C)) ) )
      (setq Lst (apply conc (flip (by length sort Lst))))
      (let Res
         (mapcar
            '((C)
               (prog1 (or (find <> Lst (circ C)) C)
                  (setq Lst (delete @ Lst)) ) )
            Str )
         (prinl Str " " Res " (" (cnt = Str Res) ")") ) ) )

Output:

: (bestShuffle "abracadabra")
abracadabra raarababadc (0)

: (bestShuffle "seesaw")
seesaw essewa (0)

: (bestShuffle "elk")
elk lke (0)

: (bestShuffle "grrrrrr")
grrrrrr rgrrrrr (5)

: (bestShuffle "up")
up pu (0)

: (bestShuffle "a")
a a (1)

\end{verbatim}

\section*{Binary digits}

\begin{verbatim}

: (bin 5)
-> "101"

: (bin 50)
-> "110010"

: (bin 9000)
-> "10001100101000"

\end{verbatim}

\section*{Binary search}

\begin{verbatim}

(de recursiveSearch (Val Lst Len)
   (unless (=0 Len)
      (let (N (inc (/ Len 2))  L (nth Lst N))
         (cond
            ((= Val (car L)) Val)
            ((> Val (car L))
               (recursiveSearch Val (cdr L) (- Len N)) )
            (T (recursiveSearch Val Lst (dec N))) ) ) ) )

Output:

: (recursiveSearch 5 (2 3 5 8 "abc" "klm" "xyz" (7) (a b)) 9)
-> 5
: (recursiveSearch '(a b) (2 3 5 8 "abc" "klm" "xyz" (7) (a b)) 9)
-> (a b)
: (recursiveSearch (9) (2 3 5 8 "abc" "klm" "xyz" (7) (a b)) 9)
-> NIL

(de iterativeSearch (Val Lst Len)
   (use (N L)
      (loop
         (T (=0 Len))
         (setq
            N (inc (/ Len 2))
            L (nth Lst N) )
         (T (= Val (car L)) Val)
         (if (> Val (car L))
            (setq Lst (cdr L)  Len (- Len N))
            (setq Len (dec N)) ) ) ) )

Output:

: (iterativeSearch 5 (2 3 5 8 "abc" "klm" "xyz" (7) (a b)) 9)
-> 5
: (iterativeSearch '(a b) (2 3 5 8 "abc" "klm" "xyz" (7) (a b)) 9)
-> (a b)
: (iterativeSearch (9) (2 3 5 8 "abc" "klm" "xyz" (7) (a b)) 9)
-> NIL

\end{verbatim}

\section*{Binary strings}

\begin{verbatim}

Byte strings are represented in PicoLisp as lists of numbers. They can be
maniplated easily with the built-in list functionality.

I/O of raw bytes is done via the 'wr' (write) and 'rd' (read) functions. The
following creates a file consisting of 256 bytes, with values from 0 to 255:

: (out "rawfile"
   (mapc wr (range 0 255)) )

Looking at a hex dump of that file:

: (hd "rawfile")
00000000  00 01 02 03 04 05 06 07 08 09 0A 0B 0C 0D 0E 0F  ................
00000010  10 11 12 13 14 15 16 17 18 19 1A 1B 1C 1D 1E 1F  ................
00000020  20 21 22 23 24 25 26 27 28 29 2A 2B 2C 2D 2E 2F   !"#\$\%\&'()*+,-./
00000030  30 31 32 33 34 35 36 37 38 39 3A 3B 3C 3D 3E 3F  0123456789:;<=>?
...


To read part of that file, an external tool like 'dd' might be used:

: (in '(dd "skip=32" "bs=1" "count=16" "if=rawfile")
   (make
      (while (rd 1)
         (link @) ) ) )
-> (32 33 34 35 36 37 38 39 40 41 42 43 44 45 46 47)

Now such byte lists can be assigned the normal way ('let', 'setq' etc.), they
can be compared with '=', '>', '>=' etc, and manipulated with all internal map-,
filter-, concatenation-, reversal-, pattern matching, and other functions.

If desired, a string containing meaningful values can also be converted to
a transient symbol, e.g. the example above

: (pack (mapcar char (32 33 34 35 36 37 38 39 40 41 42 43 44 45 46 47)))
-> " !\"#\$\%\&'()*+,-./"

\end{verbatim}

\section*{Bitmap}

\begin{verbatim}

For time critical applications this would be done with inline-C in PicoLisp,
but especially for small bitmaps the following makes sense.

# Create an empty image of 120 x 90 pixels
(setq *Ppm (make (do 90 (link (need 120)))))

# Fill an image with a given color
(de ppmFill (Ppm R G B)
   (for Y Ppm
      (map
         '((X) (set X (list R G B)))
         Y ) ) )

# Set pixel with a color
(de ppmSetPixel (Ppm X Y R G B)
   (set (nth Ppm Y X) (list R G B)) )

# Get the color of a pixel
(de ppmGetPixel (Ppm X Y)
   (get Ppm Y X) )

\end{verbatim}

\section*{Bitmap/Bitmap/Bézier curves/Cubic}

\begin{verbatim}

This uses the 'brez' line drawing function from
[[Bitmap/Bresenham's line algorithm#PicoLisp]].

(scl 6)

(de cubicBezier (Img N X1 Y1 X2 Y2 X3 Y3 X4 Y4)
   (let (R (* N N N)  X X1  Y Y1  DX 0  DY 0)
      (for I N
         (let
            (J (- N I)
               A (*/ 1.0 J J J R)
               B (*/ 3.0 I J J R)
               C (*/ 3.0 I I J R)
               D (*/ 1.0 I I I R) )
            (brez Img
               X
               Y
               (setq DX
                  (-
                     (+ (*/ A X1 1.0) (*/ B X2 1.0) (*/ C X3 1.0) (*/ D X4 1.0))
                     X ) )
               (setq DY
                  (-
                     (+ (*/ A Y1 1.0) (*/ B Y2 1.0) (*/ C Y3 1.0) (*/ D Y4 1.0))
                     Y ) ) )
            (inc 'X DX)
            (inc 'Y DY) ) ) ) )

Test:

(let Img (make (do 200 (link (need 300 0))))       # Create image 300 x 200
   (cubicBezier Img 24 20 120 540 33 -225 33 285 100)
   (out "img.pbm"                                  # Write to bitmap file
      (prinl "P1")
      (prinl 300 " " 200)
      (mapc prinl Img) ) )

(call 'display "img.pbm")

\end{verbatim}

\section*{Bitmap/Bitmap/Bézier curves/Quadratic}

\begin{verbatim}

This uses the 'brez' line drawing function from
[[Bitmap/Bresenham's line algorithm#PicoLisp]].

(scl 6)

(de quadBezier (Img N X1 Y1 X2 Y2 X3 Y3)
   (let (R (* N N)  X X1  Y Y1  DX 0  DY 0)
      (for I N
         (let (J (- N I)  A (*/ 1.0 J J R)  B (*/ 2.0 I J R)  C (*/ 1.0 I I R))
            (brez Img X Y
               (setq DX (- (+ (*/ A X1 1.0) (*/ B X2 1.0) (*/ C X3 1.0)) X))
               (setq DY (- (+ (*/ A Y1 1.0) (*/ B Y2 1.0) (*/ C Y3 1.0)) Y)) )
            (inc 'X DX)
            (inc 'Y DY) ) ) ) )

Test:

(let Img (make (do 200 (link (need 300 0))))       # Create image 300 x 200
   (quadBezier Img 12 20 100 300 -80 260 180)
   (out "img.pbm"                                  # Write to bitmap file
      (prinl "P1")
      (prinl 300 " " 200)
      (mapc prinl Img) ) )

(call 'display "img.pbm")

\end{verbatim}

\section*{Bitmap/Bresenham's line algorithm}

\begin{verbatim}

(de brez (Img X Y DX DY)
   (let SX
      (cond
         ((=0 DX) 0)
         ((gt0 DX) 1)
         (T (setq DX (- DX)) -1) )
      (let SY
         (cond
            ((=0 DY) 0)
            ((gt0 DY) 1)
            (T (setq DY (- DY)) -1) )
         (if (>= DX DY)
            (let E (- (* 2 DY) DX)
               (do DX
                  (set (nth Img Y X) 1)
                  (when (ge0 E)
                     (inc 'Y SY)
                     (dec 'E (* 2 DX)) )
                  (inc 'X SX)
                  (inc 'E (* 2 DY)) ) )
            (let E (- (* 2 DX) DY)
               (do DY
                  (set (nth Img Y X) 1)
                  (when (ge0 E)
                     (inc 'X SX)
                     (dec 'E (* 2 DY)) )
                  (inc 'Y SY)
                  (inc 'E (* 2 DX)) ) ) ) ) ) )

(let Img (make (do 90 (link (need 120 0))))        # Create image 120 x 90
   (brez Img 10 10 100 30)                         # Draw five lines
   (brez Img 10 10 100 50)
   (brez Img 10 10 100 70)
   (brez Img 10 10 60 70)
   (brez Img 10 10 20 70)
   (out "img.pbm"                                  # Write to bitmap file
      (prinl "P1")
      (prinl 120 " " 90)
      (mapc prinl Img) ) )

\end{verbatim}

\section*{Bitmap/Flood fill}

\begin{verbatim}

Using the format of [[Bitmap#PicoLisp|Bitmap]], a minimal recursive solution:

(de ppmFloodFill (Ppm X Y Color)
   (let Target (get Ppm Y X)
      (recur (X Y)
         (when (= Target (get Ppm Y X))
            (set (nth Ppm Y X) Color)
            (recurse (dec X) Y)
            (recurse (inc X) Y)
            (recurse X (dec Y))
            (recurse X (inc Y)) ) ) )
   Ppm )

Test using 'ppmRead' from [[Bitmap/Read a PPM file#PicoLisp]] and 'ppmWrite'
from [[Bitmap/Write a PPM file#PicoLisp]], filling the white area with red:

(ppmWrite
   (ppmFloodFill (ppmRead "Unfilledcirc.ppm") 192 128 (255 0 0))
   "Filledcirc.ppm" )

\end{verbatim}

\section*{Bitmap/Histogram}

\begin{verbatim}

{{trans|Forth}}

(de histogram (Pgm)
   (let H (need 256 0)
      (for L Pgm
         (for G L
            (inc (nth H (inc G))) ) )
      H ) )

\end{verbatim}

\section*{Bitmap/Midpoint circle algorithm}

\begin{verbatim}

(de midPtCircle (Img CX CY Rad)
   (let (F (- 1 Rad)  DdFx 0  DdFy (* -2 Rad)  X 0  Y Rad)
      (set (nth Img (+ CY Rad) CX) 1)
      (set (nth Img (- CY Rad) CX) 1)
      (set (nth Img CY (+ CX Rad)) 1)
      (set (nth Img CY (- CX Rad)) 1)
      (while (> Y X)
         (when (ge0 F)
            (dec 'Y)
            (inc 'F (inc 'DdFy 2)) )
         (inc 'X)
         (inc 'F (inc (inc 'DdFx 2)))
         (set (nth Img (+ CY Y) (+ CX X)) 1)
         (set (nth Img (+ CY Y) (- CX X)) 1)
         (set (nth Img (- CY Y) (+ CX X)) 1)
         (set (nth Img (- CY Y) (- CX X)) 1)
         (set (nth Img (+ CY X) (+ CX Y)) 1)
         (set (nth Img (+ CY X) (- CX Y)) 1)
         (set (nth Img (- CY X) (+ CX Y)) 1)
         (set (nth Img (- CY X) (- CX Y)) 1) ) ) )

(let Img (make (do 120 (link (need 120 0))))       # Create image 120 x 120
   (midPtCircle Img 60 60 50)                      # Draw circle
   (out "img.pbm"                                  # Write to bitmap file
      (prinl "P1")
      (prinl 120 " " 120)
      (mapc prinl Img) ) )

\end{verbatim}

\section*{Bitmap/PPM conversion through a pipe}

\begin{verbatim}

# Create an empty image of 120 x 90 pixels
(setq *Ppm (make (do 90 (link (need 120)))))

# Fill background with green color
(ppmFill *Ppm 0 255 0)

# Draw a diagonal line
(for I 80 (ppmSetPixel *Ppm I I 0 0 0))

# Write to "img.jpg" through a pipe
(ppmWrite *Ppm '("convert" "-" "img.jpg"))

\end{verbatim}

\section*{Bitmap/Read a PPM file}

\begin{verbatim}

(de ppmRead (File)
   (in File
      (unless (and `(hex "5036") (rd 2))  # P6
         (quit "Wrong file format" File) )
      (rd 1)
      (let (DX 0  DY 0  Max 0  C)
         (while (>= 9 (setq C (- (rd 1) `(char "0"))) 0)
            (setq DX (+ (* 10 DX) C)) )
         (while (>= 9 (setq C (- (rd 1) `(char "0"))) 0)
            (setq DY (+ (* 10 DY) C)) )
         (while (>= 9 (setq C (- (rd 1) `(char "0"))) 0)
            (setq Max (+ (* 10 Max) C)) )
         (prog1
            (make (do DY (link (need DX))))
            (for Y @
               (map
                  '((X) (set X (list (rd 1) (rd 1) (rd 1))))
                  Y ) ) ) ) ) )

Read a color image "img.ppm", convert and write to "img.pgm":

(pgmWrite (ppm->pgm (ppmRead "img.ppm")) "img.pgm")

\end{verbatim}

\section*{Bitmap/Read an image through a pipe}

\begin{verbatim}

(setq *Ppm (ppmRead '("convert" "img.jpg" "ppm:-")))

\end{verbatim}

\section*{Bitmap/Write a PPM file}

\begin{verbatim}

(de ppmWrite (Ppm File)
   (out File
      (prinl "P6")
      (prinl (length (car Ppm)) " " (length Ppm))
      (prinl 255)
      (for Y Ppm (for X Y (apply wr X))) ) )

\end{verbatim}

\section*{Bitwise IO}

\begin{verbatim}

(de write7bitwise (Lst)
   (let (Bits 0  Byte)
      (for N Lst
         (if (=0 Bits)
            (setq Bits 7  Byte (* 2 N))
            (wr (| Byte (>> (dec 'Bits) N)))
            (setq Byte (>> (- Bits 8) N)) ) )
      (unless (=0 Bits)
         (wr Byte) ) ) )

(de read7bitwise ()
   (make
      (let (Bits 0  Byte)
         (while (rd 1)
            (let N @
               (link
                  (if (=0 Bits)
                     (>> (one Bits) N)
                     (| Byte (>> (inc 'Bits) N)) ) )
               (setq Byte (\& 127 (>> (- Bits 7) N))) ) )
         (when (= 7 Bits)
            (link Byte) ) ) ) )


(out 'a (write7bitwise (127 0 127 0 127 0 127 0 127)))
(hd 'a)
(in 'a (println (read7bitwise)))

(out 'a (write7bitwise (0 127 0 127 0 127 0 127 0)))
(hd 'a)
(in 'a (println (read7bitwise)))

(out 'a (write7bitwise (mapcar char (chop "STRING"))))
(hd 'a)
(println (mapcar char (in 'a (read7bitwise))))

Output:

00000000  FE 03 F8 0F E0 3F 80 FE                          .....?..
(127 0 127 0 127 0 127 0)
00000000  01 FC 07 F0 1F C0 7F 00                          .......
(0 127 0 127 0 127 0 127)
00000000  A7 52 94 99 D1 C0                                .R....
("S" "T" "R" "I" "N" "G")

\end{verbatim}

\section*{Bitwise operations}

\begin{verbatim}

PicoLisp has no specific word size. Numbers grow to arbitrary length. Therefore,
bitwise NOT, logical (non-arithmetic) SHIFTs, and rotate operations do not make
sense.

Bitwise AND:

: (\& 6 3)
-> 2

: (\& 7 3 1)
-> 1

Bitwise AND-Test (tests if all bits in the first argument are set in the
following arguments):

: (bit? 1 2)
-> NIL

: (bit? 6 3)
-> NIL

: (bit? 6 15 255)
-> 6

Bitwise OR:

: (| 1 2)
-> 3

: (| 1 2 4 8)
-> 15

Bitwise XOR:

: (x| 2 7)
-> 5

: (x| 2 7 1)
-> 4

Shift (right with a positive count, left with a negative count):

: (>> 1 8)
-> 4

: (>> 3 16)
-> 2

: (>> -3 16)
-> 128

: (>> -1 -16)
-> -32

\end{verbatim}

\section*{Boolean values}

\begin{verbatim}

Like in all Lisps, the symbol 'NIL' denotes "false", any other value "true".

Some functions return the symbol 'T' for "true" if no other useful (non-NIL)
value is available in the given context. Note that 'NIL' and 'T' are written in
uppercase letters (PicoLisp is case-sensitive).

\end{verbatim}

\section*{Boxing the compass}

\begin{verbatim}

(scl 3)

(setq *Compass                      # Build lookup table
   (let H -16.875
      (mapcar
         '((Str)
            (cons
               (inc 'H 11.25)       # Heading in degrees
               (pack                # Compass point
                  (replace (chop Str)
                     "N" "north"
                     "E" "east"
                     "S" "south"
                     "W" "west"
                     "b" " by " ) ) ) )
         '("N" "NbE" "N-NE" "NEbN" "NE" "NEbE" "E-NE" "EbN"
            "E" "EbS" "E-SE" "SEbE" "SE" "SEbS" "S-SE" "SbE"
            "S" "SbW" "S-SW" "SWbS" "SW" "SWbW" "W-SW" "WbS"
            "W" "WbN" "W-NW" "NWbW" "NW" "NWbN" "N-NW" "NbW"
            "N" ) ) ) )

(de heading (Deg)
   (rank (\% Deg 360.00) *Compass) )

(for I (range 0 32)
   (let H (* I 11.25)
      (case (\% I 3)
         (1 (inc 'H 5.62))
         (2 (dec 'H 5.62)) )
      (tab (3 1 -18 8)
         (inc (\% I 32))
         NIL
         (cdr (heading H))
         (round H 2) ) ) )

Output:

  1 north                 0.00
  2 north by east        16.87
  3 north-northeast      16.88
  4 northeast by north   33.75
  5 northeast            50.62
  6 northeast by east    50.63
  7 east-northeast       67.50
  8 east by north        84.37
  9 east                 84.38
 10 east by south       101.25
 11 east-southeast      118.12
 12 southeast by east   118.13
 13 southeast           135.00
 14 southeast by south  151.87
 15 south-southeast     151.88
 16 south by east       168.75
 17 south               185.62
 18 south by west       185.63
 19 south-southwest     202.50
 20 southwest by south  219.37
 21 southwest           219.38
 22 southwest by west   236.25
 23 west-southwest      253.12
 24 west by south       253.13
 25 west                270.00
 26 west by north       286.87
 27 west-northwest      286.88
 28 northwest by west   303.75
 29 northwest           320.62
 30 northwest by north  320.63
 31 north-northwest     337.50
 32 north by west       354.37
  1 north               354.38

\end{verbatim}

\section*{Break OO privacy}

\begin{verbatim}

PicoLisp uses [http://software-lab.de/doc/ref.html#transient "transient
symbols"] for variables, functions, methods etc. inaccessible from other parts
of the program. Lexically, a transient symbol is enclosed by double quotes.

The only way to access a transient symbol outside its namespace is to search for
its name in other (public) structures. This is done by the
'[http://software-lab.de/doc/refL.html#loc loc]' function.

(class +Example)
# "_name"

(dm T (Name)
   (=: "_name" Name) )

(dm string> ()
   (pack "Hello, I am " (: "_name")) )

(====)  # Close transient scope

(setq Foo (new '(+Example) "Eric"))

Test:

: (string> Foo)                        # Access via method call
-> "Hello, I am Eric"

: (get Foo '"_name")                   # Direct access doesn't work
-> NIL

: (get Foo (loc "_name" +Example))     # Locating the transient symbol works
-> "Eric"

: (put Foo (loc "_name" +Example) "Edith")
-> "Edith"

: (string> Foo)                        # Ditto
-> "Hello, I am Edith"

: (get Foo '"_name")
-> NIL

: (get Foo (loc "_name" +Example))
-> "Edith"

\end{verbatim}

\section*{Brownian tree}

\begin{verbatim}

(load "@lib/simul.l")

(de brownianTree (File Size Cnt)
   (let Img (grid Size Size)
      (put Img (/ Size 2) (/ Size 2) 'pix T)
      (use (P Q)
         (do Cnt
            (setq P (get Img (rand 1 Size) (rand 1 Size)))
            (loop
               (setq Q ((if2 (rand T) (rand T) north east south west) P))
               (T (; Q pix) (put P 'pix T))
               (setq P (or Q (get Img (rand 1 Size) (rand 1 Size)))) ) ) )
      (out "img.pbm"
         (prinl "P1")
         (prinl Size " " Size)
         (for L Img
            (for This L
               (prin (if (: pix) 1 0)) )
            (prinl) ) ) ) )

Use:

(brownianTree "img.pbm" 300 9000)
(call 'display "img.pbm")

\end{verbatim}

\section*{Bulls and cows}

\begin{verbatim}

{IanOsgood}

\end{verbatim}

\section*{Bulls and cows/Player}

\begin{verbatim}

(load "@lib/simul.l")

(de bullsAndCows ()
   (let Choices (shuffle (mapcan permute (subsets 4 (range 1 9))))
      (use (Guess Bulls Cows)
         (loop
            (prinl "Guessing " (setq Guess (pop 'Choices)))
            (prin "How many bulls and cows? ")
            (setq Bulls (read)  Cows (read))
            (setq Choices
               (filter
                  '((C)
                     (let B (cnt = Guess C)
                        (and
                           (= Bulls B)
                           (= Cows (- (length (sect Guess C)) B)) ) ) )
                  Choices ) )
            (NIL Choices "No matching solution")
            (NIL (cdr Choices) (pack "The answer is " (car Choices))) ) ) ) )

Output:

: (bullsAndCows)
Guessing 4217
How many bulls and cows? 0 2
Guessing 5762
How many bulls and cows? 1 1
Guessing 9372
How many bulls and cows? 0 1
Guessing 7864
How many bulls and cows? 1 2
Guessing 8754
How many bulls and cows? 0 2
-> "The answer is 2468"

\end{verbatim}

\chapter{Rosetta Code Tasks starting with C}

\section*{CSV to HTML translation}

\begin{verbatim}

Simple solution

(load "@lib/http.l")

(in "text.csv"
   (<table> 'myStyle NIL NIL
      (prinl)
      (while (split (line) ",")
         (<row> NIL (ht:Prin (pack (car @))) (ht:Prin (pack (cadr @))))
         (prinl) ) ) )

Output:

<table class="myStyle">
<tr><td>Character</td><td>Speech</td></tr>
<tr><td>The multitude</td><td>The messiah! Show us the messiah!</td></tr>
<tr><td>Brians mother</td><td>\&lt;angry\&gt;Now you listen here! He's not the messiah; he's a very naughty boy! Now go away!\&lt;/angry\&gt;</td></tr>
<tr><td>The multitude</td><td>Who are you?</td></tr>
<tr><td>Brians mother</td><td>I'm his mother; that's who!</td></tr>
<tr><td>The multitude</td><td>Behold his mother! Behold his mother!</td></tr>
</table>

Extra credit solution

(load "@lib/http.l")

(in "text.csv"
   (when (split (line) ",")
      (<table> 'myStyle NIL (mapcar '((S) (list NIL (pack S))) @)
         (prinl)
         (while (split (line) ",")
            (<row> NIL (ht:Prin (pack (car @))) (ht:Prin (pack (cadr @))))
            (prinl) ) ) ) )

Output:

<table class="myStyle"><tr><th>Character</th><th>Speech</th></tr>
<tr><td>The multitude</td><td>The messiah! Show us the messiah!</td></tr>
<tr><td>Brians mother</td><td>\&lt;angry\&gt;Now you listen here! He's not the messiah; he's a very naughty boy! Now go away!\&lt;/angry\&gt;</td></tr>
<tr><td>The multitude</td><td>Who are you?</td></tr>
<tr><td>Brians mother</td><td>I'm his mother; that's who!</td></tr>
<tr><td>The multitude</td><td>Behold his mother! Behold his mother!</td></tr>
</table>

\end{verbatim}

\section*{Caesar cipher}

\begin{verbatim}

(setq *Letters (apply circ (mapcar char (range 65 90))))

(de caesar (Str Key)
   (pack
      (mapcar '((C) (cadr (nth (member C *Letters) Key)))
         (chop (uppc Str)) ) ) )

Test:

: (caesar "IBM" 25)
-> "HAL"
: (caesar @ 1)
-> "IBM"

: (caesar "The quick brown fox jumped over the lazy dog's back" 7)
-> "AOLXBPJRIYVDUMVEQBTWLKVCLYAOLSHGFKVNZIHJR"
: (caesar @ (- 26 7))
-> "THEQUICKBROWNFOXJUMPEDOVERTHELAZYDOGSBACK"

\end{verbatim}

\section*{Calendar}

\begin{verbatim}

This "calendar" is nicely formated, and fits into 20 columns ;-)

(de cal (Year)
   (prinl "====== " Year " ======")
   (for Dat (range (date Year 1 1) (date Year 12 31))
      (let D (date Dat)
         (tab (3 3 4 8)
            (when (= 1 (caddr D))
               (get *Mon (cadr D)) )
            (caddr D)
            (day Dat *Day)
            (when (=0 (\% (inc Dat) 7))
               (pack "Week " (week Dat)) ) ) ) ) )

(cal 1969)

Output:

====== 1969 ======
Jan  1 Wed
     2 Thu
     3 Fri
     4 Sat
     5 Sun
     6 Mon  Week 2
     7 Tue
....
    28 Sat
    29 Sun
    30 Mon Week 27
Jul  1 Tue
     2 Wed
     3 Thu
     4 Fri
....
    25 Thu
    26 Fri
    27 Sat
    28 Sun
    29 Mon Week 53
    30 Tue
    31 Wed

\end{verbatim}

\section*{Calendar - for "real" programmers}

\begin{verbatim}

The "CALENDAR.L" source file:

(DE CAL (YEAR)
   (PRINL "====== " YEAR " ======")
   (FOR DAT (RANGE (DATE YEAR 1 1) (DATE YEAR 12 31))
      (LET D (DATE DAT)
         (TAB (3 3 4 8)
            (WHEN (= 1 (CADDR D))
               (GET `(INTERN (PACK (MAPCAR CHAR (42 77 111 110)))) (CADR D)) )
            (CADDR D)
            (DAY DAT `(INTERN (PACK (MAPCAR CHAR (42 68 97 121)))))
            (WHEN (=0 (\% (INC DAT) 7))
               (PACK (CHAR 87) "EEk " (WEEK DAT)) ) ) ) ) )

(CAL 1969)
(BYE)

Then it can be executed with this command line:

\$ pil -'load (list "awk" "{print tolower(\$0)}" "CALENDAR.L")'

Output:

====== 1969 ======
Jan  1 Wed
     2 Thu
     3 Fri
     4 Sat
     5 Sun
     6 Mon  Week 2
     7 Tue
....
    28 Sat
    29 Sun
    30 Mon Week 27
Jul  1 Tue
     2 Wed
     3 Thu
     4 Fri
....
    25 Thu
    26 Fri
    27 Sat
    28 Sun
    29 Mon Week 53
    30 Tue
    31 Wed

\end{verbatim}

\section*{Call a foreign-language function}

\begin{verbatim}

The easiest is to inline the C code. Another possibility would be to write it
into a separate shared object file (see "Call a function in a shared library").

There are differences between the 32-bit and 64-bit versions. While the 64-bit
can interface directly to C functions, requires the 32-bit function some glue
code.

# 32-bit version

(load "@lib/gcc.l")

(gcc "str" NIL                # The 'gcc' function passes all text
   'duptest )                 # until /**/ to the C compiler

any duptest(any ex) {
   any x = evSym(cdr(ex));    // Accept a symbol (string)
   char str[bufSize(x)];      // Create a buffer to unpack the name
   char *s;

   bufString(x, str);         // Upack the string
   s = strdup(str);           // Make a duplicate
   x = mkStr(s);              // Build a new Lisp string
   free(s);                   // Dispose the duplicate
   return x;
}
/**/

(println 'Duplicate (duptest "Hello world!"))

# 64-bit version

(load "@lib/native.l")

(gcc "str" NIL
   (duptest (Str) duptest 'S Str) )

#include <stdlib.h>
#include <string.h>

char *duptest(char *str) {
   static char *s;

   if (s)         // To avoid having to worry about free(),
      free(s);    // We simply dispose the result of the last call
   return s = strdup(str);
}
/**/

(println 'Duplicate (duptest "Hello world!"))

Output in both cases:

Duplicate "Hello world!"

\end{verbatim}

\section*{Call a function}

\begin{verbatim}

When calling a funcion in PicoLisp directly (does this mean "in a statement
context"?), it is always surrounded by parentheses, with or without arguments,
and for any kind of arguments (evaluated or not):

(foo)
(bar 1 'arg 2 'mumble)

When a function is used in a "first class context" (e.g. passed to another
function), then it is not yet _called_. It is simply _used_. Technically, a
function can be either a _number_ (a built-in function) or a _list_ (a
Lisp-level function) in PicoLisp):

(mapc println Lst)  # The value of 'printlin' is a number
(apply '((A B C) (foo (+ A (* B C)))) (3 5 7))  # A list is passed

Any argument to a function may be evaluated or not, depending on the function.
For example, 'setq' evaluates every second argument

(setq A (+ 3 4)  B (* 3 4))

i.e. the first argument 'A' is not evaluated, the second evaluates to 7, 'B' is
not evaluated, then the fourth evaluates to 12.

\end{verbatim}

\section*{Call a function from a foreign language}

\begin{verbatim}

Calling a PicoLisp function from another program requires a running interpreter.
There are several possibilities, like IPC via fifo's or sockets using the PLIO
(PicoLisp-I/O) protocol, but the easiest is calling the interpreter in a pipe.
This is relatively efficient, as the interpreter's startup time is quite short.

If there is a file "query.l"

(let (Str "Here am I"  Len (format (opt)))  # Get length from command line
   (unless (>= (size Str) Len)              # Check buffer size
      (prinl Str) ) )                       # Return string if OK

then the C function 'Query' could be

int Query(char *Data, size_t *Length) {
   FILE *fp;
   char buf[64];

   sprintf(buf, "/usr/bin/picolisp query.l \%d -bye", *Length);
   if (!(fp = popen(buf, "r")))
      return 0;
   fgets(Data, *Length, fp);
   *Length = strlen(Data);
   return pclose(fp) >= 0 \&\& *Length != 0;
}

\end{verbatim}

\section*{Call a function in a shared library}

\begin{verbatim}

This differs between the 32-bit and 64-bit versions. While the 64-bit version
can interface directly to C functions (in external libraries or not), requires
the 32-bit function some glue code.

For the 32-bit version, we need some glue code:

(load "@lib/gcc.l")

(gcc "x11" '("-lX11") 'xOpenDisplay 'xCloseDisplay)

#include <X11/Xlib.h>

any xOpenDisplay(any ex) {
   any x = evSym(cdr(ex));    // Get display name
   char display[bufSize(x)];  // Create a buffer for the name

   bufString(x, display);     // Upack the name
   return boxCnt((long)XOpenDisplay(display));
}

any xCloseDisplay(any ex) {
   return boxCnt(XCloseDisplay((Display*)evCnt(ex, cdr(ex))));
}
/**/

# With that we can open and close the display:
: (setq Display (xOpenDisplay ":0.7"))   # Wrong
-> 0
: (setq Display (xOpenDisplay ":0.0"))   # Correct
-> 158094320
: (xCloseDisplay Display)
-> 0

In the 64-bit version, we can call the library directly:

: (setq Display (native "/usr/lib/libX11.so.6" "XOpenDisplay" 'N ":0.0"))
-> 6502688
: (native "/usr/lib/libX11.so.6" "XCloseDisplay" 'I Display)
-> 0

\end{verbatim}

\section*{Catalan numbers}

\begin{verbatim}

# Factorial
(de fact (N)
   (if (=0 N)
      1
      (* N (fact (dec N))) ) )

# Directly
(de catalanDir (N)
   (/ (fact (* 2 N)) (fact (inc N)) (fact N)) )

# Recursively
(de catalanRec (N)
   (if (=0 N)
      1
      (cache '(NIL) (pack (char (hash N)) N)  # Memoize
         (sum
            '((I) (* (catalanRec I) (catalanRec (- N I 1))))
            (range 0 (dec N)) ) ) ) )

# Alternatively
(de catalanAlt (N)
   (if (=0 N)
      1
      (*/ 2 (dec (* 2 N)) (catalanAlt (dec N)) (inc N)) ) )

# Test
(for (N 0 (> 15 N) (inc N))
   (tab (2 4 8 8 8)
      N
      " => "
      (catalanDir N)
      (catalanRec N)
      (catalanAlt N) ) )

Output:

 0 =>        1       1       1
 1 =>        1       1       1
 2 =>        2       2       2
 3 =>        5       5       5
 4 =>       14      14      14
 5 =>       42      42      42
 6 =>      132     132     132
 7 =>      429     429     429
 8 =>     1430    1430    1430
 9 =>     4862    4862    4862
10 =>    16796   16796   16796
11 =>    58786   58786   58786
12 =>   208012  208012  208012
13 =>   742900  742900  742900
14 =>  2674440 2674440 2674440

\end{verbatim}

\section*{Character codes}

\begin{verbatim}

: (char "a")
-> 97
: (char "字")
-> 23383
: (char 23383)
-> "字"
: (chop "文字")
-> ("文" "字")
: (mapcar char @)
-> (25991 23383)

\end{verbatim}

\section*{Character matching}

\begin{verbatim}

: (pre? "ab" "abcd")
-> "abcd"
: (pre? "xy" "abcd")
-> NIL

: (sub? "bc" "abcd")
-> "abcd"
: (sub? "xy" "abcd")
-> NIL

: (tail (chop "cd") (chop "abcd"))
-> ("c" "d")
: (tail (chop "xy") (chop "abcd"))
-> NIL


(de positions (Pat Str)
   (setq Pat (chop Pat))
   (make
      (for ((I . L) (chop Str) L (cdr L))
         (and (head Pat L) (link I)) ) ) )

: (positions "bc" "abcdabcd")
-> (2 6)

\end{verbatim}

\section*{Chat server (!)}

\begin{verbatim}

{misc/chat}

Output:

After starting the above script, connect to the chat server from two terminals:

           Terminal 1            |           Terminal 2
---------------------------------+---------------------------------
\$ telnet localhost 4004          |
Trying ::1...                    |
Trying 127.0.0.1...              |
Connected to localhost.          |
Escape character is '^]'.        |
Please enter your name: Ben      |
                                 | \$ telnet localhost 4004
                                 | Trying ::1...
                                 | Trying 127.0.0.1...
                                 | Connected to localhost.
                                 | Escape character is '^]'.
                                 | Please enter your name: Tom
+++ Tom arrived +++              |
Hi Tom                           |
                                 | Ben> Hi Tom
                                 | Hi Ben
Tom> Hi Ben                      |
                                 | How are you?
Tom> How are you?                |
Thanks, fine!                    |
                                 | Ben> Thanks, fine!
                                 | See you!
Tom> See you!                    |
                                 | ^]
                                 | telnet> quit
--- Tom left ---                 |
                                 | Connection closed.
                                 | \$

\end{verbatim}

\section*{Checkpoint synchronization (!)}

\begin{verbatim}

The following solution implements each worker as a coroutine. Therefore, it
works only in the 64-bit version.

'checkpoints' takes a number of projects to do, and a number of workers. Each
worker is started with a random number of steps to do (between 2 and 5), and is
kept in a list of 'Staff' members. Whenever a worker finishes, he is removed
from that list, until it is empty and the project is done.

'worker' takes a number of steps to perform. It "works" by printing each step,
and returning NIL when done.

(de checkpoints (Projects Workers)
   (for P Projects
      (prinl "Starting project number " P ":")
      (for
         (Staff
            (mapcar
               '((I) (worker (format I) (rand 2 5)))  # Create staff of workers
               (range 1 Workers) )
            Staff                                     # While still busy
            (filter worker Staff) ) )                 # Remove finished workers
      (prinl "Project number " P " is done.") ) )

(de worker (ID Steps)
   (co ID
      (prinl "Worker " ID " has " Steps " steps to do")
      (for N Steps
         (yield ID)
         (prinl "Worker " ID " step " N) )
      NIL ) )

Output:

: (checkpoints 2 3)  # Start two projects with 3 workers
Starting project number 1:
Worker 1 has 2 steps to do
Worker 2 has 3 steps to do
Worker 3 has 5 steps to do
Worker 1 step 1
Worker 2 step 1
Worker 3 step 1
Worker 1 step 2
Worker 2 step 2
Worker 3 step 2
Worker 2 step 3
Worker 3 step 3
Worker 3 step 4
Worker 3 step 5
Project number 1 is done.
Starting project number 2:
Worker 1 has 4 steps to do
Worker 2 has 3 steps to do
Worker 3 has 2 steps to do
Worker 1 step 1
Worker 2 step 1
Worker 3 step 1
Worker 1 step 2
Worker 2 step 2
Worker 3 step 2
Worker 1 step 3
Worker 2 step 3
Worker 1 step 4
Project number 2 is done.

\end{verbatim}

\section*{Chess player}

\begin{verbatim}

See [[Chess player/PicoLisp]].

This implementation supports all chess rules (including castling, pawn promotion
and en passant), switching sides, unlimited undo/redo, and the setup, saving and
loading of board positions to/from files.

{games/chess.l}

Start:

\$ pil chess.l -main +
   +---+---+---+---+---+---+---+---+
 8 |<R>|<N>|<B>|<Q>|<K>|<B>|<N>|<R>|
   +---+---+---+---+---+---+---+---+
 7 |<P>|<P>|<P>|<P>|<P>|<P>|<P>|<P>|
   +---+---+---+---+---+---+---+---+
 6 |   | - |   | - |   | - |   | - |
   +---+---+---+---+---+---+---+---+
 5 | - |   | - |   | - |   | - |   |
   +---+---+---+---+---+---+---+---+
 4 |   | - |   | - |   | - |   | - |
   +---+---+---+---+---+---+---+---+
 3 | - |   | - |   | - |   | - |   |
   +---+---+---+---+---+---+---+---+
 2 | P | P | P | P | P | P | P | P |
   +---+---+---+---+---+---+---+---+
 1 | R | N | B | Q | K | B | N | R |
   +---+---+---+---+---+---+---+---+
     a   b   c   d   e   f   g   h

Entering moves:

: (go e2 e4)

Undo moves:

: (go -)

Redo:

: (go +)

Switch sides:

: (go)

Save position to a file:

: (ppos "file")

Load position from file:

: (load "file")

\end{verbatim}

\section*{Cholesky decomposition}

\begin{verbatim}

(scl 9)
(load "@lib/math.l")

(de cholesky (A)
   (let L (mapcar '(() (need (length A) 0)) A)
      (for (I . R) A
         (for J I
            (let S (get R J)
               (for K (inc J)
                  (dec 'S (*/ (get L I K) (get L J K) 1.0)) )
               (set (nth L I J)
                  (if (= I J)
                     (sqrt (* 1.0 S))
                     (*/ S 1.0 (get L J J)) ) ) ) ) )
      (for R L
         (for N R (prin (align 9 (round N 5))))
         (prinl) ) ) )

Test:

(cholesky
   '((25.0 15.0 -5.0) (15.0 18.0 0) (-5.0 0 11.0)) )

(prinl)

(cholesky
   (quote
      (18.0  22.0   54.0   42.0)
      (22.0  70.0   86.0   62.0)
      (54.0  86.0  174.0  134.0)
      (42.0  62.0  134.0  106.0) ) )

Output:

  5.00000  0.00000  0.00000
  3.00000  3.00000  0.00000
 -1.00000  1.00000  3.00000

  4.24264  0.00000  0.00000  0.00000
  5.18545  6.56591  0.00000  0.00000
 12.72792  3.04604  1.64974  0.00000
  9.89949  1.62455  1.84971  1.39262

\end{verbatim}

\section*{Classes}

\begin{verbatim}

(class +Rectangle)
# dx dy

(dm area> ()  # Define a a method that calculates the rectangle's area
   (* (: dx) (: dy)) )

(println  # Create a rectangle, and print its area
   (area> (new '(+Rectangle) 'dx 3 'dy 4)) )

\end{verbatim}

\section*{Closest-pair problem}

\begin{verbatim}

A brute-force solution:

(de closestPairBF (Lst)
   (let Min T
      (use (Pt1 Pt2)
         (for P Lst
            (for Q Lst
               (or
                  (== P Q)
                  (>=
                     (setq N
                        (let (A (- (car P) (car Q))  B (- (cdr P) (cdr Q)))
                           (+ (* A A) (* B B)) ) )
                     Min )
                  (setq Min N  Pt1 P  Pt2 Q) ) ) )
         (list Pt1 Pt2 (sqrt Min)) ) ) )

Test:

: (scl 6)
-> 6

: (closestPairBF
   (quote
      (0.654682 . 0.925557)
      (0.409382 . 0.619391)
      (0.891663 . 0.888594)
      (0.716629 . 0.996200)
      (0.477721 . 0.946355)
      (0.925092 . 0.818220)
      (0.624291 . 0.142924)
      (0.211332 . 0.221507)
      (0.293786 . 0.691701)
      (0.839186 . 0.728260) ) )
-> ((891663 . 888594) (925092 . 818220) 77910)

\end{verbatim}

\section*{Closures/Variable capture}

\begin{verbatim}

(setq FunList
   (make
      (for @N 10
         (link (curry (@N) () (* @N @N))) ) ) )

Test:

: ((get FunList 2))
-> 4

: ((get FunList 8))
-> 64

\end{verbatim}

\section*{Collections}

\begin{verbatim}

The direct way in PicoLisp is a linear list (other possibilities could involve
[http://software-lab.de/doc/refI.html#idx index] trees or
[http://software-lab.de/doc/ref.html#symbol property] lists).

: (setq Lst (3 4 5 6))
-> (3 4 5 6)

: (push 'Lst 2)
-> 2

: (push 'Lst 1)
-> 1

: Lst
-> (1 2 3 4 5 6)

: (insert 4 Lst 'X)
-> (1 2 3 X 4 5 6)

\end{verbatim}

\section*{Color of a screen pixel}

\begin{verbatim}

Using '[http://www.muquit.com/muquit/software/grabc/grabc.html grabc]'
as recommended in the C solution

(in '(grabc)
   (mapcar hex (cdr (line NIL 1 2 2 2))) )

Output:

73,61,205
-> (73 61 205)

\end{verbatim}

\section*{Colour bars/Display}

\begin{verbatim}

{{trans|UNIX Shell}}

(call 'clear)

(let Width (in '(tput cols) (read))
   (do (in '(tput lines) (read))
      (for B (range 0 7)
         (call 'tput 'setab B)
         (space (/ Width 8)) )
      (prinl) ) )

(call 'tput 'sgr0)   # reset

\end{verbatim}

\section*{Colour pinstripe/Printer}

\begin{verbatim}

(load "@lib/ps.l")

# Using circular lists for an endless supply of colors
#      (black  red  green blue magenta cyan yellow white)
(setq
   Red   (0    100    0     0    100    0    100   100 .)
   Green (0     0    100    0     0    100   100   100 .)
   Blue  (0     0     0    100   100   100    0    100 .) )

(call 'lpr
   (pdf "pinstripes"
      (a4)  # 595 x 842 dots
      (let (I 0  Step 1)
         (for X 595
            (color (car Red) (car Green) (car Blue)
               (vline X 0 842) )
            (when (= Step (inc 'I))
               (zero I)
               (pop 'Red)
               (pop 'Green)
               (pop 'Blue) )
            (when (=0 (\% X 72))  # 1 inch
               (zero I)
               (inc 'Step) ) ) )
      (page) ) )

\end{verbatim}

\section*{Combinations}

\begin{verbatim}

(de comb (M Lst)
   (cond
      ((=0 M) '(NIL))
      ((not Lst))
      (T
         (conc
            (mapcar
               '((Y) (cons (car Lst) Y))
               (comb (dec M) (cdr Lst)) )
            (comb M (cdr Lst)) ) ) ) )

(comb 3 (1 2 3 4 5))

\end{verbatim}

\section*{Combinations with repetitions}

\begin{verbatim}

(de combrep (N Lst)
   (cond
      ((=0 N) '(NIL))
      ((not Lst))
      (T
         (conc
            (mapcar
               '((X) (cons (car Lst) X))
               (combrep (dec N) Lst) )
            (combrep N (cdr Lst)) ) ) ) )

Output:

: (combrep 2 '(iced jam plain))
-> ((iced iced) (iced jam) (iced plain) (jam jam) (jam plain) (plain plain))

: (length (combrep 3 (range 1 10)))
-> 220

\end{verbatim}

\section*{Command-line arguments}

\begin{verbatim}

There are three ways to handle command-line arguments in PicoLisp:

1. Obtain all arguments as a list of strings via '[http://software-lab.de/doc/refA.html#argv argv]'

2. Fetch each argument individually with '[http://software-lab.de/doc/refO.html#opt opt]'

3. Use the built-in [http://software-lab.de/doc/ref.html#invoc command-line interpretation], where arguments starting with a hypen are executed as functions.

Here we use the third option, as it is not so obvious, sometimes more flexible,
and in fact the most commonly used one for application development.

We define 'c' and 'h' as functions, which retrieve their argument with 'opt',
and then '[http://software-lab.de/doc/refL.html#load load]' all remaining
command line arguments.

#!/usr/bin/picolisp /usr/lib/picolisp/lib.l

(de c ()
   (prinl "Got 'c': " (opt)) )

(de h ()
   (prinl "Got 'h': " (opt)) )

(load T)
(bye)

Output:

\$ ./myprogram -c "alpha beta" -h "gamma"
Got 'c': alpha beta
Got 'h': gamma

\end{verbatim}

\section*{Comments}

\begin{verbatim}

# The rest of the line is ignored
#{
   This is a
   multiline comment
}#
NIL
Immediately stop reading this file. Because all text in the input file following
a top-level 'NIL' is ignored.

This is typically used conditionally, with a read-macro expression like
`*Dbg
so that this text is only read if in debugging mode.

\end{verbatim}

\section*{Compile-time calculation}

\begin{verbatim}

The PicoLisp "compiler" is the so-called "reader", which converts the
human-readable source code into nested internal pointer structures. When it
runs, arbitrary expressions can be executed with the backqoute and tilde
operators ([http://software-lab.de/doc/ref.html#macro-io read macros]).

(de fact (N)
   (apply * (range 1 N)) )

(de foo ()
   (prinl "The value of fact(10) is " `(fact 10)) )

Output:

: (pp 'foo)  # Pretty-print the function
(de foo NIL
   (prinl "The value of fact(10) is " 3628800) )
-> foo

: (foo)  # Execute it
The value of fact(10) is 3628800
-> 3628800

\end{verbatim}

\section*{Compound data type}

\begin{verbatim}

(class +Point)

(dm T (X Y)
   (=: x X)
   (=: y Y) )

(setq P (new '(+Point) 3 4))

(show P)

Output:

\$52717735311266 (+Point)
   y 4
   x 3

\end{verbatim}

\section*{Concurrent computing}

\begin{verbatim}

Using background tasks

(for (N . Str) '("Enjoy" "Rosetta" "Code")
   (task (- N) (rand 1000 4000)              # Random start time 1 .. 4 sec
      Str Str                                # Closure with string value
      (println Str)                          # Task body: Print the string
      (task @) ) )                           # and stop the task

Using child processes

(for Str '("Enjoy" "Rosetta" "Code")
   (let N (rand 1000 4000)                   # Randomize
      (unless (fork)                         # Create child process
         (wait N)                            # Wait 1 .. 4 sec
         (println Str)                       # Print string
         (bye) ) ) )                         # Terminate child process

\end{verbatim}

\section*{Conditional structures}

\begin{verbatim}

(if (condition)                  # If the condition evaluates to non-NIL
   (then-do-this)                # Then execute the following expression
   (else-do-that)                # Else execute all other expressions
   (and-more) )

(ifn (condition)                 # If the condition evaluates to NIL
   (then-do-this)                # Then execute the following expression
   (else-do-that)                # Else execute all other expressions
   (and-more) )

(when (condition)                # If the condition evaluates to non-NIL
   (then-do-this)                # Then execute tall following expressions
   (and-more) )

(unless (condition)              # If the condition evaluates to NIL
   (then-do-this)                # Then execute all following expressions
   (and-more) )

(if2 (condition1) (condition2)   # If both conditions evaluate to non-NIL
   (expression-both)             # Then execute this expression
   (expression-first)            # Otherwise this for the first
   (expression-second)           # or this the second condition.
   (expression-none)             # If both are NIL, all following expressions
   (and-more) )

(cond
   ((condition1)                 # If this condition evaluates to non-NIL
      (expression 1)             # Execute these expression(s)
      (more 1) )
   ((condition2)                 # Otherwise, if this evaluates to non-NIL
      (expression 2)             # Execute these expression(s)
      (more 2) )
   (T                            # If none evaluated to non-NIL
      (expression 1)             # Execute these expression(s)
      (more 1) )

(nond
   ((condition1)                 # If this condition evaluates to NIL
      (expression 1)             # Execute these expression(s)
      (more 1) )
   ((condition2)                 # Otherwise, if this evaluates to NIL
      (expression 2)             # Execute these expression(s)
      (more 2) )
   (NIL                          # If none evaluated to NIL
      (expression 1)             # Execute these expression(s)
      (more 1) )

(case (expression)               # Evaluate the expression
   (value1                       # If it is equal to, or member of, 'value1'
      (do-this1)                 # Execute these expression(s)
      (do-that1) )
   (value2                       # Else if it is equal to, or member of, 'value2'
      (do-this2)                 # Execute these expression(s)
      (do-that2) )
   (T                            # Else execute final expression(s)
      (do-something-else) ) )

\end{verbatim}

\section*{Constrained Random Points on a Circle}

\begin{verbatim}

(let Area (make (do 31 (link (need 31 " "))))
   (use (X Y)
      (do 100
         (until
            (>=
               15
               (sqrt
                  (+
                     (* (setq X (rand -15 15)) X)
                     (* (setq Y (rand -15 15)) Y) ) )
               10 ) )
         (set (nth Area (+ 16 X) (+ 16 Y)) "#") ) )
   (mapc prinl Area) )

Output:

           #
        ##
           #  #  #  #
         ## #    #  #   #
       #      #   # # #
    #       #     # #   #
      #   #         #    #
      #                #   #
                       #
       #               #
     ##                      #
                          #
   #                      # #
  ###                      #
 #                         # #
##                          # #
  #                        #
 #                        #
 ## #
                           #
   #
     #                 #
     #
   ###                  #   #
      ###           # #    #
    #      #
         #  #   ##
                 # #
      #     #           #
                # #  #

\end{verbatim}

\section*{Constrained genericity}

\begin{verbatim}

(class +Eatable)

(dm eat> ()
   (prinl "I'm eatable") )


(class +FoodBox)
# obj

(dm set> (Obj)
   (unless (method 'eat> Obj)                # Check if the object is eatable
      (quit "Object is not eatable" Obj) )
   (=: obj Obj) )                            # If so, set the object


(let (Box (new '(+FoodBox))  Eat (new '(+Eatable))  NoEat (new '(+Bla)))
   (set> Box Eat)       # Works
   (set> Box NoEat) )   # Gives an error

Output:

\$384320489 -- Object is not eatable

? (show Box)
\$384320487 (+FoodBox)
   obj \$384320488

? (show Box 'obj)
\$384320488 (+Eatable)

? (show NoEat)
\$384320489 (+Bla)

\end{verbatim}

\section*{Conway's Game of Life}

\begin{verbatim}

This example uses 'grid' and 'disp' from "lib/simul.l". These functions
maintain an array of multiply linked objects, and are also used in the chess
program and other games in the distribution.

{misc/life.l}

\end{verbatim}

\section*{Copy a string}

\begin{verbatim}

(setq Str1 "abcdef")
(setq Str2 Str1)                       # Create a reference to that symbol
(setq Str3 (name Str1))                # Create new symbol with name "abcdef"

\end{verbatim}

\section*{Count occurrences of a substring}

\begin{verbatim}

(de countSubstring (Str Sub)
   (let (Cnt 0  H (chop Sub))
      (for (S (chop Str)  S  (cdr S))
         (when (head H S)
            (inc 'Cnt)
            (setq S (map prog2 H S)) ) )
      Cnt ) )

Test:

: (countSubstring "the three truths" "th")
-> 3

: (countSubstring "ababababab" "abab")
-> 2

\end{verbatim}

\section*{Count the Coins}

\begin{verbatim}

{{trans|C}}

(de coins (Sum Coins)
   (let (Buf (mapcar '((N) (cons 1 (need (dec N) 0))) Coins)  Prev)
      (do Sum
         (zero Prev)
         (for L Buf
            (inc (rot L) Prev)
            (setq Prev (car L)) ) )
      Prev ) )

Test:

(for Coins '((100 50 25 10 5 1) (200 100 50 20 10 5 2 1))
   (println (coins 100 (cddr Coins)))
   (println (coins (* 1000 100) Coins))
   (println (coins (* 10000 100) Coins))
   (println (coins (* 100000 100) Coins))
   (prinl) )

Output:

242
13398445413854501
1333983445341383545001
133339833445334138335450001

4562
10056050940818192726001
99341140660285639188927260001
992198221207406412424859964272600001

\end{verbatim}

\section*{Counting in Factors}

\begin{verbatim}

This is the 'factor' function from [[Prime decomposition#PicoLisp]].

(de factor (N)
   (make
      (let (D 2  L (1 2 2 . (4 2 4 2 4 6 2 6 .))  M (sqrt N))
         (while (>= M D)
            (if (=0 (\% N D))
               (setq M (sqrt (setq N (/ N (link D)))))
               (inc 'D (pop 'L)) ) )
         (link N) ) ) )

(for N 20
   (prinl N ": " (glue " * " (factor N))) )

Output:

1: 1
2: 2
3: 3
4: 2 * 2
5: 5
6: 2 * 3
7: 7
8: 2 * 2 * 2
9: 3 * 3
10: 2 * 5
11: 11
12: 2 * 2 * 3
13: 13
14: 2 * 7
15: 3 * 5
16: 2 * 2 * 2 * 2
17: 17
18: 2 * 3 * 3
19: 19
20: 2 * 2 * 5

\end{verbatim}

\section*{Counting in octal}

\begin{verbatim}

(for (N 0  T  (inc N))
   (prinl (oct N)) )

\end{verbatim}

\section*{Create a file}

\begin{verbatim}

(out "output.txt")                     # Empty output
(call 'mkdir "docs")                   # Call external
(out "/output.txt")
(call 'mkdir "/docs")

\end{verbatim}

\section*{Create a two-dimensional array at runtime}

\begin{verbatim}

(de 2dimTest (DX DY)
   (let A (make (do DX (link (need DY))))
      (set (nth A 3 3) 999)            # Set A[3][3] to 999
      (mapc println A)                 # Print all
      (get A 3 3) ) )                  # Return A[3][3]

(2dimTest 5 5)

Output:

(NIL NIL NIL NIL NIL)
(NIL NIL NIL NIL NIL)
(NIL NIL 999 NIL NIL)
(NIL NIL NIL NIL NIL)
(NIL NIL NIL NIL NIL)
-> 999

\end{verbatim}

\section*{Create an HTML table}

\begin{verbatim}

(load "@lib/xhtml.l")

(<table> NIL NIL '(NIL (NIL "X") (NIL "Y") (NIL "Z"))
   (for N 3
      (<row> NIL N 124 456 789) ) )

\end{verbatim}

\section*{Create an object at a given address}

\begin{verbatim}

: (setq IntSpace 12345)          # Integer
-> 12345

: (setq Address (adr 'IntSpace)) # Encoded machine address
-> -2969166782547

: (set (adr Address) 65535)      # Set this address to a new value
-> 65535

: IntSpace                       # Show the new value
-> 65535

\end{verbatim}

\chapter{Rosetta Code Tasks starting with D}

\section*{DNS}

\begin{verbatim}

(make
   (in '(host "www.kame.net")
      (while (from "address ")
         (link (till "^J" T)) ) ) )

Output:

-> ("203.178.141.194" "2001:200:dff:fff1:216:3eff:feb1:44d7")

\end{verbatim}

\section*{Date format}

\begin{verbatim}

(let (Date (date)  Lst (date Date))
   (prinl (dat\$ Date "-"))             # 2010-02-19
   (prinl                              # Friday, February 19, 2010
      (day Date)
      ", "
      (get *MonFmt (cadr Lst))
      " "
      (caddr Lst)
      ", "
      (car Lst) ) )

\end{verbatim}

\section*{Date manipulation}

\begin{verbatim}

(de timePlus12 (Str)
   (use (@Mon @Day @Year @Time @Zone)
      (and
         (match
            '(@Mon " " @Day " " @Year " " @Time " " @Zone)
            (chop Str) )
         (setq @Mon (index (pack @Mon) *MonFmt))
         (setq @Day (format @Day))
         (setq @Year (format @Year))
         (setq @Time
            (case (tail 2 @Time)
               (("a" "m") (\$tim (head -2 @Time)))
               (("p" "m") (+ `(time 12 0) (\$tim (head -2 @Time))))
               (T (\$tim @Time)) ) )
         (let? Date (date @Year @Mon @Day)
            (when (>= (inc '@Time `(time 12 0)) 86400)
               (dec '@Time 86400)
               (inc 'Date) )
            (pack (dat\$ Date "-") " " (tim\$ @Time T) " " @Zone) ) ) ) )

\end{verbatim}

\section*{Day of the week}

\begin{verbatim}

(for (Y 2008 (>= 2121 Y) (inc Y))
   (when (= "Sunday" (day (date Y 12 25)))
      (printsp Y) ) )

Output:

2011 2016 2022 2033 2039 2044 2050 2061 2067 2072 2078 2089 2095 2101 2107 2112 2118

\end{verbatim}

\section*{Deal cards for FreeCell}

\begin{verbatim}

Using the random generator from [[Linear congruential generator#PicoLisp]]:

(setq *MsSeed 11982)

(de msRand ()
   (>> 16
      (setq *MsSeed
         (\& (+ 2531011 (* 214013 *MsSeed)) `(dec (** 2 31))) ) ) )

(let L
   (make
      (for Num (range 13 1)
         (for Suit '((32 . "♠") (31 . "♥") (31 . "♦") (32 . "♣"))
            (link (cons (get '`(chop "A23456789TJQK") Num) Suit)) ) ) )
   (for I 51
      (xchg
         (nth L I)
         (nth L (- 52 (\% (msRand) (- 53 I)))) ) )
   (for C L
      (prin "  ^[[" (cadr C) "m" (cddr C) "^[[m" (car C))
      (at (0 . 8) (prinl)) )
   (prinl) )

\end{verbatim}

\section*{Decision tables (!)}

\begin{verbatim}

We allow ourselves a luxurious user interface:

(de yes? (Cond)
   (out NIL (prin (car Cond) "? "))
   (in NIL
      (use Reply
         (loop
            (setq Reply (read))
            (T (member Reply '(T Y YES Yes y yes true 1))
               T )
            (T (member Reply '(NIL N NO No n no false 0)))
            (prinl "Please answer 'Yes' or 'No'") ) ) ) )

The decision table used in the example:

(de *Conditions
   ("Printer does not print"                T   T   T   T  NIL NIL NIL NIL)
   ("A red light is flashing"               T   T  NIL NIL  T   T  NIL NIL)
   ("Printer is unrecognised"               T  NIL  T  NIL  T  NIL  T  NIL) )

(de *Actions
   ("Check the power cable"                NIL NIL  T)
   ("Check the printer-computer cable"      T  NIL  T)
   ("Ensure printer software is installed"  T  NIL  T  NIL  T  NIL  T)
   ("Check/replace ink"                     T   T  NIL NIL  T   T)
   ("Check for paper jam"                  NIL  T  NIL  T) )

The decision can be made directly on the condition and action data, without the
need to create intermediate tables:

(de decide ()
   (let Reply (mapcar yes? *Conditions)
      (extract and
         (apply pick (append *Conditions *Actions)
            '(@
               (unless (pick '((Flg) (<> Flg (next))) Reply)
                  (rest) ) ) )
         (mapcar car *Actions) ) ) )

Output:

: (decide)
Printer does not print? y
A red light is flashing? y
Printer is unrecognised? n
-> ("Check/replace ink" "Check for paper jam")

: (decide)
Printer does not print? n
A red light is flashing? y
Printer is unrecognised? y
-> ("Ensure printer software is installed" "Check/replace ink")

: (decide)
Printer does not print? n
A red light is flashing? n
Printer is unrecognised? n
-> NIL

\end{verbatim}

\section*{Deepcopy}

\begin{verbatim}

A shallow copy can be done with '[http://software-lab.de/doc/refC.html#copy
copy]'. This function takes care of cons pairs and lists, no matter whether they
are cyclic, or end in NIL or some other data structure.

For a known depth, it might be used in combination with other list functions.
For example, to copy a non-cyclic structure of depth 2 with
'[http://software-lab.de/doc/refM.html#mapcar mapcar]':

(mapcar copy List)

Copying non-cyclic structures of arbitrary depth and list-termination could be
handled with a custom function (using
'[http://software-lab.de/doc/refC.html#cons cons]'):

(de deepCopy (X)
   (if (atom X)
      X
      (cons (deepCopy (car X)) (deepCopy (cdr X))) ) )

Test:

: (setq A '((a . b) (c d e) f g . e))
-> ((a . b) (c d e) f g . e)

: (setq B (deepCopy A))
-> ((a . b) (c d e) f g . e)

: A
-> ((a . b) (c d e) f g . e)

: B
-> ((a . b) (c d e) f g . e)

: (= A B)
-> T              # A and its copy B are structure-equal
: (== A B)
-> NIL            # but they are not identical (pointer-equal)

: (cadr A)
-> (c d e)

: (cadr B)
-> (c d e)

: (== (cadr A) (cadr B))
-> NIL            # The same holds for sub-structures

For cyclic structures, the above 'deepCopy' function could be extended, to
remember already visited structures and their copies in a mark list:

(de deepCopy (X)
   (let Mark NIL
      (recur (X)
         (cond
            ((atom X) X)
            ((asoq X Mark) (cdr @))
            (T
               (prog1 (cons)
                  (push 'Mark (cons X @))
                  (set @ (recurse (car X)))
                  (con @ (recurse (cdr X))) ) ) ) ) ) )

Test:

: (setq A '(a b .)  B (deepCopy A))
-> (a b .)
: A
-> (a b .)
: B
-> (a b .)

: (= A B)
-> T              # A and its copy B are structure-equal

: (== A B)
-> NIL            # but they are not identical (pointer-equal)

\end{verbatim}

\section*{Define a primitive data type}

\begin{verbatim}

{{trans|Java}}

(class +BoundedInt)
# value lower upper

(dm T (Low Up)
   (=: lower (min Low Up))
   (=: upper (max Low Up)) )

(de "checkBounds" (Val)
   (if (>= (: upper) Val (: lower))
      Val
      (throw 'boundedIntOutOfBounds
         (pack
            "value " Val
            " is out of bounds [" (: lower) "," (: upper) "]" ) ) ) )

(dm set> (Val)
   (=: value ("checkBounds" Val)) )

(dm +> (Val)
   (=: value ("checkBounds" (+ Val (: value)))) )

(dm val> ()
   (: value) )

(de main ()
   (let (A (new '(+BoundedInt) 1 10)  B (new '(+BoundedInt) 1 10))
      (set> A 6)
      (when (catch 'boundedIntOutOfBounds (set> B 12) NIL)
         (prinl @) )
      (set> B 9)
      (when (catch 'boundedIntOutOfBounds (+> A (val> B)) NIL)
         (prinl @) ) ) )

Output:

: (main)
value 12 is out of bounds [1,10]
value 15 is out of bounds [1,10]

\end{verbatim}

\section*{Delegates}

\begin{verbatim}

(class +Delegator)
# delegate

(dm operation> ()
   (if (: delegate)
      (thing> @)
      "default implementation" ) )


(class +Delegate)
# thing

(dm T (Msg)
   (=: thing Msg) )

(dm thing> ()
   (: thing) )


(let A (new '(+Delegator))
   # Without a delegate
   (println (operation> A))

   # With delegate that does not implement 'thing>'
   (put A 'delegate (new '(+Delegate)))
   (println (operation> A))

   # With delegate that implements 'thing>'
   (put A 'delegate (new '(+Delegate) "delegate implementation"))
   (println (operation> A)) )

Output:

"default implementation"
NIL
"delegate implementation"

\end{verbatim}

\section*{Delete a file}

\begin{verbatim}

(call 'rm "input.txt")
(call 'rmdir "docs")
(call 'rm "/input.txt")
(call 'rmdir "/docs")

\end{verbatim}

\section*{Detect division by zero}

\begin{verbatim}

(catch '("Div/0") (/ A B))

\end{verbatim}

\section*{Determine if a string is numeric}

\begin{verbatim}

The 'format' function can be used for that. It returns NIL if the given string
is not a legal number

: (format "123")
-> 123

: (format "123a45")
-> NIL

: (format "-123.45" 4)
-> 1234500

\end{verbatim}

\section*{Determine if only one instance is running}

\begin{verbatim}

Calling 'killall'

One possibility is to send a zero-signal with 'killall', and check the return
value. This is useful if each application is started by a hash-bang script (the
first line is e.g. "#!/usr/bin/picolisp /usr/lib/picolisp/lib.l"). In that way,
each application has its own name which can be passed to 'killall'.

\$ cat myScript
#!/usr/bin/picolisp /usr/lib/picolisp/lib.l

(wait 120000)
(bye)

\$ ./myScript \&  # Start in the background
[1] 26438

\$ pil +
: (call "killall" "-0" "-q" "myScript")
-> T

Using a mutex

Another possibility is to 'acquire' a mutex on program start, and never release
it.

: (acquire "running1")
-> 30817  # A successful call returns the PID

A second application trying to acquire the same mutex would receive 'NIL'

\end{verbatim}

\section*{Determine the maximum height and width of a window}

\begin{verbatim}

The following works on ErsatzLisp, the Java version of PicoLisp.

(let Frame (java "javax.swing.JFrame" T "Window")
   (java Frame 'setExtendedState
      (java (public "javax.swing.JFrame" 'MAXIMIZED_BOTH)) )
   (java Frame 'setVisible T)
   (wait 200)
   (let Size (java (java Frame 'getContentPane) 'getSize)
      (prinl "Width: " (java (public Size 'width)))
      (prinl "Height: " (java (public Size 'height))) )
   (java Frame 'dispose) )

Output (on a 1024x768 screen):

Width: 1010
Height: 735

\end{verbatim}

\section*{Dinesman's multiple-dwelling problem}

\begin{verbatim}

Using Pilog (PicoLisp Prolog). The problem can be modified by changing just the
'dwelling' rule (the "Problem statement"). This might involve the names and
number of dwellers (the list in the first line), and statements about who does
(or does not) live on the top floor (using the 'topFloor' predicate), the bottom
floor (using the 'bottomFloor' predicate), on a higher floor (using the
'higherFloor' predicate) or on an adjecent floor (using the 'adjacentFloor'
predicate). The logic follows an implied AND, and statements may be arbitrarily
combined using OR and NOT (using the 'or' and 'not' predicates), or any other
Pilog (Prolog) built-in predicates. If the problem statement has several
solutions, they will be all generated.

# Problem statement
(be dwelling (@Tenants)
   (permute (Baker Cooper Fletcher Miller Smith) @Tenants)
   (not (topFloor Baker @Tenants))
   (not (bottomFloor Cooper @Tenants))
   (not (or ((topFloor Fletcher @Tenants)) ((bottomFloor Fletcher @Tenants))))
   (higherFloor Miller Cooper @Tenants)
   (not (adjacentFloor Smith Fletcher @Tenants))
   (not (adjacentFloor Fletcher Cooper @Tenants)) )

# Utility rules
(be topFloor (@Tenant @Lst)
   (equal (@ @ @ @ @Tenant) @Lst) )

(be bottomFloor (@Tenant @Lst)
   (equal (@Tenant @ @ @ @) @Lst) )

(be higherFloor (@Tenant1 @Tenant2 @Lst)
   (append @ @Rest @Lst)
   (equal (@Tenant2 . @Higher) @Rest)
   (member @Tenant1 @Higher) )

(be adjacentFloor (@Tenant1 @Tenant2 @Lst)
   (append @ @Rest @Lst)
   (or
      ((equal (@Tenant1 @Tenant2 . @) @Rest))
      ((equal (@Tenant2 @Tenant1 . @) @Rest)) ) )

Output:

: (? (dwelling @Result))
 @Result=(Smith Cooper Baker Fletcher Miller)  # Only one solution
-> NIL

\end{verbatim}

\section*{Dining philosophers}

\begin{verbatim}

This following solution uses the built-in fininte state machine function
'[http://software-lab.de/doc/refS.html#state state]'. Deadlocks are avoided, as
each philosopher releases the first fork if he doesn't succeed to obtain the
second fork, and waits for a random time.

Another solution, using the Chandy/Misra method, can be found
[http://logand.com/sw/phil.l here].

{misc/dining.l}

Output:

<pre>Aristotle: hungry
Aristotle: rightFork
Kant: hungry
Kant: rightFork
Spinoza: hungry
Spinoza: rightFork
Marx: hungry
Marx: rightFork
Russell: hungry
Marx: hungry
Spinoza: hungry
Kant: hungry
Russell: hungry
Aristotle: eating
...

\end{verbatim}

\section*{Discordian date}

\begin{verbatim}

{{trans|Python}}

(de disdate (Year Month Day)
   (let? Date (date Year Month Day)
      (let (Leap (date Year 2 29)  D (- Date (date Year 1 1)))
         (if (and Leap (= 2 Month) (= 29 Day))
            (pack "St. Tib's Day, YOLD " (+ Year 1166))
            (and Leap (>= D 60) (dec 'D))
            (pack
               (get
                  '("Chaos" "Discord" "Confusion" "Bureaucracy" "The Aftermath")
                  (inc (/ D 73)) )
               " "
               (inc (\% D 73))
               ", YOLD "
               (+ Year 1166) ) ) ) ) )

\end{verbatim}

\section*{Distributed programming}

\begin{verbatim}

# Server

(task (port 12321)                     # Background server task
   (let? Sock (accept @)
      (unless (fork)                   # Handle request in child process
         (in Sock
            (while (rd)                # Handle requests
               (out Sock
                  (pr (eval @)) ) ) )  # Evaluate and send reply
         (bye) )                       # Exit child process
      (close Sock) ) )                 # Close socket in parent process

# Client

(let? Sock (connect "localhost" 12321)
   (out Sock (pr '*Pid))               # Query PID from server
   (println 'PID (in Sock (rd)))       # Receive and print reply
   (out Sock (pr '(* 3 4)))            # Request some calculation
   (println 'Result (in Sock (rd)))    # Print result
   (close Sock) )                      # Close connection to server

Output:

PID 18372
Result 12

\end{verbatim}

\section*{Documentation}

\begin{verbatim}

PicoLisp doesn't yet support inline documentation directly in the code. However,
it has built-in runtime documentation via the
'[http://software-lab.de/doc/refD.html#doc doc]' function. This requires no
external tools, except that the interpreter must have been started in debug
mode.

: (doc 'car)         # View documentation of a function

: (doc '+Entity)     # View documentation of a class

: (doc '+ 'firefox)  # Explicitly specify a browser

\end{verbatim}

\section*{Dot product (!)}

\begin{verbatim}

(de dotProduct (A B)
   (sum * A B) )

(dotProduct (1 3 -5) (4 -2 -1))

Output:

-> 3

\end{verbatim}

\section*{Doubly-linked list/Definition}

\begin{verbatim}

For the list of double-cell structures described in
[[Doubly-linked list/Element definition#PicoLisp]],
we define a header structure, containing one pointer to the start
and one to the end of the list.

            +------------> start
            |
         +--+--+-----+
         |  |  |  ---+---> end
         +-----+-----+

# Build a doubly-linked list
(de 2list @
   (let Prev NIL
      (let L
         (make
            (while (args)
               (setq Prev (chain (list (next) Prev))) ) )
         (cons L Prev) ) ) )

(setq *DLst (2list 'was 'it 'a 'cat 'I 'saw))

For output of the example data, see [[Doubly-linked list/Traversal#PicoLisp]].

\end{verbatim}

\section*{Doubly-linked list/Element definition}

\begin{verbatim}

We use (in addition to the header structure described in
[[Doubly-linked list/Definition#PicoLisp]])
two cells per doubly-linked list element:

         +-----+-----+     +-----+-----+
         | Val |  ---+---> |  |  |  ---+---> next
         +-----+-----+     +--+--+-----+
                              |
                     prev <---+

With that, 'cddr' can be used to access the next, and 'cadr' to access the
previous element.

# 'cons' an element to a doubly-linked list
(de 2cons (X DLst)
   (let L (car DLst)                  # Get current data list
      (set DLst (cons X NIL L))       # Prepend two new cons pairs
      (if L                           # Unless DLst was empty
         (set (cdr L) (car DLst))     # set new 'prev' link
         (con DLst (car DLst)) ) ) )  # otherwise set 'end' link

# We prepend 'not' to the list in the previous example
(2cons 'not *DLst)

For output of the example data, see [[Doubly-linked list/Traversal#PicoLisp]].

\end{verbatim}

\section*{Doubly-linked list/Element insertion}

\begin{verbatim}

This works with the structures described in
[[Doubly-linked list/Definition#PicoLisp]] and
[[Doubly-linked list/Element definition#PicoLisp]].

# Insert an element X at position Pos
(de 2insert (X Pos DLst)
   (let (Lst (nth (car DLst) (dec (* 2 Pos)))  New (cons X (cadr Lst) Lst))
      (if (cadr Lst)
         (con (cdr @) New)
         (set DLst New) )
      (if (cdr Lst)
         (set @ New)
         (con DLst New) ) ) )

(setq *DL (2list 'A 'B))      # Build a two-element doubly-linked list
(2insert 'C 2 *DL)            # Insert C at position 2

For output of the example data, see [[Doubly-linked list/Traversal#PicoLisp]].

\end{verbatim}

\section*{Doubly-linked list/Traversal}

\begin{verbatim}

# Print the elements a doubly-linked list
(de 2print (DLst)
   (for (L (car DLst) L (cddr L))
      (printsp (car L)) )
   (prinl) )

# Print the elements a doubly-linked list in reverse order
(de 2printReversed (DLst)
   (for (L (cdr DLst) L (cadr L))
      (printsp (car L)) )
   (prinl) )

Output for the example data produced in
[[Doubly-linked list/Definition#PicoLisp]] and
[[Doubly-linked list/Element definition#PicoLisp]]:

: (2print *DLst)                 # Print the list
not was it a cat I saw

: (2printReversed *DLst)         # Print it in reversed order
saw I cat a it was not

Output for the example data produced in
[[Doubly-linked list/Element insertion#PicoLisp]]:

: (2print *DL)                   # Print the list
A C B

: (2printReversed *DL)           # Print it in reversed order
B C A

\end{verbatim}

\section*{Draw a clock}

\begin{verbatim}

This is an animated ASCII drawing of the "Berlin-Uhr", a clock built to display
the time according to the principles of set theory, which is installed in Berlin
since 1975.

See [http://www.surveyor.in-berlin.de/berlin/uhr/indexe.html
www.surveyor.in-berlin.de/berlin/uhr/indexe.html] and
[http://www.cs.utah.edu/~hatch/berlin_uhr.html
www.cs.utah.edu/~hatch/berlin_uhr.html].

(de draw Lst
   (for L Lst
      (for X L
         (cond
            ((num? X) (space X))
            ((sym? X) (prin X))
            (T (do (car X) (prin (cdr X)))) ) )
      (prinl) ) )

(de bigBox (N)
   (do 2
      (prin "|")
      (for I 4
         (prin (if (> I N) "          |" " ======== |")) )
      (prinl) ) )

(call 'clear)          # Clear screen
(call "tput" "civis")  # Set cursor invisible

(push '*Bye '(call "tput" "cnorm"))  # Set cursor visible on exit

(loop
   (call "tput" "cup" 0 0)  # Cursor to top left
   (let Time (time (time))
      (draw (20 (5 . _)) (19 / 5 \\))
      (if (onOff (NIL))
         (draw (18 / 7 \\) (18 \\ 7 /))
         (draw (18 / 2 (3 . "#") 2 \\) (18 \\ 2 (3 . "#") 2 /)) )
      (draw
         (19 \\ (5 . _) /)
         (+ (10 . -) + (10 . -) + (10 . -) + (10 . -) +) )
      (bigBox (/ (car Time) 5))
      (draw (+ (10 . -) + (10 . -) + (10 . -) + (10 . -) +))
      (bigBox (\% (car Time) 5))
      (draw (+ (43 . -) +))
      (do 2
         (prin "|")
         (for I `(range 5 55 5)
            (prin
               (cond
                  ((> I (cadr Time)) "   |")
                  ((=0 (\% I 3)) " # |")
                  (T " = |") ) ) )
         (prinl) )
      (draw (+ (43 . -) +))
      (bigBox (\% (cadr Time) 5))
      (draw (+ (10 . -) + (10 . -) + (10 . -) + (10 . -) +)) )
   (wait 1000) )

The six '#' characters in the "circle" on top toggle on/off every second. This
is the display at 17:46:

                    _____
                   /     \
                  /  ###  \
                  \  ###  /
                   \_____/
+----------+----------+----------+----------+
| ======== | ======== | ======== |          |
| ======== | ======== | ======== |          |
+----------+----------+----------+----------+
| ======== | ======== |          |          |
| ======== | ======== |          |          |
+-------------------------------------------+
| = | = | # | = | = | # | = | = | # |   |   |
| = | = | # | = | = | # | = | = | # |   |   |
+-------------------------------------------+
| ======== |          |          |          |
| ======== |          |          |          |
+----------+----------+----------+----------+

\end{verbatim}

\section*{Draw a cuboid}

\begin{verbatim}

# Using ASCII

(de cuboid (DX DY DZ)
   (cubLine (inc DY) "+" DX "-" 0)
   (for I DY
      (cubLine (- DY I -1) "/" DX " " (dec I) "|") )
   (cubLine 0 "+" DX "-" DY "|")
   (do (- (* 4 DZ) DY 2)
      (cubLine 0 "|" DX " " DY "|") )
   (cubLine 0 "|" DX " " DY "+")
   (for I DY
      (cubLine 0 "|" DX " " (- DY I) "/") )
   (cubLine 0 "+" DX "-" 0) )

(de cubLine (N C DX D DY E)
   (space N)
   (prin C)
   (do (dec (* 9 DX)) (prin D))
   (prin C)
   (space DY)
   (prinl E) )

Output:

: (cuboid 2 3 4)
    +-----------------+
   /                 /|
  /                 / |
 /                 /  |
+-----------------+   |
|                 |   |
|                 |   |
|                 |   |
|                 |   |
|                 |   |
|                 |   |
|                 |   |
|                 |   |
|                 |   |
|                 |   |
|                 |   |
|                 |   +
|                 |  /
|                 | /
|                 |/
+-----------------+

: (cuboid 1 1 1)
  +--------+
 /        /|
+--------+ |
|        | |
|        | +
|        |/
+--------+

: (cuboid 6 2 1)
   +-----------------------------------------------------+
  /                                                     /|
 /                                                     / |
+-----------------------------------------------------+  |
|                                                     |  +
|                                                     | /
|                                                     |/
+-----------------------------------------------------+

# Using OpenGL

# Based on cube.io by Mike Austin

(load "@lib/openGl.l")

(setq *AngleX -26.0 *AngleY 74.0)
(setq *LastX 0 *LastY 0)

(glutInit)
(glutInitDisplayMode (| GLUT_RGBA GLUT_DOUBLE GLUT_DEPTH))
(glutInitWindowSize 512 512)
(glutInitWindowPosition 10 50)
(glutCreateWindow "PicoLisp Cube")

(glClearColor 1.0 1.0 1.0 1.0)	# The background color
(glEnable GL_DEPTH_TEST)
(glEnable GL_LIGHTING)
(glEnable GL_LIGHT0)
(glDisable GL_CULL_FACE)

(glEnable GL_BLEND)
(glBlendFunc GL_SRC_ALPHA GL_ONE_MINUS_SRC_ALPHA)
(glEnable GL_LINE_SMOOTH)
(glHint GL_LINE_SMOOTH_HINT GL_NICEST)
(glLineWidth 2.0)

(mouseFunc
   '((Btn State X Y)
      (setq *LastX X  *LastY Y) ) )

(motionFunc
   '((X Y)
      (inc '*AngleX (* (- Y *LastY) 1.0))
      (inc '*AngleY (* (- X *LastX) 1.0))
      (setq *LastX X  *LastY Y)
      (glutPostRedisplay) ) )

(reshapeFunc
   '((Width Height)
      (glMatrixMode GL_PROJECTION)
      (glLoadIdentity)
      (gluPerspective 45.0 (*/ Width 1.0 Height) 1.0 10.0)
      (glMatrixMode GL_MODELVIEW)
      (glViewport 0 0 Width Height) ) )

(displayPrg
	(glClear (| GL_COLOR_BUFFER_BIT GL_DEPTH_BUFFER_BIT))
	(glLoadIdentity)
	(glTranslatef 0.0 0.0 -3.0)
	(glRotatef *AngleX 1 0 0)
	(glRotatef *AngleY 0 1 0)
	(glutSolidCube 1.0)
	
	(glDisable GL_LIGHTING)
	(glColor4f 0.4 0.4 0.4 1.0)
	(glutWireCube 1.002)
	(glEnable GL_LIGHTING)
	
	(glFlush)
	(glutSwapBuffers) )

(glutMainLoop)

\end{verbatim}

\section*{Draw a sphere}

\begin{verbatim}

This is for the 64-bit version.

(load "@lib/openGl.l")

(glutInit)
(glutInitDisplayMode (| GLUT_RGBA GLUT_DOUBLE GLUT_ALPHA GLUT_DEPTH))
(glutInitWindowSize 400 400)
(glutCreateWindow "Sphere")

(glEnable GL_LIGHTING)
(glEnable GL_LIGHT0)
(glLightiv GL_LIGHT0 GL_POSITION (10 10 -10 0))

(glEnable GL_COLOR_MATERIAL)
(glColorMaterial GL_FRONT_AND_BACK GL_AMBIENT_AND_DIFFUSE)

(glClearColor 0.3 0.3 0.5 0)
(glColor4f 0.0 0.8 0.0 1.0)

(displayPrg
   (glClear (| GL_COLOR_BUFFER_BIT GL_DEPTH_BUFFER_BIT))
   (glutSolidSphere 0.9 40 32)
   (glFlush)
   (glutSwapBuffers) )

# Exit upon mouse click
(mouseFunc '((Btn State X Y) (bye)))
(glutMainLoop)

\end{verbatim}

\section*{Dragon curve}

\begin{verbatim}

{{trans|Forth}}

This uses the 'brez' line drawing function from
[[Bitmap/Bresenham's line algorithm#PicoLisp]].

# Need some turtle graphics
(load "@lib/math.l")

(setq
   *TurtleX 100      # X position
   *TurtleY  75      # Y position
   *TurtleA 0.0 )    # Angle

(de fd (Img Len)  # Forward
   (let (R (*/ *TurtleA pi 180.0)  DX (*/ (cos R) Len 1.0)  DY (*/ (sin R) Len 1.0))
      (brez Img *TurtleX *TurtleY DX DY)
      (inc '*TurtleX DX)
      (inc '*TurtleY DY) ) )

(de rt (A)  # Right turn
   (inc '*TurtleA A) )

(de lt (A)  # Left turn
   (dec '*TurtleA A) )


# Dragon curve stuff
(de *DragonStep . 4)

(de dragon (Img Depth Dir)
   (if (=0 Depth)
      (fd Img *DragonStep)
      (rt Dir)
      (dragon Img (dec Depth) 45.0)
      (lt (* 2 Dir))
      (dragon Img (dec Depth) -45.0)
      (rt Dir) ) )

# Run it
(let Img (make (do 200 (link (need 300 0))))       # Create image 300 x 200
   (dragon Img 10 45.0)                            # Build dragon curve
   (out "img.pbm"                                  # Write to bitmap file
      (prinl "P1")
      (prinl 300 " " 200)
      (mapc prinl Img) ) )

\end{verbatim}

\section*{Dynamic variable names}

\begin{verbatim}

(de userVariable ()
   (prin "Enter a variable name: ")
   (let Var (line T)                                  # Read transient symbol
      (prin "Enter a value: ")
      (set Var (read))                                # Set symbol's value
      (println 'Variable Var 'Value (val Var)) ) )    # Print them

Output:
Enter a variable name: Tom
Enter a value: 42
Variable "Tom" Value 42
-> 42

\end{verbatim}

\chapter{Rosetta Code Tasks starting with E}

\section*{EBNF parser}

\begin{verbatim}

(de EBNF
   "expr  : term ( ( PLUS | MINUS )  term )* ;"
   "term  : factor ( ( MULT | DIV ) factor )* ;"
   "factor   : NUMBER ;" )

(for E EBNF
   (use (@S @E)
      (unless (and (match '(@S : @E ;) (str E)) (not (cdr @S)))
         (quit "Invalid EBNF" E) )
      (put (car @S) 'ebnf @E) ) )


(de matchEbnf (Pat)
   (cond
      ((asoq Pat '((PLUS . +) (MINUS . -) (MULT . *) (DIV . /)))
         (let Op (cdr @)
            (when (= Op (car *Lst))
               (pop '*Lst)
               Op ) ) )
      ((== 'NUMBER Pat)
         (cond
            ((num? (car *Lst))
               (pop '*Lst)
               @ )
            ((and (= "-" (car *Lst)) (num? (cadr *Lst)))
               (setq *Lst (cddr *Lst))
               (- @) ) ) )
      ((get Pat 'ebnf) (parseLst @))
      ((atom Pat))
      (T
         (loop
            (T (matchEbnf (pop 'Pat)) @)
            (NIL Pat)
            (NIL (== '| (pop 'Pat)))
            (NIL Pat) ) ) ) )

(de parseLst (Pat)
   (let (P (pop 'Pat)  X (matchEbnf P))
      (loop
         (NIL Pat)
         (if (n== '* (cadr Pat))
            (if (matchEbnf (pop 'Pat))
               (setq X (list @ X))
               (throw) )
            (loop
               (NIL *Lst)
               (NIL (matchEbnf (car Pat)))
               (setq X (list @ X (or (matchEbnf P) (throw)))) )
            (setq Pat (cddr Pat)) ) )
      X ) )

(de parseEbnf (Str)
   (let *Lst (str Str "")
      (catch NIL
         (parseLst (get 'expr 'ebnf)) ) ) )

Output:

: (parseEbnf "1 + 2 * -3 / 7 - 3 * 4")
-> (- (+ 1 (/ (* 2 -3) 7)) (* 3 4))

\end{verbatim}

\section*{Echo server}

\begin{verbatim}

(setq Port (port 12321))

(loop
   (setq Sock (listen Port))           # Listen
   (NIL (fork) (close Port))           # Accepted
   (close Sock) )                      # Parent: Close socket and continue

# Child:
(prinl (stamp) " -- (Pid " *Pid ") Client connected from " *Adr)

(in Sock
   (until (eof)                        # Echo lines
      (out Sock (prinl (line))) ) )

(prinl (stamp) " -- (Pid " *Pid ") Client disconnected")
(bye)                                  # Terminate child

\end{verbatim}

\section*{Element-wise operations}

\begin{verbatim}

(de elementWiseMatrix (Fun Mat1 Mat2)
   (mapcar '((L1 L2) (mapcar Fun L1 L2)) Mat1 Mat2) )

(de elementWiseScalar (Fun Mat Scalar)
   (elementWiseMatrix Fun Mat (circ (circ Scalar))) )

Test:

(let (S 10  M '((7 11 13) (17 19 23) (29 31 37)))
   (println (elementWiseScalar + M S))
   (println (elementWiseScalar - M S))
   (println (elementWiseScalar * M S))
   (println (elementWiseScalar / M S))
   (println (elementWiseScalar ** M S))
   (prinl)
   (println (elementWiseMatrix + M M))
   (println (elementWiseMatrix - M M))
   (println (elementWiseMatrix * M M))
   (println (elementWiseMatrix / M M))
   (println (elementWiseMatrix ** M M)) )

Output:

((17 21 23) (27 29 33) (39 41 47))
((-3 1 3) (7 9 13) (19 21 27))
((70 110 130) (170 190 230) (290 310 370))
((0 1 1) (1 1 2) (2 3 3))
((282475249 25937424601 137858491849) (2015993900449 6131066257801 ...

((14 22 26) (34 38 46) (58 62 74))
((0 0 0) (0 0 0) (0 0 0))
((49 121 169) (289 361 529) (841 961 1369))
((1 1 1) (1 1 1) (1 1 1))
((823543 285311670611 302875106592253) (827240261886336764177 ...

\end{verbatim}

\section*{Empty program}

\begin{verbatim}

(de foo ())

\end{verbatim}

\section*{Empty string}

\begin{verbatim}

The empty string is represented by '[http://software-lab.de/doc/ref.html#nilSym
NIL]' in PicoLisp. During input, two subsequent double qoutes '""' return the
symbol NIL.

# To assign a variable an empty string:
(off String)
(setq String "")
(setq String NIL)

# To check for an empty string:
(or String ..)
(ifn String ..)
(unless String ..)

# or a non-empty string:
(and String ..)
(if String ..)
(when String ..)

\end{verbatim}

\section*{Ensure that a file exists}

\begin{verbatim}

(if (info "file.txt")
   (prinl "Size: " (car @) " bytes, last modified " (stamp (cadr @) (cddr @)))
   (prinl "File doesn't exist") )

\end{verbatim}

\section*{Enumerations}

\begin{verbatim}

Enumerations are not very useful in a symbolic language like PicoLisp. If
desired, an 'enum' function could be defined:

(de enum "Args"
   (mapc def "Args" (range 1 (length "Args"))) )

: (enum A B C D E F)
-> F

: A
-> 1
: B
-> 2
: F
-> 6

\end{verbatim}

\section*{Environment variables}

\begin{verbatim}

: (sys "TERM")
-> "xterm"

: (sys "SHELL")
-> "/bin/bash"

\end{verbatim}

\section*{Equilibrium index}

\begin{verbatim}

(de equilibria (Lst)
   (make
      (let Sum 0
         (for ((I . L) Lst L (cdr L))
            (and (= Sum (sum prog (cdr L))) (link I))
            (inc 'Sum (car L)) ) ) ) )

Output:

: (equilibria (-7 1 5 2 -4 3 0))
-> (4 7)

: (equilibria (make (do 10000 (link (rand -10 10)))))
-> (4091 6174 6198 7104 7112 7754)

\end{verbatim}

\section*{Ethiopian multiplication}

\begin{verbatim}

(de halve (N)
   (/ N 2) )

(de double (N)
   (* N 2) )

(de even? (N)
   (not (bit? 1 N)) )

(de ethiopian (X Y)
   (let R 0
      (while (>= X 1)
         (or (even? X) (inc 'R Y))
         (setq
            X (halve X)
            Y (double Y) ) )
      R ) )

\end{verbatim}

\section*{Euler Method}

\begin{verbatim}

(load "@lib/math.l")

(de euler (F Y A B H)
   (while (> B A)
      (prinl (round A) " " (round Y))
      (inc 'Y (*/ H (F A Y) 1.0))
      (inc 'A H) ) )

(de newtonCoolingLaw (A B)
   (*/ -0.07 (- B 20.) 1.0) )

(euler newtonCoolingLaw 100.0 0 100.0 2.0)
(euler newtonCoolingLaw 100.0 0 100.0 5.0)
(euler newtonCoolingLaw 100.0 0 100.0 10.0)

Output:

...
0.000 100.000
10.000 44.000
20.000 27.200
30.000 22.160
40.000 20.648
50.000 20.194
60.000 20.058
70.000 20.018
80.000 20.005
90.000 20.002

\end{verbatim}

\section*{Evaluate binomial coefficients}

\begin{verbatim}

(de binomial (N K)
   (let f '((N) (apply * (range 1 N)))
      (/ (f N) (* (f (- N K)) (f K))) ) )

Output:

: (binomial 5 3)
-> 10

\end{verbatim}

\section*{Even or odd}

\begin{verbatim}

PicoLisp doesn't have a built-in predicate for that. Using
'[http://software-lab.de/doc/refB.html#bit? bit?]' is the easiest and most
efficient. The bit test with 1 will return NIL if the number is even.

: (bit? 1 3)
-> 1  # Odd

: (bit? 1 4)
-> NIL  # Even

\end{verbatim}

\section*{Events}

\begin{verbatim}

PicoLisp supports events from timers (via
'[http://software-lab.de/doc/refT.html#task task]' and
'[http://software-lab.de/doc/refA.html#alarm alarm]'),
file descriptors (also 'task') and various
'[http://software-lab.de/doc/refS.html#*Sig1 signals]'.
This will print a message after one second, then terminate the program after
another four seconds:

(alarm 1
   (prinl "Exit in 4 seconds")
   (alarm 4 (bye)) )

\end{verbatim}

\section*{Evolutionary algorithm}

\begin{verbatim}
This example uses 'gen', the genetic function in "lib/simul.l"

(load "@lib/simul.l")

(setq *Target (chop "METHINKS IT IS LIKE A WEASEL"))

# Generate random character
(de randChar ()
   (if (=0 (rand 0 26))
      " "
      (char (rand `(char "A") `(char "Z"))) ) )

# Fitness function (Hamming distance)
(de fitness (A)
   (cnt = A *Target) )

# Genetic algorithm
(gen
   (make                               # Parent population
      (do 100                             # C = 100 children
         (link
            (make
               (do (length *Target)
                  (link (randChar)) ) ) ) ) )
   '((A)                               # Termination condition
      (prinl (maxi fitness A))            # Print the fittest element
      (member *Target A) )                # and check if solution is found
   '((A B)                             # Recombination function
      (mapcar
         '((C D) (if (rand T) C D))       # Pick one of the chars
         A B ) )
   '((A)                               # Mutation function
      (mapcar
         '((C)
            (if (=0 (rand 0 10))          # With a proability of 10\%
               (randChar)                 # generate a new char, otherwise
               C ) )                      # return the current char
         A ) )
   fitness )                           # Selection function

Output:

RQ ASLWWWI ANSHPNABBAJ ZLTKX
DETGGNGHWITIKSXLIIEBA WAATPC
CETHINWS ITKESQGIKE A WSAGHO
METHBNWS IT NSQLIKE A WEAEWL
METHINKS IT ISCLIKE A WVASEL
METHINKS IT ISOLIKE A WEASEL
METHINKS IT IS LIKE A WEASEL

\end{verbatim}

\section*{Exceptions}

\begin{verbatim}

[http://software-lab.de/doc/refC.html#catch catch], [http://software-lab.de/doc/refT.html#throw throw]
(and [http://software-lab.de/doc/refF.html#finally finally]) can be used for
exception handling. 'throw' will transfer control to a 'catch' environment that
was set up with the given label.

(catch 'thisLabel          # Catch this label
   (println 1)             # Do some processing (print '1')
   (throw 'thisLabel 2)    # Abort processing and return '2'
   (println 3) )           # This is never reached

Output:

1        # '1' is printed
-> 2     # '2' is returned

\end{verbatim}

\section*{Exceptions/Catch an exception thrown in a nested call}

\begin{verbatim}

(de foo ()
   (for Tag '(U0 U1)
      (catch 'U0
         (bar Tag) ) ) )

(de bar (Tag)
   (baz Tag) )

(de baz (Tag)
   (throw Tag) )

(mapc trace '(foo bar baz))
(foo)

Output:

 foo :
  bar : U0
   baz : U0
  bar : U1
   baz : U1
[x:13] !? (throw Tag)
U1 -- Tag not found
?                          # Debug prompt

\end{verbatim}

\section*{Executable library}

\begin{verbatim}

There is no formal difference between libraries and other executable files in
PicoLisp. Any function in a library can be called from the command line by
prefixing it with '-'. Create an executable file (chmod +x) "hailstone.l":

#!/usr/bin/picolisp /usr/lib/picolisp/lib.l

(de hailstone (N)
   (make
      (until (= 1 (link N))
         (setq N
            (if (bit? 1 N)
               (inc (* N 3))
               (/ N 2) ) ) ) ) )

(de hailtest ()
   (let L (hailstone 27)
      (test 112 (length L))
      (test (27 82 41 124) (head 4 L))
      (test (8 4 2 1) (tail 4 L)) )
   (let N (maxi '((N) (length (hailstone N))) (range 1 100000))
      (test 77031 N)
      (test 351 (length (hailstone N))) )
   (println 'OK)
   (bye) )

and an executable file (chmod +x) "test.l":

#!/usr/bin/picolisp /usr/lib/picolisp/lib.l

(load "hailstone.l")

(let Len NIL
   (for N 100000
      (accu 'Len (length (hailstone N)) 1) )
   (let M (maxi cdr Len)
      (prinl "The hailstone length returned most often is " (car M))
      (prinl "It is returned " (cdr M) " times") ) )
(bye)

Test:

\$ ./hailstone.l -hailtest
OK

\$ ./test.l
The hailstone length returned most often is 72
It is returned 1467 times

\end{verbatim}

\section*{Execute Brain****}

\begin{verbatim}

This solution uses a doubly-linked list for the cell space. That list consists
of a single cell initially, and grows automatically in both directions. The
value in each cell is unlimited.

(off "Program")

(de compile (File)
   (let Stack NIL
      (setq "Program"
         (make
            (in File
               (while (char)
                  (case @
                     (">"
                        (link
                           '(setq Data
                              (or
                                 (cddr Data)
                                 (con (cdr Data) (cons 0 (cons Data))) ) ) ) )
                     ("<"
                        (link
                           '(setq Data
                              (or
                                 (cadr Data)
                                 (set (cdr Data) (cons 0 (cons NIL Data))) ) ) ) )
                     ("+" (link '(inc Data)))
                     ("-" (link '(dec Data)))
                     ("." (link '(prin (char (car Data)))))
                     ("," (link '(set Data (char (read)))))
                     ("["
                        (link
                           '(setq Code
                              ((if (=0 (car Data)) cdar cdr) Code) ) )
                        (push 'Stack (chain (cons))) )
                     ("]"
                        (unless Stack
                           (quit "Unbalanced ']'") )
                        (link
                           '(setq Code
                              ((if (n0 (car Data)) cdar cdr) Code) ) )
                        (let (There (pop 'Stack)  Here (cons There))
                           (chain (set There Here)) ) ) ) ) ) ) )
      (when Stack
         (quit "Unbalanced '['") ) ) )

(de execute ()
   (let Data (cons 0 (cons))              # Create initial cell
      (for (Code "Program"  Code)         # Run program
         (eval (pop 'Code)) )
      (while (cadr Data)                  # Find beginning of data
         (setq Data @) )
      (filter prog Data '(T NIL .)) ) )   # Return data space

Output:

: (compile "hello.bf")
-> NIL

: (execute)
Goodbye, World!
-> (0 10 33 44 71 87 98 100 114 121)

\end{verbatim}

\section*{Execute HQ9+}

\begin{verbatim}

(de hq9+ (Code)
   (let Accu 0
      (for C (chop Code)
         (case C
            ("H" (prinl "Hello, world"))
            ("Q" (prinl Code))
            ("9"
               (for (N 99 (gt0 N))
                  (prinl N " bottles of beer on the wall")
                  (prinl N " bottles of beer")
                  (prinl "Take one down, pass it around")
                  (prinl (dec 'N) " bottles of beer on the wall")
                  (prinl) ) )
            ("+" (inc 'Accu)) ) )
      Accu ) )

\end{verbatim}

\section*{Execute a Markov algorithm}

\begin{verbatim}

(de markov (File Text)
   (use (@A @Z R)
      (let Rules
         (make
            (in File
               (while (skip "#")
                  (when (match '(@A " " "-" ">" " " @Z) (replace (line) "@" "#"))
                     (link (cons (clip @A) (clip @Z))) ) ) ) )
         (setq Text (chop Text))
         (pack
            (loop
               (NIL
                  (find
                     '((R) (match (append '(@A) (car R) '(@Z)) Text))
                     Rules )
                  Text )
               (T (= "." (cadr (setq R @)))
                  (append @A (cddr R) @Z) )
               (setq Text (append @A (cdr R) @Z)) ) ) ) ) )

Output:

: (markov "r1" "I bought a B of As from T S.")
-> "I bought a bag of apples from my brother."

: (markov "r2" "I bought a B of As from T S.")
-> "I bought a bag of apples from T shop."

: (markov "r3" "I bought a B of As W my Bgage from T S.")
-> "I bought a bag of apples with my money from T shop."

: (markov "r4" "_1111*11111_")
-> "11111111111111111111"

: (markov "r5" "000000A000000")
-> "00011H1111000"

\end{verbatim}

\section*{Execute a system command}

\begin{verbatim}

(call "ls")

\end{verbatim}

\section*{Exponentiation operator}

\begin{verbatim}

This uses Knuth's algorithm (The Art of Computer Programming, Vol. 2, page 442)

(de ** (X N)  # N th power of X
   (let Y 1
      (loop
         (when (bit? 1 N)
            (setq Y (* Y X)) )
         (T (=0 (setq N (>> 1 N)))
            Y )
         (setq X (* X X)) ) ) )

\end{verbatim}

\section*{Extend your language (!)}

\begin{verbatim}

(undef 'if2)  # Undefine the built-in 'if2'

(de if2 "P"
   (if (eval (pop '"P"))
      (eval ((if (eval (car "P")) cadr caddr) "P"))
      (if (eval (car "P"))
         (eval (cadddr "P"))
         (run (cddddr "P")) ) ) )

Usage:

(if2 (condition1isTrue) (condition2isTrue)
   (bothConditionsAreTrue)             # A single expression in each of the
   (firstConditionIsTrue)              # first three branches
   (secondConditionIsTrue)
   (noConditionIsTrue)                 # The final branch may contain
   (...) )                             # an arbitrary number of expressions

As another example of language extension, see [[Anonymous recursion#PicoLisp]].

\end{verbatim}

\section*{Extreme floating point values}

\begin{verbatim}

PicoLisp has only very limited built-in floating point support, and handles the
rest by calling native (typically C) libraries. Minus zero and negative infinity
cannot be represented, while NaN is represented by NIL

(load "@lib/math.l")

: (exp 1000.0)  # Too large for IEEE floats
-> T

: (+ 1 2 NIL 3)  # NaN propagates
-> NIL

\end{verbatim}

\chapter{Rosetta Code Tasks starting with F}

\section*{Factorial}

\begin{verbatim}

(de fact (N)
   (if (=0 N)
      1
      (* N (fact (dec N))) ) )

or

(de fact (N)
   (apply * (range 1 N)) )

\end{verbatim}

\section*{Factors of a Mersenne number}

\begin{verbatim}

(de **Mod (X Y N)
   (let M 1
      (loop
         (when (bit? 1 Y)
            (setq M (\% (* M X) N)) )
         (T (=0 (setq Y (>> 1 Y)))
            M )
         (setq X (\% (* X X) N)) ) ) )

(de prime? (N)
   (or
      (= N 2)
      (and
         (> N 1)
         (bit? 1 N)
         (for (D 3  T  (+ D 2))
            (T (> D (sqrt N)) T)
            (T (=0 (\% N D)) NIL) ) ) ) )

(de mFactor (P)
   (let (Lim (sqrt (dec (** 2 P)))  K 0  Q)
      (loop
         (setq Q (inc (* 2 (inc 'K) P)))
         (T (>= Q Lim) NIL)
         (T
            (and
               (member (\% Q 8) (1 7))
               (prime? Q)
               (= 1 (**Mod 2 P Q)) )
            Q ) ) ) )

Output:

: (for P (2 3 4 5 7 11 13 17 19 23 29 31 37 41 43 47 53 929)
   (prinl
      "M" P " = 2**" P "-1 is "
      (cond
         ((not (prime? P)) "not prime")
         ((mFactor P) (pack "composite with factor " @))
         (T "prime") ) ) )
M2 = 2**2-1 is prime
M3 = 2**3-1 is prime
M4 = 2**4-1 is not prime
M5 = 2**5-1 is prime
M7 = 2**7-1 is prime
M11 = 2**11-1 is composite with factor 23
M13 = 2**13-1 is prime
M17 = 2**17-1 is prime
M19 = 2**19-1 is prime
M23 = 2**23-1 is composite with factor 47
M29 = 2**29-1 is composite with factor 233
M31 = 2**31-1 is prime
M37 = 2**37-1 is composite with factor 223
M41 = 2**41-1 is composite with factor 13367
M43 = 2**43-1 is composite with factor 431
M47 = 2**47-1 is composite with factor 2351
M53 = 2**53-1 is composite with factor 6361
M929 = 2**929-1 is composite with factor 13007

\end{verbatim}

\section*{Factors of an integer}

\begin{verbatim}

(de factors (N)
   (filter
      '((D) (=0 (\% N D)))
      (range 1 N) ) )

\end{verbatim}

\section*{Fast Fourier transform}

\begin{verbatim}

{{works with|PicoLisp|3.1.0.3}}

# apt-get install libfftw3-dev

(scl 4)

(de FFTW_FORWARD . -1)
(de FFTW_ESTIMATE . 64)

(de fft (Lst)
   (let
      (Len (length Lst)
         In (native "libfftw3.so" "fftw_malloc" 'N (* Len 16))
         Out (native "libfftw3.so" "fftw_malloc" 'N (* Len 16))
         P (native "libfftw3.so" "fftw_plan_dft_1d" 'N
            Len In Out FFTW_FORWARD FFTW_ESTIMATE ) )
      (struct In NIL (cons 1.0 (apply append Lst)))
      (native "libfftw3.so" "fftw_execute" NIL P)
      (prog1 (struct Out (make (do Len (link (1.0 . 2)))))
         (native "libfftw3.so" "fftw_destroy_plan" NIL P)
         (native "libfftw3.so" "fftw_free" NIL Out)
         (native "libfftw3.so" "fftw_free" NIL In) ) ) )

Test:

(for R (fft '((1.0 0) (1.0 0) (1.0 0) (1.0 0) (0 0) (0 0) (0 0) (0 0)))
   (tab (6 8)
      (round (car R))
      (round (cadr R)) ) )

Output:

 4.000   0.000
 1.000  -2.414
 0.000   0.000
 1.000  -0.414
 0.000   0.000
 1.000   0.414
 0.000   0.000
 1.000   2.414

\end{verbatim}

\section*{Fibonacci n-step number sequences}

\begin{verbatim}

(de nacci (Init Cnt)
   (let N (length Init)
      (make
         (made Init)
         (do (- Cnt N)
            (link (apply + (tail N (made)))) ) ) ) )

Test:
# Fibonacci
: (nacci (1 1) 10)
-> (1 1 2 3 5 8 13 21 34 55)

# Tribonacci
: (nacci (1 1 2) 10)
-> (1 1 2 4 7 13 24 44 81 149)

# Tetranacci
: (nacci (1 1 2 4) 10)
-> (1 1 2 4 8 15 29 56 108 208)

# Lucas
: (nacci (2 1) 10)
-> (2 1 3 4 7 11 18 29 47 76)

# Decanacci
: (nacci (1 1 2 4 8 16 32 64 128 256) 15)
-> (1 1 2 4 8 16 32 64 128 256 512 1023 2045 4088 8172)

\end{verbatim}

\section*{Fibonacci sequence}

\begin{verbatim}

Recursive

(de fibo (N)
   (if (> 2 N)
      1
      (+ (fibo (dec N)) (fibo (- N 2))) ) )

Recursive with Cache

Using a recursive version doesn't need to be slow, as the following shows:

(de fibo (N)
   (cache '(NIL) (pack (char (hash N)) N)  # Use a cache to accelerate
      (if (> 2 N)
         N
         (+ (fibo (dec N)) (fibo (- N 2))) ) ) )

(bench (fibo 1000))

Output:

0.012 sec
-> 43466557686937456435688527675040625802564660517371780402481729089536555417949
05189040387984007925516929592259308032263477520968962323987332247116164299644090
6533187938298969649928516003704476137795166849228875

\end{verbatim}

\section*{File IO}

\begin{verbatim}

# Using a variable

(let V (in "input.txt" (till))
   (out "output.txt" (prin V)) )

# Skipping intermediate variable

(in "input.txt"
   (out "output.txt"
      (echo) ) )

\end{verbatim}

\section*{File modification time}

\begin{verbatim}

(let File "test.file"
   (and
      (info File)
      (prinl (stamp (cadr @) (cddr @))) ) # Print date and time in UTC
   (call 'touch File) )                   # Set modification time to "now"

\end{verbatim}

\section*{File size}

\begin{verbatim}

(println (car (info "input.txt")))
(println (car (info "/input.txt")))

\end{verbatim}

\section*{Filter}

\begin{verbatim}

(filter '((N) (not (bit? 1 N)))
   (1 2 3 4 5 6 7 8 9) )

Output:

-> (2 4 6 8)

\end{verbatim}

\section*{Find Common Directory Path}

\begin{verbatim}

(de commonPath (Lst Chr)
   (glue Chr
      (make
         (apply find
            (mapcar '((L) (split (chop L) Chr)) Lst)
            '(@ (or (pass <>) (nil (link (next))))) ) ) ) )

Output:

(commonPath
   (quote
      "/home/user1/tmp/coverage/test"
      "/home/user1/tmp/covert/operator"
      "/home/user1/tmp/coven/members" )
   "/" )

-> "/home/user1/tmp"

\end{verbatim}

\section*{Find first and last set bit of a long integer}

\begin{verbatim}

(de msb (N)
   (dec (length (bin (abs N)))) )

(de lsb (N)
   (length (stem (chop (bin N)) "1")) )

Test:

(for N (1 42 717368321110468608 291733167875766667063796853374976)
   (tab (33 6 6) N (lsb N) (msb N)) )

Output:

                                1     0     0
                               42     1     5
               717368321110468608    11    59
291733167875766667063796853374976    20   107

\end{verbatim}

\section*{Find limit of recursion}

\begin{verbatim}

The 64-bit and the 32-bit version behave slightly different. While the 32-bit
version imposes no limit on its own, and relies on the 'ulimit' setting of the
caller, the 64-bit version segments the available stack (likewise depending on
'ulimit') and allows each (co)routine a maximal stack size as configured by
'[http://software-lab.de/doc/refS.html#stack stack]'.

# 32-bit version

\$ ulimit -s
8192
\$ pil +
: (let N 0 (recur (N) (recurse (msg (inc N)))))
...
730395
730396
730397
Segmentation fault

# 64-bit version

\$ ulimit -s
unlimited
\$ pil +
: (stack)  # The default stack segment size is 4 MB
-> 4

: (co 'a (yield y))  # Start a dummy coroutine
-> 7

: (let N 0 (recur (N) (recurse (println (inc N)))))
...
43642
43643
43644
Stack overflow
?

\end{verbatim}

\section*{Find the missing permutation}

\begin{verbatim}

(setq *PermList
   (mapcar chop
      (quote
         "ABCD" "CABD" "ACDB" "DACB" "BCDA" "ACBD" "ADCB" "CDAB"
         "DABC" "BCAD" "CADB" "CDBA" "CBAD" "ABDC" "ADBC" "BDCA"
         "DCBA" "BACD" "BADC" "BDAC" "CBDA" "DBCA" "DCAB" ) ) )

(let (Lst (chop "ABCD")  L Lst)
   (recur (L)  # Permute
      (if (cdr L)
         (do (length L)
            (recurse (cdr L))
            (rot L) )
         (unless (member Lst *PermList)  # Check
            (prinl Lst) ) ) ) )

Output:

DBAC

\end{verbatim}

\section*{First class environments}

\begin{verbatim}

Runtime environments can be controlled with the
'[http://software-lab.de/doc/refJ.html#job job]' function:

(let Envs
   (mapcar
      '((N) (list (cons 'N N) (cons 'Cnt 0)))  # Build environments
      (range 1 12) )
   (while (find '((E) (job E (> N 1))) Envs)   # Until all values are 1:
      (for E Envs
         (job E                                # Use environment 'E'
            (prin (align 4 N))
            (unless (= 1 N)
               (inc 'Cnt)                      # Increment step count
               (setq N
                  (if (bit? 1 N)               # Calculate next hailstone value
                     (inc (* N 3))
                     (/ N 2) ) ) ) ) )
      (prinl) )
   (prinl (need 48 '=))
   (for E Envs                                 # For each environment 'E'
      (job E
         (prin (align 4 Cnt)) ) )              # print the step count
   (prinl) )

Output:

   1   2   3   4   5   6   7   8   9  10  11  12
   1   1  10   2  16   3  22   4  28   5  34   6
   1   1   5   1   8  10  11   2  14  16  17   3
   1   1  16   1   4   5  34   1   7   8  52  10
   1   1   8   1   2  16  17   1  22   4  26   5
   1   1   4   1   1   8  52   1  11   2  13  16
   1   1   2   1   1   4  26   1  34   1  40   8
   1   1   1   1   1   2  13   1  17   1  20   4
   1   1   1   1   1   1  40   1  52   1  10   2
   1   1   1   1   1   1  20   1  26   1   5   1
   1   1   1   1   1   1  10   1  13   1  16   1
   1   1   1   1   1   1   5   1  40   1   8   1
   1   1   1   1   1   1  16   1  20   1   4   1
   1   1   1   1   1   1   8   1  10   1   2   1
   1   1   1   1   1   1   4   1   5   1   1   1
   1   1   1   1   1   1   2   1  16   1   1   1
   1   1   1   1   1   1   1   1   8   1   1   1
   1   1   1   1   1   1   1   1   4   1   1   1
   1   1   1   1   1   1   1   1   2   1   1   1
================================================
   0   1   7   2   5   8  16   3  19   6  14   9

\end{verbatim}

\section*{First-class functions}

\begin{verbatim}

(load "@lib/math.l")

(de compose (F G)
   (curry (F G) (X)
      (F (G X)) ) )

(de cube (X)
   (pow X 3.0) )

(de cubeRoot (X)
   (pow X 0.3333333) )

(mapc
   '((Fun Inv)
      (prinl (format ((compose Inv Fun) 0.5) *Scl)) )
   '(sin  cos  cube)
   '(asin acos cubeRoot) )

Output:

0.500001
0.499999
0.500000

\end{verbatim}

\section*{First-class functions/Use numbers analogously}

\begin{verbatim}

(load "@lib/math.l")

(de multiplier (N1 N2)
   (curry (N1 N2) (X)
      (*/ N1 N2 X `(* 1.0 1.0)) ) )

(let (X 2.0  Xi 0.5  Y 4.0  Yi 0.25  Z (+ X Y)  Zi (*/ 1.0 1.0 Z))
   (mapc
      '((Num Inv)
         (prinl (format ((multiplier Inv Num) 0.5) *Scl)) )
      (list X Y Z)
      (list Xi Yi Zi) ) )

Output:

0.500000
0.500000
0.500001

\end{verbatim}

\section*{Five weekends}

\begin{verbatim}

(setq Lst
   (make
      (for Y (range 1900 2100)
         (for M (range 1 12)
            (and
               (date Y M 31)
               (= "Friday" (day (date Y M 1)))
               (link (list (get *Mon M) Y)) ) ) ) ) )

(prinl "There are " (length Lst) " months with five weekends:")
(mapc println (head 5 Lst))
(prinl "...")
(mapc println (tail 5 Lst))
(prinl)
(setq Lst (diff (range 1900 2100) (uniq (mapcar cadr Lst))))
(prinl "There are " (length Lst) " years with no five-weekend months:")
(println Lst)

Output:

There are 201 months with five weekends:
(Mar 1901)
(Aug 1902)
(May 1903)
(Jan 1904)
(Jul 1904)
...
(Mar 2097)
(Aug 2098)
(May 2099)
(Jan 2100)
(Oct 2100)

There are 29 years with no five-weekend months:
(1900 1906 1917 1923 1928 1934 1945 1951 1956 1962 1973 1979 1984 1990 2001 2007
2012 2018 2029 2035 2040 2046 2057 2063 2068 2074 2085 2091 2096)

\end{verbatim}

\section*{FizzBuzz}

\begin{verbatim}

We could simply use '[http://software-lab.de/doc/refA.html#at at]' here:

(for N 100
   (prinl
      (or (pack (at (0 . 3) "Fizz") (at (0 . 5) "Buzz")) N) ) )

Or do it the standard way:

(for N 100
   (prinl
      (cond
         ((=0 (\% N 15)) "FizzBuzz")
         ((=0 (\% N 3)) "Fizz")
         ((=0 (\% N 5)) "Buzz")
         (T N) ) ) )

\end{verbatim}

\section*{Flatten a list}

\begin{verbatim}

(de flatten (X)
   (make                               # Build a list
      (recur (X)                       # recursively over 'X'
         (if (atom X)
            (link X)                   # Put atoms into the result
            (mapc recurse X) ) ) ) )   # or recurse on sub-lists

More succinct (by armadillo):

(de flatten (X)
   (fish atom X) )

\end{verbatim}

\section*{Flow-control structures}

\begin{verbatim}

As this task asks for the documentation of common flow control structures, we
refer here to the online documentation for more complete descriptions and
examples.

Relevant functions are:

# fork
[http://software-lab.de/doc/refF.html#fork fork] creates a child process

# task
[http://software-lab.de/doc/refT.html#task task] installs a background task
consisting of an environment and a list of executable expressions

# alarm
[http://software-lab.de/doc/refA.html#alarm alarm] schedules a timer, which
runs a given list of executable expressions when it expires

# abort
[http://software-lab.de/doc/refA.html#abort abort] runs a given list of
executable expressions, and aborts processing it if it takes longer than
a given time

# quit
[http://software-lab.de/doc/refQ.html#quit quit] immediately stops all
execution and returns to the top level read-eval-print loop, optionally
signaling an error

# wait
[http://software-lab.de/doc/refW.html#wait wait] delays current processing
(optionally to a maximal time) until an optionally given condition
evaluates to non-NIL

# sync
[http://software-lab.de/doc/refS.html#sync sync] synchronizes with other
processes of the same family

# protect
[http://software-lab.de/doc/refP.html#protect protect] delays the processing
of signals while a given list of executable expressions is executed

# catch
[http://software-lab.de/doc/refC.html#catch catch] prepares for receiving a
'throw' while running a given list of executable expressions

# throw
[http://software-lab.de/doc/refT.html#throw throw] causes a non-local jump
to a specified 'catch' environment

# bye
[http://software-lab.de/doc/refB.html#bye bye] exits the interpreter

# finally
[http://software-lab.de/doc/refF.html#finally finally] specifies a list of
executable expressions, to be run when current processing is done, even if
a 'throw' or 'bye' was executed, or an error occurred.

\end{verbatim}

\section*{Floyd's triangle}

\begin{verbatim}

Calculate widths relative to lower left corner

(de floyd (N)
   (let LLC (/ (* N (dec N)) 2)
      (for R N
         (for C R
            (prin
               (align
                  (length (+ LLC C))
                  (+ C (/ (* R (dec R)) 2)) ) )
            (if (= C R) (prinl) (space)) ) ) ) )

Pre-calculate all rows, and take format from last one

(de floyd (N)
   (let
      (Rows
         (make
            (for ((I . L) (range 1 (/ (* N (inc N)) 2))  L)
               (link (cut I 'L)) ) )
         Fmt (mapcar length (last Rows)) )
      (map inc (cdr Fmt))
      (for R Rows
         (apply tab R Fmt) ) ) )

Output in both cases:

: (floyd 5)
 1
 2  3
 4  5  6
 7  8  9 10
11 12 13 14 15

: (floyd 14)
 1
 2  3
 4  5  6
 7  8  9 10
11 12 13 14 15
16 17 18 19 20 21
22 23 24 25 26 27 28
29 30 31 32 33 34 35 36
37 38 39 40 41 42 43 44  45
46 47 48 49 50 51 52 53  54  55
56 57 58 59 60 61 62 63  64  65  66
67 68 69 70 71 72 73 74  75  76  77  78
79 80 81 82 83 84 85 86  87  88  89  90  91
92 93 94 95 96 97 98 99 100 101 102 103 104 105

\end{verbatim}

\section*{Forest fire}

\begin{verbatim}

(load "@lib/simul.l")

(scl 3)

(de forestFire (Dim ProbT ProbP ProbF)
   (let Grid (grid Dim Dim)
      (for Col Grid
         (for This Col
            (=: tree (> ProbT (rand 0 1.0))) ) )
      (loop
         (disp Grid NIL
            '((This)
               (cond
                  ((: burn) "# ")
                  ((: tree) "T ")
                  (T ". ") ) ) )
         (wait 1000)
         (for Col Grid
            (for This Col
               (=: next
                  (cond
                     ((: burn) NIL)
                     ((: tree)
                        (if
                           (or
                              (find  # Neighbor burning?
                                 '((Dir) (get (Dir This) 'burn))
                                 (quote
                                    west east south north
                                    ((X) (south (west X)))
                                    ((X) (north (west X)))
                                    ((X) (south (east X)))
                                    ((X) (north (east X))) ) )
                              (> ProbF (rand 0 1.0)) )
                           'burn
                           'tree ) )
                     (T (and (> ProbP (rand 0 1.0)) 'tree)) ) ) ) )
         (for Col Grid
            (for This Col
               (if (: next)
                  (put This @ T)
                  (=: burn)
                  (=: tree) ) ) ) ) ) )

Use:

(forestFire 26 0.5 0.01 0.001)

\end{verbatim}

\section*{Fork}

\begin{verbatim}

(unless (fork)                         # In child process
   (println *Pid)                      # Print the child's PID
   (bye) )                             # and terminate

\end{verbatim}

\section*{Formal power series}

\begin{verbatim}

With a 'lazy' function, as a frontend to '[http://software-lab.de/doc/refC.html#cache cache]',

(de lazy Args
   (def (car Args)
      (list (cadr Args)
         (cons 'cache (lit (cons))
            (list 'pack (list 'char (list 'hash (caadr Args))) (caadr Args))
            (cddr Args) ) ) ) )

we can build a formal power series functionality:

(scl 20)

(de fpsOne (N)
   (if (=0 N) 1.0 0) )

(de fpsInverse (N X)
   (last
      (make
         (let Res1 (- (link (*/ 1.0 1.0 (X 0))))
            (for I N
               (link
                  (*/
                     (sum '((Res J) (*/ (X J) Res 1.0))
                        (made)
                        (range I 1) )
                     Res1
                     1.0 ) ) ) ) ) ) )

(de fpsAdd (N X Y)
   (+ (X N) (Y N)) )

(de fpsSub (N X Y)
   (- (X N) (Y N)) )

(de fpsMul (N X Y)
   (sum
      '((I)
         (*/ (X I) (Y (- N I)) 1.0) )
      (range 0 N) ) )

(de fpsDiv (N X Y)
   (sum
      '((I)
         (*/ (X I) (fpsInverse (- N I) Y) 1.0) )
      (range 0 N) ) )

(de fpsDifferentiate (N)
   (curry (X) (N)
      (* (X (inc N)) N) ) )

(de fpsIntegrate (X)
   (curry (X) (N)
      (or
         (=0 N)
         (*/ (X (dec N)) N) ) ) )

(lazy fpsSin (N)
   ((fpsIntegrate fpsCos) N) )

(lazy fpsCos (N)
   (fpsSub N fpsOne (fpsIntegrate fpsSin)) )

(lazy fpsTan (N)
   (fpsDiv N fpsSin fpsCos) )

(lazy fpsExp (N)
   (if (=0 N)
      1.0
      ((fpsIntegrate fpsExp) N) ) )

Test:

(prin "SIN:")
(for N (range 1 11 2)
   (prin " " (round (fpsSin N) 9)) )
(prinl)

(prin "COS:")
(for N (range 0 10 2)
   (prin " " (round (fpsCos N) 9)) )
(prinl)

(prin "TAN:")
(for N (range 1 13 2)
   (prin " " (round (fpsTan N) 7)) )
(prinl)

(prin "EXP:")
(for N (range 0 6)
   (prin " " (round (fpsExp N) 7)) )
(prinl)

Output:

SIN: 1.000000000 -0.166666667 0.008333333 -0.000198413 0.000002756 -0.000000025
COS: 1.000000000 -0.500000000 0.041666667 -0.001388889 0.000024802 -0.000000276
TAN: 1.0000000 0.3333333 0.1333333 0.0539683 0.0218695 0.0088632 0.0035921
EXP: 1.0000000 1.0000000 0.5000000 0.1666667 0.0416667 0.0083333 0.0013889

\end{verbatim}

\section*{Formatted numeric output}

\begin{verbatim}

(pad 9 (format 7125 3))
(pad 9 (format 7125 3 ","))  # European format

\end{verbatim}

\section*{Forward difference}

\begin{verbatim}

(de fdiff (Lst)
   (mapcar - (cdr Lst) Lst) )

(for (L (90 47 58 29 22 32 55 5 55 73) L (fdiff L))
   (println L) )

Output:

(90 47 58 29 22 32 55 5 55 73)
(-43 11 -29 -7 10 23 -50 50 18)
(54 -40 22 17 13 -73 100 -32)
(-94 62 -5 -4 -86 173 -132)
(156 -67 1 -82 259 -305)
(-223 68 -83 341 -564)
(291 -151 424 -905)
(-442 575 -1329)
(1017 -1904)
(-2921)

\end{verbatim}

\section*{Four bit adder}

\begin{verbatim}

(de halfAdder (A B)  #> (Carry . Sum)
   (cons
      (and A B)
      (xor A B) ) )

(de fullAdder (A B C)  #> (Carry . Sum)
   (let (Ha1 (halfAdder C A)  Ha2 (halfAdder (cdr Ha1) B))
      (cons
         (or (car Ha1) (car Ha2))
         (cdr Ha2) ) ) )

(de 4bitsAdder (A4 A3 A2 A1  B4 B3 B2 B1)  #> (V S4 S3 S2 S1)
   (let
      (Fa1 (fullAdder A1 B1)
         Fa2 (fullAdder A2 B2 (car Fa1))
         Fa3 (fullAdder A3 B3 (car Fa2))
         Fa4 (fullAdder A4 B4 (car Fa3)) )
      (list
         (car Fa4)
         (cdr Fa4)
         (cdr Fa3)
         (cdr Fa2)
         (cdr Fa1) ) ) )

Output:

: (4bitsAdder NIL NIL NIL T  NIL NIL NIL T)
-> (NIL NIL NIL T NIL)

: (4bitsAdder NIL T NIL NIL  NIL NIL T T)
-> (NIL NIL T T T)

: (4bitsAdder NIL T T T  NIL T T T)
-> (NIL T T T NIL)

: (4bitsAdder T T T T  NIL NIL NIL T)
-> (T NIL NIL NIL NIL)

\end{verbatim}

\section*{Fractal tree}

\begin{verbatim}

This uses the 'brez' line drawing function from
[[Bitmap/Bresenham's line algorithm#PicoLisp]].

(load "@lib/math.l")

(de fractalTree (Img X Y A D)
   (unless (=0 D)
      (let (R (*/ A pi 180.0)  DX (*/ (cos R) D 0.2)  DY (*/ (sin R) D 0.2))
         (brez Img X Y DX DY)
         (fractalTree Img (+ X DX) (+ Y DY) (+ A 30.0) (dec D))
         (fractalTree Img (+ X DX) (+ Y DY) (- A 30.0) (dec D)) ) ) )

(let Img (make (do 300 (link (need 400 0))))       # Create image 400 x 300
   (fractalTree Img 200 300 -90.0 10)              # Draw tree
   (out "img.pbm"                                  # Write to bitmap file
      (prinl "P1")
      (prinl 400 " " 300)
      (mapc prinl Img) ) )

\end{verbatim}

\section*{Function composition}

\begin{verbatim}

(de compose (F G)
   (curry (F G) (X)
      (F (G X)) ) )

(def 'a (compose inc dec))
(def 'b (compose 'inc 'dec))
(def 'c (compose '((A) (inc A)) '((B) (dec B))))

: (a 7)
-> 7

: (b 7)
-> 7

: (c 7)
-> 7

\end{verbatim}

\section*{Function definition}

\begin{verbatim}

(de multiply (A B)
   (* A B) )

\end{verbatim}

\section*{Function frequency}

\begin{verbatim}

(let Freq NIL
   (for "L" (filter pair (extract getd (all)))
      (for "F"
         (filter atom
            (fish '((X) (or (circ? X) (getd X)))
               "L" ) )
         (accu 'Freq "F" 1) ) )
   (for X (head 10 (flip (by cdr sort Freq)))
      (tab (-7 4) (car X) (cdr X)) ) )

Output, for the system in debug mode plus the above code:

quote   310
car     236
cdr     181
setq    148
let     136
if      127
and     124
cons    110
cadr     80
or       76

If the condition in the 5th line (getd X) is replaced with (sym? X), then all
symbols are counted, and the output is

X       566
quote   310
car     236
cdr     181
C       160
N       157
L       155
Lst     152
setq    148
T       144

And if it is replaced with (num? X), it is

1        71
0        38
2        27
3        17
7         9
-1        9
100       8
48        6
43        6
12        6

\end{verbatim}

\chapter{Rosetta Code Tasks starting with G}

\section*{GUI component interaction}

\begin{verbatim}

The standard PicoLisp GUI is HTTP based. Connect your browser to
http://localhost:8080 after starting the following script.

#!/usr/bin/picolisp /usr/lib/picolisp/lib.l

(load "@ext.l" "@lib/http.l" "@lib/xhtml.l" "@lib/form.l")

(de start ()
   (and (app) (zero *Number))
   (action
      (html 0 "Increment" "@lib.css" NIL
         (form NIL
            (gui '(+Var +NumField) '*Number 20 "Value")
            (gui '(+JS +Button) "increment"
               '(inc '*Number) )
            (gui '(+Button) "random"
               '(ask "Reset to a random value?"
                  (setq *Number (rand)) ) ) ) ) ) )

(server 8080 "!start")
(wait)

\end{verbatim}

\section*{GUI enabling/disabling of controls}

\begin{verbatim}

The standard PicoLisp GUI is HTTP based. Connect your browser to
http://localhost:8080 after starting the following script.

#!/usr/bin/picolisp /usr/lib/picolisp/lib.l

(load "@ext.l" "@lib/http.l" "@lib/xhtml.l" "@lib/form.l")

(de start ()
   (and (app) (zero *Number))
   (action
      (html 0 "Enable/Disable" "@lib.css" NIL
         (form NIL
            (gui '(+Var +Able +NumField) '*Number '(=0 *Number) 20 "Value")
            (gui '(+Able +JS +Button) '(> 10 *Number) "increment"
               '(inc '*Number) )
            (gui '(+Able +JS +Button) '(gt0 *Number) "decrement"
               '(dec '*Number) ) ) ) ) )

(server 8080 "!start")
(wait)

\end{verbatim}

\section*{Gamma function}

\begin{verbatim}

{{trans|Ada}}

(scl 28)

(de *A
   ~(flip
      (1.00000000000000000000  0.57721566490153286061 -0.65587807152025388108
      -0.04200263503409523553  0.16653861138229148950 -0.04219773455554433675
      -0.00962197152787697356  0.00721894324666309954 -0.00116516759185906511
      -0.00021524167411495097  0.00012805028238811619 -0.00002013485478078824
      -0.00000125049348214267  0.00000113302723198170 -0.00000020563384169776
       0.00000000611609510448  0.00000000500200764447 -0.00000000118127457049
       0.00000000010434267117  0.00000000000778226344 -0.00000000000369680562
       0.00000000000051003703 -0.00000000000002058326 -0.00000000000000534812
       0.00000000000000122678 -0.00000000000000011813  0.00000000000000000119
       0.00000000000000000141 -0.00000000000000000023  0.00000000000000000002 ) ) )

(de gamma (X)
   (let (Y (- X 1.0)  Sum (car *A))
      (for A (cdr *A)
         (setq Sum (+ A (*/ Sum Y 1.0))) )
      (*/ 1.0 1.0 Sum) ) )

Output:

: (for I (range 1 10)
   (prinl (round (gamma (*/ I 1.0 3)) 14)) )
2.67893853470775
1.35411793942640
1.00000000000000
0.89297951156925
0.90274529295093
1.00000000000000
1.19063934875900
1.50457548825154
1.99999999999397
2.77815847933858

\end{verbatim}

\section*{Generator}

\begin{verbatim}

Coroutines are available only in the 64-bit version.

(de powers (M)
   (co (intern (pack 'powers M))
      (for (I 0 (inc 'I))
         (yield (** I M)) ) ) )

(de filtered (N M)
   (co 'filtered
      (let (V (powers N)  F (powers M))
         (loop
            (if (> V F)
               (setq F (powers M))
               (and (> F V) (yield V))
               (setq V (powers N)) ) ) ) ) )

(do 20 (filtered 2 3))
(do 10 (println (filtered 2 3)))

Output:

529
576
625
676
784
841
900
961
1024
1089

\end{verbatim}

\section*{Generic swap}

\begin{verbatim}

[http://software-lab.de/doc/refX.html#xchg xchg] works with any data type
(let (A 1  B 2)
   (xchg 'A 'B)
   (println A B) )

(let (Lst1 '(a b c)  Lst2 '(d e f))
   (xchg (cdr Lst1) (cdr Lst2))
   (println Lst1 Lst2) )

Output:

2 1
(a e c) (d b f)

\end{verbatim}

\section*{Globally replace text in several files}

\begin{verbatim}

(for File '(a.txt b.txt c.txt)
   (call 'mv File (tmp File))
   (out File
      (in (tmp File)
         (while (echo "Goodbye London!")
            (prin "Hello New York!") ) ) ) )

\end{verbatim}

\section*{Go Fish}

\begin{verbatim}

(de *Ranks
   Ace 2 3 4 5 6 7 8 9 10 Jack Queen King )

(de goFish ()
   (let
      (Ocean (by '(NIL (rand)) sort (mapcan '((R) (need 4 R)) *Ranks))
         Your (cut 9 'Ocean)
         Mine (cut 9 'Ocean)
         YouHave NIL
         YouDont NIL
         YourBooks NIL
         MyBooks NIL
         Reply NIL
         Options NIL
         Request NIL )
      (loop
         (prin "Your Books: ")
         (println YourBooks)
         (prin "My Books:   ")
         (println MyBooks)
         (T (nor Your Mine Ocean)
            (let (Y (length YourBooks)  M (length MyBooks))
               (prinl
                  (cond
                     ((= Y M) "Tie game")
                     ((> Y M) "You won!")
                     (T "I won!") ) ) ) )
         (prin "You have ")
         (println Your)
         (prinl "I have " (length Mine) " cards")
         (loop
            (prin
               (if Ocean
                  "Ask for a rank, lay down a book, or 'draw' a card: "
                  "Ask for a rank or lay down a book: " ) )
            (T (member (setq Reply (read)) *Ranks)
               (ifn (filter = Mine (circ Reply))
                  (prinl
                     "   I don't have any card of rank "
                     (push 'YouHave Reply) )
                  (prin "   I give you ")
                  (println @)
                  (setq
                     Mine (diff Mine @)
                     Your (append @ Your)
                     YouHave (append @ YouHave)
                     YouDont (diff YouDont @) ) ) )
            (T (and Ocean (== 'draw Reply))
               (prinl "   You draw a " (push 'Your (pop 'Ocean)))
               (off YouDont) )
            (cond
               ((atom Reply)
                  (prin "   The rank must be one of ")
                  (println *Ranks) )
               ((and (cdddr Reply) (member (car Reply) *Ranks) (not (cdr (uniq Reply))))
                  (prin "   You lay down the book ")
                  (println (push 'YourBooks Reply))
                  (setq
                     Your (diff Your Reply)
                     YouHave (diff YouHave Reply) ) )
               (T (prinl "   A book consists of four ranks, e.g. (7 7 7 7)")) ) )
         (cond
            ((setq Options (diff (rot Mine) YouDont))
               (setq Request
                  (car
                     (or
                        (sect
                           (filter
                              '((Opt) (= 3 (cnt = Mine (circ Opt))))
                              Options )
                           YouHave )
                        (sect Options YouHave)
                        Options ) ) )
               (loop
                  (prin "Please give me all your " Request "s (or NIL): ")
                  (NIL (setq Reply (read))
                     (push 'YouDont Request)
                     (ifn Ocean
                        (prinl "   I pass")
                        (prinl "   I draw a card")
                        (push 'Mine (pop 'Ocean)) ) )
                  (T (and (pair Reply) (member Request Reply) (not (cdr (uniq Reply))))
                     (setq
                        Your (diff Your Reply)
                        YouHave (diff YouHave Reply)
                        Mine (append Reply Mine) ) )
                  (prinl "   I expect a list of " Request "s") ) )
            (Ocean
               (prinl "   I draw a card")
               (push 'Mine (pop 'Ocean)) )
            (T (prinl "   I pass")) )
         (while (find '((R) (= 4 (cnt = Mine (circ R)))) *Ranks)
            (let B (need 4 @)
               (prin "   I lay down the book ")
               (println (push 'MyBooks B))
               (setq Mine (diff Mine B)) ) )
         (prinl) ) ) )

\end{verbatim}

\section*{Gray code}

\begin{verbatim}

(de grayEncode (N)
   (bin (x| N (>> 1 N))) )

(de grayDecode (G)
   (bin
      (pack
         (let X 0
            (mapcar
               '((C) (setq X (x| X (format C))))
               (chop G) ) ) ) ) )

Test:

(prinl "       Binary     Gray  Decoded")
(for I (range 0 31)
   (let G (grayEncode I)
      (tab (4 9 9 9) I (bin I) G (grayDecode G)) ) )

Output:

       Binary     Gray  Decoded
   0        0        0        0
   1        1        1        1
   2       10       11        2
   3       11       10        3
   4      100      110        4
   5      101      111        5
   6      110      101        6
   7      111      100        7
   8     1000     1100        8
   9     1001     1101        9
  10     1010     1111       10
  11     1011     1110       11
  12     1100     1010       12
  13     1101     1011       13
  14     1110     1001       14
  15     1111     1000       15
  16    10000    11000       16
  17    10001    11001       17
  18    10010    11011       18
  19    10011    11010       19
  20    10100    11110       20
  21    10101    11111       21
  22    10110    11101       22
  23    10111    11100       23
  24    11000    10100       24
  25    11001    10101       25
  26    11010    10111       26
  27    11011    10110       27
  28    11100    10010       28
  29    11101    10011       29
  30    11110    10001       30
  31    11111    10000       31

\end{verbatim}

\section*{Grayscale image}

\begin{verbatim}

# Convert color image (PPM) to greyscale image (PGM)
(de ppm->pgm (Ppm)
   (mapcar
      '((Y)
         (mapcar
            '((C)
               (/
                  (+
                     (* (car C) 2126)  # Red
                     (* (cadr C) 7152)  # Green
                     (* (caddr C) 722) )  # Blue
                  10000 ) )
            Y ) )
      Ppm ) )

# Convert greyscale image (PGM) to color image (PPM)
(de pgm->ppm (Pgm)
   (mapcar
      '((Y)
         (mapcar
            '((G) (list G G G))
            Y ) )
      Pgm ) )

# Write greyscale image (PGM) to file
(de pgmWrite (Pgm File)
   (out File
      (prinl "P5")
      (prinl (length (car Pgm)) " " (length Pgm))
      (prinl 255)
      (for Y Pgm (apply wr Y)) ) )

# Create an empty image of 120 x 90 pixels
(setq *Ppm (make (do 90 (link (need 120)))))

# Fill background with green color
(ppmFill *Ppm 0 255 0)

# Draw a diagonal line
(for I 80 (ppmSetPixel *Ppm I I 0 0 0))


# Convert to greyscale image (PGM)
(setq *Pgm (ppm->pgm *Ppm))

# Write greyscale image to .pgm file
(pgmWrite *Pgm "img.pgm")

# Convert to color image and write to .ppm file
(ppmWrite (pgm->ppm *Pgm) "img.ppm")

\end{verbatim}

\section*{Greatest common divisor}

\begin{verbatim}

(de gcd (A B)
   (until (=0 B)
      (let M (\% A B)
         (setq A B B M) ) )
   (abs A) )

\end{verbatim}

\section*{Greatest element of a list}

\begin{verbatim}

: (max 2 4 1 3)               # Return the maximal argument
-> 4
: (apply max (2 4 1 3))       # Apply to a list
-> 4
: (maxi abs (2 -4 -1 3))      # Maximum according to given function
-> -4

\end{verbatim}

\section*{Greatest subsequential sum (!)}

\begin{verbatim}

(maxi '((L) (apply + L))
   (mapcon '((L) (maplist reverse (reverse L)))
      (-1 -2 3 5 6 -2 -1 4 -4 2 -1) ) )

Output:

-> (3 5 6 -2 -1 4)

\end{verbatim}

\section*{Guess the number}

\begin{verbatim}

(de guessTheNumber ()
   (let Number (rand 1 9)
      (loop
         (prin "Guess the number: ")
         (T (= Number (read))
            (prinl "Well guessed!") )
         (prinl "Sorry, this was wrong") ) ) )

\end{verbatim}

\section*{Guess the number/With Feedback}

\begin{verbatim}

{{trans|PureBasic}}

(de guessTheNumber ()
   (use (Low High Guess)
      (until
         (and
            (prin "Enter low limit : ")
            (setq Low (read))
            (prin "Enter high limit: ")
            (setq High (read))
            (> High Low) ) )
      (seed (time))
      (let Number (rand Low High)
         (loop
            (prin "Guess what number I have: ")
            (T (= Number (setq Guess (read)))
               (prinl "You got it!") )
            (prinl
               "Your guess is too "
               (if (> Number Guess) "low" "high")
               "." ) ) ) ) )

Output:

: (guessTheNumber)
Enter low limit : 1
Enter high limit: 64
Guess what number I have: 32
Your guess is too high.
Guess what number I have: 16
Your guess is too low.
Guess what number I have: 24
You got it!

\end{verbatim}

\section*{Guess the number/With Feedback (Player)}

\begin{verbatim}

{{trans|PureBasic}}

(de guessTheNumber (Min Max)
   (prinl "Think of a number between " Min " and " Max ".")
   (prinl "On every guess of mine you should state whether my guess was")
   (prinl "too high, too low, or equal to your number by typing 'h', 'l', Or '='")
   (use Guess
      (loop
         (NIL (> Max Min)
            (prinl "I think somthing is strange here...") )
         (prin
            "My guess is "
            (setq Guess (+ Min (/ (- Max Min) 2)))
            ",is this correct? " )
         (flush)
         (NIL
            (case (uppc (car (line)))
               ("H" (setq Max Guess))
               ("L" (setq Min Guess))
               ("=" (nil (prinl "I did it!")))
               (T (prinl "I do not understand that...")) ) ) ) ) )

Output:

: (guessTheNumber 1 99)
Think of a number between 1 and 99.
On every guess of mine you should state whether my guess was
too high, too low, or equal to your number by typing 'h', 'l', Or '='
My guess is 50,is this correct? h
My guess is 25,is this correct? h
My guess is 13,is this correct? l
My guess is 19,is this correct? l
My guess is 22,is this correct? =
I did it!

\end{verbatim}

\chapter{Rosetta Code Tasks starting with H}

\section*{HTTP}

\begin{verbatim}

(load "@lib/http.l")

(client "rosettacode.org" 80 NIL       # Connect to rosettacode
   (out NIL (echo)) )                  # Echo to standard output

\end{verbatim}

\section*{HTTPS}

\begin{verbatim}

PicoLisp has no functionality for communicating with a HTTPS server (only for
the other direction), but it is easy to use an external tool

(in '(curl "https://sourceforge.net")  # Open a pipe to 'curl'
   (out NIL (echo)) )                  # Echo to standard output

\end{verbatim}

\section*{HTTPS/Authenticated}

\begin{verbatim}

(let (User "Bill"  Pass "T0p5ecRet"  Url "https://www.example.com")
   (in (list 'curl "-u" (pack User ': Pass) Url)
      (while (line)
         (doSomeProcessingWithLine @) ) ) )

\end{verbatim}

\section*{HTTPS/Client-authenticated}

\begin{verbatim}

(in '(curl "-E" "myCert.pem" "https://www.example.com")
   (while (line)
      (doSomeProcessingWithLine @) ) )

\end{verbatim}

\section*{Hailstone sequence}

\begin{verbatim}

(de hailstone (N)
   (make
      (until (= 1 (link N))
         (setq N
            (if (bit? 1 N)
               (inc (* N 3))
               (/ N 2) ) ) ) ) )

(let L (hailstone 27)
   (println 27 (length L) (head 4 L) '- (tail 4 L)) )

(let N (maxi '((N) (length (hailstone N))) (range 1 100000))
   (println N (length (hailstone N))) )

Output:

27 112 (27 82 41 124) - (8 4 2 1)
77031 351

\end{verbatim}

\section*{Hamming numbers (!)}

\begin{verbatim}

(de hamming (N)
   (let (L (1)  H)
      (do N
         (for (X L X (cadr X))      # Find smallest result
            (setq H (car X)) )
         (idx 'L H NIL)             # Remove it
         (for I (2 3 5)             # Generate next results
            (idx 'L (* I H) T) ) )
      H ) )

(println (make (for N 20 (link (hamming N)))))
(println (hamming 1691))
(println (hamming 1000000))

Output:

(1 2 3 4 5 6 8 9 10 12 15 16 18 20 24 25 27 30 32 36)
2125764000
519312780448388736089589843750000000000000000000000000000000000000000000000000000000
# (took almost 2 hours)

\end{verbatim}

\section*{Handle a signal}

\begin{verbatim}

Put the following into a file, set it to executable, and run it

#!/usr/bin/picolisp /usr/lib/picolisp/lib.l

(push '*Bye '(println (*/ (usec) 1000000)) '(prinl))

(let Cnt 0
   (loop
      (println (inc 'Cnt))
      (wait 500) ) )

\end{verbatim}

\section*{Happy numbers}

\begin{verbatim}

(de happy? (N)
   (let Seen NIL
      (loop
         (T (= N 1) T)
         (T (member N Seen))
         (setq N
            (sum '((C) (** (format C) 2))
               (chop (push 'Seen N)) ) ) ) ) )

(let H 0
   (do 8
      (until (happy? (inc 'H)))
      (printsp H) ) )

Output:

1 7 10 13 19 23 28 31

\end{verbatim}

\section*{Hash from two arrays}

\begin{verbatim}

(let (Keys '(one two three)  Values (1 2 3))
   (mapc println
      (mapcar cons Keys Values) ) )

Output:

(one . 1)
(two . 2)
(three . 3)

\end{verbatim}

\section*{Haversine formula}

\begin{verbatim}

(scl 12)
(load "@lib/math.l")

(de haversine (Th1 Ph1 Th2 Ph2)
   (setq
      Ph1 (*/ (- Ph1 Ph2) pi 180.0)
      Th1 (*/ Th1 pi 180.0)
      Th2 (*/ Th2 pi 180.0) )
   (let
      (DX (- (*/ (cos Ph1) (cos Th1) 1.0) (cos Th2))
         DY (*/ (sin Ph1) (cos Th1) 1.0)
         DZ (- (sin Th1) (sin Th2)) )
      (* `(* 2 6371)
         (asin
            (/
               (sqrt (+ (* DX DX) (* DY DY) (* DZ DZ)))
               2 ) ) ) ) )

Test:

(prinl
   "Haversine distance: "
   (round (haversine 36.12 -86.67 33.94 -118.4))
   " km" )

Output:

Haversine distance: 2,886.444 km

\end{verbatim}

\section*{Hello world/Graphical}

\begin{verbatim}

(call 'dialog "--msgbox" "Goodbye, World!" 5 20)

\end{verbatim}

\section*{Hello world/Line printer}

\begin{verbatim}

(out '(lpr "-P" "Printer01")
   (prinl "Hello world") )

\end{verbatim}

\section*{Hello world/Newline omission}

\begin{verbatim}

(prin "Goodbye, world")

\end{verbatim}

\section*{Hello world/Standard error}

\begin{verbatim}

(out 2 (prinl "Goodbye, World!"))

\end{verbatim}

\section*{Hello world/Text}

\begin{verbatim}

(prinl "Goodbye, World!")

\end{verbatim}

\section*{Hello world/Web server}

\begin{verbatim}

Contents of the file "goodbye.l":

(html 0 "Bye" NIL NIL
   "Goodbye, World!" )

Start server:

\$ pil @lib/http.l @lib/xhtml.l -'server 8080 "goodbye.l"' -wait

\end{verbatim}

\section*{Here document}

\begin{verbatim}

We can use the '[http://software-lab.de/doc/refH.html#here here]' function:

(out "file.txt"                        # Write to "file.txt"
   (prinl "### This is before the text ###")
   (here "TEXT-END")
   (prinl "### This is after the text ###") )
"There must be some way out of here", said the joker to the thief
"There's too much confusion, I can't get no relief"
TEXT-END

(in "file.txt" (echo))                 # Show "file.txt"

Output:

### This is before the text ###
"There must be some way out of here", said the joker to the thief
"There's too much confusion, I can't get no relief"
### This is after the text ###

\end{verbatim}

\section*{Higher-order functions}

\begin{verbatim}

: (de first (Fun)
   (Fun) )
-> first

: (de second ()
   "second" )
-> second

: (first second)
-> "second"

: (de add (A B)
   (+ A B) )
-> add

: (add 1 2)
-> 3

: (de call-it (Fun X Y)
   (Fun X Y) )
-> call-it

: (call-it add 1 2)
-> 3

: (mapcar inc (1 2 3 4 5))
-> (2 3 4 5 6)

: (mapcar + (1 2 3) (4 5 6))
-> (5 7 9)

:  (mapcar add (1 2 3) (4 5 6))
-> (5 7 9)

\end{verbatim}

\section*{History variables}

\begin{verbatim}

(de setH ("Var" Val)
   (when (val "Var")
      (with "Var"
         (=: history (cons @ (: history))) ) )
   (set "Var" Val) )

(de restoreH ("Var")
   (set "Var" (pop (prop "Var" 'history))) )

Test:

: (setH 'A "Hello world")
-> "Hello world"

: (setH 'A '(a b c d))
-> (a b c d)

: (setH 'A 123)
-> 123

: A
-> 123

: (get 'A 'history)
-> ((a b c d) "Hello world")

: (restoreH 'A)
-> (a b c d)

: (restoreH 'A)
-> "Hello world"

: A
-> "Hello world"

: (restoreH 'A)
-> NIL

\end{verbatim}

\section*{Hofstadter Figure-Figure sequences (!)}

\begin{verbatim}

(setq *RNext 2)

(de ffr (N)
   (cache '(NIL) (pack (char (hash N)) N)
      (if (= 1 N)
         1
         (+ (ffr (dec N)) (ffs (dec N))) ) ) )

(de ffs (N)
   (cache '(NIL) (pack (char (hash N)) N)
      (if (= 1 N)
         2
         (let S (inc (ffs (dec N)))
            (when (= S (ffr *RNext))
               (inc 'S)
               (inc '*RNext) )
            S ) ) ) )

Test:

: (mapcar ffr (range 1 10))
-> (1 3 7 12 18 26 35 45 56 69)

: (=
   (range 1 1000)
   (sort (conc (mapcar ffr (range 1 40)) (mapcar ffs (range 1 960)))) )
-> T

\end{verbatim}

\section*{Hofstadter Q sequence}

\begin{verbatim}

(de q (N)
   (cache '(NIL) (pack (char (hash N)) N)
      (if (>= 2 N)
         1
         (+
            (q (- N (q (dec N))))
            (q (- N (q (- N 2)))) ) ) ) )

Test:

: (mapcar q (range 1 10))
-> (1 1 2 3 3 4 5 5 6 6)

: (q 1000)
-> 502

: (let L (mapcar q (range 1 100000)) (!)
   (cnt < (cdr L) L) )
-> 49798

\end{verbatim}

\section*{Hofstadter-Conway \$10,000 sequence}

\begin{verbatim}

(de hofcon (N)
   (cache '(NIL) (pack (char (hash N)) N)
      (if (>= 2 N)
         1
         (+
            (hofcon (hofcon (dec N)))
            (hofcon (- N (hofcon (dec N)))) ) ) ) )

(scl 20)

(de sequence (M)
   (let (Lim 4  Max 0  4k\$ 0)
      (for (N 3 (>= M N) (inc N))
         (let V (*/ (hofcon N) 1.0 N)
            (setq Max (max Max V))
            (when (>= V 0.55)
               (setq 4k\$ N) )
            (when (= N Lim)
               (prinl
                  "Maximum between " (/ Lim 2)
                  " and " Lim
                  " was " (format Max `*Scl) )
               (inc 'Lim Lim)
               (zero Max) ) ) )
      (prinl
         "Win with " (inc 4k\$)
         " (the task requests 'n >= p')" ) ) )

(sequence (** 2 20))

Output:

Maximum between 2 and 4 was 0.66666666666666666667
Maximum between 4 and 8 was 0.66666666666666666667
Maximum between 8 and 16 was 0.63636363636363636364
Maximum between 16 and 32 was 0.60869565217391304348
Maximum between 32 and 64 was 0.59090909090909090909
Maximum between 64 and 128 was 0.57608695652173913043
Maximum between 128 and 256 was 0.56741573033707865169
Maximum between 256 and 512 was 0.55945945945945945946
Maximum between 512 and 1024 was 0.55493741307371349096
Maximum between 1024 and 2048 was 0.55010087424344317418
Maximum between 2048 and 4096 was 0.54746289264756644805
Maximum between 4096 and 8192 was 0.54414474786396381303
Maximum between 8192 and 16384 was 0.54244270878036220067
Maximum between 16384 and 32768 was 0.54007109751158709445
Maximum between 32768 and 65536 was 0.53878402058425570614
Maximum between 65536 and 131072 was 0.53704365699986594575
Maximum between 131072 and 262144 was 0.53602006781156104419
Maximum between 262144 and 524288 was 0.53464543107811232092
Maximum between 524288 and 1048576 was 0.53377922996336783427
Win with 1490 (the task requests 'n >= p')

\end{verbatim}

\section*{Holidays related to Easter}

\begin{verbatim}

(load "@lib/cal.l")  # For 'easter' function

(de dayMon (Dat)
   (let D (date Dat)
      (list (day Dat *Day) " " (align 2 (caddr D)) " " (get *Mon (cadr D))) ) )

(for Y (append (range 400 2100 100) (range 2010 2020))
   (let E (easter Y)
      (prinl
         (align 4 Y)
         " Easter: " (dayMon E)
         ", Ascension: " (dayMon (+ E 39))
         ", Pentecost: " (dayMon (+ E 49))
         ", Trinity: " (dayMon (+ E 56))
         ", Corpus: " (dayMon (+ E 60)) ) ) )

Output:

 400 Easter: Sun  2 Apr, Ascension: Thu 11 May, Pentecost: Sun 21 May, Trinity: Sun 28 May, Corpus: Thu  1 Jun
 500 Easter: Sun  4 Apr, Ascension: Thu 13 May, Pentecost: Sun 23 May, Trinity: Sun 30 May, Corpus: Thu  3 Jun
 600 Easter: Sun 13 Apr, Ascension: Thu 22 May, Pentecost: Sun  1 Jun, Trinity: Sun  8 Jun, Corpus: Thu 12 Jun
 700 Easter: Sun 15 Apr, Ascension: Thu 24 May, Pentecost: Sun  3 Jun, Trinity: Sun 10 Jun, Corpus: Thu 14 Jun
 800 Easter: Sun 23 Apr, Ascension: Thu  1 Jun, Pentecost: Sun 11 Jun, Trinity: Sun 18 Jun, Corpus: Thu 22 Jun
 900 Easter: Sun 28 Mar, Ascension: Thu  6 May, Pentecost: Sun 16 May, Trinity: Sun 23 May, Corpus: Thu 27 May
1000 Easter: Sun 30 Mar, Ascension: Thu  8 May, Pentecost: Sun 18 May, Trinity: Sun 25 May, Corpus: Thu 29 May
1100 Easter: Sun  8 Apr, Ascension: Thu 17 May, Pentecost: Sun 27 May, Trinity: Sun  3 Jun, Corpus: Thu  7 Jun
1200 Easter: Sun  9 Apr, Ascension: Thu 18 May, Pentecost: Sun 28 May, Trinity: Sun  4 Jun, Corpus: Thu  8 Jun
1300 Easter: Sun 18 Apr, Ascension: Thu 27 May, Pentecost: Sun  6 Jun, Trinity: Sun 13 Jun, Corpus: Thu 17 Jun
1400 Easter: Sun 20 Apr, Ascension: Thu 29 May, Pentecost: Sun  8 Jun, Trinity: Sun 15 Jun, Corpus: Thu 19 Jun
1500 Easter: Sun  1 Apr, Ascension: Thu 10 May, Pentecost: Sun 20 May, Trinity: Sun 27 May, Corpus: Thu 31 May
1600 Easter: Sun  2 Apr, Ascension: Thu 11 May, Pentecost: Sun 21 May, Trinity: Sun 28 May, Corpus: Thu  1 Jun
1700 Easter: Sun 11 Apr, Ascension: Thu 20 May, Pentecost: Sun 30 May, Trinity: Sun  6 Jun, Corpus: Thu 10 Jun
1800 Easter: Sun 13 Apr, Ascension: Thu 22 May, Pentecost: Sun  1 Jun, Trinity: Sun  8 Jun, Corpus: Thu 12 Jun
1900 Easter: Sun 15 Apr, Ascension: Thu 24 May, Pentecost: Sun  3 Jun, Trinity: Sun 10 Jun, Corpus: Thu 14 Jun
2000 Easter: Sun 23 Apr, Ascension: Thu  1 Jun, Pentecost: Sun 11 Jun, Trinity: Sun 18 Jun, Corpus: Thu 22 Jun
2100 Easter: Sun 28 Mar, Ascension: Thu  6 May, Pentecost: Sun 16 May, Trinity: Sun 23 May, Corpus: Thu 27 May
2010 Easter: Sun  4 Apr, Ascension: Thu 13 May, Pentecost: Sun 23 May, Trinity: Sun 30 May, Corpus: Thu  3 Jun
2011 Easter: Sun 24 Apr, Ascension: Thu  2 Jun, Pentecost: Sun 12 Jun, Trinity: Sun 19 Jun, Corpus: Thu 23 Jun
2012 Easter: Sun  8 Apr, Ascension: Thu 17 May, Pentecost: Sun 27 May, Trinity: Sun  3 Jun, Corpus: Thu  7 Jun
2013 Easter: Sun 31 Mar, Ascension: Thu  9 May, Pentecost: Sun 19 May, Trinity: Sun 26 May, Corpus: Thu 30 May
2014 Easter: Sun 20 Apr, Ascension: Thu 29 May, Pentecost: Sun  8 Jun, Trinity: Sun 15 Jun, Corpus: Thu 19 Jun
2015 Easter: Sun  5 Apr, Ascension: Thu 14 May, Pentecost: Sun 24 May, Trinity: Sun 31 May, Corpus: Thu  4 Jun
2016 Easter: Sun 27 Mar, Ascension: Thu  5 May, Pentecost: Sun 15 May, Trinity: Sun 22 May, Corpus: Thu 26 May
2017 Easter: Sun 16 Apr, Ascension: Thu 25 May, Pentecost: Sun  4 Jun, Trinity: Sun 11 Jun, Corpus: Thu 15 Jun
2018 Easter: Sun  1 Apr, Ascension: Thu 10 May, Pentecost: Sun 20 May, Trinity: Sun 27 May, Corpus: Thu 31 May
2019 Easter: Sun 21 Apr, Ascension: Thu 30 May, Pentecost: Sun  9 Jun, Trinity: Sun 16 Jun, Corpus: Thu 20 Jun
2020 Easter: Sun 12 Apr, Ascension: Thu 21 May, Pentecost: Sun 31 May, Trinity: Sun  7 Jun, Corpus: Thu 11 Jun

\end{verbatim}

\section*{Horizontal sundial calculations}

\begin{verbatim}

{{trans|ALGOL 68}}

(load "@lib/math.l")

(de prompt (Str . Arg)
   (prin Str " => ")
   (set (car Arg) (in NIL (read))) )

(use (Lat Lng Ref)
   (prompt "Enter latitude      " Lat)
   (prompt "Enter longitude     " Lng)
   (prompt "Enter legal meridian" Ref)
   (prinl)
   (let Slat (sin (*/ Lat pi 180.0))
      (prinl "    sine of latitude:   " (round Slat))
      (prinl "    diff longitude:     " (round (- Lng Ref)))
      (prinl)
      (prinl "Hour, sun hour angle, dial hour line angle from 6am to 6pm")
      (for H (range -6 6)
         (let Hra (- (* 15.0 H) (- Lng Ref))
            (let Hla (*/ (atan (*/ Slat (tan (*/ Hra pi 180.0)) 1.0)) 180.0 pi)
               (prinl
                  "HR="
                  (align 3 H)
                  "; HRA="
                  (align 8 (round Hra))
                  "; HLA="
                  (align 8 (round Hla)) ) ) ) ) ) )

Output:

Enter latitude       => -4.95
Enter longitude      => -150.5
Enter legal meridian => -150.          # Don't omit the '.' here

    sine of latitude:   -0.086
    diff longitude:     -0.500

Hour, sun hour angle, dial hour line angle from 6am to 6pm
HR= -6; HRA= -89.500; HLA=  84.225
HR= -5; HRA= -74.500; HLA=  17.283
HR= -4; HRA= -59.500; HLA=   8.334
HR= -3; HRA= -44.500; HLA=   4.847
HR= -2; HRA= -29.500; HLA=   2.795
HR= -1; HRA= -14.500; HLA=   1.278
HR=  0; HRA=   0.500; HLA=  -0.043
HR=  1; HRA=  15.500; HLA=  -1.371
HR=  2; HRA=  30.500; HLA=  -2.910
HR=  3; HRA=  45.500; HLA=  -5.018
HR=  4; HRA=  60.500; HLA=  -8.671
HR=  5; HRA=  75.500; HLA= -18.451
HR=  6; HRA=  90.500; HLA=  84.225

\end{verbatim}

\section*{Horner's rule for polynomial evaluation}

\begin{verbatim}

(de horner (Coeffs X)
   (let Res 0
      (for C (reverse Coeffs)
         (setq Res (+ C (* X Res))) ) ) )

: (horner (-19.0 7.0 -4.0 6.0) 3.0)
-> 128

\end{verbatim}

\section*{Host introspection}

\begin{verbatim}

(in (cmd)                              # Inspect ELF header
   (rd 4)                              # Skip "7F" and 'E', 'L' and 'F'
   (prinl
      (case (rd 1)                     # Get EI_CLASS byte
         (1 "32 bits")
         (2 "64 bits")
         (T "Bad EI_CLASS") ) )
   (prinl
      (case (rd 1)                     # Get EI_DATA byte
         (1 "Little endian")
         (2 "Big endian")
         (T "Bad EI_DATA") ) ) )

\end{verbatim}

\section*{Hostname}

\begin{verbatim}

This will just print the hostname:

(call 'hostname)

To use it as a string in a program:

(in '(hostname) (line T))

\end{verbatim}

\section*{Huffman coding (!)}

\begin{verbatim}

Using a cons cells (freq . char) for leaves, and two cells (freq left . right)
for nodes.

(de prio (Idx)
   (while (cadr Idx) (setq Idx @))
   (car Idx) )

(let (A NIL  P NIL  L NIL)
   (for C (chop "this is an example for huffman encoding")
      (accu 'A C 1) )                  # Count characters
   (for X A                            # Build index tree as priority queue
      (idx 'P (cons (cdr X) (car X)) T) )
   (while (or (cadr P) (cddr P))       # Remove entries, insert as nodes
      (let (A (car (idx 'P (prio P) NIL))  B (car (idx 'P (prio P) NIL)))
         (idx 'P (cons (+ (car A) (car B)) A B) T) ) )
   (setq P (car P))
   (recur (P L)                        # Traverse and print
      (if (atom (cdr P))
         (prinl (cdr P)  " " L)
         (recurse (cadr P) (cons 0 L))
         (recurse (cddr P) (cons 1 L)) ) ) )

Output:

n 000
m 0100
o 1100
s 0010
c 01010
d 11010
g 00110
l 10110
p 01110
r 11110
t 00001
u 10001
a 1001
  101
e 0011
f 1011
i 0111
x 01111
h 11111

\end{verbatim}

\chapter{Rosetta Code Tasks starting with I}

\section*{IPC via named pipe}

\begin{verbatim}

(call 'mkfifo "in" "out")              # Create pipes

(zero *Cnt)                            # Initialize byte counter

(unless (fork)                         # Handle "out" pipe
   (loop
      (out "out"
         (sync)
         (tell)
         (prinl *Cnt) ) ) )

(unless (fork)                         # Handle "in" pipe
   (let P (open "in")
      (loop
         (in P                         # Open twice, to avoid broken pipes
            (while (rd 1)                 # (works on Linux, perhaps not POSIX)
               (tell 'inc ''*Cnt) ) ) ) ) )

(push '*Bye '(call 'rm "in" "out"))    # Remove pipes upon exit
(wait)                                 # (Terminate with Ctrl-C)

Test:

\$ line <out
0
\$ echo abc >in
\$ line <out
4
\$ echo äöü >in
\$ line <out
11

\end{verbatim}

\section*{Identity matrix}

\begin{verbatim}

(de identity (Size)
   (let L (need Size (1) 0)
      (make
         (do Size
            (link (copy (rot L))) ) ) ) )

Test:

: (identity 3)
-> ((1 0 0) (0 1 0) (0 0 1))

: (mapc println (identity 5))
(1 0 0 0 0)
(0 1 0 0 0)
(0 0 1 0 0)
(0 0 0 1 0)
(0 0 0 0 1)

\end{verbatim}

\section*{Image convolution}

\begin{verbatim}

(scl 3)

(de ppmConvolution (Ppm Kernel)
   (let (Len (length (car Kernel))  Radius (/ Len 2))
      (make
         (chain (head Radius Ppm))
         (for (Y Ppm  T  (cdr Y))
            (NIL (nth Y Len)
               (chain (tail Radius Y)) )
            (link
               (make
                  (chain (head Radius (get Y (inc Radius))))
                  (for (X (head Len Y) T)
                     (NIL (nth X 1 Len)
                        (chain (tail Radius (get X (inc Radius)))) )
                     (link
                        (make
                           (for C 3
                              (let Val 0
                                 (for K Len
                                    (for L Len
                                       (inc 'Val
                                          (* (get X K L C) (get Kernel K L)) ) ) )
                                 (link (min 255 (max 0 (*/ Val 1.0)))) ) ) ) )
                     (map pop X) ) ) ) ) ) ) )

Test using 'ppmRead' from [[Bitmap/Read a PPM file#PicoLisp]] and 'ppmWrite'
from [[Bitmap/Write a PPM file#PicoLisp]]:

# Sharpen
(ppmWrite
   (ppmConvolution
      (ppmRead "Lenna100.ppm")
      '((-1.0 -1.0 -1.0) (-1.0 +9.0 -1.0) (-1.0 -1.0 -1.0)) )
   "a.ppm" )

# Blur
(ppmWrite
   (ppmConvolution
      (ppmRead "Lenna100.ppm")
      '((0.1 0.1 0.1) (0.1 0.1 0.1) (0.1 0.1 0.1)) )
   "b.ppm" )

\end{verbatim}

\section*{Image Noise}

\begin{verbatim}

This solution works on ErsatzLisp, the Java version of PicoLisp. It creates a
'JFrame' window, and calls inlined Java code to handle the image.

(javac "ImageNoise" "JPanel" NIL
      "java.util.*"
      "java.awt.*" "java.awt.image.*" "javax.swing.*" )
   int DX, DY;
   int[] Pixels;
   MemoryImageSource Source;
   Image Img;
   Random Rnd;

   public ImageNoise(int dx, int dy) {
      DX = dx;
      DY = dy;
      Pixels = new int[DX * DY];
      Source = new MemoryImageSource(DX, DY, Pixels, 0, DX);
      Source.setAnimated(true);
      Img = createImage(Source);
      Rnd = new Random();
   }

   public void paint(Graphics g) {update(g);}
   public void update(Graphics g) {g.drawImage(Img, 0, 0, this);}

   public void draw() {
      for (int i = 0; i < Pixels.length; ++i) {
         int c = Rnd.nextInt(255);
         Pixels[i] = 0xFF000000 | c<<16 | c<<8 | c;
      }
      Source.newPixels();
      paint(getGraphics());
   }
/**/

(de imageNoise (DX DY Fps)
   (let
      (Frame (java "javax.swing.JFrame" T "Image Noise")
         Noise (java "ImageNoise" T DX DY)
         Button (java "javax.swing.JButton" T "OK") )
      (java Frame "add" Noise)
      (java Frame "add" "South" Button)
      (java Button "addActionListener"
         (interface "java.awt.event.ActionListener"
            'actionPerformed '((Ev) (bye)) ) )
      (java Frame "setSize" DX DY)
      (java Frame "setVisible" T)
      (task (/ -1000 Fps) 0
         Image Noise
         (java Image "draw") ) ) )

# Start with 25 frames per second
(imageNoise 320 240 25)

\end{verbatim}

\section*{Include a file}

\begin{verbatim}

The function '[http://software-lab.de/doc/refL.html#load load]' is used for
recursively executing the contents of files.

(load "file1.l" "file2.l" "file3.l")

\end{verbatim}

\section*{Increment a numerical string}

\begin{verbatim}

(format (inc (format "123456")))

\end{verbatim}

\section*{Infinity}

\begin{verbatim}

The symbol '[http://software-lab.de/doc/refT.html#T T]' is used to represent
infinite values, e.g. for the length of circular lists, and is greater than any
other value in comparisons. PicoLisp has only very limited floating point
support (scaled bignum arithmetics), but some functions return 'T' for infinite
results.

(load "@lib/math.l")

: (exp 1000.0)
-> T

\end{verbatim}

\section*{Inheritance/Multiple}

\begin{verbatim}

(class +Camera)

(class +MobilePhone)

(class +CameraPhone +Camera +MobilePhone)
(class +Camera)

(class +MobilePhone)

(class +CameraPhone +Camera +MobilePhone)

\end{verbatim}

\section*{Inheritance/Single}

\begin{verbatim}

(class +Animal)

(class +Dog +Animal)

(class +Cat +Animal)

(class +Lab +Dog)

(class +Collie +Dog)

: (dep '+Animal)
+Animal
   +Cat
   +Dog
      +Collie
      +Lab

\end{verbatim}

\section*{Input loop}

\begin{verbatim}

This reads all lines in a file, and returns them as a list of lists

(in "file.txt"
   (make
      (until (eof)
         (link (line)) ) ) )

\end{verbatim}

\section*{Integer comparison}

\begin{verbatim}

(prin "Please enter two values: ")

(in NIL  # Read from standard input
   (let (A (read) B (read))
      (prinl
         "The first one is "
         (cond
            ((> A B) "greater than")
            ((= A B) "equal to")
            (T "less than") )
         " the second." ) ) )

Output:

Please enter two values: 4 3
The first one is greater than the second.

\end{verbatim}

\section*{Integer sequence}

\begin{verbatim}

(for (I 1 T (inc I))
   (printsp I) )

\end{verbatim}

\section*{Interactive programming}

\begin{verbatim}

\$ pil +

: (de f (Str1 Str2 Sep)
   (pack Str1 Sep Sep Str2) )
-> f

: (f "Rosetta" "Code" ":")
-> "Rosetta::Code"

\end{verbatim}

\section*{Introspection}

\begin{verbatim}

(unless (>= (version T) (3 0 1))       # Check version (only in the 64-bit version)
   (bye) )

# (setq bloop -7)                      # Uncomment this to get the output '7'

(and
   (num? bloop)                        # When 'bloop' is bound to a number
   (getd 'abs)                         # and 'abs' defined as a function
   (println (abs bloop)) )             # then print the absolute value

\end{verbatim}

\section*{Inverted index}

\begin{verbatim}

Assuming three files "file1", "file2" and "file3":

\$ cat file1
it is what it is

\$ cat file2
what is it

\$ cat file3
it is a banana

we can read them into a binary tree in the global variable '*MyIndex'

(off *MyIndex)

(use Word
   (for File '("file1" "file2" "file3")
      (in File
         (while (skip)
            (if (idx '*MyIndex (setq Word (till " ^I^J^M" T)) T)
               (push1 (car @) File)
               (set Word (cons File)) ) ) ) ) )

(de searchFor @
   (apply sect
      (extract
         '((Word) (val (car (idx '*MyIndex Word))))
         (rest) ) ) )

Output:

: (searchFor "what" "is" "it")
-> ("file2" "file1")

: (searchFor "a" "banana")
-> ("file3")

: (searchFor "it" "is")
-> ("file3" "file2" "file1")

\end{verbatim}

\section*{Inverted syntax}

\begin{verbatim}

We define a read macro for reverted syntax

(de rv Prg
   (append (last Prg) (head -1 Prg)) )

Test:

(de needUmbrella (Raining)
   `(rv                                # Inverted syntax
      (on *NeedUmbrella)
      (println 'Need 'an 'umbrella)
      (when Raining) ) )

(de keepUmbrella (Raining)
   `(rv                                # Inverted syntax
      (on *KeepUmbrella)
      (println 'Still 'need 'an 'umbrella)
      (while Raining) ) )

Output:

: (pp 'needUmbrella)
(de needUmbrella (Raining)
   (when Raining                       # Traditional syntax
      (on *NeedUmbrella)
      (println 'Need 'an 'umbrella) ) )

: (pp 'keepUmbrella)
(de keepUmbrella (Raining)
   (while Raining                      # Traditional syntax
      (on *KeepUmbrella)
      (println 'Still 'need 'an 'umbrella) ) )

\end{verbatim}

\chapter{Rosetta Code Tasks starting with J}

\section*{JSON (!)}

\begin{verbatim}

PicoLisp has no JSON library, but it is easy enough to write one. The following
supports only fixpoint numbers (no floating point, as it doesn't exist in
PicoLisp), no arrays (uses lists instead), and no unicode escapes (UTF-8 can be
used directly):

(de checkJson (X Item)
   (unless (= X Item)
      (quit "Bad JSON" Item) ) )

(de readJson ()
   (case (read "_")
      ("{"
         (make
            (for (X (readJson) (not (= "}" X)) (readJson))
               (checkJson ":" (readJson))
               (link (cons X (readJson)))
               (T (= "}" (setq X (readJson))))
               (checkJson "," X) ) ) )
      ("["
         (make
            (link T)  # Array marker
            (for (X (readJson) (not (= "]" X)) (readJson))
               (link X)
               (T (= "]" (setq X (readJson))))
               (checkJson "," X) ) ) )
      (T
         (let X @
            (if (and (= "-" X) (format (peek)))
               (- (read))
               X ) ) ) ) )

(de printJson (Item)  # For simplicity, without indentation
   (cond
      ((atom Item) (if Item (print @) (prin "{}")))
      ((=T (car Item))
         (prin "[")
         (map
            '((X)
               (printJson (car X))
               (and (cdr X) (prin ", ")) )
            (cdr Item) )
         (prin "]") )
      (T
         (prin "{")
         (map
            '((X)
               (print (caar X))
               (prin ": ")
               (printJson (cdar X))
               (and (cdr X) (prin ", ")) )
            Item )
         (prin "}") ) ) )

This reads/prints JSON from/to files, pipes, sockets etc. To read from a string,
a pipe can be used:

: (pipe (prinl "{ \"foo\": 1, \"bar\": [10, \"apples\"] }")
   (readJson) )
-> (("foo" . 1) ("bar" T 10 "apples"))

: (printJson
   (quote
      ("name" . "Smith")
      ("age" . 25)
      ("address"
         ("street" . "21 2nd Street")
         ("city" . "New York")
         ("state" . "NY")
         ("zip" . "10021") )
      ("phone" T "212 555-1234" "646 555-4567") ) )
{"name": "Smith", "age": 25, ... {"street": ... "phone": ["212 555-1234", ...

\end{verbatim}

\section*{Jensen's Device}

\begin{verbatim}

(scl 6)

(de jensen (I Lo Hi Term)
   (let Temp 0
      (set I Lo)
      (while (>= Hi (val I))
         (inc 'Temp (Term))
         (inc I) )
      Temp ) )

(let I (box)  # Create indirect reference
   (format
      (jensen I 1 100 '(() (*/ 1.0 (val I))))
      *Scl ) )

Output:

-> "5.187383"

\end{verbatim}

\section*{Joystick position}

\begin{verbatim}

This is for the 64-bit version.

Note: The code is not yet tested with a real joystick (I don't have one), it was
just simulated with dummy functions. Can somebody having a joystick please test
it, and remove this message?

(load "@lib/openGl.l")

(setq *JoyX 0.0  *JoyY 0.0)

(glutInit)
(glutInitDisplayMode (| GLUT_RGBA GLUT_DOUBLE GLUT_ALPHA GLUT_DEPTH))
(glutInitWindowSize 400 400)
(glutCreateWindow "Joystick")

(glClearColor 0.3 0.3 0.5 0)

(displayPrg
   (glClear GL_COLOR_BUFFER_BIT)
   (glBegin GL_LINES)
   (glVertex2f *JoyX (- *JoyY 0.1))  # Draw crosshair
   (glVertex2f *JoyX (+ *JoyY 0.1))
   (glVertex2f (- *JoyX 0.1) *JoyY)
   (glVertex2f (+ *JoyX 0.1) *JoyY)
   (glEnd)
   (glFlush)
   (glutSwapBuffers) )

# Track joystick position
(native `*GlutLib "glutJoystickFunc" NIL
   (lisp 'joystickFunc
      '((Btn X Y Z)
         (msg                          # Display buttons
            (make
               (for (B 1 (n0 Btn) (inc B))
                  (and (bit? 1 Btn) (link B))
                  (setq Btn (>> 1 Btn)) ) ) )
         (setq                         # Move crosshair
            *JoyX (*/ X 1.0 1000)
            *JoyY (*/ Y 1.0 1000) )
         (glutPostRedisplay) ) )
   100 )

# Exit upon mouse click
(mouseFunc '((Btn State X Y) (bye)))
(glutMainLoop)

\end{verbatim}

\section*{Jump anywhere}

\begin{verbatim}

PicoLisp supports non-local jumps to a previously setup environment (see
[[Exceptions#PicoLisp|exceptions]]) via
'[http://software-lab.de/doc/refC.html#catch catch]' and
'[http://software-lab.de/doc/refT.html#throw throw]', or to some location in
another coroutine with '[http://software-lab.de/doc/refY.html#yield yield]' (see
[[Generator#PicoLisp|generator]]).

'[http://software-lab.de/doc/refQ.html#quit quit]' is similar to 'throw', but
doesn't require a corresponding 'catch', as it directly jumps to the error
handler (where the program may catch that error again).

There is no 'go' or 'goto' function in PicoLisp, but it can be emulated with
normal list processing functions. This allows "jumps" to arbitrary locations
within (the same or other) functions. The following example implements a "loop":

(de foo (N)
   (prinl "This is 'foo'")
   (printsp N)
   (or (=0 (dec 'N)) (run (cddr foo))) )

Test:

: (foo 7)
This is 'foo'
7 6 5 4 3 2 1 -> 0

\end{verbatim}

\chapter{Rosetta Code Tasks starting with K}

\section*{Kaprekar numbers}

\begin{verbatim}

(de kaprekar (N)
   (let L (cons 0 (chop (* N N)))
      (for ((I . R) (cdr L) R (cdr R))
         (NIL (gt0 (format R)))
         (T (= N (+ @ (format (head I L)))) N) ) ) )

Output:

: (filter kaprekar (range 1 10000))
-> (1 9 45 55 99 297 703 999 2223 2728 4879 4950 5050 5292 7272 7777 9999)

: (cnt kaprekar (range 1 1000000))
-> 54

\end{verbatim}

\section*{Keyboard macros}

\begin{verbatim}

The 'fkey' function associates a key with an executable body. Some common key
codes are predefined in "lib/term.l". Here we use 'F1' to store the value 1 in
a global variable, 'Up' and 'Down' arrows to increment or decrement that value,
and 'Home' to print the current value to the console.

(load "@lib/term.l")

(fkey *XtF1
   (prinl "Initialized value to " (setq *Number 1)) )

(fkey *XtUp
   (prinl "Incremented to " (inc '*Number)) )

(fkey *XtDown
   (prinl "Decremented to " (dec '*Number)) )

(fkey *XtHome
   (prinl "Current value is " *Number) )

Output when hitting 'F1', 'Down', 'Up', 'Up' and 'Home':

Initialized value to 1
Decremented to 0
Incremented to 1
Incremented to 2
Current value is 2

\end{verbatim}

\section*{Knapsack problem/0-1 (!)}

\begin{verbatim}

(de *Items
   ("map" 9 150)                    ("compass" 13 35)
   ("water" 153 200)                ("sandwich" 50 160)
   ("glucose" 15 60)                ("tin" 68 45)
   ("banana" 27 60)                 ("apple" 39 40)
   ("cheese" 23 30)                 ("beer" 52 10)
   ("suntan cream" 11 70)           ("camera" 32 30)
   ("t-shirt" 24 15)                ("trousers" 48 10)
   ("umbrella" 73 40)               ("waterproof trousers" 42 70)
   ("waterproof overclothes" 43 75) ("note-case" 22 80)
   ("sunglasses" 7 20)              ("towel" 18 12)
   ("socks" 4 50)                   ("book" 30 10) )

# Dynamic programming solution
(de knapsack (Lst W)
   (when Lst
      (cache '*KnapCache (pack (length Lst) ":" W)
         (let X (knapsack (cdr Lst) W)
            (if (ge0 (- W (cadar Lst)))
               (let Y (cons (car Lst) (knapsack (cdr Lst) @))
                  (if (> (sum caddr X) (sum caddr Y)) X Y) )
               X ) ) ) ) )

(let K (knapsack *Items 400)
   (for I K
      (apply tab I (3 -24 6 6) NIL) )
   (tab (27 6 6) NIL (sum cadr K) (sum caddr K)) )

Output:

   map                          9   150
   compass                     13    35
   water                      153   200
   sandwich                    50   160
   glucose                     15    60
   banana                      27    60
   suntan cream                11    70
   waterproof trousers         42    70
   waterproof overclothes      43    75
   note-case                   22    80
   sunglasses                   7    20
   socks                        4    50
                              396  1030

\end{verbatim}

\section*{Knapsack problem/Bounded}

\begin{verbatim}

(de *Items
   ("map" 9 150 1)                     ("compass" 13 35 1)
   ("water" 153 200 3)                 ("sandwich" 50 60 2)
   ("glucose" 15 60 2)                 ("tin" 68 45 3)
   ("banana" 27 60 3)                  ("apple" 39 40 3)
   ("cheese" 23 30 1)                  ("beer" 52 10 3)
   ("suntan cream" 11 70 1)            ("camera" 32 30 1)
   ("t-shirt" 24 15 2)                 ("trousers" 48 10 2)
   ("umbrella" 73 40 1)                ("waterproof trousers" 42 70 1)
   ("waterproof overclothes" 43 75 1)  ("note-case" 22 80 1)
   ("sunglasses" 7 20 1)               ("towel" 18 12 2)
   ("socks" 4 50 1)                    ("book" 30 10 2) )

# Dynamic programming solution
(de knapsack (Lst W)
   (when Lst
      (cache '*KnapCache (pack (length Lst) ":" W)
         (let X (knapsack (cdr Lst) W)
            (if (ge0 (- W (cadar Lst)))
               (let Y (cons (car Lst) (knapsack (cdr Lst) @))
                  (if (> (sum caddr X) (sum caddr Y)) X Y) )
               X ) ) ) ) )

(let K
   (knapsack
      (mapcan                                   # Expand multiple items
         '((X) (need (cadddr X) NIL X))
         *Items )
      400 )
   (for I K
      (apply tab I (3 -24 6 6) NIL) )
   (tab (27 6 6) NIL (sum cadr K) (sum caddr K)) )

Output:

   map                          9   150
   compass                     13    35
   water                      153   200
   glucose                     15    60
   glucose                     15    60
   banana                      27    60
   banana                      27    60
   banana                      27    60
   cheese                      23    30
   suntan cream                11    70
   waterproof overclothes      43    75
   note-case                   22    80
   sunglasses                   7    20
   socks                        4    50
                              396  1010

\end{verbatim}

\section*{Knapsack problem/Continuous}

\begin{verbatim}

(scl 2)

(de *Items
   ("beef" 3.8 36.0)
   ("pork" 5.4 43.0)
   ("ham" 3.6 90.0)
   ("greaves" 2.4 45.0)
   ("flitch" 4.0 30.0)
   ("brawn" 2.5 56.0)
   ("welt" 3.7 67.0)
   ("salami" 3.0 95.0)
   ("sausage" 5.9 98.0) )

(let K
   (make
      (let Weight 0
         (for I (by '((L) (*/ (caddr L) -1.0 (cadr L))) sort *Items)
            (T (= Weight 15.0))
            (inc 'Weight (cadr I))
            (T (> Weight 15.0)
               (let W (- (cadr I) Weight -15.0)
                  (link (list (car I) W (*/ W (caddr I) (cadr I)))) ) )
            (link I) ) ) )
   (for I K
      (tab (3 -9 8 8)
         NIL
         (car I)
         (format (cadr I) *Scl)
         (format (caddr I) *Scl) ) )
   (tab (12 8 8)
      NIL
      (format (sum cadr K) *Scl)
      (format (sum caddr K) *Scl) ) )

Output:

   salami       3.00   95.00
   ham          3.60   90.00
   brawn        2.50   56.00
   greaves      2.40   45.00
   welt         3.50   63.38
               15.00  349.38

\end{verbatim}

\section*{Knapsack problem/Unbounded}

\begin{verbatim}

Brute force solution

(de *Items
   ("panacea"  3  25  3000)
   ("ichor"    2  15  1800)
   ("gold"    20   2  2500) )

(de knapsack (Lst W V)
   (when Lst
      (let X (knapsack (cdr Lst) W V)
         (if (and (ge0 (dec 'W (cadar Lst))) (ge0 (dec 'V (caddar Lst))))
            (maxi
               '((L) (sum cadddr L))
               (list
                  X
                  (cons (car Lst) (knapsack (cdr Lst) W V))
                  (cons (car Lst) (knapsack Lst W V)) ) )
            X ) ) ) )

(let K (knapsack *Items 250 250)
   (for (L K  L)
      (let (N 1  X)
         (while (= (setq X (pop 'L)) (car L))
            (inc 'N) )
         (apply tab X (4 2 8 5 5 7) N "x") ) )
   (tab (14 5 5 7) NIL (sum cadr K) (sum caddr K) (sum cadddr K)) )

Output:

  15 x   ichor    2   15   1800
  11 x    gold   20    2   2500
                250  247  54500

\end{verbatim}

\section*{Knight's tour}

\begin{verbatim}

(load "@lib/simul.l")

# Build board
(grid 8 8)

# Generate legal moves for a given position
(de moves (Tour)
   (extract
      '((Jump)
         (let? Pos (Jump (car Tour))
            (unless (memq Pos Tour)
               Pos ) ) )
      (quote  # (taken from "games/chess.l")
         ((This) (: 0 1  1  0 -1  1  0 -1  1))        # South Southwest
         ((This) (: 0 1  1  0 -1  1  0  1  1))        # West Southwest
         ((This) (: 0 1  1  0 -1 -1  0  1  1))        # West Northwest
         ((This) (: 0 1  1  0 -1 -1  0 -1 -1))        # North Northwest
         ((This) (: 0 1 -1  0 -1 -1  0 -1 -1))        # North Northeast
         ((This) (: 0 1 -1  0 -1 -1  0  1 -1))        # East Northeast
         ((This) (: 0 1 -1  0 -1  1  0  1 -1))        # East Southeast
         ((This) (: 0 1 -1  0 -1  1  0 -1  1)) ) ) )  # South Southeast

# Build a list of moves, using Warnsdorff’s algorithm
(let Tour '(b1)  # Start at b1
   (while
      (mini
         '((P) (length (moves (cons P Tour))))
         (moves Tour) )
      (push 'Tour @) )
   (flip Tour) )

Output:

-> (b1 a3 b5 a7 c8 b6 a8 c7 a6 b8 d7 f8 h7 g5 h3 g1 e2 c1 a2 b4 c2 a1 b3 a5 b7
d8 c6 d4 e6 c5 a4 c3 d1 b2 c4 d2 f1 h2 f3 e1 d3 e5 f7 h8 g6 h4 g2 f4 d5 e7 g8
h6 g4 e3 f5 d6 e8 g7 h5 f6 e4 g3 h1 f2)

\end{verbatim}

\section*{Knuth's algorithm S}

\begin{verbatim}

(de s_of_n_creator (@N)
   (curry (@N (I . 0) (Res)) (Item)
      (cond
         ((>= @N (inc 'I)) (push 'Res Item))
         ((>= @N (rand 1 I)) (set (nth Res (rand 1 @N)) Item)) )
      Res ) )

(let Freq (need 10 0)
   (do 100000
      (let S_of_n (s_of_n_creator 3)
         (for I (mapc S_of_n (0 1 2 3 4 5 6 7 8 9))
            (inc (nth Freq (inc I))) ) ) )
   Freq )

Output:

-> (30003 29941 29918 30255 29848 29875 30056 29839 30174 30091)

\end{verbatim}

\section*{Knuth shuffle}

\begin{verbatim}

(de shuffle (Lst)
   (make
      (for (N (length Lst) (gt0 N))
         (setq Lst
            (conc
               (cut (rand 0 (dec 'N)) 'Lst)
               (prog (link (car Lst)) (cdr Lst)) ) ) ) ) )

\end{verbatim}

\chapter{Rosetta Code Tasks starting with L}

\section*{LZW compression}

\begin{verbatim}

(de lzwCompress (Lst)
   (let (Codes 255  Dict)
      (balance 'Dict
         (make
            (for C Codes
               (link (cons (char C) C)) ) ) )
      (make
         (let W (pop 'Lst)
            (for C Lst
               (let WC (pack W C)
                  (if (lup Dict WC)
                     (setq W WC)
                     (link (cdr (lup Dict W)))
                     (idx 'Dict (cons WC (inc 'Codes)) T)
                     (setq W C) ) ) )
            (and W (link (cdr (lup Dict W)))) ) ) ) )

(de lzwDecompress (Lst)
   (let (Codes 255  Dict)
      (balance 'Dict
         (make
            (for C Codes
               (link (list C (char C))) ) ) )
      (make
         (let W NIL
            (for N Lst
               (let WC (if (lup Dict N) (cdr @) (cons (last W) W))
                  (chain (reverse WC))
                  (when W
                     (idx 'Dict (cons (inc 'Codes) (cons (last WC) W)) T) )
                  (setq W WC) ) ) ) ) ) )

Test:

: (lzwCompress (chop "TOBEORNOTTOBEORTOBEORNOT"))
-> (84 79 66 69 79 82 78 79 84 256 258 260 265 259 261 263)

: (pack (lzwDecompress @))
-> "TOBEORNOTTOBEORTOBEORNOT"

\end{verbatim}

\section*{Last Fridays of year}

\begin{verbatim}

(de lastFridays (Y)
   (for M `(range 1 12)
      (prinl
         (dat\$
            (find '((D) (= "Friday" (day D)))
               (mapcar '((D) (date Y M D)) `(range 31 22)) )
            "-" ) ) ) )

Test:

: (lastFridays 2012)
2012-01-27
2012-02-24
2012-03-30
2012-04-27
2012-05-25
2012-06-29
2012-07-27
2012-08-31
2012-09-28
2012-10-26
2012-11-30
2012-12-28

\end{verbatim}

\section*{Last letter-first letter (!)}

\begin{verbatim}

(de pokemonChain (File)
   (let Names (make (in File (while (read) (link @))))
      (for Name Names
         (let C (last (chop Name))
            (set Name
               (filter '((Nm) (pre? C Nm)) Names) ) ) )
      (let Res NIL
         (for Name Names
            (let Lst NIL
               (recur (Name Lst)
                  (if (or (memq Name Lst) (not (val (push 'Lst Name))))
                     (when (> (length Lst) (length Res))
                        (setq Res Lst) )
                     (mapc recurse (val Name) (circ Lst)) ) ) ) )
         (flip Res) ) ) )

Test:

: (pokemonChain "pokemon.list")
-> (machamp poliwrath haxorus scrafty yamask kangaskhan nidoking gabite emboar
registeel landorus seaking girafarig gulpin noctowl loudred darmanitan nosepass
simisear rufflet tyrogue exeggcute emolga audino)

: (length @)
-> 24

\end{verbatim}

\section*{Leap year}

\begin{verbatim}

(de isLeapYear (Y)
   (bool (date Y 2 29)) )

Output:

: (isLeapYear 2010)
-> NIL

: (isLeapYear 2008)
-> T

: (isLeapYear 1600)
-> T

: (isLeapYear 1700)
-> NIL

\end{verbatim}

\section*{Least common multiple}

\begin{verbatim}

Using 'gcd' from [[Greatest common divisor#PicoLisp]]:

(de lcm (A B)
   (abs (*/ A B (gcd A B))) )

\end{verbatim}

\section*{Letter frequency}

\begin{verbatim}

(let Freq NIL
   (in "file.txt"
      (while (char) (accu 'Freq @ 1)) )
   (sort Freq) )

For a "file.txt":

abcd
cdef

Output:

-> (("^J" . 2) ("a" . 1) ("b" . 1) ("c" . 2) ("d" . 2) ("e" . 1) ("f" . 1))

\end{verbatim}

\section*{Levenshtein distance}

\begin{verbatim}

Translation of the pseudo-code in the Wikipedia article:

(de levenshtein (A B)
   (let D
      (cons
         (range 0 (length A))
         (mapcar
            '((I) (cons I (copy A)))
            (range 1 (length B)) ) )
      (for (J . Y) B
         (for (I . X) A
            (set
               (nth D (inc J) (inc I))
               (if (= X Y)
                  (get D J I)
                  (inc
                     (min
                        (get D J (inc I))
                        (get D (inc J) I)
                        (get D J I) ) ) ) ) ) ) ) )

or, using 'map' to avoid list indexing:

(de levenshtein (A B)
   (let D
      (cons
         (range 0 (length A))
         (mapcar
            '((I) (cons I (copy A)))
            (range 1 (length B)) ) )
      (map
         '((B Y)
            (map
               '((A X P)
                  (set (cdr P)
                     (if (= (car A) (car B))
                        (car X)
                        (inc (min (cadr X) (car P) (car X))) ) ) )
               A
               (car Y)
               (cadr Y) ) )
         B
         D ) ) )

Output in both cases:

: (levenshtein (chop "kitten") (chop "sitting"))
-> 3

\end{verbatim}

\section*{Linear congruential generator}

\begin{verbatim}

(zero *BsdSeed *MsSeed)

(de bsdRand ()
   (setq *BsdSeed
      (\& (+ 12345 (* 1103515245 *BsdSeed)) `(dec (** 2 31))) ) )

(de msRand ()
   (>> 16
      (setq *MsSeed
         (\& (+ 2531011 (* 214013 *MsSeed)) `(dec (** 2 31))) ) ) )

Output:

: (do 7 (printsp (bsdRand)))
12345 1406932606 654583775 1449466924 229283573 1109335178 1051550459 -> 1051550459

: (do 12 (printsp (msRand)))
38 7719 21238 2437 8855 11797 8365 32285 10450 30612 5853 28100 -> 28100

\end{verbatim}

\section*{List comprehensions}

\begin{verbatim}

PicoLisp doesn't have list comprehensions.
We might use a generator function, pipe, coroutine or pilog predicate.

# Using a generator function

(de pythag (N)
   (job '((X . 1) (Y . 1) (Z . 0))
      (loop
         (when (> (inc 'Z) N)
            (when (> (inc 'Y) N)
               (setq Y (inc 'X)) )
            (setq Z Y) )
         (T (> X N))
         (T (= (+ (* X X) (* Y Y)) (* Z Z))
            (list X Y Z) ) ) ) )

(while (pythag 20)
   (println @) )

# Using a pipe

(pipe
   (for X 20
      (for Y (range X 20)
         (for Z (range Y 20)
            (when (= (+ (* X X) (* Y Y)) (* Z Z))
               (pr (list X Y Z)) ) ) ) )
   (while (rd)
      (println @) ) )

# Using a coroutine

Coroutines are available only in the 64-bit version.

(de pythag (N)
   (co 'pythag
      (for X N
         (for Y (range X N)
            (for Z (range Y N)
               (when (= (+ (* X X) (* Y Y)) (* Z Z))
                  (yield (list X Y Z)) ) ) ) ) ) )

(while (pythag 20)
   (println @) )

Output in all three cases:

(3 4 5)
(5 12 13)
(6 8 10)
(8 15 17)
(9 12 15)
(12 16 20)

# Using Pilog

{{works with|PicoLisp|3.0.9.7}}

(be pythag (@N @X @Y @Z)
   (for @X @N)
   (for @Y @X @N)
   (for @Z @Y @N)
   (@ let (X (-> @X)  Y (-> @Y)  Z (-> @Z))
       (= (+ (* X X) (* Y Y)) (* Z Z)) ) )

Test:

: (? (pythag 20 @X @Y @Z))
 @X=3 @Y=4 @Z=5
 @X=5 @Y=12 @Z=13
 @X=6 @Y=8 @Z=10
 @X=8 @Y=15 @Z=17
 @X=9 @Y=12 @Z=15
 @X=12 @Y=16 @Z=20
-> NIL

\end{verbatim}

\section*{Literals/Floating point}

\begin{verbatim}

PicoLisp does not support floating point literals in the base language, only
fixed point (scaled) decimal integers of unlimited size and precision. See
[http://software-lab.de/doc/ref.html#num-io Numbers] in the reference.

\end{verbatim}

\section*{Literals/Integer}

\begin{verbatim}

In the strict sense of this task, PicoLisp reads only integers at bases which
are a power of ten (scaled fixpoint numbers). This is controlled via the global
variable '[http://software-lab.de/doc/refS.html#*Scl *Scl]':

: (setq *Scl 4)
-> 4

: 123.456789
-> 1234568

However, the reader is normally augmented by read macros, which can read any
base or any desired format. Read macros are not executed at runtime, but
intially when the sources are read.

: '(a `(hex "7F") b `(oct "377") c)
-> (a 127 b 255 c)

In addition to standard formats like
'[http://software-lab.de/doc/refH.html#hex hex]' (hexadecimal) and
'[http://software-lab.de/doc/refO.html#oct oct]' (octal),
there are also more esoteric formats like
'[http://software-lab.de/doc/refF.html#fmt64 fmt64]' (base 64) and
'[http://software-lab.de/doc/refH.html#hax hax]' (hexadecimal numbers
coded with alphabetic characters).

\end{verbatim}

\section*{Literals/String}

\begin{verbatim}

PicoLisp doesn't have a string data type. Instead, symbols are used. Certain
uninterned symbols, called
[http://software-lab.de/doc/ref.html#transient "transient symbols"],
however, look and behave like strings on other languages.

Syntactically, transient symbols (called "strings" in the following)  are
surrounded by double quotes.

: "ab\"cd"
-> "ab\"cd"

Double quotes in strings are escaped with a backslash.

ASCII control characters can be written using the hat ('^') character:

: "ab^Icd^Jef"  # Tab, linefeed

There is no special character type or representation. Individual characters are
handled as single-character strings:

: (chop "abc")
-> ("a" "b" "c")

: (pack (reverse @))
-> "cba"

A limited handling of here-strings is available with the
'[http://software-lab.de/doc/refH.html#here here]' function.

\end{verbatim}

\section*{Logical operations}

\begin{verbatim}

(de logic (A B)
   (prin "A AND B is ")
   (println (and A B))
   (prin "A OR B is ")
   (println (or A B))
   (prin "A XOR B is ")
   (println (xor A B))
   (prin "NOT A is ")
   (println (not A)) )

\end{verbatim}

\section*{Long multiplication}

\begin{verbatim}

: (* (** 2 64) (** 2 64))
-> 340282366920938463463374607431768211456

\end{verbatim}

\section*{Longest common subsequence (!)}

\begin{verbatim}

(de commonSequences (A B)
   (when A
      (conc
         (when (member (car A) B)
            (mapcar '((L) (cons (car A) L))
               (cons NIL (commonSequences (cdr A) (cdr @))) ) )
         (commonSequences (cdr A) B) ) ) )

(maxi length
   (commonSequences
      (chop "thisisatest")
      (chop "testing123testing") ) )

Output:

-> ("t" "s" "i" "t" "e" "s" "t")

\end{verbatim}

\section*{Longest string challenge}

\begin{verbatim}

Not sure if this meets the spirit. I would implement it the same way if there
were no "restrictions":

(mapc prinl
   (maxi '((L) (length (car L)))
      (by length group
         (in NIL
            (make (until (eof) (link (line)))) ) ) ) )

Another solution avoids 'group', and builds an associative buffer of lines
instead:

(let Buf NIL
   (in NIL
      (until (eof)
         (let (Line (line)  Len (length Line))
            (if (assoc Len Buf)
               (conc @ (cons Line))
               (push 'Buf (cons Len (cons Line))) ) ) ) )
   (mapc prinl (cdr (maxi car Buf))) )

\end{verbatim}

\section*{Look-and-say sequence}

\begin{verbatim}

(de las (Lst)
   (make
      (while Lst
         (let (N 1  C)
            (while (= (setq C (pop 'Lst)) (car Lst))
               (inc 'N) )
            (link N C) ) ) ) )

Usage:

: (las (1))
-> (1 1)
: (las @)
-> (2 1)
: (las @)
-> (1 2 1 1)
: (las @)
-> (1 1 1 2 2 1)
: (las @)
-> (3 1 2 2 1 1)
: (las @)
-> (1 3 1 1 2 2 2 1)
: (las @)
-> (1 1 1 3 2 1 3 2 1 1)
: (las @)
-> (3 1 1 3 1 2 1 1 1 3 1 2 2 1)

\end{verbatim}

\section*{Loop over multiple arrays simultaneously}

\begin{verbatim}

(mapc prinl
   '(a b c)
   '(A B C)
   (1 2 3) )

The length of the first argument list controls the operation. If subsequent
lists are longer, their remaining values are ignored. If they are shorter, NIL
is passed to the function.

\end{verbatim}

\section*{Loops/Break}

\begin{verbatim}

Literally:

(use R
   (loop
      (println (setq R (rand 1 19)))
      (T (= 10 R))
      (println (rand 1 19)) ) )

Shorter:

(until (= 10 (println (rand 1 19)))
   (println (rand 1 19)) )

\end{verbatim}

\section*{Loops/Continue}

\begin{verbatim}

PicoLisp doesn't have an explicit 'continue' functionality. It can always be
emulated with a conditional expression.

(for I 10
   (print I)
   (if (=0 (\% I 5))
      (prinl)
      (prin ", ") ) )

\end{verbatim}

\section*{Loops/Do-while}

\begin{verbatim}

Literally:

(let Val 0
   (loop
      (println (inc 'Val))
      (T (=0 (\% Val 6))) ) )

Shorter:

(let Val 0
   (until (=0 (\% (println (inc 'Val)) 6))) )

or:

(for (Val 0  (n0 (\% (println (inc 'Val)) 6))))

\end{verbatim}

\section*{Loops/Downward for}

\begin{verbatim}

(for (I 10 (ge0 I) (dec I))
   (println I) )

or:

(mapc println (range 10 0))

\end{verbatim}

\section*{Loops/For}

\begin{verbatim}

(for N 5
   (do N (prin "*"))
   (prinl) )

\end{verbatim}

\section*{Loops/For with a specified step}

\begin{verbatim}

(for (N 1 (> 10 N) (+ N 2))
   (printsp N) )

\end{verbatim}

\section*{Loops/Foreach}

\begin{verbatim}

(mapc println '(Apple Banana Coconut))

\end{verbatim}

\section*{Loops/Infinite}

\begin{verbatim}

(loop (prinl "SPAM"))

\end{verbatim}

\section*{Loops/N plus one half}

\begin{verbatim}

(for N 10
   (prin N)
   (T (= N 10))
   (prin ", ") )

\end{verbatim}

\section*{Loops/Nested}

\begin{verbatim}

(for Lst (make (do 10 (link (make (do 10 (link (rand 1 20)))))))
   (T
      (for N Lst
         (printsp N)
         (T (= N 20) T) ) ) )

or:

(catch NIL
   (for Lst (make (do 10 (link (make (do 10 (link (rand 1 20)))))))
      (for N Lst
         (printsp N)
         (and (= N 20) (throw)) ) ) )

\end{verbatim}

\section*{Loops/While}

\begin{verbatim}

(let N 1024
   (while (gt0 N)
      (println N)
      (setq N (/ N 2)) ) )

\end{verbatim}

\section*{Lucas-Lehmer test}

\begin{verbatim}

(de prime? (N)
   (or
      (= N 2)
      (and
         (> N 1)
         (bit? 1 N)
         (for (D 3  T  (+ D 2))
            (T (> D (sqrt N)) T)
            (T (=0 (\% N D)) NIL) ) ) ) )

(de mersenne? (P)
   (or
      (= P 2)
      (let (MP (dec (>> (- P) 1))  S 4)
         (do (- P 2)
            (setq S (\% (- (* S S) 2) MP)) )
         (=0 S) ) ) )

Output:

: (for N 10000
   (and (prime? N) (mersenne? N) (println N)) )
2
3
5
7
13
17
19
31
61
89
107
127
521
607
1279
2203
2281
3217
4253
4423
9689
9941

\end{verbatim}

\section*{Luhn test of credit card numbers}

\begin{verbatim}

(de luhn (Num)  # 'Num' may be a number or a string
   (=0
      (\%
         (sum
            '((C F)
               (setq C (- (char C) 48))
               (if F
                  C                               # Odd
                  (+ (/ C 5) (\% (* 2 C) 10)) ) )  # Even
            (flip (chop Num))
            '(T NIL .) )
         10 ) ) )

Output:

: (mapcar luhn (49927398716 49927398717 1234567812345678 1234567812345670))
-> (0 NIL NIL 0)

\end{verbatim}

\chapter{Rosetta Code Tasks starting with M}

\section*{MD5}

\begin{verbatim}

(let Str "The quick brown fox jumped over the lazy dog's back"
   (pack
      (mapcar '((B) (pad 2 (hex B)))
         (native "libcrypto.so" "MD5" '(B . 16) Str (length Str) '(NIL (16))) ) ) )

Output:

-> "E38CA1D920C4B8B8D3946B2C72F01680"

\end{verbatim}

\section*{MD5/Implementation}

\begin{verbatim}

This is an implementation of the pseudo-code in the Wikipedia article. Special
care had to be taken with modulo 32-bit arithmetics, as PicoLisp supports only
numbers of unspecified size.

(scl 12)
(load "@lib/math.l")  # For 'sin'

(de *Md5-R
   7 12 17 22  7 12 17 22  7 12 17 22  7 12 17 22
   5  9 14 20  5  9 14 20  5  9 14 20  5  9 14 20
   4 11 16 23  4 11 16 23  4 11 16 23  4 11 16 23
   6 10 15 21  6 10 15 21  6 10 15 21  6 10 15 21 )

(de *Md5-K
   ~(make
      (for I 64
         (link
            (/ (* (abs (sin (* I 1.0))) `(** 2 32)) 1.0) ) ) ) )

(de mod32 (N)
   (\& N `(hex "FFFFFFFF")) )

(de not32 (N)
   (x| N `(hex "FFFFFFFF")) )

(de add32 @
   (mod32 (pass +)) )

(de leftRotate (X C)
   (| (mod32 (>> (- C) X)) (>> (- 32 C) X)) )

(de md5 (Str)
   (let Len (length Str)
      (setq Str
         (conc
            (need
               (- 8 (* 64 (/ (+ Len 1 8 63) 64)))  # Pad to 64-8 bytes
               (conc
                  (mapcar char (chop Str))   # Works only with ASCII characters
                  (cons `(hex "80")) )       # '1' bit
               0 )                           # Pad with '0'
            (make
               (setq Len (* 8 Len))
               (do 8
                  (link (\& Len 255))
                  (setq Len (>> 8 Len )) ) ) ) ) )
   (let
      (H0 `(hex "67452301")
         H1 `(hex "EFCDAB89")
         H2 `(hex "98BADCFE")
         H3 `(hex "10325476") )
      (while Str
         (let
            (A H0  B H1  C H2  D H3
               W (make
                  (do 16
                     (link
                        (apply |
                           (mapcar >> (0 -8 -16 -24) (cut 4 'Str)) ) ) ) ) )
               (use (Tmp F G)
                  (for I 64
                     (cond
                        ((>= 16 I)
                           (setq
                              F (| (\& B C) (\& (not32 B) D))
                              G I ) )
                        ((>= 32 I)
                           (setq
                              F (| (\& D B) (\& (not32 D) C))
                              G (inc (\& (inc (* 5 (dec I))) 15)) ) )
                        ((>= 48 I)
                           (setq
                              F (x| B C D)
                              G (inc (\& (+ 5 (* 3 (dec I))) 15)) ) )
                        (T
                           (setq
                              F (x| C (| B (not32 D)))
                              G (inc (\& (* 7 (dec I)) 15)) ) ) )
                     (setq
                        Tmp D
                        D C
                        C B
                        B
                        (add32 B
                           (leftRotate
                              (add32 A F (get *Md5-K I) (get W G))
                              (get *Md5-R I) ) )
                        A Tmp ) ) )
               (setq
                  H0 (add32 H0 A)
                  H1 (add32 H1 B)
                  H2 (add32 H2 C)
                  H3 (add32 H3 D) ) ) )
      (pack
         (make
            (for N (list H0 H1 H2 H3)
               (do 4  # Convert to little endian hex string
                  (link (pad 2 (hex (\& N 255))))
                  (setq N (>> 8 N)) ) ) ) ) ) )

Output:

: (md5 "")
-> "D41D8CD98F00B204E9800998ECF8427E"
: (md5 "a")
-> "0CC175B9C0F1B6A831C399E269772661"
: (md5 "abc")
-> "900150983CD24FB0D6963F7D28E17F72"
: (md5 "message digest")
-> "F96B697D7CB7938D525A2F31AAF161D0"
: (md5 "abcdefghijklmnopqrstuvwxyz")
-> "C3FCD3D76192E4007DFB496CCA67E13B"
: (md5 "ABCDEFGHIJKLMNOPQRSTUVWXYZabcdefghijklmnopqrstuvwxyz0123456789")
-> "D174AB98D277D9F5A5611C2C9F419D9F"
: (md5 "12345678901234567890123456789012345678901234567890123456789012345678901234567890")
-> "57EDF4A22BE3C955AC49DA2E2107B67A"

\end{verbatim}

\section*{Making the application start up from a main routine}

\begin{verbatim}

PicoLisp automatically starts a REPL (read-eval-print loop), which is a kind of
event loop. If that is not desired, call (wait), or pass -wait on the command
line. Per convention, the GUI event loop is started by calling (go), or by
passing -go on the command line.

\end{verbatim}

\section*{Man or boy test}

\begin{verbatim}

As PicoLisp uses exclusively shallow dynamic binding, stack frames have to be
explicitly constructed.

(de a (K X1 X2 X3 X4 X5)
   (let (@K (cons K)  B (cons))  # Explicit frame
      (set B
         (curry (@K B X1 X2 X3 X4) ()
            (a (dec @K) (car B) X1 X2 X3 X4) ) )
      (if (gt0 (car @K)) ((car B)) (+ (X4) (X5))) ) )

(a 10 '(() 1) '(() -1) '(() -1) '(() 1) '(() 0))

Output:

-> -67

\end{verbatim}

\section*{Mandelbrot set}

\begin{verbatim}

(scl 6)

(let Ppm (make (do 300 (link (need 400))))
   (for (Y . Row) Ppm
      (for (X . @) Row
         (let (ZX 0  ZY 0  CX (*/ (- X 250) 1.0 150)  CY (*/ (- Y 150) 1.0 150)  C 570)
            (while (and (> 4.0 (+ (*/ ZX ZX 1.0) (*/ ZY ZY 1.0))) (gt0 C))
               (let Tmp (- (*/ ZX ZX 1.0) (*/ ZY ZY 1.0) (- CX))
                  (setq
                     ZY (+ (*/ 2 ZX ZY 1.0) CY)
                     ZX Tmp ) )
               (dec 'C) )
            (set (nth Ppm Y X) (list 0 C C)) ) ) )
   (out "img.ppm"
      (prinl "P6")
      (prinl 400 " " 300)
      (prinl 255)
      (for Y Ppm (for X Y (apply wr X))) ) )

\end{verbatim}

\section*{Map range}

\begin{verbatim}

(scl 1)

(de mapRange (Val A1 A2 B1 B2)
   (+ B1 (*/ (- Val A1) (- B2 B1) (- A2 A1))) )


(for Val (range 0 10.0 1.0)
   (prinl
      (format (mapRange Val 0 10.0 -1.0 0) *Scl) ) )

Output:

-1.0
-0.9
-0.8
-0.7
-0.6
-0.5
-0.4
-0.3
-0.2
-0.1
0.0

\end{verbatim}

\section*{Matrix multiplication (!)}

\begin{verbatim}

(de matMul (Mat1 Mat2)
   (mapcar
      '((Row)
         (apply mapcar Mat2
            '(@ (sum * Row (rest))) ) )
      Mat1 ) )

(matMul
   '((1 2 3) (4 5 6))
   '((6 -1) (3 2) (0 -3)) )

Output:

-> ((12 -6) (39 -12))

\end{verbatim}

\section*{Matrix transposition (!)}

\begin{verbatim}

(de matTrans (Mat)
   (apply mapcar Mat list) )

(matTrans '((1 2 3) (4 5 6)))

Output:

-> ((1 4) (2 5) (3 6))

\end{verbatim}

\section*{Matrix-exponentiation operator (!)}

\begin{verbatim}

Uses the 'matMul' function from [[Matrix multiplication#PicoLisp]]

(de matIdent (N)
   (let L (need N (1) 0)
      (mapcar '(() (copy (rot L))) L) ) )

(de matExp (Mat N)
   (let M (matIdent (length Mat))
      (do N
         (setq M (matMul M Mat)) )
      M ) )

(matExp '((3 2) (2 1)) 3)

Output:

-> ((55 34) (34 21))

\end{verbatim}

\section*{Maze generation}

\begin{verbatim}

This solution uses 'grid' from "lib/simul.l" to generate the two-dimensional
structure.

{misc/maze.l}

Output:

: (display (maze 11 8))
   +   +---+---+---+---+---+---+---+---+---+---+
 8 |           |       |                       |
   +   +   +   +   +   +   +---+   +---+---+   +
 7 |   |   |       |   |   |       |       |   |
   +---+   +---+---+   +   +   +---+   +   +   +
 6 |   |       |       |   |           |   |   |
   +   +---+   +---+   +---+---+---+   +   +---+
 5 |       |       |               |   |       |
   +---+   +---+   +---+---+---+   +---+---+   +
 4 |   |       |       |       |   |           |
   +   +---+   +---+   +---+   +   +   +---+   +
 3 |       |       |   |       |   |       |   |
   +   +---+---+   +   +   +   +   +---+   +   +
 2 |       |       |   |   |   |   |       |   |
   +   +   +   +---+   +   +---+   +   +---+   +
 1 |   |               |               |
   +---+---+---+---+---+---+---+---+---+---+---+
     a   b   c   d   e   f   g   h   i   j   k

\end{verbatim}

\section*{Maze solving}

\begin{verbatim}

(de shortestPath (Goal This Maze)
   (let (Path NIL  Best NIL  Dir " > ")
      (recur (This Path Dir)
         (when (and This (not (: mark)))
            (push 'Path (cons This Dir))
            (if (== Goal This)
               (unless (and Best (>= (length Path) (length Best)))
                  (setq Best Path) )
               (=: mark T)
               (recurse (: west) Path " > ")
               (recurse (: east) Path " < ")
               (recurse (: south) Path " \^ ")
               (recurse (: north) Path " v ")
               (=: mark NIL) ) ) )
      (disp Maze 0
         '((Fld) (if (asoq Fld Best) (cdr @) "   ")) ) ) )

Using the maze produced in [[Maze generation#PicoLisp]], this finds the shortest
path from the top-left cell 'a8' to the bottom-right exit 'k1':

: (shortestPath 'a8 'k1 (maze 11 8))
   +   +---+---+---+---+---+---+---+---+---+---+
 8 | >   >   v | >   v |                       |
   +   +   +   +   +   +   +---+   +---+---+   +
 7 |   |   | >   ^ | v |   |       |       |   |
   +---+   +---+---+   +   +   +---+   +   +   +
 6 |   |       |     v |   |           |   |   |
   +   +---+   +---+   +---+---+---+   +   +---+
 5 |       |       | >   >   >   v |   |       |
   +---+   +---+   +---+---+---+   +---+---+   +
 4 |   |       |       |       | v | >   >   v |
   +   +---+   +---+   +---+   +   +   +---+   +
 3 |       |       |   |       | v | ^   < | v |
   +   +---+---+   +   +   +   +   +---+   +   +
 2 |       |       |   |   |   | v | >   ^ | v |
   +   +   +   +---+   +   +---+   +   +---+   +
 1 |   |               |         >   ^ |     >
   +---+---+---+---+---+---+---+---+---+---+---+
     a   b   c   d   e   f   g   h   i   j   k

\end{verbatim}

\section*{Median filter}

\begin{verbatim}

(de ppmMedianFilter (Radius Ppm)
   (let Len (inc (* 2 Radius))
      (make
         (chain (head Radius Ppm))
         (for (Y Ppm  T  (cdr Y))
            (NIL (nth Y Len)
               (chain (tail Radius Y)) )
            (link
               (make
                  (chain (head Radius (get Y (inc Radius))))
                  (for (X (head Len Y) T)
                     (NIL (nth X 1 Len)
                        (chain (tail Radius (get X (inc Radius)))) )
                     (link
                        (cdr
                           (get
                              (sort
                                 (mapcan
                                    '((Y)
                                       (mapcar
                                          '((C)
                                             (cons
                                                (+
                                                   (* (car C) 2126)     # Red
                                                   (* (cadr C) 7152)    # Green
                                                   (* (caddr C) 722) )  # Blue
                                                C ) )
                                          (head Len Y) ) )
                                    X ) )
                              (inc Radius) ) ) )
                     (map pop X) ) ) ) ) ) ) )

Test using 'ppmRead' from [[Bitmap/Read a PPM file#PicoLisp]] and 'ppmWrite'
from [[Bitmap/Write a PPM file#PicoLisp]]:

(ppmWrite (ppmMedianFilter 2 (ppmRead "Lenna100.ppm")) "a.ppm")

\end{verbatim}

\section*{Memory allocation}

\begin{verbatim}

Only the heap can be explicitly controlled in PicoLisp. Usually this is not
necessary, as it happens automatically.

But if desired, memory can be pre-allocated by calling gc with a single numeric
argument, specifying the desired number of megabytes that should be reserved.
When that argument is zero, the heap size is decreased (as far as possible).

\end{verbatim}

\section*{Memory layout of a data structure}

\begin{verbatim}

PicoLisp can handle bit fields or bit structures only as bignums. They can be
manipulated with '[http://software-lab.de/doc/ref_.html#\& \&]',
'[http://software-lab.de/doc/ref_.html#| |]' and
'[http://software-lab.de/doc/refX.html#x| x|]',
or tested with '[http://software-lab.de/doc/refB.html#bit? bit?]'.

# Define bit constants
(for (N . Mask) '(CD RD TD DTR SG DSR RTS CTS RI)
   (def Mask (>> (- 1 N) 1)) )

# Test if Clear to send
(when (bit? CTS Data)
   ... )

\end{verbatim}

\section*{Menu}

\begin{verbatim}

(de choose (Prompt Items)
   (use N
      (loop
         (for (I . Item) Items
            (prinl I ": " Item) )
         (prin Prompt " ")
         (NIL (setq N (in NIL (read))))
         (T (>= (length Items) N 1) (get Items N)) ) ) )

(choose "Which is from the three pigs?"
   '("fee fie" "huff and puff" "mirror mirror" "tick tock") )

Output:

1: fee fie
2: huff and puff
3: mirror mirror
4: tick tock
Which is from the three pigs? 2
-> "huff and puff"

\end{verbatim}

\section*{Metaprogramming}

\begin{verbatim}

As in any Lisp, metaprogramming is an essential aspect of PicoLisp.
In most cases normal functions are used to extend the language
(see [[Extend your language#PicoLisp]]),
[http://software-lab.de/doc/ref.html#macro-io read-macros] operate on
the source level, and also runtime
[http://software-lab.de/doc/refM.html#macro macros] are used occasionally.

\end{verbatim}

\section*{Metered concurrency}

\begin{verbatim}

(let Sem (tmp "sem")
   (for U 4  # Create 4 concurrent units
      (unless (fork)
         (ctl Sem
            (prinl "Unit " U " aquired the semaphore")
            (wait 2000)
            (prinl "Unit " U " releasing the semaphore") )
         (bye) ) ) )

\end{verbatim}

\section*{Metronome}

\begin{verbatim}

A short beep (440 Hz, 40 msec) is produced in a child process, while a
"pendulum" is swinging left and right. Hitting any key will stop it.

(de metronome (Bpm)
   (if (fork)
      (let Pid @
         (for Pendulum '(" /" . ("^H^H\\ " "^H^H /" .))
            (tell Pid 'call "/usr/bin/beep" "-f" 440 "-l" 40)
            (prin Pendulum)
            (T (key (*/ 30000 Bpm)) (tell Pid 'bye)) )
         (prinl) )
      (wait) ) )

Test:

: (metronome 60)
 /
-> NIL  # A key was hit

\end{verbatim}

\section*{Miller-Rabin primality test}

\begin{verbatim}

(de longRand (N)
   (use (R D)
      (while (=0 (setq R (abs (rand)))))
      (until (> R N)
         (unless (=0 (setq D (abs (rand))))
            (setq R (* R D)) ) )
      (\% R N) ) )

(de **Mod (X Y N)
   (let M 1
      (loop
         (when (bit? 1 Y)
            (setq M (\% (* M X) N)) )
         (T (=0 (setq Y (>> 1 Y)))
            M )
         (setq X (\% (* X X) N)) ) ) )

(de _prim? (N D S)
   (use (A X R)
      (while (> 2 (setq A (longRand N))))
      (setq R 0  X (**Mod A D N))
      (loop
         (T
            (or
               (and (=0 R) (= 1 X))
               (= X (dec N)) )
            T )
         (T
            (or
               (and (> R 0) (= 1 X))
               (>= (inc 'R) S) )
            NIL )
         (setq X (\% (* X X) N)) ) ) )

(de prime? (N K)
   (default K 50)
   (and
      (> N 1)
      (bit? 1 N)
      (let (D (dec N)  S 0)
         (until (bit? 1 D)
            (setq
               D  (>> 1 D)
               S  (inc S) ) )
         (do K
            (NIL (_prim? N D S))
            T ) ) ) )

Output:

: (filter '((I) (prime? I)) (range 937 1000))
-> (937 941 947 953 967 971 977 983 991 997)

: (prime? 4547337172376300111955330758342147474062293202868155909489)
-> T

: (prime? 4547337172376300111955330758342147474062293202868155909393)
-> NIL

\end{verbatim}

\section*{Minesweeper game}

\begin{verbatim}

# NIL    Hidden: Empty field
# T      Hidden: Mine
# 0-8    Marked: Empty field
# ?      Marked: Mine

(de minesweeper (DX DY Density)
   (default Density 20)
   (setq *Field (make (do DY (link (need DX)))))
   (use (X Y)
      (do (prinl "Number of mines: " (*/ DX DY Density 100))
         (while
            (get *Field
               (setq Y (rand 1 DY))
               (setq X (rand 1 DX)) ) )
         (set (nth *Field Y X) T) ) )
   (showMines) )

(de showMines ()
   (for L *Field
      (for F L
         (prin (if (flg? F) "." F)) )
      (prinl) ) )

(de *NeighborX -1  0 +1 -1  +1 -1  0 +1)
(de *NeighborY -1 -1 -1  0   0 +1 +1 +1)

(de c (X Y)
   (if (=T (get *Field Y X))
      "KLABOOM!! You hit a mine."
      (let Visit NIL
         (recur (X Y)
            (when
               (=0
                  (set (nth *Field Y X)
                     (cnt
                        '((DX DY)
                           (=T (get *Field (+ Y DY) (+ X DX))) )
                        *NeighborX
                        *NeighborY ) ) )
               (mapc
                  '((DX DY)
                     (and
                        (get *Field (inc 'DY Y))
                        (nth @ (inc 'DX X))
                        (not (member (cons DX DY) Visit))
                        (push 'Visit (cons DX DY))
                        (recurse DX DY) ) )
                  *NeighborX
                  *NeighborY ) ) ) )
      (showMines) ) )

(de m (X Y)
   (set (nth *Field Y X) '?)
   (showMines)
   (unless (fish =T *Field)
      "Congratulations! You won!!" ) )

Output:

: (minesweeper 6 4)
Number of mines: 5
......
......
......
......
-> NIL

: (c 6 4)
......
...122
...100
...100
-> NIL

# ... omitted ...

: (c 1 4)
.201..
.20122
121100
01.100
-> NIL

# ... omitted ...

: (m 1 1)
?201..
.20122
121100
01.100
-> NIL

# ... omitted ...

: (m 3 4)
?201??
?20122
121100
01?100
-> "Congratulations! You won!!"

\end{verbatim}

\section*{Modular exponentiation}

\begin{verbatim}

The following function is taken from "lib/rsa.l":

(de **Mod (X Y N)
   (let M 1
      (loop
         (when (bit? 1 Y)
            (setq M (\% (* M X) N)) )
         (T (=0 (setq Y (>> 1 Y)))
            M )
         (setq X (\% (* X X) N)) ) ) )

Test:

: (**Mod
   2988348162058574136915891421498819466320163312926952423791023078876139
   2351399303373464486466122544523690094744975233415544072992656881240319
   10000000000000000000000000000000000000000 )
-> 1527229998585248450016808958343740453059
\end{verbatim}

\section*{Monte Carlo methods}

\begin{verbatim}

(de carloPi (Scl)
   (let (Dim (** 10 Scl)  Dim2 (* Dim Dim)  Pi 0)
      (do (* 4 Dim)
         (let (X (rand 0 Dim)  Y (rand 0 Dim))
            (when (>= Dim2 (+ (* X X) (* Y Y)))
               (inc 'Pi) ) ) )
      (format Pi Scl) ) )

(for N 6
   (prinl (carloPi N)) )

Output:

3.4
3.23
3.137
3.1299
3.14360
3.140964

\end{verbatim}

\section*{Monty Hall problem}

\begin{verbatim}

(de montyHall (Keep)
   (let (Prize (rand 1 3)  Choice (rand 1 3))
      (if Keep                    # Keeping the first choice?
         (= Prize Choice)         # Yes: Monty's choice doesn't matter
         (<> Prize Choice) ) ) )  # Else: Win if your first choice was wrong

(prinl
   "Strategy KEEP    -> "
   (let Cnt 0
      (do 10000 (and (montyHall T) (inc 'Cnt)))
      (format Cnt 2) )
   " \%" )

(prinl
   "Strategy SWITCH  -> "
   (let Cnt 0
      (do 10000 (and (montyHall NIL) (inc 'Cnt)))
      (format Cnt 2) )
   " \%" )

Output:

Strategy KEEP    -> 33.01 \%
Strategy SWITCH  -> 67.73 \%

\end{verbatim}

\section*{Morse code}

\begin{verbatim}

The following simply uses the 'beep' pc-speaker beeper utility.

# *Morse *Dit *Dah

(balance '*Morse
   (mapcar
      '((L)
         (def (car L)
            (mapcar = (chop (cadr L)) '("." .)) ) )
      (quote
         ("!"  "---.")     ("\"" ".-..-.")   ("\$" "...-..-")   ("'" ".----.")
         ("(" "-.--.")     (")" "-.--.-")    ("+" ".-.-.")     ("," "--..--")
         ("-" "-....-")    ("." ".-.-.-")    ("/" "-..-.")
         ("0" "-----")     ("1" ".----")     ("2" "..---")     ("3" "...--")
         ("4" "....-")     ("5" ".....")     ("6" "-....")     ("7" "--...")
         ("8" "---..")     ("9" "----.")
         (":" "---...")    (";" "-.-.-.")    ("=" "-...-")     ("?" "..--..")
         ("@" ".--.-.")
         ("A" ".-")        ("B" "-...")      ("C" "-.-.")      ("D" "-..")
         ("E" ".")         ("F" "..-.")      ("G" "--.")       ("H" "....")
         ("I" "..")        ("J" ".---")      ("K" "-.-")       ("L" ".-..")
         ("M" "--")        ("N" "-.")        ("O" "---")       ("P" ".--.")
         ("Q" "--.-")      ("R" ".-.")       ("S" "...")       ("T" "-")
         ("U" "..-")       ("V" "...-")      ("W" ".--")       ("X" "-..-")
         ("Y" "-.--")      ("Z" "--..")
         ("[" "-.--.")     ("]" "-.--.-")    ("_" "..--.-") ) ) )

# Words per minute
(de wpm (N)
   (setq *Dit (*/ 1200 N)  *Dah (* 3 *Dit)) )

(wpm 20)

# Morse a string
(de morse (Str)
   (for C (chop Str)
      (cond
         ((sp? C) (wait (+ *Dah *Dit)))         # White space: Pause
         ((idx '*Morse (uppc C))                # Known character
            (for Flg (val (car @))
               (call "/usr/bin/beep" "-D" *Dit "-l" (if Flg *Dit *Dah)) ) )
         (T (call "/usr/bin/beep" "-f" 370)) )  # Unkown character
      (wait (- *Dah *Dit)) ) )

(morse "Hello world!")

\end{verbatim}

\section*{Mouse position}

\begin{verbatim}

The following works in an XTerm window. After calling (mousePosition), click
into the current terminal window. The returned value is (X . Y), where X is the
column and Y the line number.

(de mousePosition ()
   (prog2
      (prin "^[[?9h")  # Mouse reporting on
      (and
         (= "^[" (key))
         (key 200)
         (key 200)
         (key)
         (cons
            (- (char (key)) 32)
            (- (char (key)) 32) ) )
      (prin "^[[?9l") ) )  # Mouse reporting off

Output:

: (mousePosition)
-> (7 . 3)

\end{verbatim}

\section*{Multiline shebang}

\begin{verbatim}

We can use a multi-line comment #{ ... }# to hide the shell commands from Lisp.
The opening #{ in turn is a coment for the shell.

#!/bin/bash
#{
exec pil \$0 foo bar
# }#

# Lisp code
(println (cadr (file)) (opt) (opt))
(bye)

Output:

\$ ./myScript
"myScript" "foo" "bar"

\end{verbatim}

\section*{Multiple distinct objects}

\begin{verbatim}

Create 5 distinct (empty) objects:

: (make (do 5 (link (new))))
-> (\$384717187 \$384717189 \$384717191 \$384717193 \$384717195)

Create 5 anonymous symbols with the values 1 .. 5:

: (mapcar box (range 1 5))
-> (\$384721107 \$384721109 \$384721111 \$384721113 \$384721115)
: (val (car @))
-> 1
: (val (cadr @@))
-> 2

\end{verbatim}

\section*{Multiple regression}

\begin{verbatim}

(scl 20)

# Matrix transposition
(de matTrans (Mat)
   (apply mapcar Mat list) )

# Matrix multiplication
(de matMul (Mat1 Mat2)
   (mapcar
      '((Row)
         (apply mapcar Mat2
            '(@ (sum */ Row (rest) (1.0 .))) ) )
      Mat1 ) )

# Matrix identity
(de matIdent (N)
   (let L (need N (1.0) 0)
      (mapcar '(() (copy (rot L))) L) ) )

# Reduced row echelon form
(de reducedRowEchelonForm (Mat)
   (let (Lead 1  Cols (length (car Mat)))
      (for (X Mat X (cdr X))
         (NIL
            (loop
               (T (seek '((R) (n0 (get R 1 Lead))) X)
                  @ )
               (T (> (inc 'Lead) Cols)) ) )
         (xchg @ X)
         (let D (get X 1 Lead)
            (map
               '((R) (set R (*/ (car R) 1.0 D)))
               (car X) ) )
         (for Y Mat
            (unless (== Y (car X))
               (let N (- (get Y Lead))
                  (map
                     '((Dst Src)
                        (inc Dst (*/ N (car Src) 1.0)) )
                     Y
                     (car X) ) ) ) )
         (T (> (inc 'Lead) Cols)) ) )
   Mat )

{{trans|JavaScript}}

(de matInverse (Mat)
   (let N (length Mat)
      (unless (= N (length (car Mat)))
         (quit "can't invert a non-square matrix") )
      (mapc conc Mat (matIdent N))
      (mapcar '((L) (tail N L)) (reducedRowEchelonForm Mat)) ) )

(de columnVector (Ary)
   (mapcar cons Ary) )

(de regressionCoefficients (Mat X)
   (let Xt (matTrans X)
      (matMul (matMul (matInverse (matMul Xt X)) Xt) Mat) ) )

(setq
   Y (columnVector (1.0 2.0 3.0 4.0 5.0))
   X (columnVector (2.0 1.0 3.0 4.0 5.0)) )

(round (caar (regressionCoefficients Y X)) 17)

Output:

-> "0.98181818181818182"

\end{verbatim}

\section*{Multiplication tables}

\begin{verbatim}

(de mulTable (N)
   (space 4)
   (for X N
      (prin (align 4 X)) )
   (prinl)
   (prinl)
   (for Y N
      (prin (align 4 Y))
      (space (* (dec Y) 4))
      (for (X Y (>= N X) (inc X))
         (prin (align 4 (* X Y))) )
      (prinl) ) )

(mulTable 12)

Output:

       1   2   3   4   5   6   7   8   9  10  11  12

   1   1   2   3   4   5   6   7   8   9  10  11  12
   2       4   6   8  10  12  14  16  18  20  22  24
   3           9  12  15  18  21  24  27  30  33  36
   4              16  20  24  28  32  36  40  44  48
   5                  25  30  35  40  45  50  55  60
   6                      36  42  48  54  60  66  72
   7                          49  56  63  70  77  84
   8                              64  72  80  88  96
   9                                  81  90  99 108
  10                                     100 110 120
  11                                         121 132
  12                                             144

\end{verbatim}

\section*{Multisplit}

\begin{verbatim}

(de multisplit (Str Sep)
   (setq Sep (mapcar chop Sep))
   (make
      (for (S (chop Str) S)
         (let L
            (make
               (loop
                  (T (find head Sep (circ S))
                     (link
                        (list
                           (- (length Str) (length S))
                           (pack (cut (length @) 'S)) ) ) )
                  (link (pop 'S))
                  (NIL S (link NIL)) ) )
            (link (pack (cdr (rot L))))
            (and (car L) (link @)) ) ) ) )

(println (multisplit "a!===b=!=c" '("==" "!=" "=")))
(println (multisplit "a!===b=!=c" '("=" "!=" "==")))

Output:

("a" (1 "!=") NIL (3 "==") "b" (6 "=") NIL (7 "!=") "c")
("a" (1 "!=") NIL (3 "=") NIL (4 "=") "b" (6 "=") NIL (7 "!=") "c")

\end{verbatim}

\section*{Mutex}

\begin{verbatim}

PicoLisp uses several mechanisms of interprocess communication, mainly within
the same process family (children of the same parent process) for database
synchronization (e.g.
'[http://software-lab.de/doc/refL.html#lock lock]',
'[http://software-lab.de/doc/refS.html#sync sync]' or
'[http://software-lab.de/doc/refT.html#tell tell]'.

For a simple synchronization of unrelated PicoLisp processes the
'[http://software-lab.de/doc/refA.html#acquire acquire]' /
'[http://software-lab.de/doc/refR.html#release release]' function pair
can be used.

\end{verbatim}

\section*{Mutual recursion}

\begin{verbatim}

(de f (N)
   (if (=0 N)
      1
      (- N (m (f (dec N)))) ) )

(de m (N)
   (if (=0 N)
      0
      (- N (f (m (dec N)))) ) )

\end{verbatim}

\chapter{Rosetta Code Tasks starting with N}

\section*{N-queens problem}

\begin{verbatim}

(load "@lib/simul.l")

(de queens (N)
   (let (R (range 1 N)  Cnt 0)
      (for L (permute (range 1 N))
         (when
            (= N
               (length (uniq (mapcar + L R)))
               (length (uniq (mapcar - L R))) )
            (inc 'Cnt) ) )
      Cnt ) )

This alternative version does not first pre-generate all permutations with
'permute', but creates them recursively. Also, it directly checks for
duplicates, instead of calling 'uniq' and 'length'. This is much faster.

(de queens (N)
   (let (R (range 1 N)  L (copy R)  X L  Cnt 0)
      (recur (X)  # Permute
         (if (cdr X)
            (do (length X)
               (recurse (cdr X))
               (rot X) )
            (or
               (seek  # Direct check for duplicates
                  '((L) (member (car L) (cdr L)))
                  (mapcar + L R) )
               (seek
                  '((L) (member (car L) (cdr L)))
                  (mapcar - L R) )
               (inc 'Cnt) ) ) )
      Cnt ) )

Output in both cases:

: (queens 8)
-> 92

\end{verbatim}

\section*{Named parameters}

\begin{verbatim}

PicoLisp uses normally positional parameters, but
'[http://software-lab.de/doc/refB.html#bind bind]' can be used
to establish bindings to passed names.

Passing symbol-value pairs

(de foo @
   (bind (rest)  # Bind symbols in CARs to values in CDRs
      (println 'Bar 'is Bar)
      (println 'Mumble 'is Mumble) ) )

(foo '(Bar . 123) '(Mumble . "def"))

Passing a name list followed by values

(de foo @
   (bind (next)                # Save all symbols in first argument
      (mapc set (arg) (rest))  # then bind them to remaining arguments
      (println 'Bar 'is Bar)
      (println 'Mumble 'is Mumble) ) )

(foo '(Bar Mumble) 123 "def")

Output in both cases:

Bar is 123
Mumble is "def"

\end{verbatim}

\section*{Narcissist}

\begin{verbatim}

(de narcissist (Str)
   (= Str (str narcissist)) )

Output:

: (narcissist "(Str) (= Str (str narcissist))")
-> T

\end{verbatim}

\section*{Natural sorting}

\begin{verbatim}

This parser takes care of features 1,2,3,4,5 and 8:

(de parseNatural (Str)
   (clip
      (make
         (for (L (chop Str)  L)
            (cond
               ((sp? (car L))
                  (link " ")
                  (while (and L (sp? (car L)))
                     (pop 'L) ) )
               ((>= "9" (car L) "0")
                  (link
                     (format
                        (make
                           (loop
                              (link (pop 'L))
                              (NIL (>= "9" (car L) "0")) ) ) ) ) )
               (T
                  (let Word
                     (pack
                        (replace
                           (make
                              (loop
                                 (link (lowc (pop 'L)))
                                 (NIL L)
                                 (T (sp? (car L)))
                                 (T (>= "9" (car L) "0")) ) )
                            "ß" "ss" "ſ" "s" "ʒ" "s" ) )
                     (unless (member Word '(the it to))
                        (link Word) ) ) ) ) ) ) ) )

Test:

: (parseNatural " ^MThe abc123Defß ^I Ghi ")
-> ("abc" 123 "defss" " " "ghi")

Sorting is trivial then:

(de naturalSort (Lst)
   (by parseNatural sort Lst) )

Test:

(de *TestData
   "# Ignoring leading spaces"
   ("ignore leading spaces: 2-2" " ignore leading spaces: 2-1" "  ignore leading spaces: 2+0" "   ignore leading spaces: 2+1")

   "# Ignoring multiple adjacent spaces (m.a.s)"
   ("ignore m.a.s spaces: 2-2" "ignore m.a.s  spaces: 2-1" "ignore m.a.s   spaces: 2+0" "ignore m.a.s    spaces: 2+1")

   "# Equivalent whitespace characters"
   ("Equiv. spaces: 3-3" "Equiv.^Mspaces: 3-2" "Equiv.^Acspaces: 3-1" "Equiv.^Kbspaces: 3+0" "Equiv.^Jspaces: 3+1" "Equiv.^Ispaces: 3+2")

   "# Case Indepenent sort"
   ("cASE INDEPENENT: 3-2" "caSE INDEPENENT: 3-1" "casE INDEPENENT: 3+0" "case INDEPENENT: 3+1")

   "# Numeric fields as numerics"
   ("foo100bar99baz0.txt" "foo100bar10baz0.txt" "foo1000bar99baz10.txt" "foo1000bar99baz9.txt")

   "# Title sorts"
   ("The Wind in the Willows" "The 40th step more" "The 39 steps" "Wanda")

   "# Equivalent accented characters (and case)"
   ("Equiv. ý accents: 2-2" "Equiv. Ý accents: 2-1" "Equiv. y accents: 2+0" "Equiv. Y accents: 2+1")

   # "Separated ligatures"
   ### ("IJ ligatured ij" "no ligature")

   "# Character replacements"
   ("Start with an ʒ: 2-2" "Start with an ſ: 2-1" "Start with an ß: 2+0" "Start with an s: 2+1") )

(de pythonOut (Ttl Lst)
   (prinl Ttl)
   (prin "['" (car Lst))
   (for S (cdr Lst)
      (prin "',^J '" S) )
   (prinl "']") )

(for X *TestData
   (if (atom X)
      (prinl X)
      (pythonOut "Text strings:" X)
      (pythonOut "Normally sorted :" (sort (copy X)))
      (pythonOut "Naturally sorted:" (naturalSort X))
      (prinl) ) )

Output:

# Ignoring leading spaces
Text strings:
['ignore leading spaces: 2-2',
 ' ignore leading spaces: 2-1',
 '  ignore leading spaces: 2+0',
 '   ignore leading spaces: 2+1']
Normally sorted :
['   ignore leading spaces: 2+1',
 '  ignore leading spaces: 2+0',
 ' ignore leading spaces: 2-1',
 'ignore leading spaces: 2-2']
Naturally sorted:
['  ignore leading spaces: 2+0',
 '   ignore leading spaces: 2+1',
 ' ignore leading spaces: 2-1',
 'ignore leading spaces: 2-2']

# Ignoring multiple adjacent spaces (m.a.s)
Text strings:
['ignore m.a.s spaces: 2-2',
 'ignore m.a.s  spaces: 2-1',
 'ignore m.a.s   spaces: 2+0',
 'ignore m.a.s    spaces: 2+1']
Normally sorted :
['ignore m.a.s    spaces: 2+1',
 'ignore m.a.s   spaces: 2+0',
 'ignore m.a.s  spaces: 2-1',
 'ignore m.a.s spaces: 2-2']
Naturally sorted:
['ignore m.a.s   spaces: 2+0',
 'ignore m.a.s    spaces: 2+1',
 'ignore m.a.s  spaces: 2-1',
 'ignore m.a.s spaces: 2-2']

# Equivalent whitespace characters
Text strings:
['Equiv. spaces: 3-3',
 'Equiv.
spaces: 3-2',
 'Equiv.cspaces: 3-1',
 'Equiv.bspaces: 3+0',
 'Equiv.
spaces: 3+1',
 'Equiv.	spaces: 3+2']
Normally sorted :
['Equiv.cspaces: 3-1',
 'Equiv.	spaces: 3+2',
 'Equiv.
spaces: 3+1',
 'Equiv.bspaces: 3+0',
 'Equiv.
spaces: 3-2',
 'Equiv. spaces: 3-3']
Naturally sorted:
['Equiv.bspaces: 3+0',
 'Equiv.cspaces: 3-1',
 'Equiv.
spaces: 3+1',
 'Equiv.	spaces: 3+2',
 'Equiv.
spaces: 3-2',
 'Equiv. spaces: 3-3']

# Case Indepenent sort
Text strings:
['cASE INDEPENENT: 3-2',
 'caSE INDEPENENT: 3-1',
 'casE INDEPENENT: 3+0',
 'case INDEPENENT: 3+1']
Normally sorted :
['cASE INDEPENENT: 3-2',
 'caSE INDEPENENT: 3-1',
 'casE INDEPENENT: 3+0',
 'case INDEPENENT: 3+1']
Naturally sorted:
['casE INDEPENENT: 3+0',
 'case INDEPENENT: 3+1',
 'caSE INDEPENENT: 3-1',
 'cASE INDEPENENT: 3-2']

# Numeric fields as numerics
Text strings:
['foo100bar99baz0.txt',
 'foo100bar10baz0.txt',
 'foo1000bar99baz10.txt',
 'foo1000bar99baz9.txt']
Normally sorted :
['foo1000bar99baz10.txt',
 'foo1000bar99baz9.txt',
 'foo100bar10baz0.txt',
 'foo100bar99baz0.txt']
Naturally sorted:
['foo100bar10baz0.txt',
 'foo100bar99baz0.txt',
 'foo1000bar99baz9.txt',
 'foo1000bar99baz10.txt']

# Title sorts
Text strings:
['The Wind in the Willows',
 'The 40th step more',
 'The 39 steps',
 'Wanda']
Normally sorted :
['The 39 steps',
 'The 40th step more',
 'The Wind in the Willows',
 'Wanda']
Naturally sorted:
['The 39 steps',
 'The 40th step more',
 'Wanda',
 'The Wind in the Willows']

# Equivalent accented characters (and case)
Text strings:
['Equiv. ý accents: 2-2',
 'Equiv. Ý accents: 2-1',
 'Equiv. y accents: 2+0',
 'Equiv. Y accents: 2+1']
Normally sorted :
['Equiv. Y accents: 2+1',
 'Equiv. y accents: 2+0',
 'Equiv. Ý accents: 2-1',
 'Equiv. ý accents: 2-2']
Naturally sorted:
['Equiv. y accents: 2+0',
 'Equiv. Y accents: 2+1',
 'Equiv. Ý accents: 2-1',
 'Equiv. ý accents: 2-2']

# Character replacements
Text strings:
['Start with an ʒ: 2-2',
 'Start with an ſ: 2-1',
 'Start with an ß: 2+0',
 'Start with an s: 2+1']
Normally sorted :
['Start with an s: 2+1',
 'Start with an ß: 2+0',
 'Start with an ſ: 2-1',
 'Start with an ʒ: 2-2']
Naturally sorted:
['Start with an s: 2+1',
 'Start with an ſ: 2-1',
 'Start with an ʒ: 2-2',
 'Start with an ß: 2+0']

\end{verbatim}

\section*{Non-continuous subsequences}

\begin{verbatim}

{{trans|Scheme}}

(de ncsubseq (Lst)
   (let S 0
      (recur (S Lst)
         (ifn Lst
            (and (>= S 3) '(NIL))
            (let (X (car Lst)  XS (cdr Lst))
               (ifn (bit? 1 S)  # even
                  (conc
                     (mapcar '((YS) (cons X YS))
                        (recurse (inc S) XS) )
                     (recurse S XS) )
                  (conc
                     (mapcar '((YS) (cons X YS))
                        (recurse S XS) )
                     (recurse (inc S) XS) ) ) ) ) ) ) )

\end{verbatim}

\section*{Non-decimal radices/Convert}

\begin{verbatim}

(de numToString (N Base)
   (default Base 10)
   (let L NIL
      (loop
         (let C (\% N Base)
            (and (> C 9) (inc 'C 39))
            (push 'L (char (+ C `(char "0")))) )
         (T (=0 (setq N (/ N Base)))) )
      (pack L) ) )

(de stringToNum (S Base)
   (default Base 10)
   (let N 0
      (for C (chop S)
         (when (> (setq C (- (char C) `(char "0"))) 9)
            (dec 'C 39) )
         (setq N (+ C (* N Base))) )
      N ) )

(prinl (numToString 26 16))
(prinl (stringToNum "1a" 16))
(prinl (numToString 123456789012345678901234567890 36))

Output:

"1a"
26
"byw97um9s91dlz68tsi"

\end{verbatim}

\section*{Non-decimal radices/Input}

\begin{verbatim}

(de parseNumber (S Base)
   (let N 0
      (for C (chop S)
         (when (> (setq C (- (char C) `(char "0"))) 9)
            (dec 'C 39) )
         (setq N (+ C (* N Base))) )
      N ) )

(println (parseNumber "91g5dcg2h6da7260a9f3c4a" 19))

Output:

123456789012345678901234567890

\end{verbatim}

\section*{Non-decimal radices/Output}

\begin{verbatim}

(de printNumber (N Base)
   (when (>= N Base)
      (printNumber (/ N Base) Base) )
   (let C (\% N Base)
      (and (> C 9) (inc 'C 39))
      (prin (char (+ C `(char "0")))) ) )

(printNumber 26 16))
(prinl)
(printNumber 123456789012345678901234567890 36))
(prinl)

Output:

1a
byw97um9s91dlz68tsi

\end{verbatim}

\section*{Nth root}

\begin{verbatim}

(load "@lib/math.l")

(de nthroot (N A)
   (let (X1 A  X2 (*/ A N))
      (until (= X1 X2)
         (setq
            X1 X2
            X2 (*/
               (+
                  (* X1 (dec N))
                  (*/ A 1.0 (pow X1 (* (dec N) 1.0))) )
               N ) ) )
      X2 ) )

(prinl (format (nthroot 2  2.0) *Scl))
(prinl (format (nthroot 3 12.3) *Scl))
(prinl (format (nthroot 4 45.6) *Scl))

Output:

1.414214
2.308350
2.598611

\end{verbatim}

\section*{Number names}

\begin{verbatim}

(de numName (N)
   (cond
      ((=0 N) "zero")
      ((lt0 N) (pack "minus " (numName (- N))))
      (T (numNm N)) ) )

(de numNm (N)
   (cond
      ((=0 N))
      ((> 14 N)
         (get '("one" "two" "three" "four" "five" "six" "seven" "eight" "nine" "ten" "eleven" "twelve" "thirteen") N) )
      ((= 15 N) "fifteen")
      ((= 18 N) "eighteen")
      ((> 20 N) (pack (numNm (\% N 10)) "teen"))
      ((> 100 N)
         (pack
            (get '("twen" "thir" "for" "fif" "six" "seven" "eigh" "nine") (dec (/ N 10)))
            "ty"
            (unless (=0 (\% N 10))
               (pack "-" (numNm (\% N 10))) ) ) )
      ((rank N '((100 . "hundred") (1000 . "thousand") (1000000 . "million")))
         (pack (numNm (/ N (car @))) " " (cdr @) " " (numNm (\% N (car @)))) ) ) )

\end{verbatim}

\section*{Number reversal game}

\begin{verbatim}

(load "@lib/simul.l")

(de reversalGame ()
   (let (Lst (shuffle (range 1 9))  Cnt 0)
      (while (apply < Lst)
         (setq Lst (shuffle Lst)) )
      (loop
         (printsp Lst)
         (T (apply < Lst) Cnt)
         (NIL (num? (read)))
         (setq Lst (flip Lst @))
         (inc 'Cnt) ) ) )

Output:

: (reversalGame)
(1 7 6 8 4 2 3 5 9) 4
(8 6 7 1 4 2 3 5 9) 8
(5 3 2 4 1 7 6 8 9) 6
(7 1 4 2 3 5 6 8 9) 7
(6 5 3 2 4 1 7 8 9) 6
(1 4 2 3 5 6 7 8 9) 2
(4 1 2 3 5 6 7 8 9) 4
(3 2 1 4 5 6 7 8 9) 3
(1 2 3 4 5 6 7 8 9) -> 8

\end{verbatim}

\section*{Numeric error propagation}

\begin{verbatim}

For this task, we overload the built-in arithmetic functions. If the arguments
are cons pairs, they are assumed to hold the fixpoint number in the CAR, and the
uncertainty's square in the CDR. Otherwise normal numbers are handled as usual.

The overloaded +, -, * and / operators look a bit complicated, because they must
handle an arbitrary number of arguments to be compatible with the standard
operators.

(scl 12)
(load "@lib/math.l")

# Overload arithmetic operators +, -, *, / and **
(redef + @
   (let R (next)
      (while (args)
         (let N (next)
            (setq R
               (if2 (atom R) (atom N)
                  (+ R N)                       # c + c
                  (cons (+ R (car N)) (cdr N))  # c + a
                  (cons (+ (car R) N) (cdr R))  # a + c
                  (cons                         # a + b
                     (+ (car R) (car N))
                     (+ (cdr R) (cdr N)) ) ) ) ) )
      R ) )

(redef - @
   (let R (next)
      (ifn (args)
         (- R)
         (while (args)
            (let N (next)
               (setq R
                  (if2 (atom R) (atom N)
                     (- R N)                       # c - c
                     (cons (- R (car N)) (cdr N))  # c - a
                     (cons (- (car R) N) (cdr R))  # a - c
                     (cons                         # a - b
                        (- (car R) (car N))
                        (+ (cdr R) (cdr N)) ) ) ) ) )
         R ) ) )

(redef * @
   (let R (next)
      (while (args)
         (let N (next)
            (setq R
               (if2 (atom R) (atom N)
                  (* R N)                                        # c * c
                  (cons                                          # c * a
                     (*/ R (car N) 1.0)
                     (mul2div2 (cdr N) R 1.0) )
                  (cons                                          # a * c
                     (*/ (car R) N 1.0)
                     (mul2div2 (cdr R) N 1.0) )
                  (uncMul (*/ (car R) (car N) 1.0) R N) ) ) ) )  # a * b
      R ) )

(redef / @
   (let R (next)
      (while (args)
         (let N (next)
            (setq R
               (if2 (atom R) (atom N)
                  (/ R N)                                        # c / c
                  (cons                                          # c / a
                     (*/ R 1.0 (car N))
                     (mul2div2 (cdr N) R 1.0) )
                  (cons                                          # a / c
                     (*/ (car R) 1.0 N)
                     (mul2div2 (cdr R) N 1.0) )
                  (uncMul (*/ (car R) 1.0 (car N)) R N) ) ) ) )  # a / b
      R ) )

(redef ** (A C)
   (if (atom A)
      (** A C)
      (let F (pow (car A) C)
         (cons F
            (mul2div2 (cdr A) (*/ F C (car A)) 1.0) ) ) ) )

# Utilities
(de mul2div2 (A B C)
   (*/ A B B (* C C)) )

(de uncMul (F R N)
   (cons F
      (mul2div2
         (+
            (mul2div2 (cdr R) 1.0 (car R))
            (mul2div2 (cdr N) 1.0 (car N)) )
         F
         1.0 ) ) )

# I/O conversion
(de unc (N U)
   (if U
      (cons N (*/ U U 1.0))
      (pack
         (round (car N) 10)
         " ± "
         (round (sqrt (* 1.0 (cdr N))) 8) ) ) )

Test:

(de distance (X1 Y1 X2 Y2)
   (**
      (+ (** (- X1 X2) 2.0) (** (- Y1 Y2) 2.0))
      0.5 ) )

(prinl "Distance: "
   (unc
      (distance
         (unc 100. 1.1)
         (unc 50. 1.2)
         (unc 200. 2.2)
         (unc 100. 2.3) ) ) )

Output:

Distance: 111.8033988750 ± 2.48716706

\end{verbatim}

\section*{Numerical integration}

\begin{verbatim}

(scl 6)

(de leftRect (Fun X)
   (Fun X) )

(de rightRect (Fun X H)
   (Fun (+ X H)) )

(de midRect (Fun X H)
   (Fun (+ X (/ H 2))) )

(de trapezium (Fun X H)
   (/ (+ (Fun X) (Fun (+ X H))) 2) )

(de simpson (Fun X H)
   (*/
      (+
         (Fun X)
         (* 4 (Fun (+ X (/ H 2))))
         (Fun (+ X H)) )
      6 ) )

(de square (X)
   (*/ X X 1.0) )

(de integrate (Fun From To Steps Meth)
   (let (H (/ (- To From) Steps)  Sum 0)
      (for (X From  (>= (- To H) X)  (+ X H))
         (inc 'Sum (Meth Fun X H)) )
      (*/ H Sum 1.0) ) )

(prinl (round (integrate square 3.0 7.0 30 simpson)))

Output:

105.333

\end{verbatim}

\chapter{Rosetta Code Tasks starting with O}

\section*{Object serialization}

\begin{verbatim}

The built-in function [http://software-lab.de/doc/refP.html#pr pr] serializes
any kind of data, and [http://software-lab.de/doc/refR.html#rd rd] reads it
back. This functionality is also used internally for database access and
interprocess-communication.

(class +Point)
# x y

(dm T (X Y)
   (=: x (or X 0))
   (=: y (or Y 0)) )

(dm print> ()
   (prinl "Point " (: x) "," (: y)) )

(class +Circle +Point)
# r

(dm T (X Y R)
   (super X Y)
   (=: r (or R 0)) )

(dm print> ()
   (prinl "Circle " (: x) "," (: y) "," (: r)) )

(setq
   P (new '(+Point) 3 4)
   C (new '(+Circle) 10 10 5) )

(print> P)
(print> C)

(out "objects.dat"
   (pr (val P) (getl P))
   (pr (val C) (getl C)) )

(in "objects.dat"
   (putl (setq A (box (rd))) (rd))
   (putl (setq B (box (rd))) (rd)) )

(print> A)
(print> B)

Output:

Point 3,4
Circle 10,10,5
Point 3,4
Circle 10,10,5

\end{verbatim}

\section*{Odd word problem}

\begin{verbatim}

(de oddWords ()
   (use C
      (loop
         (until (sub? (prin (setq C (char))) "!,.:;?"))
         (T (= "." C))
         (setq C (char))
         (T
            (= "."
               (prin
                  (recur (C)
                     (if (sub? C "!,.:;?")
                        C
                        (prog1 (recurse (char)) (prin C)) ) ) ) ) ) )
   (prinl) ) )

Test:

(in "txt1" (oddWords))
(in "txt2" (oddWords))

Output:

what,si,the;gninaem,of:efil.
we,era;not,ni,kansas;yna,more.

\end{verbatim}

\section*{Old lady swalllowed a fly}

\begin{verbatim}

{{trans|C}}

(de *Dict
   `(chop
      "_ha _c _e _p,/Quite absurd_f_p;_`cat,/Fancy that_fcat;_j`dog,\
         /What a hog_fdog;_l`pig,/Her mouth_qso big_fpig;_d_r,/She just \
         opened her throat_f_r;_icow,/_mhow she_ga cow;_k_o,/It_qrather \
         wonky_f_o;_a_o_bcow,_khorse.../She's dead, of course!/" )
   `(chop "_a_p_b_e ")
   `(chop "/S_t ")
   `(chop " to catch the ")
   `(chop "fly,/But _mwhy s_t fly,/Perhaps she'll die!//_ha")
   `(chop "_apig_bdog,_l`")
   `(chop "spider,/That wr_nj_ntickled inside her;_aspider_b_c")
   `(chop ", to_s a ")
   `(chop "_sed ")
   `(chop "There_qan old lady who_g")
   `(chop "_a_r_bpig,_d")
   `(chop "_acat_b_p,_")
   `(chop "_acow_b_r,_i")
   `(chop "_adog_bcat,_j")
   `(chop "I don't know ")
   `(chop "iggled and ")
   `(chop "donkey")
   `(chop "bird")
   `(chop " was ")
   `(chop "goat")
   `(chop " swallow")
   `(chop "he_gthe") )

(de oldLady (Lst Flg)
   (loop
      (let C (pop 'Lst)
         (cond
            (Flg
               (setq Flg
                  (oldLady (get *Dict (- (char C) 94))) ) )
            ((= "_" C) (on Flg))
            ((= "/" C) (prinl))
            (T (prin C)) ) )
      (NIL Lst) )
   Flg )

(oldLady (car *Dict))

\end{verbatim}

\section*{One of n lines in a file}

\begin{verbatim}

(de one-of-n (N)
   (let R 1
      (for I N
         (when (= 1 (rand 1 I))
            (setq R I) ) )
      R ) )

(let L (need 10 0)
   (do 1000000
      (inc (nth L (one-of-n 10))) )
   L )

Output:

-> (99893 100145 99532 100400 100263 100229 99732 100116 99709 99981)

\end{verbatim}

\section*{One-dimensional cellular automata}

\begin{verbatim}

(let Cells (chop "_###_##_#_#_#_#__#__")
   (do 10
      (prinl Cells)
      (setq Cells
         (make
            (link "_")
            (map
               '((L)
                  (case (head 3 L)
                     (`(mapcar chop '("___" "__#" "_#_" "#__" "###"))
                         (link "_") )
                     (`(mapcar chop '("_##" "#_#" "##_"))
                        (link "#") ) ) )
               Cells )
            (link "_") ) ) ) )

Output:

_###_##_#_#_#_#__#__
_#_#####_#_#_#______
__##___##_#_#_______
__##___###_#________
__##___#_##_________
__##____###_________
__##____#_#_________
__##_____#__________
__##________________
__##________________

\end{verbatim}

\section*{OpenGL}

\begin{verbatim}

This is for the 64-bit version.

(load "@lib/openGl.l")

(glutInit)
(glutInitWindowSize 400 300)
(glutCreateWindow "Triangle")

(displayPrg
   (glClearColor 0.3 0.3 0.3 0.0)
   (glClear (| GL_COLOR_BUFFER_BIT GL_DEPTH_BUFFER_BIT))
   (glShadeModel GL_SMOOTH)
   (glLoadIdentity)
   (glTranslatef -15.0 -15.0 0.0)
   (glBegin GL_TRIANGLES)
   (glColor3f 1.0 0.0 0.0)
   (glVertex2f 0.0 0.0)
   (glColor3f 0.0 1.0 0.0)
   (glVertex2f 30.0 0.0)
   (glColor3f 0.0 0.0 1.0)
   (glVertex2f 0.0 30.0)
   (glEnd)
   (glFlush) )

(reshapeFunc
   '((Width Height)
      (glViewport 0 0 Width Height)
      (glMatrixMode GL_PROJECTION)
      (glLoadIdentity)
      (glOrtho -30.0 30.0 -30.0 30.0 -30.0 30.0)
      (glMatrixMode GL_MODELVIEW) ) )

# Exit upon mouse click
(mouseFunc '((Btn State X Y) (bye)))

(glutMainLoop)

\end{verbatim}

\section*{Optional parameters}

\begin{verbatim}

(de sortTable (Tbl . @)
   (let (Ordering prog  Column 1  Reverse NIL)  # Set defaults
      (bind (rest)                              # Bind optional params
         (setq Tbl
            (by '((L) (Ordering (get L Column)))
               sort
               Tbl ) )
         (if Reverse (flip Tbl) Tbl) ) ) )

Output:

(de *Data ("a" "bcdef" "X") (" " "qrst" "z") ("zap" "zip" "Zot"))

: (sortTable *Data)
-> ((" " "qrst" "z") ("a" "bcdef" "X") ("zap" "zip" "Zot"))

: (sortTable *Data '(Reverse . T))
-> (("zap" "zip" "Zot") ("a" "bcdef" "X") (" " "qrst" "z"))

: (sortTable *Data '(Column . 2) '(Ordering . length))
-> (("zap" "zip" "Zot") (" " "qrst" "z") ("a" "bcdef" "X"))

: (sortTable *Data '(Ordering . uppc) '(Column . 3))
-> (("a" "bcdef" "X") (" " "qrst" "z") ("zap" "zip" "Zot"))

\end{verbatim}

\section*{Order two numerical lists}

\begin{verbatim}

The built-in comparison functions already do this (not only for lists of
numbers, but for any arbitrary data type).

: (> (1 2 0 4 4 0 0 0) (1 2 1 3 2))
-> NIL

\end{verbatim}

\section*{Ordered Partitions (!)}

\begin{verbatim}

Uses the 'comb' function from [[Combinations#PicoLisp]]

(de partitions (Args)
   (let Lst (range 1 (apply + Args))
      (recur (Args Lst)
         (ifn Args
            '(NIL)
            (mapcan
               '((L)
                  (mapcar
                     '((R) (cons L R))
                     (recurse (cdr Args) (diff Lst L)) ) )
               (comb (car Args) Lst) ) ) ) ) )

Output:

: (more (partitions (2 0 2)))
((1 2) NIL (3 4))
((1 3) NIL (2 4))
((1 4) NIL (2 3))
((2 3) NIL (1 4))
((2 4) NIL (1 3))
((3 4) NIL (1 2))
-> NIL

: (more (partitions (1 1 1)))
((1) (2) (3))
((1) (3) (2))
((2) (1) (3))
((2) (3) (1))
((3) (1) (2))
((3) (2) (1))
-> NIL

\end{verbatim}

\section*{Ordered words}

\begin{verbatim}

(in "unixdict.txt"
   (mapc prinl
      (maxi '((L) (length (car L)))
         (by length group
            (filter '((S) (apply <= S))
               (make (while (line) (link @))) ) ) ) ) )

Output:

abbott
accent
accept
access
accost
almost
bellow
billow
biopsy
chilly
choosy
choppy
effort
floppy
glossy
knotty

\end{verbatim}

\chapter{Rosetta Code Tasks starting with P}

\section*{Palindrome detection}

\begin{verbatim}

(de palindrome? (S)
   (= (setq S (chop S)) (reverse S)) )

Output:

: (palindrome? "ingirumimusnocteetconsumimurigni")
-> T

\end{verbatim}

\section*{Pangram checker}

\begin{verbatim}

(de isPangram (Str)
   (not
      (diff
         '`(chop "abcdefghijklmnopqrstuvwxyz")
         (chop (lowc Str)) ) ) )

\end{verbatim}

\section*{Parallel calculations (!)}

\begin{verbatim}

The '[http://software-lab.de/doc/refL.html#later later]' function is used in
PicoLisp to start parallel computations. The following solution calls 'later' on
the 'factor' function from [[Prime decomposition#PicoLisp]], and then
'[http://software-lab.de/doc/refW.html#wait wait]'s until all results are
available:

(let Lst
   (mapcan
      '((N)
         (later (cons)               # When done,
            (cons N (factor N)) ) )  # return the number and its factors
      (quote
         188573867500151328137405845301  # Process a collection of 12 numbers
         3326500147448018653351160281
         979950537738920439376739947
         2297143294659738998811251
         136725986940237175592672413
         3922278474227311428906119
         839038954347805828784081
         42834604813424961061749793
         2651919914968647665159621
         967022047408233232418982157
         2532817738450130259664889
         122811709478644363796375689 ) )
   (wait NIL (full Lst))  # Wait until all computations are done
   (maxi '((L) (apply min L)) Lst) )  # Result: Number in CAR, factors in CDR

Output:

-> (2532817738450130259664889 6531761 146889539 2639871491)

\end{verbatim}

\section*{Parametric polymorphism}

\begin{verbatim}

PicoLisp is dynamically-typed, so in principle every function is polymetric over
its arguments. It is up to the function to decide what to do with them. A
function traversing a tree, modifying the nodes in-place (no matter what the
type of the node is):

(de mapTree (Tree Fun)
   (set Tree (Fun (car Tree)))
   (and (cadr Tree) (mapTree @ Fun))
   (and (cddr Tree) (mapTree @ Fun)) )

Test:

(balance 'MyTree (range 1 7))          # Create a tree of numbers
-> NIL

: (view MyTree T)                      # Display it
      7
   6
      5
4
      3
   2
      1
-> NIL


: (mapTree MyTree inc)                 # Increment all nodes
-> NIL

: (view MyTree T)                      # Display the tree
      8
   7
      6
5
      4
   3
      2
-> NIL


: (balance 'MyTree '("a" "b" "c" "d" "e" "f" "g"))  # Create a tree of strings
-> NIL

: (view MyTree T)                      # Display it
      "g"
   "f"
      "e"
"d"
      "c"
   "b"
      "a"
-> NIL

: (mapTree MyTree uppc)                # Convert all nodes to upper case
-> NIL

: (view MyTree T)                      # Display the tree
      "G"
   "F"
      "E"
"D"
      "C"
   "B"
      "A"
-> NIL

\end{verbatim}

\section*{Parametrized SQL statement}

\begin{verbatim}

As PicoLisp uses normal function calls for DB manipulations, parameters are
always treated as plain data and are not executed.

(for P (collect 'jerseyNum '+Players 99)
   (put!> P 'name "Smith, Steve")
   (put!> P 'score 42)
   (put!> P 'active T) )

\end{verbatim}

\section*{Parse an IP Address}

\begin{verbatim}

# Return a cons pair of address and port: (address . port)
(de ipAddress (Adr)
   (use (@A @B @C @D @Port)
      (cond
         ((match '("[" @A "]" ":" @Port) Adr)
            (adrIPv6 (split @A ":") @Port) )
         ((match '("[" @A "]") Adr)
            (adrIPv6 (split @A ":")) )
         ((match '(@A ":" @B ":" @C) Adr)
            (adrIPv6 (cons @A @B (split @C ":"))) )
         ((match '(@A "." @B "." @C "." @D ":" @Port) Adr)
            (adrIPv4 (list @A @B @C @D) @Port) )
         ((match '(@A "." @B "." @C "." @D) Adr)
            (adrIPv4 (list @A @B @C @D)) )
         (T (quit "Bad IP address" (pack Adr))) ) ) )

(de adrIPv4 (Lst Port)
   (cons
      (sum >> (-24 -16 -8 0) (mapcar format Lst))
      (format Port) ) )

(de adrIPv6 (Lst Port)
   (cons
      (sum >>
         (-112 -96 -80 -64 -48 -32 -16 0)
         (mapcan
            '((X)
               (if X
                  (cons (hex X))
                  (need (- 9 (length Lst)) 0) ) )  # Handle '::'
            (cons (or (car Lst) "0") (cdr Lst)) ) )
      (format Port) ) )

Test:

(for A
   (quote
      "127.0.0.1"
      "127.0.0.1:80"
      "::1"
      "[::1]:80"
      "2605:2700:0:3::4713:93e3"
      "[2605:2700:0:3::4713:93e3]:80" )
   (let I (ipAddress (chop A))
      (tab (-29 34 40 7)
         A
         (hex (car I))
         (format (car I))
         (cdr I) ) ) )

Output:

127.0.0.1                                              7F000001                              2130706433
127.0.0.1:80                                           7F000001                              2130706433     80
::1                                                           1                                       1
[::1]:80                                                      1                                       1     80
2605:2700:0:3::4713:93e3       260527000000000300000000471393E3  50537416338094019778974086937420469219
[2605:2700:0:3::4713:93e3]:80  260527000000000300000000471393E3  50537416338094019778974086937420469219     80

\end{verbatim}

\section*{Parsing command-line arguments}

\begin{verbatim}

PicoLisp doesn't have a library to get options. Instead, the command line is
parsed at startup and handled in the following way: Each command line argument
is executed (interpreted) as a Lisp source file, except that if the first
character is a hypen '-', then that arguments is taken as a Lisp function call
(without the surrounding parentheses). For example, the command line

\$ ./pil abc.l -foo def.l -"bar 3 4" -'mumble "hello"' -bye

has the effect that

# The file "abc.l" is executed
# (foo) is called
# The file "def.l" is executed
# (bar 3 4) is called
# (mumble "hello") is called
# (bye) is called, resulting in program termination

Command line arguments like "-v", "-n" and "-z" can be implemented simply by
defining three functions 'v', 'n' and 'z'.

In addition to the above mechanism, the command line can also be handled
"manually", by either processing the list of arguments returned by
'[http://software-lab.de/doc/refA.html#argv argv]', or by fetching arguments
individually with '[http://software-lab.de/doc/refO.html#opt opt]'.

\end{verbatim}

\section*{Parsing/RPN calculator algorithm}

\begin{verbatim}

This is an integer-only calculator:

(de rpnCalculator (Str)
   (let (^ **  Stack)  # Define '^' from the built-in '**'
      (prinl "Token  Stack")
      (for Token (str Str "*+-/\^")
         (if (num? Token)
            (push 'Stack @)
            (set (cdr Stack)
               (Token (cadr Stack) (pop 'Stack)) ) )
         (prin Token)
         (space 6)
         (println Stack) )
      (println (car Stack)) ) )

Test (note that the top-of-stack is in the left-most position):

: (rpnCalculator "3 4 2 * 1 5 - 2 3 \^ \^ / +")
Token  Stack
3      (3)
4      (4 3)
2      (2 4 3)
*      (8 3)
1      (1 8 3)
5      (5 1 8 3)
-      (-4 8 3)
2      (2 -4 8 3)
3      (3 2 -4 8 3)
^      (8 -4 8 3)
^      (65536 8 3)
/      (0 3)
+      (3)
3
-> 3

\end{verbatim}

\section*{Parsing/RPN to infix conversion}

\begin{verbatim}

We maintain a stack of cons pairs, consisting of precedences and partial
expressions. Numbers get a "highest" precedence of '9'.

(de leftAssoc (Op)
   (member Op '("*" "/" "+" "-")) )

(de precedence (Op)
   (case Op
      ("\^" 4)
      (("*" "/") 3)
      (("+" "-") 2) ) )

(de rpnToInfix (Str)
   (let Stack NIL
      (prinl "Token  Stack")
      (for Token (str Str "_")
         (cond
            ((num? Token) (push 'Stack (cons 9 @)))  # Highest precedence
            ((not (cdr Stack)) (quit "Stack empty"))
            (T
               (let (X (pop 'Stack)  P (precedence Token))
                  (set Stack
                     (cons P
                        (pack
                           (if ((if (leftAssoc Token) < <=) (caar Stack) P)
                              (pack "(" (cdar Stack) ")")
                              (cdar Stack) )
                           " " Token " "
                           (if ((if (leftAssoc Token) <= <) (car X) P)
                              (pack "(" (cdr X) ")")
                              (cdr X) ) ) ) ) ) ) )
         (prin Token)
         (space 6)
         (println Stack) )
      (prog1 (cdr (pop 'Stack))
         (and Stack (quit "Garbage remained on stack")) ) ) )

Test (note that the top-of-stack is in the left-most position):

: (rpnToInfix "3 4 2 * 1 5 - 2 3 \^ \^ / +")
Token  Stack
3      ((9 . 3))
4      ((9 . 4) (9 . 3))
2      ((9 . 2) (9 . 4) (9 . 3))
*      ((3 . "4 * 2") (9 . 3))
1      ((9 . 1) (3 . "4 * 2") (9 . 3))
5      ((9 . 5) (9 . 1) (3 . "4 * 2") (9 . 3))
-      ((2 . "1 - 5") (3 . "4 * 2") (9 . 3))
2      ((9 . 2) (2 . "1 - 5") (3 . "4 * 2") (9 . 3))
3      ((9 . 3) (9 . 2) (2 . "1 - 5") (3 . "4 * 2") (9 . 3))
^      ((4 . "2 \^ 3") (2 . "1 - 5") (3 . "4 * 2") (9 . 3))
^      ((4 . "(1 - 5) \^ 2 \^ 3") (3 . "4 * 2") (9 . 3))
/      ((3 . "4 * 2 / (1 - 5) \^ 2 \^ 3") (9 . 3))
+      ((2 . "3 + 4 * 2 / (1 - 5) \^ 2 \^ 3"))
-> "3 + 4 * 2 / (1 - 5) \^ 2 \^ 3"

: (rpnToInfix "1 2 + 3 4 + \^ 5 6 + \^")
Token  Stack
1      ((9 . 1))
2      ((9 . 2) (9 . 1))
+      ((2 . "1 + 2"))
3      ((9 . 3) (2 . "1 + 2"))
4      ((9 . 4) (9 . 3) (2 . "1 + 2"))
+      ((2 . "3 + 4") (2 . "1 + 2"))
^      ((4 . "(1 + 2) \^ (3 + 4)"))
5      ((9 . 5) (4 . "(1 + 2) \^ (3 + 4)"))
6      ((9 . 6) (9 . 5) (4 . "(1 + 2) \^ (3 + 4)"))
+      ((2 . "5 + 6") (4 . "(1 + 2) \^ (3 + 4)"))
^      ((4 . "((1 + 2) \^ (3 + 4)) \^ (5 + 6)"))
-> "((1 + 2) \^ (3 + 4)) \^ (5 + 6)"

\end{verbatim}

\section*{Parsing/Shunting-yard algorithm}

\begin{verbatim}

Note: "^" is a meta-character and must be escaped in strings

(de operator (Op)
   (member Op '("\^" "*" "/" "+" "-")) )

(de leftAssoc (Op)
   (member Op '("*" "/" "+" "-")) )

(de precedence (Op)
   (case Op
      ("\^" 4)
      (("*" "/") 3)
      (("+" "-") 2) ) )

(de shuntingYard (Str)
   (make
      (let (Fmt (-7 -30 -4)  Stack)
         (tab Fmt "Token" "Output" "Stack")
         (for Token (str Str "_")
            (cond
               ((num? Token) (link @))
               ((= "(" Token) (push 'Stack Token))
               ((= ")" Token)
                  (until (= "(" (car Stack))
                     (unless Stack
                        (quit "Unbalanced Stack") )
                     (link (pop 'Stack)) )
                  (pop 'Stack) )
               (T
                  (while
                     (and
                        (operator (car Stack))
                        ((if (leftAssoc (car Stack)) <= <)
                           (precedence Token)
                           (precedence (car Stack)) ) )
                     (link (pop 'Stack)) )
                  (push 'Stack Token) ) )
            (tab Fmt Token (glue " " (made)) Stack) )
         (while Stack
            (when (= "(" (car Stack))
               (quit "Unbalanced Stack") )
            (link (pop 'Stack))
            (tab Fmt NIL (glue " " (made)) Stack) ) ) ) )

Output:

: (shuntingYard "3 + 4 * 2 / (1 - 5) \^ 2 \^ 3")
Token  Output                        Stack
3      3
+      3                             +
4      3 4                           +
*      3 4                           *+
2      3 4 2                         *+
/      3 4 2 *                       /+
(      3 4 2 *                       (/+
1      3 4 2 * 1                     (/+
-      3 4 2 * 1                     -(/+
5      3 4 2 * 1 5                   -(/+
)      3 4 2 * 1 5 -                 /+
^      3 4 2 * 1 5 -                 ^/+
2      3 4 2 * 1 5 - 2               ^/+
^      3 4 2 * 1 5 - 2               ^^/+
3      3 4 2 * 1 5 - 2 3             ^^/+
       3 4 2 * 1 5 - 2 3 ^           ^/+
       3 4 2 * 1 5 - 2 3 ^ ^         /+
       3 4 2 * 1 5 - 2 3 ^ ^ /       +
       3 4 2 * 1 5 - 2 3 ^ ^ / +
-> (3 4 2 "*" 1 5 "-" 2 3 "\^" "\^" "/" "+")

\end{verbatim}

\section*{Partial function application}

\begin{verbatim}

(def 'fs mapcar)
(de f1 (N) (* 2 N))
(de f2 (N) (* N N))

(de partial (F1 F2)
   (curry (F1 F2) @
      (pass F1 F2) ) )

(def 'fsf1 (partial fs f1))
(def 'fsf2 (partial fs f2))

(for S '((0 1 2 3) (2 4 6 8))
   (println (fsf1 S))
   (println (fsf2 S)) )

Output:

(0 2 4 6)
(0 1 4 9)
(4 8 12 16)
(4 16 36 64)

\end{verbatim}

\section*{Pascal's triangle}

\begin{verbatim}

{{trans|C}}

(de pascalTriangle (N)
   (for I N
      (space (* 2 (- N I)))
      (let C 1
         (for K I
            (prin (align 3 C) " ")
            (setq C (*/ C (- I K) K)) ) )
      (prinl) ) )

\end{verbatim}

\section*{Pascal's triangle/Puzzle}

\begin{verbatim}

(be number (@N @Max)
   (@C box 0)
   (repeat)
   (or
      ((@ >= (val (-> @C)) (-> @Max)) T (fail))
      ((@N inc (-> @C))) ) )

(be + (@A @B @Sum)
   (@ -> @A)
   (@ -> @B)
   (@Sum + (-> @A) (-> @B)) )

(be + (@A @B @Sum)
   (@ -> @A)
   (@ -> @Sum)
   (@B - (-> @Sum) (-> @A))
   T
   (@ ge0 (-> @B)) )

(be + (@A @B @Sum)
   (number @A @Sum)
   (@B - (-> @Sum) (-> @A)) )

#{
         151
        A   B
      40  C   D
     E  F  G    H
   X  11  Y   4   Z
}#

(be puzzle (@X @Y @Z)
   (+ @A @B 151)
   (+ 40 @C @A)
   (+ @C @D @B)
   (+ @E @F 40)
   (+ @F @G @C)
   (+ @G @H @D)
   (+ @X 11 @E)
   (+ 11 @Y @F)
   (+ @Y 4 @G)
   (+ 4 @Z @H)
   (+ @X @Z @Y) )

Output:

: (? (puzzle @X @Y @Z))
 @X=5 @Y=13 @Z=8

\end{verbatim}

\section*{Pattern matching}

\begin{verbatim}

{{trans|Prolog}}

(be color (R))
(be color (B))

(be tree (@ E))
(be tree (@P (T @C @L @X @R))
   (color @C)
   (tree @P @L)
   (call @P @X)
   (tree @P @R) )

(be bal (B (T R (T R @A @X @B) @Y @C) @Z @D (T R (T B @A @X @B) @Y (T B @C @Z @D))))
(be bal (B (T R @A @X (T R @B @Y @C)) @Z @D (T R (T B @A @X @B) @Y (T B @C @Z @D))))
(be bal (B @A @X (T R (T R @B @Y @C) @Z @D) (T R (T B @A @X @B) @Y (T B @C @Z @D))))
(be bal (B @A @X (T R @B @Y (T R @C @Z @D)) (T R (T B @A @X @B) @Y (T B @C @Z @D))))

(be balance (@C @A @X @B @S)
   (bal @C @A @X @B @S)
   T )
(be balance (@C @A @X @B (T @C @A @X @B)))

(be ins (@X E (T R E @X E)))
(be ins (@X (T @C @A @Y @B) @R)
   (@ < (-> @X) (-> @Y))
   (ins @X @A @Ao)
   (balance @C @Ao @Y @B @R)
   T )
(be ins (@X (T @C @A @Y @B) @R)
   (@ > (-> @X) (-> @Y))
   (ins @X @B @Bo)
   (balance @C @A @Y @Bo @R)
   T )
(be ins (@X (T @C @A @Y @B) (T @C @A @Y @B)))

(be insert (@X @S (T B @A @Y @B))
   (ins @X @S (T @ @A @Y @B)) )

Test:

: (? (insert 2 E @A) (insert 1 @A @B) (insert 3 @B @C))
 @A=(T B E 2 E) @B=(T B (T R E 1 E) 2 E) @C=(T B (T R E 1 E) 2 (T R E 3 E))
-> NIL

\end{verbatim}

\section*{Percentage difference between images}

\begin{verbatim}

(call "convert" "Lenna50.jpg" (tmp "Lenna50.ppm"))
(call "convert" "Lenna100.jpg" (tmp "Lenna100.ppm"))

(let (Total 0  Diff 0)
   (in (tmp "Lenna50.ppm")
      (in (tmp "Lenna100.ppm")
         (while (rd 1)
            (inc 'Diff
               (*/
                  (abs (- @ (in -1 (rd 1))))
                  1000000
                  255 ) )
            (inc 'Total) ) ) )
   (prinl "Difference is " (format (*/ Diff Total) 4) " percent") )

Output:

Difference is 1.6256 percent

\end{verbatim}

\section*{Perfect numbers}

\begin{verbatim}

(de perfect (N)
   (let C 0
      (for I (/ N 2)
         (and (=0 (\% N I)) (inc 'C I)) )
      (= C N) ) )

\end{verbatim}

\section*{Permutation test}

\begin{verbatim}

(load "@lib/simul.l")  # For 'subsets'

(scl 2)

(de _stat (A)
   (let (LenA (length A)  SumA (apply + A))
      (-
         (*/ SumA LenA)
         (*/ (- SumAB SumA) (- LenAB LenA)) ) ) )

(de permutationTest (A B)
   (let
      (AB (append A B)
         SumAB (apply + AB)
         LenAB (length AB)
         Tobs (_stat A)
         Count 0 )
      (*/
         (sum
            '((Perm)
               (inc 'Count)
               (and (>= Tobs (_stat Perm)) 1) )
            (subsets (length A) AB) )
         100.0
         Count ) ) )

(setq
   *TreatmentGroup (0.85 0.88 0.75 0.66 0.25 0.29 0.83 0.39 0.97)
   *ControlGroup   (0.68 0.41 0.10 0.49 0.16 0.65 0.32 0.92 0.28 0.98) )

(let N (permutationTest *TreatmentGroup *ControlGroup)
   (prinl "under = " (round N) "\%, over = " (round (- 100.0 N)) "\%") )

Output:

under = 87.85\%, over = 12.15\%

\end{verbatim}

\section*{Permutations}

\begin{verbatim}

(load "@lib/simul.l")

(permute (1 2 3))

Output:

-> ((1 2 3) (1 3 2) (2 1 3) (2 3 1) (3 1 2) (3 2 1))

\end{verbatim}

\section*{Permutations/Derangements}

\begin{verbatim}

(load "@lib/simul.l")  # For 'permute'

(de derangements (Lst)
   (filter
      '((L) (not (find = L Lst)))
      (permute Lst) ) )

(de subfact (N)
   (if (>= 2 N)
      (if (= 1 N) 0 1)
      (*
         (dec N)
         (+ (subfact (dec N)) (subfact (- N 2))) ) ) )

Output:

: (derangements (range 1 4))
-> ((2 1 4 3) (2 3 4 1) (2 4 1 3) (3 1 4 2) (3 4 1 2) (3 4 2 1) (4 1 2 3) (4 3 1 2) (4 3 2 1))

: (for I (range 0 9)
   (tab (2 8 8)
      I
      (length (derangements (range 1 I)))
      (subfact I) ) )
 0       1       1
 1       0       0
 2       1       1
 3       2       2
 4       9       9
 5      44      44
 6     265     265
 7    1854    1854
 8   14833   14833
 9  133496  133496
-> NIL

: (subfact 20)
-> 895014631192902121

\end{verbatim}

\section*{Pi}

\begin{verbatim}

The following script uses the spigot algorithm published by Jeremy Gibbons.
Hit Ctrl-C to stop it.

#!/usr/bin/picolisp /usr/lib/picolisp/lib.l

(de piDigit ()
   (job '((Q . 1) (R . 0) (S . 1) (K . 1) (N . 3) (L . 3))
      (while (>= (- (+ R (* 4 Q)) S) (* N S))
         (mapc set '(Q R S K N L)
            (list
               (* Q K)
               (* L (+ R (* 2 Q)))
               (* S L)
               (inc K)
               (/ (+ (* Q (+ 2 (* 7 K))) (* R L)) (* S L))
               (+ 2 L) ) ) )
      (prog1 N
         (let M (- (/ (* 10 (+ R (* 3 Q))) S) (* 10 N))
            (setq Q (* 10 Q)  R (* 10 (- R (* N S)))  N M) ) ) ) )

(prin (piDigit) ".")
(loop
   (prin (piDigit))
   (flush) )

Output:

3.14159265358979323846264338327950288419716939937510582097494459 ...

\end{verbatim}

\section*{Pick random element}

\begin{verbatim}

(get Lst (rand 1 (length Lst)))

\end{verbatim}

\section*{Pinstripe/Printer}

\begin{verbatim}

(load "@lib/ps.l")

(call 'lpr
   (pdf "pinstripes"
      (a4)  # 595 x 842 dots
      (for X 595
         (gray (if (bit? 1 X) 0 100)
            (vline X 0 842) ) )
      (page) ) )

\end{verbatim}

\section*{Play recorded sounds}

\begin{verbatim}

The obvious way is to call 'sox', the "Swiss Army knife of audio manipulation"
(man sox).

The following plays two files "a.wav" and "b.wav" simultaneously (to play them
individually, omit the "-m" flag). The first one is played with a volume of 75
percent, the second with 25 percent, starting at the 4th second, with a duration
of 6 seconds, looping 5 times.

(call 'sox
   "-m"  "-v" "0.75" "a.wav"  "-v" "0.25" "b.wav"
   "-d"
   "trim" 4 6
   "repeat" 5 )

\end{verbatim}

\section*{Playing cards}

\begin{verbatim}

{{trans|Common Lisp}}

(de *Suits
   Club Diamond Heart Spade )

(de *Pips
   Ace 2 3 4 5 6 7 8 9 10 Jack Queen King )

(de mkDeck ()
   (mapcan
      '((Pip) (mapcar cons *Suits (circ Pip)))
      *Pips ) )

(de shuffle (Lst)
   (by '(NIL (rand)) sort Lst) )

\end{verbatim}

\section*{Plot coordinate pairs}

\begin{verbatim}

[[Image: Plotxy-picoLisp.png|thumb|right|200px|Example picoLisp output]]

(load "@lib/ps.l")

(scl 1)

(de plot (PsFile DX DY Lst)
   (let (SX (length Lst)  SY (apply max Lst)  N 0 Val)
      (out PsFile
         (psHead (+ DX 20) (+ DY 40))
         (font (9 . "Helvetica"))
         (if (or (=0 SX) (=0 SY))
            (window 60 12 DX DY
               (font 24 ,"Not enough Data") )
            (setq Lst  # Build coordinates
               (let X -1
                  (mapcar
                     '((Y)
                        (cons
                           (*/ (inc 'X) DX SX)
                           (- DY (*/ Y DY SY)) ) )
                     Lst ) ) )
            (color 55 95 55  # Background color
               (let (X (+ DX 40) Y (+ DY 40))
                  (poly T  0 0  X 0  X Y  0 Y  0 0) ) )
            (window 20 20 DX DY  # Plot coordinates
               (poly NIL 0 0  0 DY  (- DX 20) DY)
               (color 76 24 24
                  (poly NIL (caar Lst) (cdar Lst) (cdr Lst)) ) )
            (window 4 4 60 12 (ps (format SY *Scl)))
            (for X SX
               (window (+ 6 (*/ (dec X) DX SX)) (+ 24 DY) 30 12
                  (ps (format (dec X)) 0) ) ) )
         (page) ) ) )

(plot "plot.ps" 300 200 (2.7 2.8 31.4 38.1 58.0 76.2 100.5 130.0 149.3 180.0))
(call 'display "plot.ps")

\end{verbatim}

\section*{Pointers and references}

\begin{verbatim}

The internal PicoLisp machinery consists completely of pointers. Any data item
(except numbers) is a pointer that points to a cell, which in turn consists of
two pointers ("cons pair").

The pointers are not evident to the programmer. The development environment
presents them as high level structures (symbols or lists). However, the pointer
value (the address) can be obtained with the
'[http://software-lab.de/doc/refA.html#adr adr]' function.

"Dereferencing" a pointer is done with the
'[http://software-lab.de/doc/refC.html#car car]' or
'[http://software-lab.de/doc/refV.html#val val]' functions. They return the data
item at the memory location (CAR or VAL part of a cell). With
'[http://software-lab.de/doc/refS.html#set set]', a value can be stored in the
referred location.

There is no meaningful pointer arithmetics, except functions like
'[http://software-lab.de/doc/refC.html#cdr cdr]' or
'[http://software-lab.de/doc/refN.html#nth nth]', which advance the pointer to
the next (linked to) location(s).

: (setq L (1 a 2 b 3 c))         # Create a list of 6 items in 'L'
-> (1 a 2 b 3 c)

: (nth L 4)                      # Get a pointer to the 4th item
-> (b 3 c)

: (set (nth L 4) "Hello")        # Store "Hello" in that location
-> "Hello"

: L                              # Look at the modified list in 'L'
-> (1 a 2 "Hello" 3 c)

\end{verbatim}

\section*{Polynomial long division (!)}

\begin{verbatim}

(de degree (P)
   (let I NIL
      (for (N . C) P
         (or (=0 C) (setq I N)) )
      (dec I) ) )

(de divPoly (N D)
   (if (lt0 (degree D))
      (quit "Div/0" D)
      (let (Q NIL Diff)
         (while (ge0 (setq Diff (- (degree N) (degree D))))
            (setq Q (need (- -1 Diff) Q 0))
            (let E D
               (do Diff (push 'E 0))
               (let F (/ (get N (inc (degree N))) (get E (inc (degree E))))
                  (set (nth Q (inc Diff)) F)
                  (setq N (mapcar '((N E) (- N (* E F))) N E)) ) ) )
         (list Q N) ) ) )

Output:

: (divPoly (-42 0 -12 1) (-3 1 0 0))
-> ((-27 -9 1) (-123 0 0 0))

\end{verbatim}

\section*{Polymorphic copy}

\begin{verbatim}

Any object can be copied by transferring the value and the property list. If we
create an object 'A':

: (setq A (new '(+Cls1 +Cls2) 'attr1 123  'attr2 "def"  'attr3 (4 2 0)  'attr4 T
-> \$385603635

: (show A)
\$385603635 (+Cls1 +Cls2)
   attr4
   attr3 (4 2 0)
   attr2 "def"
   attr1 123
-> \$385603635

Then we can easily copy it to a new object 'B':

(putl (setq B (new (val A))) (getl A))

Inspecting 'B':

: (show B)
\$385346595 (+Cls1 +Cls2)
   attr1 123
   attr2 "def"
   attr3 (4 2 0)
   attr4
-> \$385346595

\end{verbatim}

\section*{Polymorphism}

\begin{verbatim}

(class +Point)
# x y

(dm T (X Y)
   (=: x (or X 0))
   (=: y (or Y 0)) )

(dm print> ()
   (prinl "Point " (: x) "," (: y)) )

(class +Circle +Point)
# r

(dm T (X Y R)
   (super X Y)
   (=: r (or R 0)) )

(dm print> ()
   (prinl "Circle " (: x) "," (: y) "," (: r)) )


(setq
   P (new '(+Point) 3 4)
   C (new '(+Circle) 10 10 5) )

(print> P)
(print> C)

Output:

Point 3,4
Circle 10,10,5

\end{verbatim}

\section*{Power set}

\begin{verbatim}

(de powerset (Lst)
   (ifn Lst
      (cons)
      (let L (powerset (cdr Lst))
         (conc
            (mapcar '((X) (cons (car Lst) X)) L)
            L ) ) ) )

\end{verbatim}

\section*{Pragmatic directives}

\begin{verbatim}

PicoLisp makes no formal difference between any normal and "specific" operation
of the language. Any possible desired effect can be achieved by calling a
function or setting a variable. For example, function calls can be traced with
the '[http://software-lab.de/doc/refT.html#trace trace]' function.

\end{verbatim}

\section*{Price Fraction}

\begin{verbatim}

(scl 2)

(de price (Pr)
   (format
      (cdr
         (rank Pr
            (quote
               (0.00 . 0.10)
               (0.06 . 0.18)
               (0.11 . 0.26)
               (0.16 . 0.32)
               (0.21 . 0.38)
               (0.26 . 0.44)
               (0.31 . 0.50)
               (0.36 . 0.54)
               (0.41 . 0.58)
               (0.46 . 0.62)
               (0.51 . 0.66)
               (0.56 . 0.70)
               (0.61 . 0.74)
               (0.66 . 0.78)
               (0.71 . 0.82)
               (0.76 . 0.86)
               (0.81 . 0.90)
               (0.86 . 0.94)
               (0.91 . 0.98)
               (0.96 . 1.00) ) ) )
      *Scl ) )

(for N (0.3793 0.4425 0.0746 0.6918 0.2993 0.5486 0.7848 0.9383 0.2292)
   (prinl (price N)) )

Output:

0.54
0.58
0.18
0.78
0.44
0.66
0.86
0.98
0.38

\end{verbatim}

\section*{Primality by trial division}

\begin{verbatim}

(de prime? (N)
   (or
      (= N 2)
      (and
         (> N 1)
         (bit? 1 N)
         (for (D 3  T  (+ D 2))
            (T (> D (sqrt N)) T)
            (T (=0 (\% N D)) NIL) ) ) ) )

\end{verbatim}

\section*{Prime decomposition}

\begin{verbatim}

The following solution generates a sequence of "trial divisors" (2 3 5 7 11 13
17 19 23 29 31 37 ..), as described by Donald E. Knuth, "The Art of Computer
Programming", Vol.2, p.365.

(de factor (N)
   (make
      (let (D 2  L (1 2 2 . (4 2 4 2 4 6 2 6 .))  M (sqrt N))
         (while (>= M D)
            (if (=0 (\% N D))
               (setq M (sqrt (setq N (/ N (link D)))))
               (inc 'D (pop 'L)) ) )
         (link N) ) ) )

(factor 1361129467683753853853498429727072845823)

Output:

-> (3 11 31 131 2731 8191 409891 7623851 145295143558111)

\end{verbatim}

\section*{Priority queue}

\begin{verbatim}

The following implementation imposes no limits. It uses a
[http://software-lab.de/doc/refI.html#idx binary tree] for storage. The priority
levels may be numeric, or of any other type.

# Insert item into priority queue
(de insertPQ (Queue Prio Item)
   (idx Queue (cons Prio Item) T) )

# Remove and return top item from priority queue
(de removePQ (Queue)
   (cdar (idx Queue (peekPQ Queue) NIL)) )

# Find top element in priority queue
(de peekPQ (Queue)
   (let V (val Queue)
      (while (cadr V)
         (setq V @) )
      (car V) ) )

# Merge second queue into first
(de mergePQ (Queue1 Queue2)
   (balance Queue1 (sort (conc (idx Queue1) (idx Queue2)))) )

Test:

# Two priority queues
(off Pq1 Pq2)

# Insert into first queue
(insertPQ 'Pq1 3 '(Clear drains))
(insertPQ 'Pq1 4 '(Feed cat))

# Insert into second queue
(insertPQ 'Pq2 5 '(Make tea))
(insertPQ 'Pq2 1 '(Solve RC tasks))
(insertPQ 'Pq2 2 '(Tax return))

# Merge second into first queue
(mergePQ 'Pq1 'Pq2)

# Remove and print all items from first queue
(while Pq1
   (println (removePQ 'Pq1)) )

Output:

(Solve RC tasks)
(Tax return)
(Clear drains)
(Feed cat)
(Make tea)

\end{verbatim}

\section*{Probabilistic choice}

\begin{verbatim}

(let (Count 1000000  Denom 27720  N Denom)
   (let Probs
      (mapcar
         '((I S)
            (prog1 (cons N (*/ Count I) 0 S)
               (dec 'N (/ Denom I)) ) )
         (range 5 12)
         '(aleph beth gimel daleth he waw zayin heth) )
      (do Count
         (inc (cddr (rank (rand 1 Denom) Probs T))) )
      (let Fmt (-6 12 12)
         (tab Fmt NIL "Probability" "Result")
         (for X Probs
            (tab Fmt
               (cdddr X)
               (format (cadr X) 6)
               (format (caddr X) 6) ) ) ) ) )

Output:

       Probability      Result
aleph     0.200000    0.199760
beth      0.166667    0.166878
gimel     0.142857    0.142977
daleth    0.125000    0.124983
he        0.111111    0.111200
waw       0.100000    0.100173
zayin     0.090909    0.090591
heth      0.083333    0.063438

\end{verbatim}

\section*{Program termination}

\begin{verbatim}

Calling 'bye', optionally with a numeric code, terminates the program.

This will execute all pending 'finally' expressions, close all open files and/
or pipes, flush standard output, and execute all expressions in the global
variable '*Bye' before exiting.

(push '*Bye '(prinl "Goodbye world!"))
(bye)

Output:

Goodbye world!
\$

\end{verbatim}

\section*{Pythagorean triples}

\begin{verbatim}

{{trans|C}}

(for (Max 10  (>= 100000000 Max)  (* Max 10))
   (let (Total 0  Prim 0  In (3 4 5))
      (recur (In)
         (let P (apply + In)
            (when (>= Max P)
               (inc 'Prim)
               (inc 'Total (/ Max P))
               (for Row
                  (quote
                     (( 1 -2 2) ( 2 -1 2) ( 2 -2 3))
                     (( 1  2 2) ( 2  1 2) ( 2  2 3))
                     ((-1  2 2) (-2  1 2) (-2  2 3)) )
                  (recurse
                     (mapcar '((U) (sum * U In)) Row) ) ) ) ) )
      (prinl "Up to " Max ": " Total " triples, " Prim " primitives.") ) )

Output:

Up to 10: 0 triples, 0 primitives.
Up to 100: 17 triples, 7 primitives.
Up to 1000: 325 triples, 70 primitives.
Up to 10000: 4858 triples, 703 primitives.
Up to 100000: 64741 triples, 7026 primitives.
Up to 1000000: 808950 triples, 70229 primitives.
Up to 10000000: 9706567 triples, 702309 primitives.
Up to 100000000: 113236940 triples, 7023027 primitives.

\end{verbatim}

\chapter{Rosetta Code Tasks starting with Q}

\section*{Queue/Definition}

\begin{verbatim}

The built-in function 'fifo' maintains a queue in a circular list, with direct
access to the first and the last cell

(off Queue)                # Clear Queue
(fifo 'Queue 1)            # Store number '1'
(fifo 'Queue 'abc)         # an internal symbol 'abc'
(fifo 'Queue "abc")        # a transient symbol "abc"
(fifo 'Queue '(a b c))     # and a list (a b c)
Queue                      # Show the queue

Output:

->((a b c) 1 abc "abc" .)

\end{verbatim}

\section*{Queue/Usage}

\begin{verbatim}

Using the implementation from [[FIFO]]:
(println (fifo 'Queue))    # Retrieve the number '1'
(println (fifo 'Queue))    # Retrieve an internal symbol 'abc'
(println (fifo 'Queue))    # Retrieve a transient symbol "abc"
(println (fifo 'Queue))    # and a list (abc)
(println (fifo 'Queue))    # Queue is empty -> NIL

Output:

1
abc
"abc"
(a b c)
NIL

\end{verbatim}

\section*{Quine}

\begin{verbatim}

Using 'quote' (= 'lambda' in PicoLisp)

('((X) (list (lit X) (lit X))) '((X) (list (lit X) (lit X))))

Output:

-> ('((X) (list (lit X) (lit X))) '((X) (list (lit X) (lit X))))

Using 'let'

(let X '(list 'let 'X (lit X) X) (list 'let 'X (lit X) X))

Output:

-> (let X '(list 'let 'X (lit X) X) (list 'let 'X (lit X) X))

\end{verbatim}

\chapter{Rosetta Code Tasks starting with R}

\section*{RSA code}

\begin{verbatim}

PicoLisp comes with an RSA library. Usage:

(load "@lib/rsa.l")

# Generate 100-digit keys (private . public)
: (setq Keys (rsaKey 100))
-> (14394597526321726957429995133376978449624406217727317004742182671030....

# Encrypt
: (setq CryptText
   (encrypt (car Keys)
      (chop "The quick brown fox jumped over the lazy dog's back") ) )
-> (72521958974980041245760752728037044798830723189142175108602418861716...

# Decrypt
: (pack (decrypt Keys CryptText))
-> "The quick brown fox jumped over the lazy dog's back"

\end{verbatim}

\section*{Random number generator (device)}

\begin{verbatim}

: (in "/dev/urandom" (rd 4))
-> 2917110327

\end{verbatim}

\section*{Random number generator (included)}

\begin{verbatim}

PicoLisp uses a linear congruential generator in the built-in (rand) function,
with a multiplier suggested in Knuth's "Seminumerical Algorithms". See the
[http://software-lab.de/doc/refR.html#rand documentation].

\end{verbatim}

\section*{Random numbers}

\begin{verbatim}

{{trans|C}}

(load "@lib/math.l")

(de randomNormal ()  # Normal distribution, centered on 0, std dev 1
   (*/
      (sqrt (* -2.0 (log (rand 0 1.0))))
      (cos (*/ 2.0 pi (rand 0 1.0) `(* 1.0 1.0)))
      1.0 ) )

(seed (time))                                      # Randomize

(let Result
   (make                                           # Build list
      (do 1000                                     # of 1000 elements
         (link (+ 1.0 (/ (randomNormal) 2))) ) )
   (for N (head 7 Result)                          # Print first 7 results
      (prin (format N *Scl) " ") ) )

Output:

1.500334 1.212931 1.095283 0.433122 0.459116 1.302446 0.402477

\end{verbatim}

\section*{Range expansion}

\begin{verbatim}

(de rangeexpand (Str)
   (make
      (for S (split (chop Str) ",")
         (if (index "-" (cdr S))
            (chain
               (range
                  (format (head @ S))
                  (format (tail (- -1 @) S)) ) )
            (link (format S)) ) ) ) )

Output:

: (rangeexpand "-6,-3--1,3-5,7-11,14,15,17-20")
-> (-6 -3 -2 -1 3 4 5 7 8 9 10 11 14 15 17 18 19 20)

\end{verbatim}

\section*{Range extraction}

\begin{verbatim}

(de rangeextract (Lst)
   (glue ","
      (make
         (while Lst
            (let (N (pop 'Lst)  M N)
               (while (= (inc M) (car Lst))
                  (setq M (pop 'Lst)) )
               (cond
                  ((= N M) (link N))
                  ((= (inc N) M) (link N M))
                  (T (link (list N '- M))) ) ) ) ) ) )

Output:

: (rangeextract
   (0 1 2 4 6 7 8 11 12 14 15 16 17 18 19 20 21 22
      23 24 25 27 28 29 30 31 32 33 35 36 37 38 39 ) )

-> "0-2,4,6-8,11,12,14-25,27-33,35-39"

\end{verbatim}

\section*{Rate counter}

\begin{verbatim}

[http://software-lab.de/doc/refU.html#usec usec] returns a relative time in
microseconds. This can be used, for example, to measure the time between two key
strokes

(prin "Hit a key ... ")
(key)
(prinl)
(let Usec (usec)
   (prin "Hit another key ... ")
   (key)
   (prinl)
   (prinl "This took " (format (- (usec) Usec) 6) " seconds") )

Output:

Hit a key ...
Hit another key ...
This took 3.132058 seconds

The [http://software-lab.de/doc/refB.html#bench bench] benchmark function could
also be used. Here we measure the time until a key is pressed

(bench (key))
1.761 sec
-> "a"

\end{verbatim}

\section*{Ray-casting algorithm}

\begin{verbatim}

(scl 4)

(de intersects (Px Py Ax Ay Bx By)
   (when (> Ay By)
      (xchg 'Ax 'Bx)
      (xchg 'Ay 'By) )
   (when (or (= Py Ay) (= Py By))
      (inc 'Py) )
   (and
      (>= Py Ay)
      (>= By Py)
      (>= (max Ax Bx) Px)
      (or
         (> (min Ax Bx) Px)
         (= Ax Px)
         (and
            (<> Ax Bx)
            (>=
               (*/ (- Py Ay) 1.0 (- Px Ax))            # Blue
               (*/ (- By Ay) 1.0 (- Bx Ax)) ) ) ) ) )  # Red

(de inside (Pt Poly)
   (let Res NIL
      (for Edge Poly
         (when (apply intersects Edge (car Pt) (cdr Pt))
            (onOff Res) ) )
      Res ) )

Test data:

(de Square
   ( 0.0  0.0  10.0  0.0)
   (10.0  0.0  10.0 10.0)
   (10.0 10.0   0.0 10.0)
   ( 0.0 10.0   0.0  0.0) )

(de SquareHole
   ( 0.0  0.0  10.0  0.0)
   (10.0  0.0  10.0 10.0)
   (10.0 10.0   0.0 10.0)
   ( 0.0 10.0   0.0  0.0)
   ( 2.5  2.5   7.5  2.5)
   ( 7.5  2.5   7.5  7.5)
   ( 7.5  7.5   2.5  7.5)
   ( 2.5  7.5   2.5  2.5) )

(de Strange
   ( 0.0  0.0   2.5  2.5)
   ( 2.5  2.5   0.0 10.0)
   ( 0.0 10.0   2.5  7.5)
   ( 2.5  7.5   7.5  7.5)
   ( 7.5  7.5  10.0 10.0)
   (10.0 10.0  10.0  0.0)
   (10.0  0.0   2.5  2.5) )

(de Exagon
   ( 3.0  0.0   7.0  0.0)
   ( 7.0  0.0  10.0  5.0)
   (10.0  5.0   7.0 10.0)
   ( 7.0 10.0   3.0 10.0)
   ( 3.0 10.0   0.0  5.0)
   ( 0.0  5.0   3.0  0.0) )

Output:

: (inside (5.0 . 5.0) Square)
-> T
: (inside (5.0 . 8.0) Square)
-> T
: (inside (-10.0 . 5.0) Square)
-> NIL
: (inside (0.0 . 5.0) Square)
-> NIL
: (inside (10.0 . 5.0) Square)
-> T
: (inside (8.0 . 5.0) Square)
-> T
: (inside (10.0 . 10.0) Square)
-> NIL

: (inside (5.0 . 5.0) SquareHole)
-> NIL
: (inside (5.0 . 8.0) SquareHole)
-> T
: (inside (-10.0 . 5.0) SquareHole)
-> NIL
: (inside (0 . 5.0) SquareHole)
-> NIL
: (inside (10.0 . 5.0) SquareHole)
-> T
: (inside (8.0 . 5.0) SquareHole)
-> T
: (inside (10.0 . 10.0) SquareHole)
-> NIL

: (inside (5.0 . 5.0) Strange)
-> T
: (inside (5.0 . 8.0) Strange)
-> NIL
: (inside (-10.0 . 5.0) Strange)
-> NIL
: (inside (0 . 5.0) Strange)
-> NIL
: (inside (10.0 . 5.0) Strange)
-> T
: (inside (8.0 . 5.0) Strange)
-> T
: (inside (10.0 . 10.0) Strange)
-> NIL

: (inside (5.0 . 5.0) Exagon)
-> T
: (inside (5.0 . 8.0) Exagon)
-> T
: (inside (-10.0 . 5.0) Exagon)
-> NIL
: (inside (0.0 . 5.0) Exagon)
-> NIL
: (inside (10.0 . 5.0) Exagon)
-> T
: (inside (8.0 . 5.0) Exagon)
-> T
: (inside (10.0 . 10.0) Exagon)
-> NIL

\end{verbatim}

\section*{Read a configuration file}

\begin{verbatim}

'read' supports only a single comment character. Therefore, we use a pipe to
filter the comments.

(de rdConf (File)
   (pipe (in File (while (echo "#" ";") (till "^J")))
      (while (read)
         (set @ (or (line T) T)) ) ) )

Test:

(off FULLNAME FAVOURITEFRUIT NEEDSPEELING SEEDSREMOVED OTHERFAMILY)
(rdConf "conf.txt")

Output:

: (list FULLNAME FAVOURITEFRUIT NEEDSPEELING SEEDSREMOVED OTHERFAMILY)
-> ("Foo Barber" "banana" T NIL "Rhu Barber, Harry Barber")

\end{verbatim}

\section*{Read a specific line from a file}

\begin{verbatim}

(in "file.txt"
   (do 6 (line))
   (or (line) (quit "No 7 lines")) )

\end{verbatim}

\section*{Read entire file}

\begin{verbatim}

Using '[http://software-lab.de/doc/refT.html#till till]' is the shortest way:

   (in "file" (till NIL T))

To read the file into a list of characters:

   (in "file" (till NIL))

or, more explicit:

   (in "file" (make (while (char) (link @))))

Encoding is always assumed to be UTF-8.

\end{verbatim}

\section*{Read a file line by line}

\begin{verbatim}

(in "foobar.txt"
   (while (line)
      (process @) ) )

\end{verbatim}

\section*{Real constants and functions}

\begin{verbatim}

PicoLisp has only limited floating point support (scaled bignum arithmetics). It
can handle real numbers with as many positions after the decimal point as
desired, but is practically limited by the precision of the C-library functions
(about 16 digits). The default precision is six, and can be changed with
'[http://software-lab.de/doc/refS.html#scl scl]':

(scl 12)  # 12 places after decimal point
(load "@lib/math.l")

(prinl (format (exp 1.0) *Scl))        # e, exp
(prinl (format pi *Scl))               # pi

(prinl (format (pow 2.0 0.5) *Scl))    # sqare root
(prinl (format (sqrt (* 2.0 1.0)) *Scl))

(prinl (format (log 2.0) *Scl))        # logarithm
(prinl (format (exp 4.0) *Scl))        # exponential

(prinl (format (abs -7.2) *Scl))       # absolute value
(prinl (abs -123))

(prinl (format (pow 3.0 4.0) *Scl))    # power

Output:

2.718281828459
3.141592653590
1.414213562373
1.414213562373
0.693147180560
54.598150033144
7.200000000000
123
81.000000000000

"floor" and "ceiling" are currently not available.

\end{verbatim}

\section*{Record sound (!)}

\begin{verbatim}

(in '(rec -q -c1 -tu16 - trim 0 2)  # Record 2 seconds
   (make
      (while (rd 2)
         (link @) ) ) )

Output:

-> (16767 19071 17279 ... 5503 9343 14719)  # 96000 numbers

\end{verbatim}

\section*{Reduced row echelon form}

\begin{verbatim}

(de reducedRowEchelonForm (Mat)
   (let (Lead 1  Cols (length (car Mat)))
      (for (X Mat X (cdr X))
         (NIL
            (loop
               (T (seek '((R) (n0 (get R 1 Lead))) X)
                  @ )
               (T (> (inc 'Lead) Cols)) ) )
         (xchg @ X)
         (let D (get X 1 Lead)
            (map
               '((R) (set R (/ (car R) D)))
               (car X) ) )
         (for Y Mat
            (unless (== Y (car X))
               (let N (- (get Y Lead))
                  (map
                     '((Dst Src)
                        (inc Dst (* N (car Src))) )
                     Y
                     (car X) ) ) ) )
         (T (> (inc 'Lead) Cols)) ) )
   Mat )

Output:

(reducedRowEchelonForm
   '(( 1  2  -1   -4) ( 2  3  -1  -11) (-2  0  -3   22)) )
-> ((1 0 0 -8) (0 1 0 1) (0 0 1 -2))

\end{verbatim}

\section*{Regular expressions}

\begin{verbatim}

1. Calling the C library

PicoLisp doesn't have built-in regex functionality.
It is easy to call the native C library.

(let (Pat "a[0-9]z"  String "a7z")
   (use Preg
      (native "@" "regcomp" 'I '(Preg (64 B . 64)) Pat 1)  # Compile regex
      (when (=0 (native "@" "regexec" 'I (cons NIL (64) Preg) String 0 0 0))
         (prinl "String \"" String "\" matches regex \"" Pat "\"") ) ) )

Output:

String "a7z" matches pattern "a[0-9]z"

2. Using Pattern Matching

Regular expressions are static and inflexible. Another possibility is
dynamic pattern matching, where arbitrary conditions can be programmed.

(let String "The number <7> is incremented"
   (use (@A @N @Z)
      (and
         (match '(@A "<" @N ">"  @Z) (chop String))
         (format @N)
         (prinl @A "<" (inc @) ">" @Z) ) ) )

Output:

The number <8> is incremented

\end{verbatim}

\section*{Remote agent/Agent interface}

\begin{verbatim}

The interface logic for the PicoLisp solution is directly integrated into
the client [[Remote agent/Agent logic#PicoLisp]].

\end{verbatim}

\section*{Remote agent/Agent logic}

\begin{verbatim}

This is the client. For the server, see [[Remote agent/Simulation#PicoLisp]].

# Global variables:
#  '*Sock' is the TCP socket to the server
#  '*Dir' is a circular list of direction structures
#  '*World' holds the explored world
#  '*Ball' is the ball found in current field
#  '*Todo' is the list of mismatching fields and balls

(load "@lib/simul.l")

(de *Dir .
   ((north south . extendNorth) (east west . extendEast)
      (south north . extendSouth) (west east . extendWest) . ) )

(de gameClient (Host Port)
   (unless (setq *Sock (connect Host Port))
      (quit "Can't connect to " (cons Host Port)) )
   (in *Sock
      (when (= "A" (char (rd 1)))  # Greeting
         (out *Sock (prin "A"))
         (with (def (box) (cons (cons) (cons)))
            # Explore the world
            (setq *World (cons (cons This)))
            (off *Ball *Todo)
            (let (Turns 4  Color T)  # Initially 4 turns, unknown color
               (recur (This Turns Color)
                  (setThis Color)
                  (turnLeft)
                  (do Turns
                     (ifn (and (not (get This (caar *Dir))) (goForward))
                        (turnRight)
                        (let Next @
                           (unless ((caar *Dir) This)
                              ((cddar *Dir)) )  # Extend world
                           (put This (caar *Dir) ((caar *Dir) This))
                           (put ((caar *Dir) This) (cadar *Dir) This)
                           (if (get ((caar *Dir) This) 'field)
                              (do 2 (turnRight))
                              (recurse ((caar *Dir) This) 3 Next) )
                           (setThis (goForward)) )  # Final color on return
                        (turnLeft) ) ) ) )
            # Establish the walls
            (for Col *World
               (for This Col
                  (set This
                     (cons
                        (cons (: west) (: east))
                        (cons (: south) (: north)) ) ) ) )
            (prinl "Initial state:")
            (showWorld)
            (prin "Moving balls ... ")
            # Move balls to proper fields
            (for X *Todo
               (findField                    # Move to next field
                  (== This (car X)) )
               (getBall)                     # Pick the ball
               (findField                    # Find a suitable field
                  (unless (: ball)
                     (= (: field) (cdr X)) ) )
               (prin (cdr X))
               (flush)
               (dropBall (cdr X)) )          # Drop the ball
            (prinl "Final state:")
            (showWorld) ) ) ) )

# Set color and ball in field
(de setThis (Color)
   (=: field Color)
   (=: ball *Ball)
   (and
      *Ball
      (<> @ Color)
      (push1 '*Todo (cons This *Ball)) ) )

# Commands to server
(de goForward ()
   (out *Sock (prin "\^"))
   (in *Sock
      (let F (char (rd 1))
         (cond
            ((= "|" F) (off *Ball F) (rd 1))
            ((= "." (setq *Ball (uppc (char (rd 1)))))
               (off *Ball) )
            (T (rd 1)) )
         F ) ) )

(de turnRight ()
   (out *Sock (prin ">"))
   (pop '*Dir)
   (rd 1) )

(de turnLeft ()
   (out *Sock (prin "<"))
   (do 3 (pop '*Dir))
   (rd 1) )

(de getBall ()
   (out *Sock (prin "@"))
   (case (char (rd 1))
      ("s" (quit "No ball in sector"))
      ("A" (quit "Agent full"))
      ("." (=: ball NIL))
      (T (quit "Unexpected event" @)) ) )

(de dropBall (Ball)
   (out *Sock (prin "!"))
   (case (char (rd 1))
      ("a" (quit "No ball in agent"))
      ("S" (quit "Sector full"))
      ("." (=: ball Ball))
      ("+" (rd 1) (prinl " ... Game over!"))
      (T (quit "Unexpected event" @)) ) )

# Extend world to the north
(de extendNorth ()
   (let Last NIL
      (for Col *World
         (let (Old (last Col)  New (def (box) (cons (cons Last) (cons Old))))
            (conc Col (cons New))
            (and Last (con (car (val @)) New))
            (setq Last (con (cdr (val Old)) New)) ) ) ) )

# Extend world to the east
(de extendEast ()
   (conc *World
      (cons
         (let Last NIL
            (mapcar
               '((Old)
                  (let New (def (box) (cons (cons Old) (cons Last)))
                     (and Last (con (cdr (val @)) New))
                     (setq Last (con (car (val Old)) New)) ) )
               (last *World) ) ) ) ) )

# Extend world to the south
(de extendSouth ()
   (let Last NIL
      (map
         '((Lst)
            (push Lst
               (let
                  (Old (caar Lst)
                     New (def (box) (cons (cons Last) (cons NIL Old))) )
                  (and Last (con (car (val @)) New))
                  (setq Last (set (cdr (val Old)) New)) ) ) )
         *World ) ) )

# Extend world to the west
(de extendWest ()
   (push '*World
      (let Last NIL
         (mapcar
            '((Old)
               (let New (def (box) (cons (cons NIL Old) (cons Last)))
                  (and Last (con (cdr (val @)) New))
                  (setq Last (set (car (val Old)) New)) ) )
            (car *World) ) ) ) )

# Find matching field
(de findField Prg
   (setq This
      (catch NIL
         (recur (This)
            (unless (: mark)
               (and (run Prg) (throw NIL This))
               (finally (=: mark NIL)
                  (=: mark T)
                  (do 4
                     (when ((caar *Dir) This)
                        (goForward)
                        (recurse ((caar *Dir) This))
                        (do 2 (turnRight))
                        (goForward)
                        (do 2 (turnRight)) )
                     (turnRight) ) ) ) )
         (quit "Can't find field") ) ) )

# Visualize (debug)
(de showWorld ()
   (disp *World 0
      '((This)
         (pack " "
            (: field)
            (if (: ball) (lowc @) " ") ) ) ) )

Output:

: (gameClient "picolisp.com" 54545)
Initial state:
   +---+---+---+---+---+---+---+---+
 8 | G   G   Y   Yr| Y   Yb  G   R |
   +   +---+---+   +   +---+---+   +
 7 | Y | Y | B   Gy  Bg  Y   B | Gg|
   +---+   +   +---+   +   +   +   +
 6 | Gb| Gy  G   R   B   Y | B   Bg|
   +   +---+   +   +---+---+---+   +
 5 | R | B   G | B | R | B   R   Yg|
   +   +---+   +   +   +   +   +   +
 4 | B   B | G | Y   B   Bg| Bg  R |
   +---+   +   +---+   +   +   +   +
 3 | G | Y   Gr  R | B   B   Br  B |
   +   +   +---+---+---+   +   +---+
 2 | G   Rr  B | Gy  Y | Bg| Bb  B |
   +---+   +---+   +   +   +   +   +
 1 | R   R   Gb| Bg| G   G   R | Yg|
   +---+---+---+---+---+---+---+---+
     a   b   c   d   e   f   g   h
Moving balls ... GBGRYYBBRGGGYGRGG ... Game over!
Final state:
   +---+---+---+---+---+---+---+---+
 8 | G   Gg  Y   Y | Y   Y   Gg  R |
   +   +---+---+   +   +---+---+   +
 7 | Y | Yy| B   Gg  B   Yy  B | Gg|
   +---+   +   +---+   +   +   +   +
 6 | G | Gg  Gg  R   Bb  Y | B   B |
   +   +---+   +   +---+---+---+   +
 5 | Rr| B   G | B | Rr| B   R   Y |
   +   +---+   +   +   +   +   +   +
 4 | Bb  Bb| G | Y   B   B | B   R |
   +---+   +   +---+   +   +   +   +
 3 | G | Y   G   Rr| B   B   B   B |
   +   +   +---+---+---+   +   +---+
 2 | G   Rr  B | G   Yy| B | Bb  B |
   +---+   +---+   +   +   +   +   +
 1 | R   R   G | B | Gg  Gg  R | Y |
   +---+---+---+---+---+---+---+---+
     a   b   c   d   e   f   g   h

\end{verbatim}

\section*{Remote agent/Simulation}

\begin{verbatim}

This is the server. For the client, see [[Remote agent/Agent logic#PicoLisp]].

# Global variables:
#  '*Port' is the port where the server is listening
#  '*Sock' is the TCP socket after a client connected
#  '*World' holds the current world
#  '*Agent' is the field where the agent is in
#  '*Ball' is the ball the agent is holding
#  '*Dir' is a circular list of directions (north east south west .)

(load "@lib/simul.l")

# The server port
(setq *Port (port 54545))

# Return a random Field
(de randomField ()
   (get *World (rand 1 DX) (rand 1 DY)) )

# Create a world of size 'DX' * 'DY' with 'Balls' and 'Walls'
(de makeWorld (DX DY Balls Walls)
   (when (>= Balls (* DX DY))
      (quit "Too many balls") )
   (when (>= Walls (* (dec DX) (dec DY)))
      (quit "Too many walls") )
   (for Column (setq *World (grid DX DY))          # Initialize fields
      (for This Column
         (let Color (get '(R G Y B) (rand 1 4))
            (=: field Color)                       # Set field color
            (when (ge0 (dec 'Balls))
               (until
                  (with (randomField DX DY)        # Find a field without ball
                     (unless (: ball)              # and set a ball
                        (=: ball Color) ) ) ) ) ) ) )
   (do Walls                              # Create walls
      (until
         (let
            (Field (randomField DX DY)    # Try random field
               F (if (rand T) car cdr)    # and random side
               G (if (rand T) '(car set . con) '(cdr con . set))
               Old ((car G) (F (val Field))) )
            (when Old
               ((cadr G) (F (val Field)) NIL)  # Remove connections to neighbor
               ((cddr G) (F (val Old)) NIL)
               (or
                  (reachable? Field (* DX DY))  # Field still reachable?
                  (nil                          # No: Restore connections
                     ((cadr G) (F (val Field)) Old)
                     ((cddr G) (F (val Old)) Field) ) ) ) ) ) ) )

# Test whether a field is reachable
(de reachable? (Field Fields)
   (let Visited NIL
      (recur (Field)
         (when (and Field (not (memq Field Visited)))
            (push 'Visited Field)
            (recurse (west Field))
            (recurse (east Field))
            (recurse (south Field))
            (recurse (north Field)) ) )
      (= Fields (length Visited)) ) )

# Test for ending condition
(de ending? ()
   (nor
      *Ball
      (find
         '((Column)
            (find
               '((This)
                  (and (: ball) (n== (: field) (: ball))) )
               Column ) )
         *World ) ) )

# Initialize for a new game
(de newGame (DX DY Balls Walls)
   (makeWorld DX DY Balls Walls)
   (setq
      *Agent (randomField DX DY)
      *Dir (do (rand 1 4) (rot '(north east south west .))) ) )

# Start the game server
(de gameServer (DX DY Balls Walls)
   (loop
      (setq *Sock (listen *Port))
      (NIL (fork) (close *Port))
      (close *Sock) )
   (seed *Pid)  # Ensure private random sequence
   (in *Sock
      (out *Sock (prin "A"))  # Greeting
      (when (= "A" (char (rd 1)))
         (newGame DX DY Balls Walls)
         (and *Dbg (showWorld))
         (while (rd 1)
            (out *Sock
               (case (char @)  # Command character
                  ("\^"  # Forward
                     (ifn ((car *Dir) *Agent)  # Hit wall?
                        (prin "|")             # Yes: Bump event
                        (with (setq *Agent @)  # Else go to new position
                           (prin (: field))
                           (and (: ball) (prin (lowc @))) ) ) )
                  (">"  # Turn right
                     (pop '*Dir) )
                  ("<"  # Turn left
                     (do 3 (pop '*Dir)) )
                  ("@"  # Get ball
                     (with *Agent
                        (cond
                           ((not (: ball)) (prin "s"))  # No ball in sector
                           (*Ball (prin "A"))           # Agent full
                           (T
                              (setq *Ball (: ball))
                              (=: ball) ) ) ) )
                  ("!"  # Drop ball
                     (with *Agent
                        (cond
                           ((not *Ball) (prin "a"))  # No ball in agent
                           ((: ball) (prin "S"))     # Sector full
                           (T (=: ball *Ball)
                              (off *Ball)
                              (and (ending?) (prin "+")) ) ) ) ) )  # Game over
               (prin ".") ) ) ) )  # Stop event
   (bye) )

# Visualize (debug)
(de showWorld ()
   (disp *World 0
      '((This)
         (pack
            (if (== *Agent This) "*" " ")
            (: field)
            (if (: ball) (lowc @) " ") ) ) ) )

An online demo version of this server runs on port 54545 of "picolisp.com". It
can be used for testing.

For local tests, you can start also it interactively:

: (newGame 8 8 20 40) (showWorld)
   +---+---+---+---+---+---+---+---+
 8 | R   Y | B | R   R   Br| Rb  Br|
   +   +   +   +   +   +---+---+   +
 7 | Yy  G   G   Gb| Y   Gg  Rr| Y |
   +---+   +   +   +---+   +---+   +
 6 | R   Y   B   Rr *G   Y | Y   Br|
   +---+---+   +   +---+---+   +---+
 5 | B   Ry  G   R | Yy  Yy  Y | B |
   +   +---+---+   +---+   +---+   +
 4 | R | R   R   Gg  B   G   B   Y |
   +   +---+---+   +---+---+   +   +
 3 | R   Rr| Y   B   G | Yr  B | R |
   +   +   +---+---+---+   +   +---+
 2 | Y | B | B   Bb  Gr  B   B   Yy|
   +   +   +   +   +---+   +---+   +
 1 | Rr| R   G   Gr  R   G   R | G |
   +---+---+---+---+---+---+---+---+
     a   b   c   d   e   f   g   h

This displays the field colors in upper case letters, the balls in lower case
letters, and the position of the agent with an asterisk.

\end{verbatim}

\section*{Remove duplicate elements}

\begin{verbatim}

There is a built-in function

(uniq (2 4 6 1 2 3 4 5 6 1 3 5))

Output:

-> (2 4 6 1 3 5)

\end{verbatim}

\section*{Remove lines from a file}

\begin{verbatim}

(de deleteLines (File Start Cnt)
   (let L (in File (make (until (eof) (link (line)))))
      (if (> (+ (dec 'Start) Cnt) (length L))
         (quit "Not enough lines")
         (out File
            (mapc prinl (cut Start 'L))
            (mapc prinl (nth L (inc Cnt))) ) ) ) )

\end{verbatim}

\section*{Remove the first and last characters from a string/Top and tail}

\begin{verbatim}

: (pack (cdr (chop "knight")))         # Remove first character
-> "night"

: (pack (head -1 (chop "socks")))      # Remove last character
-> "sock"

: (pack (cddr (rot (chop "brooms"))))  # Remove first and last characters
-> "room"

\end{verbatim}

\section*{Rename a file}

\begin{verbatim}

(call 'mv "input.txt" "output.txt")
(call 'mv "docs" "mydocs")
(call 'mv "/input.txt" "/output.txt")
(call 'mv "/docs" "/mydocs")

\end{verbatim}

\section*{Rendezvous}

\begin{verbatim}

Rendezvous can be implemented in PicoLisp via the following function:

(de rendezvous (Pid . Exe)
   (when
      (catch '(NIL)
         (tell Pid 'setq 'Rendezvous (lit (eval Exe)))
         NIL )
      (tell Pid 'quit @) ) )  # Raise caught error in caller

The caller invokes it in the callee via the
'[http://software-lab.de/doc/refT.html#tell tell]' interprocess communication,
and it uses 'tell' in turn to communicate results (and possible errors) back to
the caller.

Use case task:

(de printLine (Str)
   (cond
      ((gt0 *Ink) (prinl *ID ": " Str) (dec '*Ink))
      (*Backup (rendezvousPrint @ Str) T)
      (T (quit "Out of Ink")) ) )

(de rendezvousPrint (Printer Str)
   (let Rendezvous NIL
      (tell Printer 'rendezvous *Pid 'printLine Str)  # Call entry point
      (unless (wait 6000 Rendezvous)                  # Block max. 1 minute
         (quit "Rendezvous timed out") ) ) )

# Start RESERVE printer process
(unless (setq *ReservePrinter (fork))
   (setq *ID 2  *Ink 5)
   (wait) )  # Run forever

# Start MAIN printer process
(unless (setq *MainPrinter (fork))
   (setq *ID 1  *Ink 5  *Backup *ReservePrinter)
   (wait) )

# Start Humpty Dumpty process
(unless (fork)
   (when
      (catch '(NIL)
         (for Line
            (quote
               "Humpty Dumpty sat on a wall."
               "Humpty Dumpty had a great fall."
               "All the king's horses and all the king's men"
               "Couldn't put Humpty together again." )
            (rendezvousPrint *MainPrinter Line) ) )
      (prinl "      Humpty Dumpty: " @ "!") )
   (bye) )

# Start Mother Goose process
(unless (fork)
   (when
      (catch '(NIL)
         (for Line
            (quote
               "Old Mother Goose"
               "When she wanted to wander,"
               "Would ride through the air"
               "On a very fine gander."
               "Jack's mother came in,"
               "And caught the goose soon,"
               "And mounting its back,"
               "Flew up to the moon." )
            (rendezvousPrint *MainPrinter Line) ) )
      (prinl "      Mother Goose: " @ "!") )
   (bye) )

# Prepare to terminate all processes upon exit
(push '*Bye '(tell 'bye))

Output:

1: Old Mother Goose
1: Humpty Dumpty sat on a wall.
1: When she wanted to wander,
1: Humpty Dumpty had a great fall.
1: Would ride through the air
2: All the king's horses and all the king's men
2: On a very fine gander.
2: Jack's mother came in,
2: And caught the goose soon,
2: And mounting its back,
      Humpty Dumpty: Out of Ink!

\end{verbatim}

\section*{Repeat a string}

\begin{verbatim}

(pack (need 5 "ha"))
-> "hahahahaha"

or:

(pack (make (do 5 (link "ha"))))
-> "hahahahaha"

\end{verbatim}

\section*{Respond to an unknown method call}

\begin{verbatim}

The function '[http://software-lab.de/doc/refT.html#try try]' is used to send a
message to an object for which it is not known whether it inherits a method for
that message or not. As opposed to the syntacically equivalent
'[http://software-lab.de/doc/refS.html#send send]' function, 'try' does not give
an error, but returns NIL. We might redefine 'send' to get an effect analog to
CLOS.

(redef send (Msg Obj . @)
   (or
      (pass try Msg Obj)
      (pass 'no-applicable-method> Obj Msg) ) )

(de no-applicable-method> (This Msg)
   (pack "No method for " Msg " on " This) )

(class +A)

(dm do-something> ()
   (pack "Do something to " This) )

Test:

: (object 'A '(+A))
-> A
: (object 'B '(+B))
-> B
: (list (send 'do-something> 'A) (send 'do-something> 'B))
-> ("Do something to A" "No method for do-something> on B")

\end{verbatim}

\section*{Return multiple values}

\begin{verbatim}

A PicoLisp function returns a single value. For multiple return values, a cons
pair or a list may be used.

(de addsub (X Y)
   (list (+ X Y) (- X Y)) )

Test:

: (addsub 4 2)
-> (6 2)
: (addsub 3 1)
-> (4 2)
: (+ (car (addsub 4 2)) (car (addsub 3 1)))
-> 10
: (sum + (addsub 4 2) (addsub 3 1))
-> 14

\end{verbatim}

\section*{Reverse a string}

\begin{verbatim}

(pack (flip (chop "äöüÄÖÜß")))

Output:

-> "ßÜÖÄüöä"

\end{verbatim}

\section*{Rock-paper-scissors}

\begin{verbatim}

(use (C Mine Your)
   (let (Rock 0  Paper 0  Scissors 0)
      (loop
         (setq Mine
            (let N (if (gt0 (+ Rock Paper Scissors)) (rand 1 @) 0)
               (seek
                  '((L) (le0 (dec 'N (caar L))))
                  '(Rock Paper Scissors .) ) ) )
         (prin "Enter R, P or S to play, or Q to quit: ")
         (loop
            (and (= "Q" (prinl (setq C (uppc (key))))) (bye))
            (T (setq Your (find '((S) (pre? C S)) '(Rock Paper Scissors))))
            (prinl "Bad input - try again") )
         (prinl
            "I say " (cadr Mine) ", You say " Your ": "
            (cond
               ((== Your (cadr Mine)) "Draw")
               ((== Your (car Mine)) "I win")
               (T "You win") ) )
         (inc Your) ) ) )

\end{verbatim}

\section*{Roman numerals}

\begin{verbatim}

(de roman (N)
   (pack
      (make
         (mapc
            '((C D)
               (while (>= N D)
                  (dec 'N D)
                  (link C) ) )
            '(M CM D CD C XC L XL X IX V IV I)
            (1000 900 500 400 100 90 50 40 10 9 5 4 1) ) ) ) )

Output:

: (roman 1009)
-> "MIX"

: (roman 1666)
-> "MDCLXVI"

\end{verbatim}

\section*{Roman numerals/Decode}

\begin{verbatim}

(de roman2decimal (Rom)
   (let L (replace (chop Rom) 'M 1000 'D 500 'C 100 'L 50 'X 10 'V 5 'I 1)
      (sum '((A B) (if (>= A B) A (- A))) L (cdr L)) ) )

Test:

: (roman2decimal "MCMXC")
-> 1990

: (roman2decimal "MMVIII")
-> 2008

: (roman2decimal "MDCLXVI")
-> 1666

\end{verbatim}

\section*{Roots of a function}

\begin{verbatim}

{{trans|Clojure}}

(de findRoots (F Start Stop Step Eps)
   (filter
      '((N) (> Eps (abs (F N))))
      (range Start Stop Step) ) )

(scl 12)

(mapcar round
   (findRoots
      '((X) (+ (*/ X X X `(* 1.0 1.0)) (*/ -3 X X 1.0) (* 2 X)))
      -1.0 3.0 0.0001 0.00000001 ) )

Output:

-> ("0.000" "1.000" "2.000")

\end{verbatim}

\section*{Roots of a quadratic function}

\begin{verbatim}

(scl 40)

(de solveQuad (A B C)
   (let SD (sqrt (- (* B B) (* 4 A C)))
      (if (lt0 B)
         (list
            (*/ (- SD B) A 2.0)
            (*/ C 2.0 (*/ A A (- SD B) `(* 1.0 1.0))) )
         (list
            (*/ C 2.0 (*/ A A (- 0 B SD) `(* 1.0 1.0)))
            (*/ (- 0 B SD) A 2.0) ) ) ) )

(mapcar round
   (solveQuad 1.0 -1000000.0 1.0)
   (6 .) )

Output:

-> ("999,999.999999" "0.000001")

\end{verbatim}

\section*{Roots of unity}

\begin{verbatim}

{{trans|C}}

(load "@lib/math.l")

(for N (range 2 10)
   (let Angle 0.0
      (prin N ": ")
      (for I N
         (let Ipart (sin Angle)
            (prin
               (round (cos Angle) 4)
               (if (lt0 Ipart) "-" "+")
               "j"
               (round (abs Ipart) 4)
               "  " ) )
         (inc 'Angle (*/ 2 pi N)) )
      (prinl) ) )

\end{verbatim}

\section*{Rosetta Code/Count examples}

\begin{verbatim}

(load "@lib/http.l")

(client "rosettacode.org" 80
   "mw/api.php?action=query\&list=categorymembers\&cmtitle=Category:Programming_Tasks\&cmlimit=500\&format=xml"
   (while (from " title=\"")
      (let Task (till "\"")
         (client "rosettacode.org" 80 (pack "wiki/" (replace Task " " "_"))
            (let Cnt 0
               (while (from "<span class=\"tocnumber\">")
                  (unless (sub? "." (till "<" T))
                     (inc 'Cnt) ) )
               (out NIL (prinl (ht:Pack Task) ": " Cnt)) ) ) ) ) )

Output (05may10):

100 doors: 79
24 game: 21
24 game/Solve: 15
99 Bottles of Beer: 95
A+B: 37
Abstract type: 29
...

\end{verbatim}

\section*{Rosetta Code/Find unimplemented tasks}

\begin{verbatim}

(load "@lib/http.l" "@lib/xm.l")

(de rosettaCategory (Cat)
   (let (Cont NIL  Xml)
      (make
         (loop
            (client "rosettacode.org" 80
               (pack
                  "mw/api.php?action=query\&list=categorymembers\&cmtitle=Category:"
                  Cat
                  "\&cmlimit=200\&format=xml"
                  Cont )
               (while (line))
               (setq Xml (and (xml?) (xml))) )
            (NIL Xml)
            (for M (body Xml 'query 'categorymembers)
               (link (attr M 'title)) )
            (NIL (attr Xml 'query-continue' categorymembers 'cmcontinue))
            (setq Cont (pack "\&cmcontinue=" @)) ) ) ) )

(de unimplemented (Task)
   (diff
      (rosettaCategory "Programming_Tasks")
      (rosettaCategory Task) ) )

\end{verbatim}

\section*{Rosetta Code/Fix code tags}

\begin{verbatim}

#!/usr/bin/picolisp /usr/lib/picolisp/lib.l

(let Lang '("ada" "awk" "c" "forth" "prolog" "python" "z80")
   (in NIL
      (while (echo "<")
         (let S (till ">" T)
            (cond
               ((pre? "code " S) (prin "<lang" (cddddr (chop S))))
               ((member S Lang) (prin "<lang " S))
               ((= S "/code") (prin "</lang"))
               ((and (pre? "/" S) (member (pack (cdr (chop S))) Lang))
                  (prin "</lang") )
               (T (prin "<" S)) ) ) ) ) )
(bye)

\end{verbatim}

\section*{Rosetta Code/Rank languages by popularity}

\begin{verbatim}

(load "@lib/http.l")

(for (I . X)
   (flip
      (sort
         (make
            (client "rosettacode.org" 80
               "mw/index.php?title=Special:Categories\&limit=5000"
               (while (from "<li><a href=\"/wiki/Category:")
                  (let Cat (till "\"")
                     (from "(")
                     (when (format (till " " T))
                        (link (cons @ (ht:Pack Cat))) ) ) ) ) ) ) )
   (prinl (align 3 I) ". " (car X) " - " (cdr X)) )

Output (07apr10):

  1. 390 - Tcl
  2. 389 - Programming_Tasks
  3. 359 - Python
  4. 344 - Ruby
  5. 326 - J
  6. 316 - OCaml
  7. 315 - C
  8. 312 - Haskell
  9. 296 - Perl
 10. 281 - Common_Lisp
...

\end{verbatim}

\section*{Rot-13}

\begin{verbatim}

(de rot13-Ch (C)
   (if
      (or
         (member C '`(apply circ (chop "ABCDEFGHIJKLMNOPQRSTUVWXYZ")))
         (member C '`(apply circ (chop "abcdefghijklmnopqrstuvwxyz"))) )
      (nth @ 14 1)
      C ) )

\end{verbatim}

\section*{Run as a daemon or service}

\begin{verbatim}

(unless (fork)
   (out "file.log"
      (println *Pid)    # First write the daemon's PID to the file
      (for N 3600       # Write count for about one hour (if not killed)
         (wait 1000)
         (println N)
         (flush) ) )
   (bye) )              # Child terminates after one hour

(bye)                   # Parent terminates immediately

\end{verbatim}

\section*{Run-length encoding}

\begin{verbatim}

(de encode (Str)
   (pack
      (make
         (for (Lst (chop Str) Lst)
            (let (N 1  C)
               (while (= (setq C (pop 'Lst)) (car Lst))
                  (inc 'N) )
               (link N C) ) ) ) ) )

(de decode (Str)
   (pack
      (make
         (let N 0
            (for C (chop Str)
               (if (>= "9" C "0")
                  (setq N (+ (format C) (* 10 N)))
                  (do N (link C))
                  (zero N) ) ) ) ) ) )

(and
   (prinl "Data:    " "WWWWWWWWWWWWBWWWWWWWWWWWWBBBWWWWWWWWWWWWWWWWWWWWWWWWBWWWWWWWWWWWWWW")
   (prinl "Encoded: " (encode @))
   (prinl "Decoded: " (decode @)) )

Output:

Data:    WWWWWWWWWWWWBWWWWWWWWWWWWBBBWWWWWWWWWWWWWWWWWWWWWWWWBWWWWWWWWWWWWWW
Encoded: 12W1B12W3B24W1B14W
Decoded: WWWWWWWWWWWWBWWWWWWWWWWWWBBBWWWWWWWWWWWWWWWWWWWWWWWWBWWWWWWWWWWWWWW

\end{verbatim}

\section*{Runtime evaluation}

\begin{verbatim}

In PicoLisp there is a formal equivalence of code and data. Almost any peace of
data is potentially executable. PicoLisp has three internal data types: Numbers,
symbols and lists. Though in certain contexts (e.g. GUI objects) also atomic
data (numbers and symbols) are evaluated as code entities, a typical executable
item is a list.

The PicoLisp reference distinguishes between two terms: An 'exe' (expression) is
an executable list, with a function as the first element, followed by arguments.
A 'prg' (program) is a list of 'exe's, to be executed sequentially.

'exe's and 'prg's are implicit in the whole runtime system. For example, the
body of a function is a 'prg', the "true" branch of an 'if' call is an 'exe',
while the "false" branch again is a 'prg'.

For explicit execution, an 'exe' can be evaluated by passing it to the function
'[http://software-lab.de/doc/refE.html#eval eval]', while a 'prg' can be handled
by '[http://software-lab.de/doc/refR.html#run run]'.

As PicoLisp uses exclusively dynamic binding, any 'exe' or 'prg' can be executed
in arbitrary contexts. The environmet can be controlled in any conceivable way,
through implicit function parameter bindings, or explicitly with the aid of
functions like '[http://software-lab.de/doc/refB.html#bind bind]',
'[http://software-lab.de/doc/refL.html#let let]' or
'[http://software-lab.de/doc/refJ.html#job job]'.

\end{verbatim}

\section*{Runtime evaluation/In an environment}

\begin{verbatim}

(let Expression '(+ X (* X X))            # Local expression
   (println
      (+
         (let X 3
            (eval Expression) )
         (let X 4
            (eval Expression) ) ) )
   (let Function (list '(X) Expression)   # Build a local function
      (println
         (+
            (Function 3)
            (Function 4) ) ) ) )

Output:

32
32

\end{verbatim}

\chapter{Rosetta Code Tasks starting with S}

\section*{S-Expressions}

\begin{verbatim}

The '[http://software-lab.de/doc/refA.html#any any]' function parses an
s-expression from a string (indentical to the way
'[http://software-lab.de/doc/refR.html#read read]' does this from an input
stream).

: (any "((data \"quoted data\" 123 4.5) (data (!@# (4.5) \"(more\" \"data)\")))")
-> ((data "quoted data" 123 5) (data (!@# (5) "(more" "data)")))

: (view @)
+---+-- data
|   |
|   +-- "quoted data"
|   |
|   +-- 123
|   |
|   +-- 5
|
+---+-- data
    |
    +---+-- !@#
        |
        +---+-- 5
        |
        +-- "(more"
        |
        +-- "data)"

Implementing a subset of 'any' explicitly:

(de readSexpr ()
   (case (skip)
      ("(" (char) (readList))
      ("\"" (char) (readString))
      (T (readAtom)) ) ) )

(de readList ()
   (make
      (loop
         (NIL (skip))
         (T (= @ ")") (char))
         (link (readSexpr)) ) ) )

(de readString ()
   (pack
      (make
         (until (= "\"" (or (peek) (quit "Unterminated string")))
            (link (char)) )
         (char) ) ) )

(de readAtom ()
   (let X
      (make
         (until (or (sp? (peek)) (member (peek) '("(" ")")))
            (link (char)) ) )
      (or (format X) (intern (pack X))) ) )

It can be used in a pipe to read from a string:

: (pipe (prin "((data \"quoted data\" 123 4.5) (data (!@# (4.5) \"(more\" \"data)\")))") (readSexpr))
-> ((data "quoted data" 123 5) (data (!@# (5) "(more" "data)")))

'[http://software-lab.de/doc/refS.html#sym sym]' does the reverse (i.e. builds a
symbol (string) from an expression).

: (sym @@)
-> "((data \"quoted data\" 123 5) (data (!@# (5) \"(more\" \"data)\")))"

Implementing a subset of the built-in printer:

(de printSexpr (Expr Fun)
   (cond
      ((pair Expr)
         (Fun "(")
         (printSexpr (car Expr) Fun)
         (for X (cdr Expr)
            (Fun " ")
            (printSexpr X Fun) )
         (Fun ")") )
      ((str? Expr)
         (Fun "\"")
         (mapc Fun (chop Expr))
         (Fun "\"") )
      (T (mapc Fun (chop Expr))) ) )

This can be used for plain printing

: (printSexpr
   '((data "quoted data" 123 4.5) (data (!@# (4.5) "(more" "data)")))
   prin )
((data "quoted data" 123 5) (data (!@# (5) "(more" "data)")))

or to collect the characters into a string:

: (pack
   (make
      (printSexpr
         '((data "quoted data" 123 4.5) (data (!@# (4.5) "(more" "data)")))
         link ) ) )
-> "((data \"quoted data\" 123 5) (data (!@# (5) \"(more\" \"data)\")))"

\end{verbatim}

\section*{SEDOLs}

\begin{verbatim}

(de sedol (Str)
   (pack Str
      (char
         (+ `(char "0")
            (\%
               (- 10
                  (\%
                     (sum
                        '((W C)
                           (cond
                              ((>= "9" C "0")
                                 (* W (format C)) )
                              ((>= "Z" (setq C (uppc C)) "A")
                                 (* W (+ 10 (- (char C) `(char "A")))) ) ) )
                        (1 3 1 7 3 9)
                        (chop Str) )
                     10 ) )
               10 ) ) ) ) )

(for S '("710889" "B0YBKJ" "406566" "B0YBLH" "228276" "B0YBKL" "557910" "B0YBKR" "585284" "B0YBKT" "B00030")
   (prinl (sedol S)) )

\end{verbatim}

\section*{SHA-1}

\begin{verbatim}

(let Str "Rosetta Code"
   (pack
      (mapcar '((B) (pad 2 (hex B)))
         (native "libcrypto.so" "SHA1" '(B . 20) Str (length Str) '(NIL (20))) ) ) )

Output:

-> "48C98F7E5A6E736D790AB740DFC3F51A61ABE2B5"

\end{verbatim}

\section*{Safe addition}

\begin{verbatim}

PicoLisp uses scaled integer arithmetics, with unlimited precision, for all
operations on real numbers. For that reason addition and subtraction are always
exact. Multiplication is also exact (unless the result is explicitly scaled by
the user), and division in combination with the remainder.

\end{verbatim}

\section*{Scope modifiers}

\begin{verbatim}

PicoLisp distinguishes between "scope" and "binding". The scope of a symbol
determines its visibility in a given context (whether or not it can be
accessed), while binding is about assigning it a value.

# Scope

In PicoLisp, the scope type of a symbol is either "internal", "transient" or
"external". It is specified lexically: Internal symbols are just normal symbols.
Transient symbols are surrounded by double quotes (and thus look like strings in
other languages), and/or with an underlined font if possible. External symbols
are surrounded by braces.

* The scope of an internal symbol is global. This means that a symbol like AB123 is always the same object, residing at a certain location in memory (pointer equality).

* A transient symbol like "AB123" is the same only within the current transient scope. This is normally a single source file, but may be further subdivided.  Within that scope it can be used like an internal symbol, but after the transient scope is closed it cannot be accessed by its name any longer. This behavior is similar to "static" identifiers in the C language.

* External symbols like {AB123} are persistent database symbols. They have a permanent identity among different processes and over time. Besides that, they have the same structure like internal and transient symbols: A value, properties and a name.

# Binding

Regardless of the scope, the binding of symbols to values is always dynamic.
This happens implicitly for function parameters, or explicitly with functions
like [http://software-lab.de/doc/refL.html#let let],
[http://software-lab.de/doc/refU.html#use use],
[http://software-lab.de/doc/refB.html#bind bind],
[http://software-lab.de/doc/refJ.html#job job] and others.
This means that the current value of a symbol is saved locally, then set to the
new value. When done, the old value is restored. Closures are created by
maintaining an explicit environment. More about that
[http://software-lab.de/doc/faq.html#dynamic here].

\end{verbatim}

\section*{Script name}

\begin{verbatim}

The function '[http://software-lab.de/doc/refC.html#cmd cmd]' returns the
command name.

: (cmd)
-> "/usr/bin/picolisp"

\end{verbatim}

\section*{Scripted Main}

\begin{verbatim}

PicoLisp normally does it the other way round: It calls main from the command
line with the '-' syntax if desired. Create an executable file (chmod +x)
"life.l":

#!/usr/bin/picolisp /usr/lib/picolisp/lib.l

(de meaningOfLife ()
   42 )

(de lifemain ()
   (prinl "Main: The meaning of life is " (meaningOfLife))
   (bye) )

and an executable file (chmod +x) "test.l":

#!/usr/bin/picolisp /usr/lib/picolisp/lib.l

(load "life.l")

(prinl "Test: The meaning of life is " (meaningOfLife))
(bye)

Test:

\$ ./life.l -lifemain
Main: The meaning of life is 42

\$ ./test.l
Test: The meaning of life is 42

\end{verbatim}

\section*{Search a list}

\begin{verbatim}

Note that in PicoLisp all indexes are one-based
(the first element has the position '1')

(de lastIndex (Item Lst)
   (- (length Lst) (index Item (reverse Lst)) -1) )

(de findNeedle (Fun Sym Lst)
   (prinl Sym " " (or (Fun Sym Lst) "not found")) )

(let Lst '(Zig Zag Wally Ronald Bush Krusty Charlie Bush Bozo)
   (findNeedle index 'Washington Lst)
   (findNeedle index 'Bush Lst)
   (findNeedle lastIndex 'Bush Lst) )

Output:

Washington not found
Bush 5
Bush 8

\end{verbatim}

\section*{Secure temporary file}

\begin{verbatim}

The 'tmp' function returns temporary file names which are exclusively for the
current process (based on the process ID). These files are automatically deleted
upon process termination. Background tasks within a single PicoLisp process is
always non-preemptive, therefore dedicated locks are usually not necessary. If
they are (e.g. because such a file name is passed to a child process), explicit
locks with the 'ctl' functions are possible.

: (out (tmp "foo") (println 123))         # Write tempfile
-> 123

: (in (tmp "foo") (read))                 # Read tempfile
-> 123

: (let F (tmp "foo")
   (ctl F                                 # Get exclusive lock
      (let N (in F (read))                # Atomic increment
         (out F (println (inc N))) ) ) )
-> 124

\end{verbatim}

\section*{Self-describing numbers}

\begin{verbatim}

(de selfDescribing (N)
   (not
      (find '((D I) (<> D (cnt = N (circ I))))
         (setq N (mapcar format (chop N)))
         (range 0 (length N)) ) ) )

Output:

: (filter selfDescribing (range 1 4000000))
-> (1210 2020 21200 3211000)

\end{verbatim}

\section*{Self-referential sequence}

\begin{verbatim}

Using 'las' from [[Look-and-say sequence#PicoLisp]]:

(de selfRefSequence (Seed)
   (let L (mapcar format (chop Seed))
      (make
         (for (Cache NIL  (not (idx 'Cache L T)))
            (setq L
               (las (flip (sort (copy (link L))))) ) ) ) ) )

(let Res NIL
   (for Seed 1000000
      (let N (length (selfRefSequence Seed))
         (cond
            ((> N (car Res)) (setq Res (list N Seed)))
            ((= N (car Res)) (queue 'Res Seed)) ) ) )
   (println 'Values: (cdr Res))
   (println 'Iterations: (car Res))
   (mapc prinl (selfRefSequence (cadr Res))) )

Output:

Values: (9009 9090 9900)
Iterations: 21
9009
2920
192210
19222110
19323110
1923123110
1923224110
191413323110
191433125110
19151423125110
19251413226110
1916151413325110
1916251423127110
191716151413326110
191726151423128110
19181716151413327110
19182716151423129110
29181716151413328110
19281716151423228110
19281716151413427110
19182716152413228110

\end{verbatim}

\section*{Send an unknown method call}

\begin{verbatim}

This can be done with the '[http://software-lab.de/doc/refS.html#send send]'
function.

(send (expression) Obj arg1 arg2)

\end{verbatim}

\section*{Send email}

\begin{verbatim}

PicoLisp has a built-in '[http://software-lab.de/doc/refM.html#mail mail]'
function. A minimal call would be

(mail "localhost" 25 "me@from.org" "you@to.org" "Subject" NIL "Hello")

Instead of "Hello" an arbitrary number of arguments may follow (possibly
containing executable expressions) for the message body.

The 6th argument (here 'NIL') may specify a list of attachments.

\end{verbatim}

\section*{Sequence of non-squares}

\begin{verbatim}

(de sqfun (N)
   (+ N (sqrt N T)) )  # 'sqrt' rounds when called with 'T'

(for I 22
   (println I (sqfun I)) )

(for I 1000000
   (let (N (sqfun I)  R (sqrt N))
      (when (= N (* R R))
         (prinl N " is square") ) ) )

Output:

1 2
2 3
3 5
4 6
5 7
6 8
7 10
8 11
9 12
10 13
11 14
12 15
13 17
14 18
15 19
16 20
17 21
18 22
19 23
20 24
21 26
22 27

\end{verbatim}

\section*{Set}

\begin{verbatim}

We may use plain lists, or '[http://software-lab.de/doc/refI.html#idx idx]'
structures for sets. A set may contain any type of data.

===Using lists===

(setq
   Set1 (1 2 3 7 abc "def" (u v w))
   Set2 (2 3 5 hello (x y z))
   Set3 (3 hello (x y z)) )


# Element tests (any non-NIL value means "yes")
: (member "def" Set1)
-> ("def" (u v w))

: (member "def" Set2)
-> NIL

: (member '(x y z) Set2)
-> ((x y z))


# Union
: (uniq (append Set1 Set2))
-> (1 2 3 7 abc "def" (u v w) 5 hello (x y z))


# Intersection
: (sect Set1 Set2)
-> (2 3)


# Difference
: (diff Set1 Set2)
-> (1 7 abc "def" (u v w))


# Test for subset
: (not (diff Set1 Set2))
-> NIL  # Set1 is not a subset of Set2

: (not (diff Set3 Set2))
-> T  # Set3 is a subset of Set2


# Test for equality
: (= (sort (copy Set1)) (sort (copy Set2)))
-> NIL

: (= (sort (copy Set2)) (sort (copy Set2)))
-> T

===Using 'idx' structures===

# Create three test-sets
(balance 'Set1 (1 2 3 7 abc "def" (u v w)))
(balance 'Set2 (2 3 5 hello (x y z)))
(balance 'Set3 (3 hello (x y z)))


# Get contents
: (idx 'Set1)
-> (1 2 3 7 abc "def" (u v w))

: (idx 'Set2)
-> (2 3 5 hello (x y z))


# Element tests (any non-NIL value means "yes")
: (idx 'Set1 "def")
-> ("def" (abc) (u v w))

: (idx 'Set2 "def")
-> NIL

: (idx 'Set2 '(x y z))
-> ((x y z))


# Union
: (use S
   (balance 'S (idx 'Set1))
   (balance 'S (idx 'Set2) T)
   (idx 'S) )
-> (1 2 3 5 7 abc "def" hello (u v w) (x y z))


# Intersection
: (sect (idx 'Set1) (idx 'Set2))
-> (2 3)


# Difference
: (diff (idx 'Set1) (idx 'Set2))
-> (1 7 abc "def" (u v w))


# Test for subset
: (not (diff (idx 'Set1) (idx 'Set2)))
-> NIL  # Set1 is not a subset of Set2

: (not (diff (idx 'Set3) (idx 'Set2)))
-> T  # Set3 is a subset of Set2


# Test for equality
: (= (idx 'Set1) (idx 'Set2))
-> NIL

: (= (idx 'Set2) (idx 'Set2))
-> T

\end{verbatim}

\section*{Set consolidation}

\begin{verbatim}

{{trans|Python}}

(de consolidate (S)
   (when S
      (let R (cons (car S))
         (for X (consolidate (cdr S))
            (if (mmeq X (car R))
               (set R (uniq (conc X (car R))))
               (conc R (cons X)) ) )
         R ) ) )

Test:

: (consolidate '((A B) (C D)))
-> ((A B) (C D))
: (consolidate '((A B) (B D)))
-> ((B D A))
: (consolidate '((A B) (C D) (D B)))
-> ((D B C A))
: (consolidate '((H I K) (A B) (C D) (D B) (F G H)))
-> ((F G H I K) (D B C A))

\end{verbatim}

\section*{Seven-sided dice from five-sided dice}

\begin{verbatim}

(de dice5 ()
   (rand 1 5) )

(de dice7 ()
   (use R
      (until (> 21 (setq R (+ (* 5 (dice5)) (dice5) -6))))
      (inc (\% R 7)) ) )

Output:

: (let R NIL
   (do 1000000 (accu 'R (dice7) 1))
   (sort R) )
-> ((1 . 142295) (2 . 142491) (3 . 143448) (4 . 143129) (5 . 142701) (6 . 143142) (7 . 142794))

\end{verbatim}

\section*{Shell one-liner}

\begin{verbatim}

\$ picolisp -'prinl "Hello world!"' -bye
Hello world!

\end{verbatim}

\section*{Short-circuit evaluation}

\begin{verbatim}

(de a (F)
   (msg 'a)
   F )

(de b (F)
   (msg 'b)
   F )

(mapc
   '((I J)
      (for Op '(and or)
         (println I Op J '-> (Op (a I) (b J))) ) )
   '(NIL NIL T T)
   '(NIL T NIL T) )

Output:

a
NIL and NIL -> NIL
a
b
NIL or NIL -> NIL
a
NIL and T -> NIL
a
b
NIL or T -> T
a
b
T and NIL -> NIL
a
T or NIL -> T
a
b
T and T -> T
a
T or T -> T

\end{verbatim}

\section*{Show the epoch}

\begin{verbatim}

The 'date' function in PicoLisp returns a day number, starting first of March of
the year zero. Calculated according to the gregorian calendar (despite that that
calendar wasn't used in 0 AD yet).

: (date 1)
-> (0 3 1)  # Year zero, March 1st

\end{verbatim}

\section*{Sierpinski carpet}

\begin{verbatim}

{{trans|Ruby}}

(de carpet (N)
   (let Carpet '("#")
      (do N
         (setq Carpet
            (conc
               (mapcar '((S) (pack S S S)) Carpet)
               (mapcar
                  '((S) (pack S (replace (chop S) "#" " ") S))
                  Carpet )
               (mapcar '((S) (pack S S S)) Carpet) ) ) ) ) )

(mapc prinl (carpet 3))

\end{verbatim}

\section*{Sierpinski triangle}

\begin{verbatim}

{{trans|Python}}

(de sierpinski (N)
   (let (D '("*")  S " ")
      (do N
         (setq
            D (conc
               (mapcar '((X) (pack S X S)) D)
               (mapcar '((X) (pack X " " X)) D) )
            S (pack S S) ) )
      D ) )

(mapc prinl (sierpinski 4))

\end{verbatim}

\section*{Sieve of Eratosthenes}

\begin{verbatim}

(de sieve (N)
   (let Sieve (range 1 N)
      (set Sieve)
      (for I (cdr Sieve)
         (when I
            (for (S (nth Sieve (* I I)) S (nth (cdr S) I))
               (set S) ) ) )
      (filter bool Sieve) ) )

Output:

: (sieve 100)
-> (2 3 5 7 11 13 17 19 23 29 31 37 41 43 47 53 59 61 67 71 73 79 83 89 97)

\end{verbatim}

\section*{Simple database}

\begin{verbatim}

The '[http://software-lab.de/doc/refR.html#rc rc]' resource file handling
function is used typically for such tasks. It also takes care of proper locking
and protection.

#!/usr/bin/pil

(de usage ()
   (prinl
      "Usage:^J\
      sdb <file> add <title> <cat> <date> ...  Add a new entry^J\
      sdb <file> get <title>                   Retrieve an entry^J\
      sdb <file> latest                        Print the latest entry^J\
      sdb <file> categories                    Print the latest for each cat^J\
      sdb <file>                               Print all, sorted by date" ) )

(de printEntry (E)
   (apply println (cdddr E) (car E) (cadr E) (datStr (caddr E))) )

(ifn (setq *File (opt))
   (usage)
   (case (opt)
      (add
         (let (Ttl (opt)  Cat (opt))
            (if (strDat (opt))
               (rc *File Ttl (cons Cat @ (argv)))
               (prinl "Bad date") ) ) )
      (get
         (let Ttl (opt)
            (when (rc *File Ttl)
               (printEntry (cons Ttl @)) ) ) )
      (latest
         (printEntry (maxi caddr (in *File (read)))) )
      (categories
         (for Cat (by cadr group (in *File (read)))
            (printEntry (maxi caddr Cat)) ) )
      (NIL
         (mapc printEntry (by caddr sort (in *File (read)))) )
      (T (usage)) ) )

(bye)

Test:

\$ sdb CDs add "Title 1" "Category 1" 2011-11-13
\$ sdb CDs add "Title 2" "Category 2" 2011-11-12
\$ sdb CDs add "Title 3" "Category 1" 2011-11-14 foo bar
\$ sdb CDs add "Title 4" "Category 2" 2011-11-15 mumble

\$ sdb CDs get "Title 3"
"Title 3" "Category 1" "2011-11-14" "foo" "bar"

\$ sdb CDs latest
"Title 4" "Category 2" "2011-11-15" "mumble"

\$ sdb CDs categories
"Title 4" "Category 2" "2011-11-15" "mumble"
"Title 3" "Category 1" "2011-11-14" "foo" "bar"

\$ sdb CDs
"Title 2" "Category 2" "2011-11-12"
"Title 1" "Category 1" "2011-11-13"
"Title 3" "Category 1" "2011-11-14" "foo" "bar"
"Title 4" "Category 2" "2011-11-15" "mumble"

\end{verbatim}

\section*{Simple quaternion type and operations (!)}

\begin{verbatim}

(scl 6)

(def 'quatCopy copy)

(de quatNorm (Q)
   (sqrt (sum * Q Q)) )

(de quatNeg (Q)
   (mapcar - Q) )

(de quatConj (Q)
   (cons (car Q) (mapcar - (cdr Q))) )

(de quatAddR (Q R)
   (cons (+ R (car Q)) (cdr Q)) )

(de quatAdd (Q1 Q2)
   (mapcar + Q1 Q2) )

(de quatMulR (Q R)
   (mapcar */ (mapcar * Q (circ R)) (1.0 .)) )

(de quatMul (Q1 Q2)
   (mapcar
      '((Ops I)
         (sum '((Op R I) (Op (*/ R (get Q2 I) 1.0))) Ops Q1 I) )
      '((+ - - -) (+ + + -) (+ - + +) (+ + - +))
      '((1 2 3 4) (2 1 4 3) (3 4 1 2) (4 3 2 1)) ) )

(de quatFmt (Q)
   (mapcar '((R S) (pack (format R *Scl) S))
      Q
      '(" + " "i + " "j + " "k") ) )

Test:

(setq
   Q (1.0 2.0 3.0 4.0)
   Q1 (2.0 3.0 4.0 5.0)
   Q2 (3.0 4.0 5.0 6.0)
   R 7.0 )

(prinl "R  = " (format R *Scl))
(prinl "Q  = " (quatFmt Q))
(prinl "Q1 = " (quatFmt Q1))
(prinl "Q2 = " (quatFmt Q2))
(prinl)
(prinl "norm(Q)  = " (format (quatNorm Q) *Scl))
(prinl "norm(Q1) = " (format (quatNorm Q1) *Scl))
(prinl "norm(Q2) = " (format (quatNorm Q2) *Scl))
(prinl "net(Q)   = " (quatFmt (quatNeg Q)))
(prinl "conj(Q)  = " (quatFmt (quatConj Q)))
(prinl "Q + R    = " (quatFmt (quatAddR Q R)))
(prinl "Q1 + Q2  = " (quatFmt (quatAdd Q1 Q2)))
(prinl "Q * R    = " (quatFmt (quatMulR Q R)))
(prinl "Q1 * Q2  = " (quatFmt (quatMul Q1 Q2)))
(prinl "Q2 * Q1  = " (quatFmt (quatMul Q2 Q1)))
(prinl (if (= (quatMul Q1 Q2) (quatMul Q2 Q1)) "Equal" "Not equal"))

Output:

R  = 7.000000
Q  = 1.000000 + 2.000000i + 3.000000j + 4.000000k
Q1 = 2.000000 + 3.000000i + 4.000000j + 5.000000k
Q2 = 3.000000 + 4.000000i + 5.000000j + 6.000000k

norm(Q)  = 5.477225
norm(Q1) = 7.348469
norm(Q2) = 9.273618
net(Q)   = -1.000000 + -2.000000i + -3.000000j + -4.000000k
conj(Q)  = 1.000000 + -2.000000i + -3.000000j + -4.000000k
Q + R    = 8.000000 + 2.000000i + 3.000000j + 4.000000k
Q1 + Q2  = 5.000000 + 7.000000i + 9.000000j + 11.000000k
Q * R    = 7.000000 + 14.000000i + 21.000000j + 28.000000k
Q1 * Q2  = -56.000000 + 16.000000i + 24.000000j + 26.000000k
Q2 * Q1  = -56.000000 + 18.000000i + 20.000000j + 28.000000k
Not equal

\end{verbatim}

\section*{Simple windowed application}

\begin{verbatim}

The standard PicoLisp GUI is HTTP based. Connect your browser to
http://localhost:8080 after starting the following script.

#!/usr/bin/picolisp /usr/lib/picolisp/lib.l

(load "@ext.l" "@lib/http.l" "@lib/xhtml.l" "@lib/form.l")

(zero *Count)

(de start ()
   (app)
   (action
      (html 0 "Clicks" NIL NIL
         (form NIL
            (gui '(+Init +TextField) "There have been no clicks yet")
            (----)
            (gui '(+JS +Button) "click me"
               '(set> (field -1)
                  (pack "Clicked " (inc '*Count) " times") ) ) ) ) ) )

(server 8080 "!start")
(wait)

\end{verbatim}

\section*{Simulate input/Keyboard}

\begin{verbatim}

PicoLisp comes with a dedicated browser GUI. A library based on web scraping (in
"lib/scrape.l") can be used to drive that GUI under program control. It allows
to read GUI pages, click on HTML links, enter text into forms, and press submit
buttons. In that way one application can control another application.

The documented [http://software-lab.de/doc/app.html#minApp demo application],
which is also available online at [http://7fach.de/8080 app.7fach.de], is used
in the following example. Keyboard input is simulated with the function 'enter'
to fill the login form's name and password fields.

(load "@lib/http.l" "@lib/scrape.l")

# Connect to the demo app at http://7fach.de/8080
(scrape "7fach.de" 80 "8080")

# Log in
(expect "'admin' logged in"
   (enter 3 "admin")       # Enter user name into 3rd field
   (enter 4 "admin")       # Enter password into 4th field
   (press "login") )       # Press the "login" button

(click "Items")         # Open "Items" dialog
(click "Spare Part")    # Click on "Spare Part" article
(prinl (value 8))       # Print the price (12.50)
(click "logout")        # Log out

Output:

12.50

The same example is used in the related task [[Simulate input/Mouse#PicoLisp]].

\end{verbatim}

\section*{Simulate input/Mouse}

\begin{verbatim}

PicoLisp comes with a dedicated browser GUI. A library based on web scraping (in
"lib/scrape.l") can be used to drive that GUI under program control. It allows
to read GUI pages, click on HTML links, enter text into forms, and press submit
buttons. In that way one application can control another application.

The documented [http://software-lab.de/doc/app.html#minApp demo application],
which is also available online at [http://7fach.de/8080 app.7fach.de], is used
in the following example. Mouse input is simulated with the functions 'click'
(click on a HTML link) and 'press' (press a submit button).

...

The same example is used in the related task [[Simulate input/Keyboard#PicoLisp]].

\end{verbatim}

\section*{Singleton}

\begin{verbatim}

As there is no physical difference between classes and objects, we can use the
class symbol itself.

(class +Singleton)

(dm message1> ()
   (prinl "This is method 1 on " This) )

(dm message2> ()
   (prinl "This is method 2 on " This) )

Output:

: (message1> '+Singleton)
This is method 1 on +Singleton
-> +Singleton

: (message2> '+Singleton)
This is method 2 on +Singleton
-> +Singleton

\end{verbatim}

\section*{Singly-linked list/Element definition}

\begin{verbatim}

In PicoLisp, the singly-linked list is the most important data structure. Many
built-in functions deal with linked lists. A list consists of interconnected
"cells". Cells are also called "cons pairs", because they are constructed with
the function '[http://software-lab.de/doc/refC.html#cons cons]'.

Each cell consists of two parts: A CAR and a CDR. Both may contain (i.e. point
to) arbitrary data (numbers, symbols, other cells, or even to itself). In the
case of a linked list, the CDR points to the rest of the list.

The CAR of a cell can be manipulated with
'[http://software-lab.de/doc/refS.html#set set]'
and the CDR with '[http://software-lab.de/doc/refC.html#con con]'.

\end{verbatim}

\section*{Singly-linked list/Element insertion}

\begin{verbatim}

Destructive operation

(de insertAfter (Item Lst New)
   (when (member Item Lst)
      (con @ (cons New (cdr @))) )
   Lst )

Non-destructive operation

(de insertAfter (Item Lst New)
   (if (index Item Lst)
      (conc (cut @ 'Lst) (cons New Lst))
      Lst ) )

Output in both cases:

: (insertAfter 'A '(A B) 'C)
-> (A C B)

: (insertAfter 'A '(X Y Z A B D E) 'C)
-> (X Y Z A C B D E)

\end{verbatim}

\section*{Singly-linked list/Traversal}

\begin{verbatim}

We might use map functions

(mapc println '(a "cde" (X Y Z) 999))

or flow control functions

(for X '(a "cde" (X Y Z) 999)
   (println X) )

Output in both cases:

a
"cde"
(X Y Z)
999

\end{verbatim}

\section*{Sleep}

\begin{verbatim}

(prinl "Sleeping..." )
(wait 2000)                # Wait for 2 seconds
(prinl "Awake!")

As wait will continue executing background events, another possibility (for a
complete stop) is calling some external program like

(prinl "Sleeping..." )
(call 'sleep 2)            # Wait for 2 seconds
(prinl "Awake!")

\end{verbatim}

\section*{Sockets}

\begin{verbatim}

(when (connect "localhost" 256)
   (out @ (prinl "hello socket world"))
   (close @) )

\end{verbatim}

\section*{Sokoban}

\begin{verbatim}

This searches for a solution, without trying for the push-optimal one. The
player moves between the pushes, however, are minimized.

(load "@lib/simul.l")

# Display board
(de display ()
   (disp *Board NIL
      '((This)
         (pack
            (if2 (== This *Pos) (memq This *Goals)
               "+"                   # Player on goal
               "@"                   # Player elsewhere
               (if (: val) "*" ".")  # On gloal
               (or (: val) " ") )    # Elsewhere
            " " ) ) ) )

# Initialize
(de main (Lst)
   (mapc
      '((B L)
         (mapc
            '((This C)
               (case C
                  (" ")
                  ("." (push '*Goals This))
                  ("@" (setq *Pos This))
                  ("\$" (=: val C) (push '*Boxes This))
                  (T (=: val C)) ) )
               B L ) )
      (setq *Board (grid (length (car Lst)) (length Lst)))
      (apply mapcar (flip (mapcar chop Lst)) list) )
   (display) )

# Generate possible push-moves
(de pushes ()
   (make
      (for Box *Boxes
         (unless (or (; (west Box) val) (; (east Box) val))
            (when (moves (east Box))
               (link (cons (cons Box (west Box)) *Pos "L" @)) )
            (when (moves (west Box))
               (link (cons (cons Box (east Box)) *Pos "R" @)) ) )
         (unless (or (; (south Box) val) (; (north Box) val))
            (when (moves (north Box))
               (link (cons (cons Box (south Box)) *Pos "D" @)) )
            (when (moves (south Box))
               (link (cons (cons Box (north Box)) *Pos "U" @)) ) ) ) ) )

# Moves of player to destination
(de moves (Dst Hist)
   (or
      (== Dst *Pos)
      (mini length
         (extract
            '((Dir)
               (with ((car Dir) Dst)
                  (cond
                     ((== This *Pos) (cons (cdr Dir)))
                     ((: val))
                     ((memq This Hist))
                     ((moves This (cons Dst Hist))
                        (cons (cdr Dir) @) ) ) ) )
            '((west . "r") (east . "l") (south . "u") (north . "d")) ) ) ) )

# Find solution
(de go (Res)
   (unless (idx '*Hist (sort (copy *Boxes)) T)  # No repeated state
      (if (find '((This) (<> "\$" (: val))) *Goals)
         (pick
            '((Psh)
               (setq  # Move
                  *Pos (caar Psh)
                  *Boxes (cons (cdar Psh) (delq *Pos *Boxes)) )
               (put *Pos 'val NIL)
               (put (cdar Psh) 'val "\$")
               (prog1 (go (append (cddr Psh) Res))
                  (setq  # Undo move
                     *Pos (cadr Psh)
                     *Boxes (cons (caar Psh) (delq (cdar Psh) *Boxes)) )
                  (put (cdar Psh) 'val NIL)
                  (put (caar Psh) 'val "\$") ) )
            (pushes) )
         (display)  # Display solution
         (pack (flip Res)) ) ) )

Test:

(main
   (quote
      "#######"
      "#     #"
      "#     #"
      "#. #  #"
      "#. \$\$ #"
      "#.\$\$  #"
      "#.#  @#"
      "#######" ) )
(prinl)
(go)

Output:

 8 # # # # # # #
 7 #           #
 6 #           #
 5 # .   #     #
 4 # .   \$ \$   #
 3 # . \$ \$     #
 2 # . #     @ #
 1 # # # # # # #
   a b c d e f g

 8 # # # # # # #
 7 #           #
 6 # @         #
 5 # *   #     #
 4 # *         #
 3 # *         #
 2 # * #       #
 1 # # # # # # #
   a b c d e f g
-> "uuulDLLulDDurrrrddlUruLLLrrddlUruLdLUUdrruulLulD"

\end{verbatim}

\section*{Sort an array of composite structures}

\begin{verbatim}

By default, the [http://software-lab.de/doc/refS.html#sort sort] function in
PicoLisp returns an ascending list (of any type)

: (sort '(("def" 456) ("abc" 789) ("ghi" 123)))
-> (("abc" 789) ("def" 456) ("ghi" 123))

To sort by a certain sub-element, the function
[http://software-lab.de/doc/refB.html#by by] can be used. For example, to
sort by the first element

: (by car sort '(("def" 456) ("abc" 789) ("ghi" 123)))
-> (("abc" 789) ("def" 456) ("ghi" 123))

or by the second element

: (by cadr sort '(("def" 456) ("abc" 789) ("ghi" 123)))
-> (("ghi" 123) ("def" 456) ("abc" 789))

\end{verbatim}

\section*{Sort an integer array}

\begin{verbatim}

The [http://software-lab.de/doc/refS.html#sort sort] function in PicoLisp
returns already by default an ascending list (of any type, not only integers):

(sort (2 4 3 1 2))
-> (1 2 2 3 4)

\end{verbatim}

\section*{Sort disjoint sublist}

\begin{verbatim}

The indices are incremented here, as PicoLisp is 1-based

(let (Values (7 6 5 4 3 2 1 0)  Indices (7 2 8))
   (mapc
      '((V I) (set (nth Values I) V))
      (sort (mapcar '((N) (get Values N)) Indices))
      (sort Indices) )
   Values )

Output:

-> (7 0 5 4 3 2 1 6)

\end{verbatim}

\section*{Sort stability}

\begin{verbatim}

The [http://software-lab.de/doc/refS.html#sort sort] function is unstable

\end{verbatim}

\section*{Sort using a custom comparator}

\begin{verbatim}

By default, the [http://software-lab.de/doc/refS.html#sort sort] function in
PicoLisp returns an ascending list (of any type). To get a result in descending
order, the "greater than" function can be supplied

: (sort '("def" "abc" "ghi") >)
-> ("ghi" "def" "abc")

or simply the result reversed (which is, btw, the most efficient way)

: (flip (sort '("def" "abc" "ghi")))
-> ("ghi" "def" "abc")

\end{verbatim}

\section*{Sorting algorithms/Bead sort}

\begin{verbatim}

The following implements a direct model of the bead sort algorithm.
Each pole is a list of 'T' symbols for the beads.

(de beadSort (Lst)
   (let Abacus (cons NIL)
      (for N Lst                                   # Thread beads on poles
         (for (L Abacus  (ge0 (dec 'N))  (cdr L))
            (or (cdr L) (queue 'L (cons)))
            (push (cadr L) T) ) )
      (make
         (while (gt0 (cnt pop (cdr Abacus)))       # Drop and count beads
            (link @) ) ) ) )

Output:

: (beadSort (5 3 1 7 4 1 1 20))
-> (20 7 5 4 3 1 1 1)

\end{verbatim}

\section*{Sorting algorithms/Bogosort}

\begin{verbatim}

(de bogosort (Lst)
   (loop
      (map
         '((L) (rot L (rand 1 (length L))))
         Lst )
      (T (apply <= Lst) Lst) ) )

Output:

: (bogosort (make (do 9 (link (rand 1 999)))))
-> (1 167 183 282 524 556 638 891 902)

: (bogosort (make (do 9 (link (rand 1 999)))))
-> (20 51 117 229 671 848 883 948 978)

: (bogosort (make (do 9 (link (rand 1 999)))))
-> (1 21 72 263 391 476 794 840 878)

\end{verbatim}

\section*{Sorting algorithms/Bubble sort}

\begin{verbatim}

(de bubbleSort (Lst)
   (use Chg
      (loop
         (off Chg)
         (for (L Lst (cdr L) (cdr L))
            (when (> (car L) (cadr L))
               (xchg L (cdr L))
               (on Chg) ) )
         (NIL Chg Lst) ) ) )

\end{verbatim}

\section*{Sorting algorithms/Cocktail sort}

\begin{verbatim}

(de cocktailSort (Lst)
   (use (Swapped L)
      (loop
         (off Swapped)
         (setq L Lst)
         (while (cdr L)
            (when (> (car L) (cadr L))
               (xchg L (cdr L))
               (on Swapped) )
            (pop 'L) )
         (NIL Swapped Lst)
         (off Swapped)
         (loop
            (setq L (prior L Lst))  # Not recommended (inefficient)
            (when (> (car L) (cadr L))
               (xchg L (cdr L))
               (on Swapped) )
            (T (== Lst L)) )
         (NIL Swapped Lst) ) ) )

Output:

: (cocktailSort (make (do 9 (link (rand 1 999)))))
-> (1 167 183 282 524 556 638 891 902)
: (cocktailSort (make (do 9 (link (rand 1 999)))))
-> (82 120 160 168 205 226 408 708 719)

\end{verbatim}

\section*{Sorting algorithms/Comb sort}

\begin{verbatim}

(de combSort (Lst)
   (let (Gap (length Lst)  Swaps NIL)
      (while (or (> Gap 1) Swaps)
         (setq Gap (max 1 (/ (* Gap 4) 5)))
         (off Swaps)
         (use Lst
            (for (G (cdr (nth Lst Gap))  G  (cdr G))
               (when (> (car Lst) (car G))
                  (xchg Lst G)
                  (on Swaps) )
               (pop 'Lst) ) ) ) )
   Lst )

Output:

: (combSort (88 18 31 44 4 0 8 81 14 78 20 76 84 33 73 75 82 5 62 70))
-> (0 4 5 8 14 18 20 31 33 44 62 70 73 75 76 78 81 82 84 88)

\end{verbatim}

\section*{Sorting algorithms/Counting sort}

\begin{verbatim}

(de countingSort (Lst Min Max)
   (let Count (need (- Max Min -1) 0)
      (for N Lst
         (inc (nth Count (- N Min -1))) )
      (make
         (map
            '((C I)
               (do (car C) (link (car I))) )
            Count
            (range Min Max) ) ) ) )

Output:

: (countingSort (5 3 1 7 4 1 1 20) 1 20)
-> (1 1 1 3 4 5 7 20)

\end{verbatim}

\section*{Sorting algorithms/Gnome sort}

\begin{verbatim}

(de gnomeSort (Lst)
   (let J (cdr Lst)
      (for (I Lst (cdr I))
         (if (>= (cadr I) (car I))
            (setq I J  J (cdr J))
            (xchg I (cdr I))
            (if (== I Lst)
               (setq I J  J (cdr J))
               (setq I (prior I Lst)) ) ) ) )
   Lst )

\end{verbatim}

\section*{Sorting algorithms/Heapsort}

\begin{verbatim}

(de heapSort (A Cnt)
   (let Cnt (length A)
      (for (Start (/ Cnt 2) (gt0 Start) (dec Start))
         (siftDown A Start (inc Cnt)) )
      (for (End Cnt (> End 1) (dec End))
         (xchg (nth A End) A)
         (siftDown A 1 End) ) )
   A )

(de siftDown (A Start End)
   (use Child
      (for (Root Start  (> End (setq Child (* 2 Root))))
         (and
            (> End (inc Child))
            (> (get A (inc Child)) (get A Child))
            (inc 'Child) )
         (NIL (> (get A Child) (get A Root)))
         (xchg (nth A Root) (nth A Child))
         (setq Root Child) ) ) )

Output:

: (heapSort (make (do 9 (link (rand 1 999)))))
-> (1 167 183 282 524 556 638 891 902)

\end{verbatim}

\section*{Sorting algorithms/Insertion sort}

\begin{verbatim}

(de insertionSort (Lst)
   (for (I (cdr Lst)  I  (cdr I))
      (for (J Lst  (n== J I)  (cdr J))
         (T (> (car J) (car I))
            (rot J (offset I J)) ) ) )
   Lst )

Output:

: (insertionSort (5 3 1 7 4 1 1 20))
-> (1 1 1 3 4 5 7 20)

\end{verbatim}

\section*{Sorting algorithms/Merge sort}

\begin{verbatim}

{IanOsgood}

\end{verbatim}

\section*{Sorting algorithms/Pancake sort}

\begin{verbatim}

(de pancake (Lst)
   (prog1 (flip Lst (index (apply max Lst) Lst))
      (for (L @  (cdr (setq Lst (cdr L)))  (cdr L))
         (con L (flip Lst (index (apply max Lst) Lst))) ) ) )

Output:

: (trace 'flip)
-> flip

: (pancake (6 7 2 1 8 9 5 3 4))
 flip : (6 7 2 1 8 9 5 3 4) 6
 flip = (9 8 1 2 7 6 5 3 4)
 flip : (8 1 2 7 6 5 3 4) 1
 flip = (8 1 2 7 6 5 3 4)
 flip : (1 2 7 6 5 3 4) 3
 flip = (7 2 1 6 5 3 4)
 flip : (2 1 6 5 3 4) 3
 flip = (6 1 2 5 3 4)
 flip : (1 2 5 3 4) 3
 flip = (5 2 1 3 4)
 flip : (2 1 3 4) 4
 flip = (4 3 1 2)
 flip : (3 1 2) 1
 flip = (3 1 2)
 flip : (1 2) 2
 flip = (2 1)
-> (9 8 7 6 5 4 3 2 1)

\end{verbatim}

\section*{Sorting algorithms/Permutation sort}

\begin{verbatim}

(de permutationSort (Lst)
   (let L Lst
      (recur (L)  # Permute
         (if (cdr L)
            (do (length L)
               (T (recurse (cdr L)) Lst)
               (rot L)
               NIL )
            (apply <= Lst) ) ) ) )

Output:

: (permutationSort (make (do 9 (link (rand 1 999)))))
-> (82 120 160 168 205 226 408 708 719)

: (permutationSort (make (do 9 (link (rand 1 999)))))
-> (108 212 330 471 667 716 739 769 938)

: (permutationSort (make (do 9 (link (rand 1 999)))))
-> (118 253 355 395 429 548 890 900 983)

\end{verbatim}

\section*{Sorting algorithms/Quicksort}

\begin{verbatim}

{IanOsgood}

\end{verbatim}

\section*{Sorting algorithms/Radix sort}

\begin{verbatim}

This is a LSD base-2 radix sort using queues:

(de radixSort (Lst)
   (let Mask 1
      (while
         (let (Pos (list NIL NIL)  Neg (list NIL NIL)  Flg)
            (for N Lst
               (queue
                  (if2 (ge0 N) (bit? Mask N)
                     (cdr Pos) Pos Neg (cdr Neg) )
                  N )
               (and (>= (abs N) Mask) (on Flg)) )
            (setq
               Lst (conc (apply conc Neg) (apply conc Pos))
               Mask (* 2 Mask) )
            Flg ) ) )
   Lst )

Output:

: (radixSort (make (do 12 (link (rand -999 999)))))
-> (-999 -930 -666 -336 -218 68 79 187 391 405 697 922)

\end{verbatim}

\section*{Sorting algorithms/Selection sort}

\begin{verbatim}

(de selectionSort (Lst)
   (map
      '((L) (and (cdr L) (xchg L (member (apply min @) L))))
      Lst )
   Lst )

\end{verbatim}

\section*{Sorting algorithms/Shell sort}

\begin{verbatim}

(de shellSort (A)
   (for (Inc (*/ (length A) 2)  (gt0 Inc)  (*/ Inc 10 22))
      (for (I Inc  (get A I)  (inc I))
         (let (Tmp @  J I)
            (while (and (>= J Inc) (> (get A (- J Inc)) Tmp))
               (set (nth A J) (get A (- J Inc)))
               (dec 'J Inc) )
            (set (nth A J) Tmp) ) ) )
   A )

Output:

: (shellSort (make (do 9 (link (rand 1 999)))))
-> (1 167 183 282 524 556 638 891 902)

: (shellSort (make (do 9 (link (rand 1 999)))))
-> (82 120 160 168 205 226 408 708 719)

: (shellSort (make (do 9 (link (rand 1 999)))))
-> (108 212 330 471 667 716 739 769 938)

\end{verbatim}

\section*{Sorting algorithms/Sleep sort}

\begin{verbatim}

# Sleeping in main process
(de sleepSort (Lst)
   (make
      (for (I . N) Lst
         (task (- I) (* N 100)  N N  I I
            (link N)
            (pop 'Lst)
            (task (- I)) ) )
      (wait NIL (not Lst)) ) )

# Sleeping in child processes
(de sleepSort (Lst)
   (make
      (for N Lst
         (task (pipe (wait (* N 100))) N N
            (link N)
            (pop 'Lst)
            (task (close @)) ) )
      (wait NIL (not Lst)) ) )

Output in both cases:

: (sleepSort (3 1 4 1 5 9 2 6 5))
-> (1 1 2 3 4 5 5 6 9)

# Just printing (no sorted result list)

Basically the C code.

(for N (3 1 4 1 5 9 2 6 5)
   (unless (fork)
      (call 'sleep N)
      (msg N)
      (bye) ) )

Output:

1
1
2
3
4
5
5
6
9

\end{verbatim}

\section*{Sorting algorithms/Stooge sort}

\begin{verbatim}

(de stoogeSort (L N)
   (default N (length L))
   (let P (nth L N)
      (when (> (car L) (car P))
         (xchg L P) ) )
   (when (> N 2)
      (let D (/ N 3)
         (stoogeSort L (- N D))
         (stoogeSort (nth L (inc D)) (- N D))
         (stoogeSort L (- N D)) ) )
   L )

Test:

: (apply < (stoogeSort (make (do 100 (link (rand))))))
-> T

\end{verbatim}

\section*{Sorting algorithms/Strand sort}

\begin{verbatim}

(de strandSort (Lst)
   (let Res NIL  # Result list
      (while Lst
         (let Sub (circ (car Lst))  # Build sublist as fifo
            (setq
               Lst (filter
                  '((X)
                     (or
                        (> (car Sub) X)
                        (nil (fifo 'Sub X)) ) )
                  (cdr Lst) )
               Res (make
                  (while (or Res Sub)  # Merge
                     (link
                        (if2 Res Sub
                           (if (>= (car Res) (cadr Sub))
                              (fifo 'Sub)
                              (pop 'Res) )
                           (pop 'Res)
                           (fifo 'Sub) ) ) ) ) ) ) )
      Res ) )

Test:

: (strandSort (3 1 5 4 2))
-> (1 2 3 4 5)

: (strandSort (3 abc 1 (d e f) 5 T 4 NIL 2))
-> (NIL 1 2 3 4 5 abc (d e f) T)

\end{verbatim}

\section*{Soundex}

\begin{verbatim}

(de soundex (Str)
   (pack
      (pad -4
         (cons
            (uppc (char (char Str)))
            (head 3
               (let Last NIL
                  (extract
                     '((C)
                        (and
                           (setq C
                              (case (uppc C)
                                 (`(chop "BFPV") "1")
                                 (`(chop "CGJKQSXZ") "2")
                                 (("D" "T") "3")
                                 ("L" "4")
                                 (("M" "N") "5")
                                 ("R" "6") ) )
                           (<> Last C)
                           (setq Last C) ) )
                     (cdr (chop Str)) ) ) ) ) ) ) )

Output:

: (mapcar soundex '("Soundex" "Example" "Sownteks" "Ekzampul"))
-> ("S532" "E251" "S532" "E251")

\end{verbatim}

\section*{Special variables}

\begin{verbatim}

PicoLisp has no special variables, but some naming conventions concerning
the "meaning" of a variable's (i.e. symbol's) value:

- Global variables start with an asterisk '*'
- Functions and other global symbols start with a lower case letter
- Locally bound symbols start with an upper case letter
- Local functions start with an underscore '_'
- Classes start with a plus-sign '+', where the first letter
   - is in lower case for abstract classes
   - and in upper case for normal classes
- Methods end with a right arrow '>'
- Class variables may be indicated by an upper case letter

For historical reasons, the global constant symbols 'T' and 'NIL' do not obey
these rules, and are written in upper case.

\end{verbatim}

\section*{Speech synthesis}

\begin{verbatim}

(call 'espeak "This is an example of speech synthesis.")

\end{verbatim}

\section*{Special characters}

\begin{verbatim}

Markup:
   () []    List
   .        Dotted pair (when surounded by white space)
   "        Transient symbol (string)
   {}       External symbol (database object)
   \        Escape for following character
   #        Comment line
   #{ }#    Comment block


Read macros:
   '        The 'quote' function
   `        Evaluate and insert a list element
   ~        Evaluate and splice a partial list
   ,        Indexed reference

Within strings:
   ^        ASCII control character
   \        At end of line: Continue on next line, skipping white space

\end{verbatim}

\section*{Spiral matrix}

\begin{verbatim}

This example uses 'grid' from "lib/simul.l", which maintains a two-dimensional
structure and is normally used for simulations and board games.

(load "@lib/simul.l")

(de spiral (N)
   (prog1 (grid N N)
      (let (Dir '(north east south west .)  This 'a1)
         (for Val (* N N)
            (=: val Val)
            (setq This
               (or
                  (with ((car Dir) This)
                     (unless (: val) This) )
                  (with ((car (setq Dir (cdr Dir))) This)
                     (unless (: val) This) ) ) ) ) ) ) )

(mapc
   '((L)
      (for This L (prin (align 3 (: val))))
      (prinl) )
   (spiral 5) )

Output:

  1  2  3  4  5
 16 17 18 19  6
 15 24 25 20  7
 14 23 22 21  8
 13 12 11 10  9

\end{verbatim}

\section*{Stable marriage problem (!)}

\begin{verbatim}

(setq
   *Boys (list
      (de abe  abi eve cath ivy jan dee fay bea hope gay)
      (de bob  cath hope abi dee eve fay bea jan ivy gay)
      (de col  hope eve abi dee bea fay ivy gay cath jan)
      (de dan  ivy fay dee gay hope eve jan bea cath abi)
      (de ed   jan dee bea cath fay eve abi ivy hope gay)
      (de fred bea abi dee gay eve ivy cath jan hope fay)
      (de gav  gay eve ivy bea cath abi dee hope jan fay)
      (de hal  abi eve hope fay ivy cath jan bea gay dee)
      (de ian  hope cath dee gay bea abi fay ivy jan eve)
      (de jon  abi fay jan gay eve bea dee cath ivy hope) )
   *Girls (list
      (de bi   bob fred jon gav ian abe dan ed col hal)
      (de bea  bob abe col fred gav dan ian ed jon hal)
      (de cath fred bob ed gav hal col ian abe dan jon)
      (de dee  fred jon col abe ian hal gav dan bob ed)
      (de eve  jon hal fred dan abe gav col ed ian bob)
      (de fay  bob abe ed ian jon dan fred gav col hal)
      (de gay  jon gav hal fred bob abe col ed dan ian)
      (de hope gav jon bob abe ian dan hal ed col fred)
      (de ivy  ian col hal gav fred bob abe ed jon dan)
      (de jan  ed hal gav abe bob jon col ian fred dan) )
   *Couples NIL )

(bind *Boys
   (while
      (find
         '((Boy) (and (val Boy) (not (asoq Boy *Couples))))
         *Boys )
      (let (Boy @  Girl (pop Boy)  Pair (find '((P) (== Girl (cdr P))) *Couples))
         (nond
            (Pair (push '*Couples (cons Boy Girl)))   # Girl is free
            ((memq Boy (memq (car Pair) (val Girl)))  # Girl prefers Boy
               (set Pair Boy) ) ) ) ) )

(for Pair *Couples
   (prinl (cdr Pair) " is engaged to " (car Pair)) )

(de checkCouples ()
   (unless
      (filter
         '((Pair)
            (let (Boy (car Pair)  Girl (cdr Pair))
               (find
                  '((B)
                     (and
                        (memq Boy (cdr (memq B (val Girl))))  # Girl prefers B
                        (memq
                           (cdr (asoq B *Couples))            # and B prefers Girl
                           (cdr (memq Girl (val B))) )
                        (prinl
                           Girl " likes " B " better than " Boy " and "
                           B " likes " Girl " better than "
                           (cdr (asoq B *Couples)) ) ) )
                  (val Girl) ) ) )
         *Couples )
      (prinl "All marriages are stable") ) )

(checkCouples)
(prinl)
(prinl "Engage fred with abi and jon with bea")
(con (asoq 'fred *Couples) 'abi)
(con (asoq 'jon *Couples) 'bea)
(checkCouples)

Output:

dee is engaged to col
fay is engaged to dan
eve is engaged to hal
gay is engaged to gav
bea is engaged to fred
jan is engaged to ed
ivy is engaged to abe
hope is engaged to ian
cath is engaged to bob
abi is engaged to jon
All marriages are stable

Engage fred with abi and jon with bea
fay likes jon better than dan and jon likes fay better than bea
eve likes jon better than hal and jon likes eve better than bea
gay likes jon better than gav and jon likes gay better than bea
bea likes fred better than jon and fred likes bea better than abi

\end{verbatim}

\section*{Stack}

\begin{verbatim}

The built-in functions [http://software-lab.de/doc/refP.html#push push] and
[http://software-lab.de/doc/refP.html#pop pop] are used to maintain a stack (of

(push 'Stack 3)
(push 'Stack 2)
(push 'Stack 1)

: Stack
-> (1 2 3)

: (pop 'Stack)
-> 1

: Stack
-> (2 3)

: (set 'Stack)  # empty
-> NIL

: Stack
-> NIL

\end{verbatim}

\section*{Stack traces}

\begin{verbatim}

PicoLisp doesn't keep full backtrace information at runtime. This is for
performance reasons. However, existing variable bindings (environments) can be
inspected with the '[http://software-lab.de/doc/refE.html#env env]' function, so
this can be used to build your own stack frames.

The following is analog to (though simpler than) the built-in
'[http://software-lab.de/doc/refT.html#trace trace]' mechanism. The function
'\$\$' (corresponds to '[http://software-lab.de/doc/ref_.html#\$ \$]' for tracing)
is inserted by 'stackAll' into every function and method definition
(corresponds to '[http://software-lab.de/doc/refT.html#traceAll traceAll]').
Then, when stopping at a '[http://software-lab.de/doc/refD.html#debug debug]'
breakpoint or an error handler, 'dumpStack' can be used to inspect the stack
contents.

As this mechanism uses 'let' to hold the stack frames, it is robust also across
catch/throw, coroutines and error handling.

(off "Stack")

(de \$\$ "Prg"
   (let "Stack" (cons (cons (car "Prg") (env)) "Stack")  # Build stack frame
      (set "Stack"
         (delq (asoq '"Stack" (car "Stack"))  # Remove self-created entries
            (delq (asoq '"Prg" (car "Stack"))
               (car "Stack") ) ) )
      (run (cdr "Prg")) ) )  # Run body

(de stackAll (Excl)
   (let *Dbg NIL
      (for "X" (all)
         (or
            (memq "X" Excl)
            (memq "X" '(\$\$ @ @@ @@@))
            (= `(char "*") (char "X"))
            (cond
               ((= `(char "+") (char "X"))
                  (for "Y" (pair (val "X"))
                     (and
                        (pair "Y")
                        (fun? (cdr "Y"))
                        (unless (== '\$\$ (caaddr "Y"))
                           (con (cdr "Y")
                              (list
                                 (cons '\$\$ (cons (car "Y" "X") (cddr "Y"))) ) ) ) ) ) )
               ((pair (getd "X"))
                  (let "Y" @
                     (unless (== '\$\$ (caadr "Y"))
                        (con "Y"
                           (list (cons '\$\$ "X" (cdr "Y"))) ) ) ) ) ) ) ) ) )

(de dumpStack ()
   (more (reverse (cdr "Stack")))
   T )

Test:

(de foo (A B)
   (let C 3
      (bar (inc 'A) (inc 'B) (inc 'C)) ) )

(de bar (A D E)
   (let (A 7  B 8  C 9)
      (! println A B C) ) )  # Set a breakpoint before (println A B C)

(stackAll)

: (foo 1 2)  # Call 'foo'
(println A B C)  # Stopped at breakpoint in 'bar'
! (dumpStack)  # Dump stack history
(foo (A . 1) (B . 2) (@ . T))  # Hit <enter> on each line to continue
(bar (B . 3) (C . 4) (A . 2) (D . 3) (E . 4) (@ . T))
-> T
!  # Hit <enter> to continue execution
7 8 9  # Output of (println A B C)
-> 9
:

\end{verbatim}

\section*{Stair-climbing puzzle}

\begin{verbatim}

(de stepUp ()
   (until (step1)  # ('step1', because 'step' is a system function)
      (stepUp) ) )

\end{verbatim}

\section*{Standard deviation}

\begin{verbatim}

(scl 2)

(de stdDev ()
   (curry ((Data)) (N)
      (push 'Data N)
      (let (Len (length Data)  M (*/ (apply + Data) Len))
         (sqrt
            (*/
               (sum
                  '((N) (*/ (- N M) (- N M) 1.0))
                  Data )
               1.0
               Len )
            T ) ) ) )

(let Fun (stdDev)
   (for N (2.0 4.0 4.0 4.0 5.0 5.0 7.0 9.0)
      (prinl (format N *Scl) " -> " (format (Fun N) *Scl)) ) )

Output:

2.00 -> 0.00
4.00 -> 1.00
4.00 -> 0.94
4.00 -> 0.87
5.00 -> 0.98
5.00 -> 1.00
7.00 -> 1.40
9.00 -> 2.00

\end{verbatim}

\section*{State name puzzle}

\begin{verbatim}

(setq *States
   (group
      (mapcar '((Name) (cons (clip (sort (chop (lowc Name)))) Name))
         (quote
            "Alabama" "Alaska" "Arizona" "Arkansas"
            "California" "Colorado" "Connecticut"
            "Delaware"
            "Florida" "Georgia" "Hawaii"
            "Idaho" "Illinois" "Indiana" "Iowa"
            "Kansas" "Kentucky" "Louisiana"
            "Maine" "Maryland" "Massachusetts" "Michigan"
            "Minnesota" "Mississippi" "Missouri" "Montana"
            "Nebraska" "Nevada" "New Hampshire" "New Jersey"
            "New Mexico" "New York" "North Carolina" "North Dakota"
            "Ohio" "Oklahoma" "Oregon"
            "Pennsylvania" "Rhode Island"
            "South Carolina" "South Dakota" "Tennessee" "Texas"
            "Utah" "Vermont" "Virginia"
            "Washington" "West Virginia" "Wisconsin" "Wyoming"
            "New Kory" "Wen Kory" "York New" "Kory New" "New Kory" ) ) ) )

(extract
   '((P)
      (when (cddr P)
         (mapcar
            '((X)
               (cons
                  (cadr (assoc (car X) *States))
                  (cadr (assoc (cdr X) *States)) ) )
            (cdr P) ) ) )
   (group
      (mapcon
         '((X)
            (extract
               '((Y)
                  (cons
                     (sort (conc (copy (caar X)) (copy (car Y))))
                     (caar X)
                     (car Y) ) )
               (cdr X) ) )
         *States ) ) )

Output:

-> ((("North Carolina" . "South Dakota") ("North Dakota" . "South Carolina")))

\end{verbatim}

\section*{Statistics/Basic}

\begin{verbatim}

The following has no limit on the number of samples. The 'statistics' function
accepts an executable body 'Prg', which it calls repeatedly to get the samples.

(scl 6)

(de statistics (Cnt . Prg)
   (prinl Cnt " numbers")
   (let (Sum 0  Sqr 0  Hist (need 10 NIL 0))
      (do Cnt
         (let N (run Prg 1)  # Get next number
            (inc 'Sum N)
            (inc 'Sqr (*/ N N 1.0))
            (inc (nth Hist (inc (/ N 0.1)))) ) )
      (let M (*/ Sum Cnt)
         (prinl "Mean:   " (round M))
         (prinl "StdDev: "
            (round
               (sqrt
                  (* 1.0
                     (- (*/ Sqr Cnt) (*/ M M 1.0)) ) ) ) ) )
      (for (I . H) Hist
         (prin (format I 1) " ")
         (do (*/ H 400 Cnt) (prin '=))
         (prinl) ) ) )

Test:

(statistics 100
   (rand 0 (dec 1.0)) )
(prinl)

(statistics 10000
   (rand 0 (dec 1.0)) )
(prinl)

(statistics 1000000
   (rand 0 (dec 1.0)) )
(prinl)

Output:
100 numbers
Mean:   0.501
StdDev: 0.284
0.1 ========================================
0.2 ====================================
0.3 ====================================================
0.4 ========================
0.5 ========================
0.6 ================================================================
0.7 ========================================================
0.8 ====================================
0.9 ========================
1.0 ============================================

10000 numbers
Mean:   0.501
StdDev: 0.288
0.1 =======================================
0.2 ========================================
0.3 =======================================
0.4 =========================================
0.5 =========================================
0.6 ========================================
0.7 =========================================
0.8 ========================================
0.9 ========================================
1.0 ========================================

1000000 numbers
Mean:   0.500
StdDev: 0.289
0.1 ========================================
0.2 ========================================
0.3 ========================================
0.4 ========================================
0.5 ========================================
0.6 ========================================
0.7 ========================================
0.8 ========================================
0.9 ========================================
1.0 ========================================

\end{verbatim}

\section*{Stem-and-leaf plot}

\begin{verbatim}

(de *Data
   12 127 28 42 39 113 42 18 44 118 44 37 113 124 37 48 127 36
   29 31 125 139 131 115 105 132 104 123 35 113 122 42 117 119
   58 109 23 105 63 27 44 105 99 41 128 121 116 125 32 61 37 127
   29 113 121 58 114 126 53 114 96 25 109 7 31 141 46 13 27 43
   117 116 27 7 68 40 31 115 124 42 128 52 71 118 117 38 27 106
   33 117 116 111 40 119 47 105 57 122 109 124 115 43 120 43 27
   27 18 28 48 125 107 114 34 133 45 120 30 127 31 116 146 )

(let L
   (group
      (mapcar
         '((N)
            (cons
               (or (format (head -1 (setq N (chop N)))) 0)
               (last N) ) )
         (sort *Data) ) )
   (for I (range (caar L) (car (last L)))
      (prinl (align 3 I) " | " (glue " " (cdr (assoc I L)))) ) )

Output:

  0 | 7 7
  1 | 2 3 8 8
  2 | 3 5 7 7 7 7 7 7 8 8 9 9
  3 | 0 1 1 1 1 2 3 4 5 6 7 7 7 8 9
  4 | 0 0 1 2 2 2 2 3 3 3 4 4 4 5 6 7 8 8
  5 | 2 3 7 8 8
  6 | 1 3 8
  7 | 1
  8 |
  9 | 6 9
 10 | 4 5 5 5 5 6 7 9 9 9
 11 | 1 3 3 3 3 4 4 4 5 5 5 6 6 6 6 7 7 7 7 8 8 9 9
 12 | 0 0 1 1 2 2 3 4 4 4 5 5 5 6 7 7 7 7 8 8
 13 | 1 2 3 9
 14 | 1 6

\end{verbatim}

\section*{Straddling checkerboard}

\begin{verbatim}

(de *Straddling
   (NIL  "H"  "O"  "L"  NIL  "M"  "E"  "S"  NIL  "R"  "T")
   ("3"  "A"  "B"  "C"  "D"  "F"  "G"  "I"  "J"  "K"  "N")
   ("7"  "P"  "Q"  "U"  "V"  "W"  "X"  "Y"  "Z"  "."  "/")
   ("79" "0"  "1"  "2"  "3"  "4"  "5"  "6"  "7"  "8"  "9") )

(de straddle (Str)
   (pack
      (mapcar
         '((C)
            (pick
               '((L)
                  (and
                     (index C (cdr L))
                     (cons (car L) (dec @)) ) )
               *Straddling ) )
         (chop (uppc Str)) ) ) )

(de unStraddle (Str)
   (pack
      (make
         (for (L (chop Str)  L)
            (let C (pop 'L)
               (setq C
                  (if (assoc C *Straddling)
                     (get (cdr @) (inc (format (pop 'L))))
                     (get (cdar *Straddling) (inc (format C))) ) )
               (link (if (= "/" C) (pop 'L) C)) ) ) ) ) )

Output:

: (straddle "One night-it was on the twentieth of March, 1888-I was returning")
-> "139539363509369743061399059745399365901344308320791798798798367430685972839363935"

: (unStraddle @)
-> "ONENIGHTITWASONTHETWENTIETHOFMARCH1888IWASRETURNING"

\end{verbatim}

\section*{String case}

\begin{verbatim}

(let Str "alphaBETA"
   (prinl (uppc Str))
   (prinl (lowc Str)) )

\end{verbatim}

\section*{String concatenation}

\begin{verbatim}

(let Str1 "First text"
   (prinl Str1 " literal")
   (let Str2 (pack Str1 " literal")
      (prinl Str2) ) )

\end{verbatim}

\section*{String interpolation (included)}

\begin{verbatim}

(let Extra "little"
   (prinl (text "Mary had a @1 lamb." Extra)) )

\end{verbatim}

\section*{String length}

\begin{verbatim}

(let Str "møøse"
   (prinl "Character Length of \"" Str "\" is " (length Str))
   (prinl "Byte Length of \"" Str "\" is " (size Str)) )

Output:

Character Length of "møøse" is 5
Byte Length of "møøse" is 7
-> 7

\end{verbatim}

\section*{Strip a set of characters from a string}

\begin{verbatim}

(de strDiff (Str1 Str2)
   (pack (diff (chop Str1) (chop Str2))) )

Test:

: (strDiff "She was a soul stripper. She took my heart!" "aei")
-> "Sh ws  soul strppr. Sh took my hrt!"

\end{verbatim}

\section*{Strip block comments}

\begin{verbatim}

(in "sample.txt"
   (while (echo "/*")
      (out "/dev/null" (echo "*/")) ) )

Output:


   function subroutine() {
    a =  b + c ;
   }



    function something() {
    }

\end{verbatim}

\section*{Strip comments from a string}

\begin{verbatim}

(for Str '("apples, pears # and bananas" "apples, pears ; and bananas")
   (prinl (car (split (chop Str) "#" ";"))) )

Output:

apples, pears
apples, pears

\end{verbatim}

\section*{Strip control codes and extended characters from a string}

\begin{verbatim}

Control characters in strings are written with a hat (^) in PicoLisp.
^? is the DEL character.

(de stripCtrl (Str)
   (pack
      (filter
         '((C)
            (nor (= "^?" C) (> " " C "^A")) )
         (chop Str) ) ) )

(de stripCtrlExt (Str)
   (pack
      (filter
         '((C) (> "^?" C "^_"))
         (chop Str) ) ) )

Test:

: (char "^?")
-> 127

: (char "^_")
-> 31

: (stripCtrl "^I^M a b c^? d äöüß")
-> " a b c d äöüß"

: (stripCtrlExt "^I^M a b c^? d äöüß")
-> " a b c d "

\end{verbatim}

\section*{Strip whitespace from a string/Top and tail}

\begin{verbatim}

(de trimLeft (Str)
   (pack (flip (trim (flip (chop Str))))) )

(de trimRight (Str)
   (pack (trim (chop Str))) )

(de trimBoth (Str)
   (pack (clip (chop Str))) )

Test:

: (trimLeft " ^G ^I trimmed left ^L ")
-> "trimmed left ^L "

: (trimRight " ^G ^I trimmed right ^L ")
-> " ^G ^I trimmed right"

: (trimBoth " ^G ^I trimmed both ^L ")
-> "trimmed both"

\end{verbatim}

\section*{Subset sum problem}

\begin{verbatim}

(de *Words
   (alliance . -624) (archbishop . -915) (balm . 397) (bonnet . 452)
   (brute . 870) (centipede . -658) (cobol . 362) (covariate . 590)
   (departure . 952) (deploy . 44) (diophantine . 645) (efferent . 54)
   (elysee . -326) (eradicate . 376) (escritoire . 856) (exorcism . -983)
   (fiat . 170) (filmy . -874) (flatworm . 503) (gestapo . 915)
   (infra . -847) (isis . -982) (lindholm . 999) (markham . 475)
   (mincemeat . -880) (moresby . 756) (mycenae . 183) (plugging . -266)
   (smokescreen . 423) (speakeasy . -745) (vein . 813) )

Minimal brute force solution:

(load "@lib/simul.l")  # For 'subsets'

(pick
   '((N)
      (find '((L) (=0 (sum cdr L)))
         (subsets N *Words) ) )
   (range 1 (length *Words)) )

Output:

-> ((archbishop . -915) (gestapo . 915))

\end{verbatim}

\section*{Substring}

\begin{verbatim}

(let Str (chop "This is a string")
   (prinl (head 4 (nth Str 6)))        # From 6 of 4 length
   (prinl (nth Str 6))                 # From 6 up to the end
   (prinl (head -1 Str))               # Minus last character
   (prinl (head 8 (member "s" Str)))   # From character "s" of length 8
   (prinl                              # From "isa" of length 8
      (head 8
         (seek '((S) (pre? "is a" S)) Str) ) ) )

Output:

is a
is a string
This is a strin
s is a s
is a str

\end{verbatim}

\section*{Subtractive generator}

\begin{verbatim}

Using a circular list (as a true "ring" buffer).

(setq
   *Bentley (apply circ (need 55))
   *Bentley2 (nth *Bentley 32) )

(de subRandSeed (S)
   (let (N 1  P (nth *Bentley 55))
      (set P S)
      (do 54
         (set (setq P (nth P 35)) N)
         (when (lt0 (setq N (- S N)))
            (inc 'N 1000000000) )
         (setq S (car P)) ) )
   (do 165 (subRand)) )

(de subRand ()
   (when (lt0 (dec *Bentley (pop '*Bentley2)))
      (inc *Bentley 1000000000) )
   (pop '*Bentley) )

Test:

(subRandSeed 292929)
(do 7 (println (subRand)))

Output:

467478574
512932792
539453717
20349702
615542081
378707948
933204586

\end{verbatim}

\section*{Sudoku}

\begin{verbatim}

(load "lib/simul.l")

### Fields/Board ###
# val lst

(setq
   *Board (grid 9 9)
   *Fields (apply append *Board) )

# Init values to zero (empty)
(for L *Board
   (for This L
      (=: val 0) ) )

# Build lookup lists
(for (X . L) *Board
   (for (Y . This) L
      (=: lst
         (make
            (let A (* 3 (/ (dec X) 3))
               (do 3
                  (inc 'A)
                  (let B (* 3 (/ (dec Y) 3))
                     (do 3
                        (inc 'B)
                        (unless (and (= A X) (= B Y))
                           (link
                              (prop (get *Board A B) 'val) ) ) ) ) ) )
            (for Dir '(`west `east `south `north)
               (for (This (Dir This)  This  (Dir This))
                  (unless (memq (:: val) (made))
                     (link (:: val)) ) ) ) ) ) ) )

# Cut connections (for display only)
(for (X . L) *Board
   (for (Y . This) L
      (when (member X (3 6))
         (con (car (val This))) )
      (when (member Y (4 7))
         (set (cdr (val This))) ) ) )

# Display board
(de display ()
   (disp *Board 0
      '((This)
         (if (=0 (: val))
            "   "
            (pack " " (: val) " ") ) ) ) )

# Initialize board
(de main (Lst)
   (for (Y . L) Lst
      (for (X . N) L
         (put *Board X (- 10 Y) 'val N) ) )
   (display) )

# Find solution
(de go ()
   (unless
      (recur (*Fields)
         (with (car *Fields)
            (if (=0 (: val))
               (loop
                  (NIL
                     (or
                        (assoc (inc (:: val)) (: lst))
                        (recurse (cdr *Fields)) ) )
                  (T (= 9 (: val)) (=: val 0)) )
               (recurse (cdr *Fields)) ) ) )
      (display) ) )

(main
   (quote
      (5 3 0 0 7 0 0 0 0)
      (6 0 0 1 9 5 0 0 0)
      (0 9 8 0 0 0 0 6 0)
      (8 0 0 0 6 0 0 0 3)
      (4 0 0 8 0 3 0 0 1)
      (7 0 0 0 2 0 0 0 6)
      (0 6 0 0 0 0 2 8 0)
      (0 0 0 4 1 9 0 0 5)
      (0 0 0 0 8 0 0 7 9) ) )

Output:

   +---+---+---+---+---+---+---+---+---+
 9 | 5   3     |     7     |           |
   +   +   +   +   +   +   +   +   +   +
 8 | 6         | 1   9   5 |           |
   +   +   +   +   +   +   +   +   +   +
 7 |     9   8 |           |     6     |
   +---+---+---+---+---+---+---+---+---+
 6 | 8         |     6     |         3 |
   +   +   +   +   +   +   +   +   +   +
 5 | 4         | 8       3 |         1 |
   +   +   +   +   +   +   +   +   +   +
 4 | 7         |     2     |         6 |
   +---+---+---+---+---+---+---+---+---+
 3 |     6     |           | 2   8     |
   +   +   +   +   +   +   +   +   +   +
 2 |           | 4   1   9 |         5 |
   +   +   +   +   +   +   +   +   +   +
 1 |           |     8     |     7   9 |
   +---+---+---+---+---+---+---+---+---+
     a   b   c   d   e   f   g   h   i

(go)
   +---+---+---+---+---+---+---+---+---+
 9 | 5   3   4 | 6   7   8 | 9   1   2 |
   +   +   +   +   +   +   +   +   +   +
 8 | 6   7   2 | 1   9   5 | 3   4   8 |
   +   +   +   +   +   +   +   +   +   +
 7 | 1   9   8 | 3   4   2 | 5   6   7 |
   +---+---+---+---+---+---+---+---+---+
 6 | 8   5   9 | 7   6   1 | 4   2   3 |
   +   +   +   +   +   +   +   +   +   +
 5 | 4   2   6 | 8   5   3 | 7   9   1 |
   +   +   +   +   +   +   +   +   +   +
 4 | 7   1   3 | 9   2   4 | 8   5   6 |
   +---+---+---+---+---+---+---+---+---+
 3 | 9   6   1 | 5   3   7 | 2   8   4 |
   +   +   +   +   +   +   +   +   +   +
 2 | 2   8   7 | 4   1   9 | 6   3   5 |
   +   +   +   +   +   +   +   +   +   +
 1 | 3   4   5 | 2   8   6 | 1   7   9 |
   +---+---+---+---+---+---+---+---+---+
     a   b   c   d   e   f   g   h   i

\end{verbatim}

\section*{Sum and product of an array}

\begin{verbatim}

(let Data (1 2 3 4 5)
   (cons
      (apply + Data)
      (apply * Data) ) )

Output:

(15 . 120)

\end{verbatim}

\section*{Sum digits of an integer}

\begin{verbatim}

(de sumDigits (N Base)
   (or
      (=0 N)
      (+ (\% N Base) (sumDigits (/ N Base) Base)) ) )

Test:

: (sumDigits 1 10)
-> 1

: (sumDigits 1234 10)
-> 10

: (sumDigits (hex "fe") 16)
-> 29

: (sumDigits (hex "f0e") 16)
-> 29

\end{verbatim}

\section*{Sum of a series}

\begin{verbatim}

(scl 9)  # Calculate with 9 digits precision

(let S 0
   (for I 1000
      (inc 'S (*/ 1.0 (* I I))) )
   (prinl (round S 6)) )  # Round result to 6 digits

1.643935

\end{verbatim}

\section*{Sum of squares}

\begin{verbatim}

: (sum '((N) (* N N)) (3 1 4 1 5 9))
-> 133
: (sum '((N) (* N N)) ())
-> 0

\end{verbatim}

\section*{Symmetric difference}

\begin{verbatim}

(de symdiff (A B)
   (uniq (conc (diff A B) (diff B A))) )

Output:

(symdiff '(John Serena Bob Mary Serena) '(Jim Mary John Jim Bob))
-> (Serena Jim)

\end{verbatim}

\section*{Synchronous concurrency}

\begin{verbatim}

PicoLisp has no threads, but synchronous background tasks and asynchronous
signal handlers, or coroutines.

# Using background tasks and signals

The following two tasks communicate via UDP, so in fact they don't need to run
within the same process and not even the same machine. "input.txt" would rather
be a device (like a named pipe or socket) than a plain file.

# Reading task (synchronous)
(task (open "input.txt")
   (let Fd @
      (if (in Fd (line T))             # More lines?
         (udp "localhost" 4444 @)      # Yes: Send next line
         (task (port T 4445)           # Else install handler
            (prinl (udp @) " lines")   # to receive and print count
            (task (close @)) )
         (udp "localhost" 4444 T)      # Send 'T' for "Done"
         (task (close Fd)) ) ) )       # Stop the task

# Printing task (asynchronous)
(sigio (setq "Sock" (port T 4444))
   (job '((Cnt . 0))
      (let? X (udp "Sock")
         (if (=T X)                    # Done?
            (prog
               (udp "localhost" 4445 Cnt) # Yes: Send count
               (sigio (close "Sock")) )   # and stop the task
            (println X)                # Else print line to stdout
            (inc 'Cnt) ) ) ) )         # and increment count

# Using coroutines

Coroutines are available only in the 64-bit version.

(co 'unit1
   (yield)                       # Allow 'unit2' to start
   (in "input.txt"               # Read the file
      (while (line T)            # Send each line
         (yield @ 'unit2) ) )    # to 'unit2'
   (prinl
      (yield NIL 'unit2)         # Send 'NIL' for "Done", receive count
      " lines" ) )

(co 'unit2
   (let Cnt 0                    # Init counter
      (while (yield NIL 'unit1)  # Receive line
         (println @)             # Print it
         (inc 'Cnt) )            # Increment count
      (yield Cnt 'unit1) ) )     # Send count to 'unit1'

\end{verbatim}

\section*{System time}

\begin{verbatim}

(stamp)

Output:

-> "2010-02-19 15:14:06"

\end{verbatim}

\chapter{Rosetta Code Tasks starting with T}

\section*{Table creation}

\begin{verbatim}

(scl 2)

(class +Account +Entity)
(rel id        (+Key +Number))
(rel created   (+Date))
(rel active    (+Bool))
(rel username  (+Key +String))
(rel balance   (+Number) 2)
(rel age       (+Number))
(rel notes     (+Blob))

(pool "account.db")  # Create database

(new! '(+Account)
   'id 12345
   'username "John Doe"
   'balance 77.22
   'created (date 2009 5 13) )

(new! '(+Account)
   'id 12346
   'username "Jane Miller"
   'active T
   'created (date 2009 5 14)
   'balance 123.75 )

(let Fmt (-13 -10 -9 -11 10)
   (tab Fmt "account_id" "created" "active" "username" "balance")
   (for This (collect 'id '+Account)
      (tab Fmt
         (: id)
         (dat\$ (: created))
         (if (: active) "Yes" "No")
         (: username)
         (money (: balance)) ) ) )

Output:

account_id   created   active   username      balance
12345        20090513  No       John Doe        77.22
12346        20090514  Yes      Jane Miller    123.75

\end{verbatim}

\section*{Table creation/Postal addresses}

\begin{verbatim}

PicoLisp has built-in database functionality, in the form of (non-relational)
entity/relations built on top of persistent objects (so-called external symbols)

Define an "address" entity, and create the database:

(class +Adr +Entity)
(rel nm (+Sn +Idx +String))            # Name [Soundex index]
(rel str (+String))                    # Street
(rel zip (+Ref +String))               # ZIP [Non-unique index]
(rel cit (+Fold +Idx +String))         # City [Folded substring index]
(rel st (+String))                     # State
(rel tel (+Fold +Ref +String))         # Phone [Folded non-unique index]
(rel em (+Ref +String))                # EMail [Non-unique index]
(rel txt (+Blob))                      # Memo
(rel jpg (+Blob))                      # Photo

(pool "address.db")  # Create database

Create a first entry, and show it:

(show
   (new! '(+Adr)  # Create a record
      'nm "FSF Inc."
      'str "51 Franklin St"
      'st "Boston, MA"
      'zip "02110-1301" ) )

Output:

{2} (+Adr)
   zip "02110-1301"
   st "Boston, MA"
   str "51 Franklin St"
   nm "FSF Inc."

Interactive "select":

(select nm zip +Adr nm "FSF")  # Select name, zip from Adr where name = FSF*
Output:
"FSF Inc." "02110-1301" {2}

\end{verbatim}

\section*{Take notes on the command line}

\begin{verbatim}

#!/usr/bin/picolisp /usr/lib/picolisp/lib.l

(load "@lib/misc.l")
(if (argv)
   (out "+notes.txt" (prinl (stamp) "^J^I" (glue " " @)))
   (and (info "notes.txt") (in "notes.txt" (echo))) )
(bye)

\end{verbatim}

\section*{Terminal Control/Determine the height and width of the terminal window}

\begin{verbatim}

(setq
   Width (in '(tput cols) (read))
   Height (in '(tput lines) (read)) )

\end{verbatim}

\section*{Terminal control/Coloured text}

\begin{verbatim}

{{trans|UNIX Shell}}

(unless (member (sys "TERM") '("linux" "xterm" "rxvt"))
   (quit "This application requires a colour terminal") )

# Coloured text
(for X '((1 . "Red") (4 . "Blue") (3 . "Yellow"))
   (call 'tput "setaf" (car X))
   (prinl (cdr X)) )

# Blinking
(out '(tput "-S")
   (prinl "setab 1^Jsetaf 3^Jblink") )
(prin "Flashing text")

(call 'tput 'sgr0)   # reset
(prinl)

\end{verbatim}

\section*{Terminal control/Cursor movement}

\begin{verbatim}

(call 'tput "cub1")                                # one position to the left
(call 'tput "cuf1")                                # one position to the right
(call 'tput "cuu1")                                # up one line
(call 'tput "cud1")                                # down one line
(call 'tput "cr")                                  # beginning of the line
(call 'tput "hpa" (sys "COLUMNS"))                 # end of the line
(call 'tput "home")                                # top left corner
(call 'tput "cup" (sys "LINES") (sys "COLUMNS"))   # bottom right corner

\end{verbatim}

\section*{Terminal control/Preserve screen}

\begin{verbatim}

#!/usr/bin/picolisp /usr/lib/picolisp/lib.l

(call 'tput "smcup")
(prinl "something")
(wait 3000)
(call 'tput "rmcup")

(bye)

\end{verbatim}

\section*{Terminal Control/Unicode output}

\begin{verbatim}

(if (sub? "UTF-8" (or (sys "LC_ALL") (sys "LC_CTYPE") (sys "LANG")))
   (prinl (char (hex "25b3")))
   (quit "UTF-8 capable terminal required") )

\end{verbatim}

\section*{Ternary logic}

\begin{verbatim}

In addition for the standard T (for "true") and NIL (for "false") we define 0
(zero, for "maybe").

(de 3not (A)
   (or (=0 A) (not A)) )

(de 3and (A B)
   (cond
      ((=T A) B)
      ((=0 A) (and B 0)) ) )

(de 3or (A B)
   (cond
      ((=T A) T)
      ((=0 A) (or (=T B) 0))
      (T B) ) )

(de 3impl (A B)
   (cond
      ((=T A) B)
      ((=0 A) (or (=T B) 0))
      (T T) ) )

(de 3equiv (A B)
   (cond
      ((=T A) B)
      ((=0 A) 0)
      (T (3not B)) ) )

Test:

(for X '(T 0 NIL)
   (println 'not X '-> (3not X)) )

(for Fun '((and . 3and) (or . 3or) (implies . 3impl) (equivalent . 3equiv))
   (for X '(T 0 NIL)
      (for Y '(T 0 NIL)
         (println X (car Fun) Y '-> ((cdr Fun) X Y)) ) ) )

Output:

not T -> NIL
not 0 -> 0
not NIL -> T
T and T -> T
T and 0 -> 0
T and NIL -> NIL
0 and T -> 0
0 and 0 -> 0
0 and NIL -> NIL
NIL and T -> NIL
NIL and 0 -> NIL
NIL and NIL -> NIL
T or T -> T
T or 0 -> T
T or NIL -> T
0 or T -> T
0 or 0 -> 0
0 or NIL -> 0
NIL or T -> T
NIL or 0 -> 0
NIL or NIL -> NIL
T implies T -> T
T implies 0 -> 0
T implies NIL -> NIL
0 implies T -> T
0 implies 0 -> 0
0 implies NIL -> 0
NIL implies T -> T
NIL implies 0 -> T
NIL implies NIL -> T
T equivalent T -> T
T equivalent 0 -> 0
T equivalent NIL -> NIL
0 equivalent T -> 0
0 equivalent 0 -> 0
0 equivalent NIL -> 0
NIL equivalent T -> NIL
NIL equivalent 0 -> 0
NIL equivalent NIL -> T

\end{verbatim}

\section*{Test a function}

\begin{verbatim}

The '[http://software-lab.de/doc/refT.html#test test]' function is
built into PicoLisp.

(de palindrome? (S)
   (= (setq S (chop S)) (reverse S)) )

(test T (palindrome? "racecar"))
(test NIL (palindrome? "ferrari"))

\end{verbatim}

\section*{Text processing/1}

\begin{verbatim}

{{trans|AWK}}

Put the following into an executable file "readings":

#!/usr/bin/picolisp /usr/lib/picolisp/lib.l

(let (NoData 0  NoDataMax -1  NoDataMaxline "!"  TotFile 0  NumFile 0)
   (let InFiles
      (glue ","
         (mapcar
            '((File)
               (in File
                  (while (split (line) "^I")
                     (let (Len (length @)  Date (car @)  TotLine 0  NumLine 0)
                        (for (L (cdr @)  L  (cddr L))
                           (if (> 1 (format (cadr L)))
                              (inc 'NoData)
                              (when (gt0 NoData)
                                 (when (= NoDataMax NoData)
                                    (setq NoDataMaxline (pack NoDataMaxline ", " Date)) )
                                 (when (> NoData NoDataMax)
                                    (setq NoDataMax NoData  NoDataMaxline Date) ) )
                              (zero NoData)
                              (inc 'TotLine (format (car L) 3))
                              (inc 'NumLine) ) )
                        (inc 'TotFile TotLine)
                        (inc 'NumFile NumLine)
                        (tab (-7 -12 -7 3 -9 3 -11 11 -11 11)
                           "Line:" Date
                           "Reject:" (- (/ (dec Len) 2) NumLine)
                           "  Accept:" NumLine
                           "  Line_tot:" (format TotLine 3)
                           "  Line_avg:"
                           (and (gt0 NumLine) (format (*/ TotLine @) 3)) ) ) ) )
               File )
            (argv) ) )
      (prinl)
      (prinl "File(s)  = " InFiles)
      (prinl "Total    = " (format TotFile 3))
      (prinl "Readings = " NumFile)
      (prinl "Average  = " (format (*/ TotFile NumFile) 3))
      (prinl)
      (prinl
         "Maximum run(s) of " NoDataMax
         " consecutive false readings ends at line starting with date(s): " NoDataMaxline ) ) )

(bye)

Then it can be called as

\$ ./readings readings.txt |tail
Line:  2004-12-29  Reject:  1  Accept: 23  Line_tot:     56.300  Line_avg:      2.448
Line:  2004-12-30  Reject:  1  Accept: 23  Line_tot:     65.300  Line_avg:      2.839
Line:  2004-12-31  Reject:  1  Accept: 23  Line_tot:     47.300  Line_avg:      2.057

File(s)  = readings.txt
Total    = 1358393.400
Readings = 129403
Average  = 10.497

Maximum run(s) of 589 consecutive false readings ends at line starting with date(s): 1993-03-05
\$

\end{verbatim}

\section*{Text processing/2}

\begin{verbatim}

Put the following into an executable file "checkReadings":

#!/usr/bin/picolisp /usr/lib/picolisp/lib.l

(load "@lib/misc.l")

(in (opt)
   (until (eof)
      (let Lst (split (line) "^I")
         (unless
            (and
               (= 49 (length Lst))     # Check total length
               (\$dat (car Lst) "-")    # Check for valid date
               (not
                  (find                # Check data format
                     '((L F)
                        (not
                           (if F                            # Alternating:
                              (format L 3)                  # Number
                              (>= 9 (format L) -9) ) ) )    # or flag
                     (cdr Lst)
                     '(T NIL .) ) ) )
            (prinl "Bad line format: " (glue " " Lst))
            (bye 1) ) ) ) )

(bye)

Then it can be called as

\$ ./checkReadings readings.txt

\end{verbatim}

\section*{Text processing/3}

\begin{verbatim}

{{trans|AWK}}

Put the following into an executable file "licenses":

#!/usr/bin/picolisp /usr/lib/picolisp/lib.l

(zero Count MaxCount)

(in (opt)
   (while (split (line) " ")
      (case (pack (cadr (setq Line @)))
         (IN
            (dec 'Count) )
         (OUT
            (let Time (cadddr Line)
               (cond
                  ((> (inc 'Count) MaxCount)
                     (setq MaxCount Count  MaxTimes Time) )
                  ((= Count MaxCount)
                     (setq MaxTimes (pack MaxTimes " and " Time)) ) ) ) ) ) ) )

(prinl "The biggest number of licenses is " MaxCount " at " MaxTimes " !")
(bye)

Then it can be called as

\$ ./licenses mlijobs.txt
The biggest number of licenses is 99 at 2008/10/03_08:39:34 and 2008/10/03_08:40:40 !

\end{verbatim}

\section*{Thiele's interpolation formula}

\begin{verbatim}

{{trans|C}}

(scl 17)
(load "@lib/math.l")

(setq
   *X-Table (range 0.0 1.55 0.05)
   *SinTable (mapcar sin *X-Table)
   *CosTable (mapcar cos *X-Table)
   *TanTable (mapcar tan *X-Table)
   *TrigRows (length *X-Table) )

(let N2 (>> 1 (* *TrigRows (dec *TrigRows)))
   (setq
      *InvSinTable (need N2)
      *InvCosTable (need N2)
      *InvTanTable (need N2) ) )

(de rho (Tbl Inv I N)
   (cond
      ((lt0 N) 0)
      ((=0 N) (get *X-Table I))
      (T
         (let Idx (+ I (>> 1 (* (- *TrigRows 1 N) (- *TrigRows N))))
            (or
               (get Inv Idx)
               (set (nth Inv Idx)  # only happens if value not computed yet
                  (+
                     (rho Tbl Inv (inc I) (- N 2))
                     (*/
                        (- (get Tbl I) (get Tbl (+ I N)))
                        1.0
                        (-
                           (rho Tbl Inv I (dec N))
                           (rho Tbl Inv (inc I) (dec N)) ) ) ) ) ) ) ) ) )

(de thiele (Tbl Inv X N)
   (if (> N *TrigRows)
      1.0
      (+
         (-
            (rho Tbl Inv 1 (dec N))
            (rho Tbl Inv 1 (- N 3)) )
         (*/
            (- X (get Tbl N))
            1.0
            (thiele Tbl Inv X (inc N)) ) ) ) )

(de iSin (X)
   (thiele *SinTable *InvSinTable X 1) )

(de iCos (X)
   (thiele *CosTable *InvCosTable X 1) )

(de iTan (X)
   (thiele *TanTable *InvTanTable 1.0 1) )

Test:

(prinl (round (* 6 (iSin 0.5)) 15))
(prinl (round (* 3 (iCos 0.5)) 15))
(prinl (round (* 4 (iTan 1.0)) 15))

Output:

3.141592653589793
3.141592653589793
3.141592653589793

\end{verbatim}

\section*{Three Dogs}

\begin{verbatim}

(let (dog "Benjamin"  Dog "Samba"  DOG "Bernie")
   (prinl "The three dogs are named " dog ", " Dog " and " DOG) )

Output:

The three dogs are named Benjamin, Samba and Bernie

\end{verbatim}

\section*{Tic-tac-toe}

\begin{verbatim}

This solution doesn't bother about the game logic, but simply uses
the alpha-beta-pruning 'game' function in the "simul" library.

{games/ttt.l}

Output:

: (main)
   +---+---+---+
 3 |   |   |   |
   +---+---+---+
 2 |   |   |   |
   +---+---+---+
 1 |   |   |   |
   +---+---+---+
     a   b   c

: (go a 1)
   +---+---+---+
 3 |   |   |   |
   +---+---+---+
 2 |   |   |   |
   +---+---+---+
 1 | T |   |   |
   +---+---+---+
     a   b   c
   +---+---+---+
 3 |   |   |   |
   +---+---+---+
 2 |   | 0 |   |
   +---+---+---+
 1 | T |   |   |
   +---+---+---+
     a   b   c

\end{verbatim}

\section*{Time a function}

\begin{verbatim}

There is a built-in function '[http://software-lab.de/doc/refB.html#bench bench]
for that. However, it measures wall-clock time, because for practical purposes
the real time needed by a task (including I/O and communication) is more meaning
There is another function, '[http://software-lab.de/doc/refT.html#tick tick]', w
also measures user time, and is used by the profiling tools.

: (bench (do 1000000 (* 3 4)))
0.080 sec
-> 12

\end{verbatim}

\section*{Tokenize a string}

\begin{verbatim}

{Arm}

\end{verbatim}

\section*{Top rank per group}

\begin{verbatim}

# Employee Name, ID, Salary, Department
(de *Employees
   ("Tyler Bennett" E10297 32000 D101)
   ("John Rappl" E21437 47000 D050)
   ("George Woltman" E00127 53500 D101)
   ("Adam Smith" E63535 18000 D202)
   ("Claire Buckman" E39876 27800 D202)
   ("David McClellan" E04242 41500 D101)
   ("Rich Holcomb" E01234 49500 D202)
   ("Nathan Adams" E41298 21900 D050)
   ("Richard Potter" E43128 15900 D101)
   ("David Motsinger" E27002 19250 D202)
   ("Tim Sampair" E03033 27000 D101)
   ("Kim Arlich" E10001 57000 D190)
   ("Timothy Grove" E16398 29900 D190) )

(de topEmployees (N)
   (let Fmt (4 -16 -7 7)
      (for Dept (by cadddr group *Employees)
         (prinl "Department " (cadddr (car Dept)) ":")
         (tab Fmt NIL "Name" "ID" "Salary")
         (for (I . D) (flip (by caddr sort Dept))
            (tab Fmt (pack I ". ") (car D) (cadr D) (caddr D))
            (T (= I N)) )
         (prinl) ) ) )

(topEmployees 3)

Output:

Department D101:
    Name            ID      Salary
 1. George Woltman  E00127   53500
 2. David McClellan E04242   41500
 3. Tyler Bennett   E10297   32000

Department D050:
    Name            ID      Salary
 1. John Rappl      E21437   47000
 2. Nathan Adams    E41298   21900

Department D202:
    Name            ID      Salary
 1. Rich Holcomb    E01234   49500
 2. Claire Buckman  E39876   27800
 3. David Motsinger E27002   19250

Department D190:
    Name            ID      Salary
 1. Kim Arlich      E10001   57000
 2. Timothy Grove   E16398   29900

\end{verbatim}

\section*{Topological sort}

\begin{verbatim}

(de sortDependencies (Lst)
   (setq Lst                              # Build a flat list
      (uniq
         (mapcan
            '((L)
               (put (car L) 'dep (cdr L)) # Store dependencies in 'dep' properties
               (copy L) )
            (mapcar uniq Lst) ) ) )       # without self-dependencies
   (make
      (while Lst
         (ifn (find '((This) (not (: dep))) Lst)   # Found non-depending lib?
            (quit "Can't resolve dependencies" Lst)
            (del (link @) 'Lst)                    # Yes: Store in result
            (for This Lst                          # and remove from 'dep's
               (=: dep (delete @ (: dep))) ) ) ) ) )

Output:

: (sortDependencies
   (quote
      (des-system-lib   std synopsys std-cell-lib des-system-lib dw02 dw01 ramlib ieee)
      (dw01             ieee dw01 dware gtech)
      (dw02             ieee dw02 dware)
      (dw03             std synopsys dware dw03 dw02 dw01 ieee gtech)
      (dw04             dw04 ieee dw01 dware gtech)
      (dw05             dw05 ieee dware)
      (dw06             dw06 ieee dware)
      (dw07             ieee dware)
      (dware            ieee dware)
      (gtech            ieee gtech)
      (ramlib           std ieee)
      (std-cell-lib     ieee std-cell-lib)
      (synopsys) ) )
-> (std synopsys ieee std-cell-lib ramlib dware dw02 gtech dw01 des-system-lib dw03 dw04 dw05 dw06 dw07)

\end{verbatim}

\section*{Towers of Hanoi}

\begin{verbatim}

(de move (N A B C)  # Use: (move 3 'left 'center 'right)
   (unless (=0 N)
      (move (dec N) A C B)
      (println 'Move 'disk 'from A 'to B)
      (move (dec N) C B A) ) )

\end{verbatim}

\section*{Trabb Pardo–Knuth algorithm [Trabb_Pardo\%E2\%80\%93Knuth_algorithm]}

\begin{verbatim}

(de f (X)
   (+ (sqrt (abs X)) (* 5 X X X)) )

(trace 'f)

(in NIL
   (prin "Input 11 numbers: ")
   (for X (reverse (make (do 11 (link (read)))))
      (when (> (f X) 400)
         (prinl "TOO LARGE") ) ) )

Test:

Input 11 numbers: 1 2 3 4 5 6 7 8 9 10 11
 f : 11
 f = 6658
TOO LARGE
 f : 10
 f = 5003
TOO LARGE
 f : 9
 f = 3648
TOO LARGE
 f : 8
 f = 2562
TOO LARGE
 f : 7
 f = 1717
TOO LARGE
 f : 6
 f = 1082
TOO LARGE
 f : 5
 f = 627
TOO LARGE
 f : 4
 f = 322
 f : 3
 f = 136
 f : 2
 f = 41
 f : 1
 f = 6

\end{verbatim}

\section*{Tree traversal}

\begin{verbatim}

(de preorder (Node Fun)
   (when Node
      (Fun (car Node))
      (preorder (cadr Node) Fun)
      (preorder (caddr Node) Fun) ) )

(de inorder (Node Fun)
   (when Node
      (inorder (cadr Node) Fun)
      (Fun (car Node))
      (inorder (caddr Node) Fun) ) )

(de postorder (Node Fun)
   (when Node
      (postorder (cadr Node) Fun)
      (postorder (caddr Node) Fun)
      (Fun (car Node)) ) )

(de level-order (Node Fun)
   (for (Q (circ Node)  Q)
      (let N (fifo 'Q)
         (Fun (car N))
         (and (cadr N) (fifo 'Q @))
         (and (caddr N) (fifo 'Q @)) ) ) )

(setq *Tree
   (1
      (2 (4 (7)) (5))
      (3 (6 (8) (9))) ) )

(for Order '(preorder inorder postorder level-order)
   (prin (align -13 (pack Order ":")))
   (Order *Tree printsp)
   (prinl) )

Output:

preorder:    1 2 4 7 5 3 6 8 9
inorder:     7 4 2 5 1 8 6 9 3
postorder:   7 4 5 2 8 9 6 3 1
level-order: 1 2 3 4 5 6 7 8 9

\end{verbatim}

\section*{Trigonometric functions}

\begin{verbatim}

(load "@lib/math.l")

(de dtor (Deg)
   (*/ Deg pi 180.0) )

(de rtod (Rad)
   (*/ Rad 180.0 pi) )

(prinl
   (format (sin (/ pi 4)) *Scl) " " (format (sin (dtor 45.0)) *Scl) )
(prinl
   (format (cos (/ pi 4)) *Scl) " " (format (cos (dtor 45.0)) *Scl) )
(prinl
   (format (tan (/ pi 4)) *Scl) " " (format (tan (dtor 45.0)) *Scl) )
(prinl
   (format (asin (sin (/ pi 4))) *Scl) " " (format (rtod (asin (sin (dtor 45.0)))) *Scl) )
(prinl
   (format (acos (cos (/ pi 4))) *Scl) " " (format (rtod (acos (cos (dtor 45.0)))) *Scl) )
(prinl
   (format (atan (tan (/ pi 4))) *Scl) " " (format (rtod (atan (tan (dtor 45.0)))) *Scl) )

Output:

0.707107 0.707107
0.707107 0.707107
1.000000 1.000000
0.785398 44.999986
0.785398 44.999986
0.785398 44.999986

\end{verbatim}

\section*{Truncatable primes (!)}

\begin{verbatim}

(load "@lib/rsa.l")  # Use the 'prime?' function from RSA package

(de truncatablePrime? (N Fun)
   (for (L (chop N) L (Fun L))
      (T (= "0" (car L)))
      (NIL (prime? (format L)))
      T ) )

(let (Left 1000000  Right 1000000)
   (until (truncatablePrime? (dec 'Left) cdr))
   (until (truncatablePrime? (dec 'Right) '((L) (cdr (rot L)))))
   (cons Left Right) )

Output:

-> (998443 . 739399)

\end{verbatim}

\section*{Truncate a file}

\begin{verbatim}

On the 64-bit version, we can call the native runtime library:

(de truncate (File Len)
   (native "@" "truncate" 'I File Len) )

Otherwise (on all versions), we call the external truncate command:

(de truncate (File Len)
   (call "truncate" "-s" Len File) )

\end{verbatim}

\section*{Truth table}

\begin{verbatim}

(de truthTable (Expr)
   (let Vars
      (uniq
         (make
            (setq Expr
               (recur (Expr)  # Convert infix to prefix notation
                  (cond
                     ((atom Expr) (link Expr))
                     ((== 'not (car Expr))
                        (list 'not (recurse (cadr Expr))) )
                     (T
                        (list
                           (cadr Expr)
                           (recurse (car Expr))
                           (recurse (caddr Expr)) ) ) ) ) ) ) )
      (for V Vars
         (prin (align -7 V)) )
      (prinl)
      (bind (mapcar cons Vars)
         (do (** 2 (length Vars))
            (for "V" Vars
               (space (if (print (val "V")) 6 4)) )
            (println (eval Expr))
            (find '(("V") (set "V" (not (val "V")))) Vars) ) ) ) )

Test:

: (truthTable (str "A and (B or C)"))
A      B      C
NIL    NIL    NIL    NIL
T      NIL    NIL    NIL
NIL    T      NIL    NIL
T      T      NIL    T
NIL    NIL    T      NIL
T      NIL    T      T
NIL    T      T      NIL
T      T      T      T

: (truthTable (str "not (Foo and (Bar or Mumble))"))
Foo    Bar    Mumble
NIL    NIL    NIL    T
T      NIL    NIL    T
NIL    T      NIL    T
T      T      NIL    NIL
NIL    NIL    T      T
T      NIL    T      NIL
NIL    T      T      T
T      T      T      NIL

: (truthTable (str "(A xor B) and (B or C)"))
A      B      C
NIL    NIL    NIL    NIL
T      NIL    NIL    NIL
NIL    T      NIL    T
T      T      NIL    NIL
NIL    NIL    T      NIL
T      NIL    T      T
NIL    T      T      T
T      T      T      NIL

: (truthTable (str "(A xor B) and ((not B) or C)"))
A      B      C
NIL    NIL    NIL    NIL
T      NIL    NIL    T
NIL    T      NIL    NIL
T      T      NIL    NIL
NIL    NIL    T      NIL
T      NIL    T      T
NIL    T      T      T
T      T      T      NIL

\end{verbatim}

\chapter{Rosetta Code Tasks starting with U}

\section*{URL decoding}

\begin{verbatim}

: (ht:Pack (chop "http\%3A\%2F\%2Ffoo\%20bar\%2F"))
-> "http://foo bar/"

\end{verbatim}

\section*{URL encoding}

\begin{verbatim}

(de urlEncodeTooMuch (Str)
   (pack
      (mapcar
         '((C)
            (if (or (>= "9" C "0") (>= "Z" (uppc C) "A"))
               C
               (list '\% (hex (char C))) ) )
         (chop Str) ) ) )

Test:

: (urlEncodeTooMuch "http://foo bar/")
-> "http\%3A\%2F\%2Ffoo\%20bar\%2F"

\end{verbatim}

\section*{Unbias a random generator}

\begin{verbatim}

(de randN (N)
   (if (= 1 (rand 1 N)) 1 0) )

(de unbiased (N)
   (use (A B)
      (while
         (=
            (setq A (randN N))
            (setq B (randN N)) ) )
      A ) )

(for N (range 3 6)
   (tab (2 1 7 2 7 2)
      N ":"
      (format
         (let S 0 (do 10000 (inc 'S (randN N))))
         2 )
      "\%"
      (format
         (let S 0 (do 10000 (inc 'S (unbiased N))))
         2 )
      "\%" ) )

Output:

 3:  33.21 \%  50.48 \%
 4:  25.06 \%  49.79 \%
 5:  20.04 \%  49.75 \%
 6:  16.32 \%  49.02 \%

\end{verbatim}

\section*{Undefined values}

\begin{verbatim}

An internal symbol is initialized to NIL. Depending on the context, this is
interpreted as "undefined". When called as a function, an error is issued:

: (myfoo 3 4)
!? (myfoo 3 4)
myfoo -- Undefined
?

The function 'default' can be used to initialize a variable if and only if its
current value is NIL:

: MyVar
-> NIL

: (default MyVar 7)
-> 7

: MyVar
-> 7

: (default MyVar 8)
-> 7

: MyVar
-> 7

\end{verbatim}

\section*{Undefined values/Check if a variable is defined}

\begin{verbatim}

New internal symbols are initialized with the value NIL. NIL is also the value
for "false", so there is never really an "undefined value".
'[http://software-lab.de/doc/refN.html#not not]' is the predicate to check for
NIL, but many other (typically flow control) functions can be used.

(if (not MyNewVariable)
   (handle value-is-NIL) )

or

(unless MyNewVariable
   (handle value-is-NIL) )

\end{verbatim}

\section*{Unicode strings}

\begin{verbatim}

PicoLisp can directly handle _only_ Unicode (UTF-8) strings. So the problem is
rather how to handle non-Unicode strings: They must be pre- or post-processed by
external tools, typically with pipes during I/O. For example, to read a line
from a file in 8859 encoding:

   (in '(iconv "-f" "ISO-8859-15" "file.txt") (line))

\end{verbatim}

\section*{Unicode variable names}

\begin{verbatim}

Variables are usually [http://software-lab.de/doc/ref.html#internal-io Internal
Symbols], and their names may contain any UTF-8 character except null-bytes.
White space, and 11 special characters (see the reference) must be escaped with
a backslash. [http://software-lab.de/doc/ref.html#transient-io Transient
Symbols] are often used as variables too, they follow the syntax of strings in
other languages.

: (setq Δ 1)
-> 1
: Δ
-> 1
: (inc 'Δ)
-> 2
: Δ
-> 2

\end{verbatim}

\section*{Update a configuration file}

\begin{verbatim}

(let Data  # Read all data
   (in "config"
      (make
         (until (eof)
            (link (trim (split (line) " "))) ) ) )
   (setq Data  # Fix comments
      (mapcar
         '((L)
            (while (head '(";" ";") (car L))
               (pop L) )
            (if (= '(";") (car L))
               L
               (cons NIL L) ) )
         Data ) )
   (let (Need NIL  Seed NIL  NBan NIL  NStr NIL  Favo NIL)
      (map
         '((L)
            (let D (mapcar uppc (cadar L))
               (cond
                  ((= '`(chop "NEEDSPEELING") D)
                     (if Need
                        (set L)
                        (on Need)
                        (unless (caar L)
                           (set (car L) '(";")) ) ) )
                  ((= '`(chop "SEEDSREMOVED") D)
                     (if Seed
                        (set L)
                        (on Seed)
                        (when (caar L)
                           (set (car L)) ) ) )
                  ((= '`(chop "NUMBEROFBANANAS") D)
                     (if NBan
                        (set L)
                        (on NBan)
                        (set (cddar L) 1024) ) )
                  ((= '`(chop "NUMBEROFSTRAWBERRIES") D)
                     (if NStr
                        (set L)
                        (on NStr) ) )
                  ((= '`(chop "FAVOURITEFRUIT") D)
                     (if Favo
                        (set L)
                        (on Favo) ) ) ) ) )
         Data )
      (unless Need
         (conc Data (cons (list NIL "NEEDSPEELING"))) )
      (unless Seed
         (conc Data (cons (list NIL "SEEDSREMOVED"))) )
      (unless NBan
         (conc Data (cons (list NIL "NUMBEROFBANANAS" 1024))) )
      (unless NStr
         (conc Data (cons (list NIL "NUMBEROFSTRAWBERRIES" 62000))) ) )
   (out "config"
      (for L Data
         (prinl (glue " " (if (car L) L (cdr L)))) ) ) )

\end{verbatim}

\section*{User input/Graphical}

\begin{verbatim}

(and
   (call 'sh "-c"
      (pack
         "dialog \
            --inputbox 'Input a string' 8 60 \
            --inputbox 'Input a number' 8 20 \
            2>"
         (tmp "dlg") ) )
   (split (in (tmp "dlg") (line)) "^I")
   (cons (pack (car @)) (format (cadr @))) )

Output:

-> ("Hello world" . 12345)

\end{verbatim}

\section*{User input/Text}

\begin{verbatim}

(in NIL  # Guarantee reading from standard input
   (let (Str (read)  Num (read))
      (prinl "The string is: \"" Str "\"")
      (prinl "The number is: " Num) ) )

\end{verbatim}

\chapter{Rosetta Code Tasks starting with V}

\section*{Van der Corput sequence}

\begin{verbatim}

(scl 6)

(de vdc (N B)
   (default B 2)
   (let (R 0  A 1.0)
      (until (=0 N)
         (inc 'R (* (setq A (/ A B)) (\% N B)))
         (setq N (/ N B)) )
      R ) )

(for B (2 3 4)
   (prinl "Base: " B)
   (for N (range 0 9)
      (prinl N ": " (round (vdc N B) 4)) ) )

Output:

Base: 2
0: 0.0000
1: 0.5000
2: 0.2500
3: 0.7500
4: 0.1250
5: 0.6250
6: 0.3750
7: 0.8750
8: 0.0625
9: 0.5625
Base: 3
0: 0.0000
1: 0.3333
2: 0.6667
3: 0.1111
4: 0.4444
5: 0.7778
6: 0.2222
7: 0.5556
8: 0.8889
9: 0.0370
Base: 4
0: 0.0000
1: 0.2500
2: 0.5000
3: 0.7500
4: 0.0625
5: 0.3125
6: 0.5625
7: 0.8125
8: 0.1250
9: 0.3750

\end{verbatim}

\section*{Variable size/Get}

\begin{verbatim}

In PicoLisp, all variables have the same size (a single cell). Therefore it
makes more sense to inspect the size of data structures. This can be done with
the '[http://software-lab.de/doc/refS.html#size size]' and
'[http://software-lab.de/doc/refL.html#length length]' functions.

\end{verbatim}

\section*{Variable size/Set}

\begin{verbatim}

In PicoLisp, all variables have the same size (a single cell). But it is
possible to create a data structure of a given minimal size with the
'[http://software-lab.de/doc/refN.html#need need]' function.

\end{verbatim}

\section*{Variable-length quantity}

\begin{verbatim}

(de numToVlq (Num)
   (let Res (cons (\& Num 127))
      (while (gt0 (setq Num (>> 7 Num)))
         (push 'Res (| 128 (\& Num 127))) )
      Res ) )

(de vlqToNum (Vlq)
   (let Res 0
      (for N Vlq
         (setq Res (| (>> -7 Res) (\& N 127))) ) ) )

(for Num (0 15 16 127 128 255 2097151 2097152)
   (let Vlq (numToVlq Num)
      (tab (12 12 12) Num (glue ":" (mapcar hex Vlq)) (vlqToNum Vlq)) ) )

Output:

           0           0           0
          15           F          15
          16          10          16
         127          7F         127
         128        81:0         128
         255       81:7F         255
     2097151    FF:FF:7F     2097151
     2097152  81:80:80:0     2097152

\end{verbatim}

\section*{Variables}

\begin{verbatim}

You can control the local bindings of symbols with functions like
'[http://software-lab.de/doc/refU.html#use use]' or
'[http://software-lab.de/doc/refL.html#let let]':

(use (A B C)
   (setq A 1  B 2  C 3)
   ... )

This is equivalent to

(let (A 1  B 2  C 3)
   ... )

The parentheses can be omitted if there is only a single variable

(use A
   (setq A ..)
   ... )

(let A 1
   ...)

Other functions that handle local bindings are

'[http://software-lab.de/doc/refL.html#let? let?]',
'[http://software-lab.de/doc/refB.html#bind bind]',
'[http://software-lab.de/doc/refJ.html#job job]',
'[http://software-lab.de/doc/refW.html#with with]' or
'[http://software-lab.de/doc/refF.html#for for]'.

\end{verbatim}

\section*{Variadic function}

\begin{verbatim}

The '@' operator causes a function to accept a variable number of arguments.
These can be accesed with the
'[http://software-lab.de/doc/refA.html#args args]',
'[http://software-lab.de/doc/refN.html#next next]',
'[http://software-lab.de/doc/refA.html#arg arg]' and
'[http://software-lab.de/doc/refR.html#rest rest]' functions.

(de varargs @
   (while (args)
      (println (next)) ) )

The '@' operator may be used in combination with normal parameters:

(de varargs (Arg1 Arg2 . @)
   (println Arg1)
   (println Arg2)
   (while (args)
      (println (next)) ) )

It is called like any other function

(varargs 'a 123 '(d e f) "hello")

also by possibly applying it to a ready-made list

(apply varargs '(a 123 (d e f) "hello"))

Output in all cases:
a
123
(d e f)
"hello"

\end{verbatim}

\section*{Vector products}

\begin{verbatim}

(de dotProduct (A B)
   (sum * A B) )

(de crossProduct (A B)
   (list
      (- (* (cadr A) (caddr B)) (* (caddr A) (cadr B)))
      (- (* (caddr A) (car B)) (* (car A) (caddr B)))
      (- (* (car A) (cadr B)) (* (cadr A) (car B))) ) )

(de scalarTriple (A B C)
   (dotProduct A (crossProduct B C)) )

(de vectorTriple (A B C)
   (crossProduct A (crossProduct B C)) )

Test:

(setq
   A ( 3   4   5)
   B ( 4   3   5)
   C (-5 -12 -13) )

: (dotProduct A B)
-> 49

: (crossProduct A B)
-> (5 5 -7)

: (scalarTriple A B C)
-> 6

: (vectorTriple A B C)
-> (-267 204 -3)

\end{verbatim}

\section*{Verify distribution uniformity/Naive}

\begin{verbatim}

The following function takes a count, and allowed deviation in per mill
(one-tenth of a percent), and a 'prg' code body (i.e. an arbitrary number of
executable expressions).

(de checkDistribution (Cnt Pm . Prg)
   (let Res NIL
      (do Cnt (accu 'Res (run Prg 1) 1))
      (let
         (N (/ Cnt (length Res))
            Min (*/ N (- 1000 Pm) 1000)
            Max (*/ N (+ 1000 Pm) 1000) )
         (for R Res
            (prinl (cdr R) " " (if (>= Max (cdr R) Min) "Good" "Bad")) ) ) ) )

Output:

: (checkDistribution 100000 5 (rand 1 7))
14299 Good
14394 Bad
14147 Bad
14418 Bad
14159 Bad
14271 Good
14312 Good

\end{verbatim}

\section*{Vigenère Cipher}

\begin{verbatim}

(de vigenereKey (Str)
   (extract
      '((C)
         (when (>= "Z" (uppc C) "A")
            (- (char (uppc C)) 65) ) )
      (chop Str) ) )

(de vigenereEncrypt (Str Key)
   (pack
      (mapcar
         '((C K)
            (char (+ 65 (\% (+ C K) 26))) )
         (vigenereKey Str)
         (apply circ (vigenereKey Key)) ) ) )

(de vigenereDecrypt (Str Key)
   (pack
      (mapcar
         '((C K)
            (char (+ 65 (\% (+ 26 (- C K)) 26))) )
         (vigenereKey Str)
         (apply circ (vigenereKey Key)) ) ) )

Test:

: (vigenereEncrypt
   "Beware the Jabberwock, my son! The jaws that bite, the claws that catch!"
   "VIGENERECIPHER" )
-> "WMCEEIKLGRPIFVMEUGXQPWQVIOIAVEYXUEKFKBTALVXTGAFXYEVKPAGY"

: (vigenereDecrypt @ "VIGENERECIPHER")
-> "BEWARETHEJABBERWOCKMYSONTHEJAWSTHATBITETHECLAWSTHATCATCH"

\end{verbatim}

\chapter{Rosetta Code Tasks starting with W}

\section*{Walk a directory/Non-recursively}

\begin{verbatim}

(for F (dir "@src/")                         # Iterate directory
   (when (match '`(chop "s@.c") (chop F))    # Matches 's*.c'?
      (println F) ) )                        # Yes: Print it

Output:

"start.c"
"ssl.c"
"subr.c"
"sym.c"
...

\end{verbatim}

\section*{Walk a directory/Recursively}

\begin{verbatim}

(let Dir "."
   (recur (Dir)
      (for F (dir Dir)
         (let Path (pack Dir "/" F)
            (cond
               ((=T (car (info Path)))             # Is a subdirectory?
                  (recurse Path) )                 # Yes: Recurse
               ((match '`(chop "s@.l") (chop F))   # Matches 's*.l'?
                  (println Path) ) ) ) ) ) )       # Yes: Print it

Output:

"./src64/sym.l"
"./src64/subr.l"
...

\end{verbatim}

\section*{Web scraping}

\begin{verbatim}

(load "@lib/http.l")

(client "tycho.usno.navy.mil" 80 "cgi-bin/timer.pl"
   (when (from "<BR>")
      (pack (trim (till "U"))) ) )

Output:

-> "Feb. 19, 18:11:37"

\end{verbatim}

\section*{Window creation}

\begin{verbatim}

{{trans|C}}

(load "@lib/openGl.l")

(glutInit)
(glutCreateWindow "Goodbye, World!")
(keyboardFunc '(() (bye)))
(glutMainLoop)

\end{verbatim}

\section*{Window creation/X11}

\begin{verbatim}

The following script works in the 32-bit version, using inlined C code

#!/usr/bin/picolisp /usr/lib/picolisp/lib.l

(load "@lib/misc.l" "@lib/gcc.l")

(gcc "x11" '("-lX11") 'simpleWin)

#include <X11/Xlib.h>

any simpleWin(any ex) {
   any x = cdr(ex);
   int dx, dy;
   Display *disp;
   int scrn;
   Window win;
   XEvent ev;

   x = cdr(ex),  dx = (int)evCnt(ex,x);
   x = cdr(x),  dy = (int)evCnt(ex,x);
   x = evSym(cdr(x));
   if (disp = XOpenDisplay(NULL)) {
      char msg[bufSize(x)];

      bufString(x, msg);
      scrn = DefaultScreen(disp);
      win = XCreateSimpleWindow(disp, RootWindow(disp,scrn), 0, 0, dx, dy,
                           1, BlackPixel(disp,scrn), WhitePixel(disp,scrn) );
      XSelectInput(disp, win, ExposureMask | KeyPressMask | ButtonPressMask);
      XMapWindow(disp, win);
      for (;;) {
         XNextEvent(disp, \&ev);
         switch (ev.type) {
         case Expose:
            XDrawRectangle(disp, win, DefaultGC(disp, scrn), 10, 10, dx-20, dy-20);
            XDrawString(disp, win, DefaultGC(disp, scrn), 30, 40, msg, strlen(msg));
            break;
         case KeyPress:
         case ButtonPress:
            XCloseDisplay(disp);
            return Nil;
         }
      }
   }
   return mkStr("Can't open Display");
}
/**/

(simpleWin 300 200 "Hello World")
(bye)

\end{verbatim}

\section*{Window management}

\begin{verbatim}

The following works on ErsatzLisp, the Java version of PicoLisp.

\$ ersatz/pil +
: (setq
   JFrame         "javax.swing.JFrame"
   MAXIMIZED_BOTH (java (public JFrame 'MAXIMIZED_BOTH))
   ICONIFIED      (java (public JFrame 'ICONIFIED))
   Win            (java JFrame T "Window") )
-> \$ JFrame

# Compare for equality
: (== Win Win)
-> T

# Set window visible
(java Win 'setLocation 100 100)
(java Win 'setSize 400 300)
(java Win 'setVisible T)

# Hide window
(java Win 'hide)

# Show again
(java Win 'setVisible T)

# Move window
(java Win 'setLocation 200 200)

# Iconify window
(java Win 'setExtendedState
   (| (java (java Win 'getExtendedState)) ICONIFIED) )

# De-conify window
(java Win 'setExtendedState
   (\& (java (java Win 'getExtendedState)) (x| (hex "FFFFFFFF") ICONIFIED)) )

# Maximize window
(java Win 'setExtendedState
   (| (java (java Win 'getExtendedState)) MAXIMIZED_BOTH) )

# Close window
(java Win 'dispose)

\end{verbatim}

\section*{Wireworld}

\begin{verbatim}

This example uses 'grid' from "lib/simul.l", which maintains a two-dimensional
structure.

(load "@lib/simul.l")

(let
   (Data (in "wire.data" (make (while (line) (link @))))
      Grid (grid (length (car Data)) (length Data)) )
   (mapc
      '((G D) (mapc put G '(val .) D))
      Grid
      (apply mapcar (flip Data) list) )
   (loop
      (disp Grid T
         '((This) (pack " " (: val) " ")) )
      (wait 1000)
      (for Col Grid
         (for This Col
            (case (=: next (: val))
               ("H" (=: next "t"))
               ("t" (=: next "."))
               ("."
                  (when
                     (>=
                        2
                        (cnt # Count neighbors
                           '((Dir) (= "H" (get (Dir This) 'val)))
                           (quote
                              west east south north
                              ((X) (south (west X)))
                              ((X) (north (west X)))
                              ((X) (south (east X)))
                              ((X) (north (east X))) ) )
                        1 )
                     (=: next "H") ) ) ) ) )
      (for Col Grid  # Update
         (for This Col
            (=: val (: next)) ) )
      (prinl) ) )

Output:

   +---+---+---+---+---+---+---+---+---+---+---+
 5 | t | H | . | . | . | . | . | . | . | . | . |
   +---+---+---+---+---+---+---+---+---+---+---+
 4 | . |   |   |   | . |   |   |   |   |   |   |
   +---+---+---+---+---+---+---+---+---+---+---+
 3 |   |   |   | . | . | . |   |   |   |   |   |
   +---+---+---+---+---+---+---+---+---+---+---+
 2 | . |   |   |   | . |   |   |   |   |   |   |
   +---+---+---+---+---+---+---+---+---+---+---+
 1 | H | t | . | . |   | . | . | . | . | . | . |
   +---+---+---+---+---+---+---+---+---+---+---+
     a   b   c   d   e   f   g   h   i   j   k

   +---+---+---+---+---+---+---+---+---+---+---+
 5 | . | t | H | . | . | . | . | . | . | . | . |
   +---+---+---+---+---+---+---+---+---+---+---+
 4 | H |   |   |   | . |   |   |   |   |   |   |
   +---+---+---+---+---+---+---+---+---+---+---+
 3 |   |   |   | . | . | . |   |   |   |   |   |
   +---+---+---+---+---+---+---+---+---+---+---+
 2 | H |   |   |   | . |   |   |   |   |   |   |
   +---+---+---+---+---+---+---+---+---+---+---+
 1 | t | . | . | . |   | . | . | . | . | . | . |
   +---+---+---+---+---+---+---+---+---+---+---+
     a   b   c   d   e   f   g   h   i   j   k

   +---+---+---+---+---+---+---+---+---+---+---+
 5 | H | . | t | H | . | . | . | . | . | . | . |
   +---+---+---+---+---+---+---+---+---+---+---+
 4 | t |   |   |   | . |   |   |   |   |   |   |
   +---+---+---+---+---+---+---+---+---+---+---+
 3 |   |   |   | . | . | . |   |   |   |   |   |
   +---+---+---+---+---+---+---+---+---+---+---+
 2 | t |   |   |   | . |   |   |   |   |   |   |
   +---+---+---+---+---+---+---+---+---+---+---+
 1 | . | H | . | . |   | . | . | . | . | . | . |
   +---+---+---+---+---+---+---+---+---+---+---+
     a   b   c   d   e   f   g   h   i   j   k

\end{verbatim}

\section*{Word wrap}

\begin{verbatim}

'[http://software-lab.de/doc/refW.html#wrap wrap]' is a built-in.

: (prinl (wrap 12 (chop "The quick brown fox jumps over the lazy dog")))
The quick
brown fox
jumps over
the lazy dog
-> "The quick^Jbrown fox^Jjumps over^Jthe lazy dog"

\end{verbatim}

\section*{Write float arrays to a text file}

\begin{verbatim}

An exponential format like "1e11" is not supported

(setq *Xprecision 3  *Yprecision 5)

(scl 7)
(mapc
   '((X Y)
      (prinl
         (round X *Xprecision)
         "  "
         (round Y *Yprecision) ) )
   (1.0 2.0 3.0)
   (1.0 1.414213562 1.732050807) )

Output:

1.000  1.00000
2.000  1.41421
3.000  1.73205

\end{verbatim}

\section*{Write to Windows event log}

\begin{verbatim}

PicoLisp doesn't run on Windows. In case of Linux, the equivalent of the event
log is the syslog. It can be written with
'[http://software-lab.de/doc/refN.html#native native]' C functions, or simply
with the 'logger' utility:

: (call 'logger "This is a test")
-> T

: (call 'logger "This" 'is "another" 'test)
-> T

\end{verbatim}

\chapter{Rosetta Code Tasks starting with X}

\section*{XML/DOM serialization}

\begin{verbatim}

(load "@lib/xm.l")

(xml? T)
(xml '(root NIL (element NIL "Some text here")))

Output:

<?xml version="1.0" encoding="utf-8"?>
<root>
   <element>Some text here</element>
</root>

\end{verbatim}

\section*{XML/Input}

\begin{verbatim}

(load "@lib/xm.l")

(mapcar
   '((L) (attr L 'Name))
   (body (in "file.xml" (xml))) )

Output:

-> ("April" "Bob" "Chad" "Dave" "Émily")

\end{verbatim}

\section*{XML/Output}

\begin{verbatim}

(load "@lib/xm.l")

(de characterRemarks (Names Remarks)
   (xml
      (cons
         'CharacterRemarks
         NIL
         (mapcar
            '((Name Remark)
               (list 'Character (list (cons 'name Name)) Remark) )
            Names
            Remarks ) ) ) )

(characterRemarks
   '("April" "Tam O'Shanter" "Emily")
   (quote
      "I'm > Tam and <= Emily"
      "Burns: \"When chapman billies leave the street ..."
      "Short \& shrift" ) )

Output:

<CharacterRemarks>
   <Character name="April">I'm > Tam and \&#60;= Emily</Character>
   <Character name="Tam O'Shanter">Burns: \&#34;When chapman billies leave the street ...</Character>
   <Character name="Emily">Short \&#38; shrift</Character>
</CharacterRemarks>

\end{verbatim}

\section*{XML/XPath}

\begin{verbatim}

(load "@lib/xm.l")

(let Sections (body (in "file.xml" (xml)))
   (pretty (car (body (car Sections))))
   (prinl)
   (for S Sections
      (for L (body S)
         (prinl (car (body L 'price))) ) )
   (make
      (for S Sections
         (for L (body S)
            (link (car (body L 'name))) ) ) ) )

Output:

(item
   ((upc . "123456789") (stock . "12"))
   (name NIL "Invisibility Cream")
   (price NIL "14.50")
   (description NIL "Makes you invisible") )
14.50
23.99
4.95
3.56
-> ("Invisibility Cream" "Levitation Salve" "Blork and Freen Instameal" "Grob winglets")

\end{verbatim}

\section*{Xiaolin Wu's line algorithm}

\begin{verbatim}

(scl 2)

(de plot (Img X Y C)
   (set (nth Img (*/ Y 1.0) (*/ X 1.0)) (- 100 C)) )

(de ipart (X)
   (* 1.0 (/ X 1.0)) )

(de iround (X)
   (ipart (+ X 0.5)) )

(de fpart (X)
   (\% X 1.0) )

(de rfpart (X)
   (- 1.0 (fpart X)) )

(de xiaolin (Img X1 Y1 X2 Y2)
   (let (DX (- X2 X1)  DY (- Y2 Y1))
      (use (Grad Xend Yend Xgap Xpxl1 Ypxl1 Xpxl2 Ypxl2 Intery)
         (when (> (abs DY) (abs DX))
            (xchg 'X1 'Y1  'X2 'Y2) )
         (when (> X1 X2)
            (xchg 'X1 'X2  'Y1 'Y2) )
         (setq
            Grad (*/ DY 1.0 DX)
            Xend (iround X1)
            Yend (+ Y1 (*/ Grad (- Xend X1) 1.0))
            Xgap (rfpart (+ X1 0.5))
            Xpxl1 Xend
            Ypxl1 (ipart Yend) )
         (plot Img Xpxl1 Ypxl1 (*/ (rfpart Yend) Xgap 1.0))
         (plot Img Xpxl1 (+ 1.0 Ypxl1) (*/ (fpart Yend) Xgap 1.0))
         (setq
            Intery (+ Yend Grad)
            Xend (iround X2)
            Yend (+ Y2 (*/ Grad (- Xend X2) 1.0))
            Xgap (fpart (+ X2 0.5))
            Xpxl2 Xend
            Ypxl2 (ipart Yend) )
         (plot Img Xpxl2 Ypxl2 (*/ (rfpart Yend) Xgap 1.0))
         (plot Img Xpxl2 (+ 1.0 Ypxl2) (*/ (fpart Yend) Xgap 1.0))
         (for (X (+ Xpxl1 1.0)  (>= (- Xpxl2 1.0) X)  (+ X 1.0))
            (plot Img X (ipart Intery) (rfpart Intery))
            (plot Img X (+ 1.0 (ipart Intery)) (fpart Intery))
            (inc 'Intery Grad) ) ) ) )

(let Img (make (do 90 (link (need 120 99))))       # Create image 120 x 90
   (xiaolin Img 10.0 10.0 110.0 80.0)              # Draw lines
   (xiaolin Img 10.0 10.0 110.0 45.0)
   (xiaolin Img 10.0 80.0 110.0 45.0)
   (xiaolin Img 10.0 80.0 110.0 10.0)
   (out "img.pgm"                                  # Write to bitmap file
      (prinl "P2")
      (prinl 120 " " 90)
      (prinl 100)
      (for Y Img (apply printsp Y)) ) )

\end{verbatim}

\chapter{Rosetta Code Tasks starting with Y}

\section*{Y combinator (!)}

\begin{verbatim}

{{trans|Common Lisp}}

(de Y (F)
   (let X (curry (F) (Y) (F (curry (Y) @ (pass (Y Y)))))
      (X X) ) )

# Factorial
(de fact (F)
   (curry (F) (N)
      (if (=0 N)
         1
         (* N (F (dec N))) ) ) )

: ((Y fact) 6)
-> 720

# Fibonacci
(de fibo (F)
   (curry (F) (N)
      (if (> 2 N)
         1
         (+ (F (dec N)) (F (- N 2))) ) ) )

: ((Y fibo) 22)
-> 28657

# Ackermann
(de ack (F)
   (curry (F) (X Y)
      (cond
         ((=0 X) (inc Y))
         ((=0 Y) (F (dec X) 1))
         (T (F (dec X) (F X (dec Y)))) ) ) )

: ((Y ack) 3 4)
-> 125

\end{verbatim}

\section*{Yahoo! Search}

\begin{verbatim}

(load "@lib/http.l")

(de yahoo (Query Page)
   (default Page 1)
   (client "search.yahoo.com" 80
      (pack
         "search?p=" (ht:Fmt Query)
         "\&b=" (inc (* 10 (dec Page))) )
      (make
         (while (from "<a class=\"yschttl spt\" href=\"")
            (link
               (make
                  (link (till "\"" T))       # Url
                  (from "<b>")
                  (link (till "<" T))        # Title
                  (from "class=\"abstr\"")
                  (from ">")
                  (link                      # Content
                     (pack
                        (make
                           (loop
                              (link (till "<" T))
                              (T (eof))
                              (T (= "</div" (till ">" T)))
                              (char) ) ) ) ) ) ) ) ) ) )

Output:

: (more (yahoo "test"))
("http://www.test.com/" "Test" "Offers practice online tests for many ...
("http://www.test.com/aboutus.htm" "Test" "Test.com has a successful ...
("http://en.wikipedia.org/wiki/Test" "Test" "YUI Test is a testing ...
("http://en.wikipedia.org/wiki/F-test" "test " "test n. A procedure for ...
...

\end{verbatim}

\section*{Yin and yang}

\begin{verbatim}

(de circle (X Y C R)
   (>=
      (* R R)
      (+
         (* (setq X (/ X 2)) X)
         (* (dec 'Y C) Y) ) ) )

(de yinYang (R)
   (for Y (range (- R) R)
      (for X (range (- 0 R R) (+ R R))
         (prin
            (cond
               ((circle X Y (- (/ R 2)) (/ R 6))
                  "#" )
               ((circle X Y (/ R 2) (/ R 6))
                  "." )
               ((circle X Y (- (/ R 2)) (/ R 2))
                  "." )
               ((circle X Y (/ R 2) (/ R 2))
                  "#" )
               ((circle X Y 0 R)
                  (if (lt0 X) "." "#") )
               (T " ") ) ) )
      (prinl) ) )

Test:

: (yinYang 18)
                                   ...
                         .....................##
                   .............................######
                 .................................######
             .......................................########
           ...........................................########
         ..........................###................##########
       ........................###########............############
       ........................###########............############
     ........................###############............############
   ............................###########............################
   ............................###########............################
   ................................###................################
 .....................................................##################
 ...................................................####################
 .................................................######################
 ...............................................########################
 .............................................##########################
......................................###################################
 ..........................#############################################
 ........................###############################################
 ......................#################################################
 ....................###################################################
 ..................#####################################################
   ................################...################################
   ................############...........############################
   ................############...........############################
     ............############...............########################
       ............############...........########################
       ............############...........########################
         ..........################...##########################
           ........###########################################
             ........#######################################
                 ......#################################
                   ......#############################
                         ..#####################
                                   ###

\end{verbatim}

\chapter{Rosetta Code Tasks starting with Z}

\section*{Zebra puzzle}

\begin{verbatim}

(be match (@House @Person @Drink @Pet @Cigarettes)
   (permute (red blue green yellow white) @House)
   (left-of @House white  @House green)

   (permute (Norwegian English Swede German Dane) @Person)
   (has @Person English  @House red)
   (equal @Person (Norwegian . @))
   (next-to @Person Norwegian  @House blue)

   (permute (tea coffee milk beer water) @Drink)
   (has @Drink tea  @Person Dane)
   (has @Drink coffee  @House green)
   (equal @Drink (@ @ milk . @))

   (permute (dog birds cats horse zebra) @Pet)
   (has @Pet dog  @Person Swede)

   (permute (Pall-Mall Dunhill Blend Blue-Master Prince) @Cigarettes)
   (has @Cigarettes Pall-Mall  @Pet birds)
   (has @Cigarettes Dunhill  @House yellow)
   (next-to @Cigarettes Blend  @Pet cats)
   (next-to @Cigarettes Dunhill  @Pet horse)
   (has @Cigarettes Blue-Master  @Drink beer)
   (has @Cigarettes Prince  @Person German)

   (next-to @Drink water  @Cigarettes Blend) )

(be has ((@A . @X) @A (@B . @Y) @B))
(be has ((@ . @X) @A (@ . @Y) @B)
   (has @X @A @Y @B) )

(be right-of ((@A . @X) @A (@ @B . @Y) @B))
(be right-of ((@ . @X) @A (@ . @Y) @B)
   (right-of @X @A @Y @B) )

(be left-of ((@ @A . @X) @A (@B . @Y) @B))
(be left-of ((@ . @X) @A (@ . @Y) @B)
   (left-of @X @A @Y @B) )

(be next-to (@X @A @Y @B) (right-of @X @A @Y @B))
(be next-to (@X @A @Y @B) (left-of @X @A @Y @B))

Test:

(pilog '((match @House @Person @Drink @Pet @Cigarettes))
   (let Fmt (-8 -11 -8 -7 -11)
      (tab Fmt "HOUSE" "PERSON" "DRINKS" "HAS" "SMOKES")
      (mapc '(@ (pass tab Fmt))
         @House @Person @Drink @Pet @Cigarettes ) ) )

Output:

HOUSE   PERSON     DRINKS  HAS    SMOKES
yellow  Norwegian  water   cats   Dunhill
blue    Dane       tea     horse  Blend
red     English    milk    birds  Pall-Mall
green   German     coffee  zebra  Prince
white   Swede      beer    dog    Blue-Master

\end{verbatim}

\section*{Zig-zag matrix}

\begin{verbatim}

This example uses 'grid' from "lib/simul.l", which maintains a two-dimensional
structure and is normally used for simulations and board games.

(load "@lib/simul.l")

(de zigzag (N)
   (prog1 (grid N N)
      (let (D '(north west  south east  .)  E '(north east .)  This 'a1)
         (for Val (* N N)
            (=: val Val)
            (setq This
               (or
                  ((cadr D) ((car D) This))
                  (prog
                     (setq D (cddr D))
                     ((pop 'E) This) )
                  ((pop 'E) This) ) ) ) ) ) )

(mapc
   '((L)
      (for This L (prin (align 3 (: val))))
      (prinl) )
   (zigzag 5) )

Output:

  1  2  6  7 15
  3  5  8 14 16
  4  9 13 17 22
 10 12 18 21 23
 11 19 20 24 25


\end{verbatim}

% Local variables:
% mode: latex
% TeX-master: "../../editor.tex"
% End:
