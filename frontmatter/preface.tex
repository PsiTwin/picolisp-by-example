%%%%%%%%%%%%%%%%%%%%%%preface.tex%%%%%%%%%%%%%%%%%%%%%%%%%%%%%%%%%%%%%%%%%
% sample preface
%
% Use this file as a template for your own input.
%
%%%%%%%%%%%%%%%%%%%%%%%% Springer %%%%%%%%%%%%%%%%%%%%%%%%%%

\preface

%% Please write your preface here
% Use the template \emph{preface.tex} together with the Springer document class SVMono (monograph-type books) or SVMult (edited books) to style your preface in the Springer layout.

% A preface\index{preface} is a book's preliminary statement, usually written by the \textit{author or editor} of a work, which states its origin, scope, purpose, plan, and intended audience, and which sometimes includes afterthoughts and acknowledgments of assistance. 

% When written by a person other than the author, it is called a foreword. The preface or foreword is distinct from the introduction, which deals with the subject of the work.

% Customarily \textit{acknowledgments} are included as last part of the preface.


\emph{Why PicoLisp?} Short answer: PicoLisp as a language is
maximizing expressive power while minimizing complexity

PicoLisp is a very simple and succinct, yet expressive language -- and
it is free (MIT/X11 License). Furthermore, PicoLisp has two
characteristic features which are not found to that extent in other
languages:

\begin{enumerate}
\item
  An integrated database
\item
  Equivalence of code and data
\end{enumerate}

These two features alone, and how they are used in combination, make
it worth to take a closer look at PicoLisp.

\begin{description}
\item[\emph{Integrated Database}]
  \href{http://software-lab.de/doc/ref.html\#dbase}{Database}
functionality is built into the core of the language. PicoLisp
\textbf{is} a database query and manipulation language.

Database entities are first class objects. They are called ``external
symbols'', because they are automatically fetched from database files
when accessed, but otherwise behave like normal symbols.

This fetching from external files is completely transparent, the
symbols ``are just there'', and there is no need (or even a function)
to read or write them explicitly.
\href{http://software-lab.de/doc/ref.html\#pilog}{Pilog} (a built-in
Prolog engine) is used as a query language.

It is possible with PicoLisp to build large multi-user databases,
distributed across many machines or in a cloud. Such a database system
can be optimally fine-tuned, because all its levels are under the
developer's control.

\item[\emph{Equivalence of Code and Data}]

  This is actually a feature of Lisp in general. However, PicoLisp
  really \emph{lives} it. It makes it easy to write things like the
  HTML, PostScript or TeX libraries, exploring a syntax of nested
  function calls. This results in very succinct and precisely
  expressed programs.

For a closer explanation, see the article
\href{!wiki?EquivalenceCodeData}{The Equivalence of Code and Data}.

\item[\emph{Expressiveness}] PicoLisp is a very expressive language. Programs
are often much shorter and concise than equivalent programs written in
other languages.

Examples of various programming tasks and their solutions, originally
published at
\href{http://rosettacode.org/wiki/Category:PicoLisp}{rosettacode.org},
can be found in this book.

\item[\emph{Efficiency}] 

PicoLisp uses (at least when compared to other
Lisps) very little memory, on disk as well as in memory (heap space).

For example, the installation size in the OpenWRT distribution is only
575 kB (uncompressed), where the statically linked interpreter with
296 kB takes the largest part. Yet, it includes the full runtime
system with interpreter, database, HTTP server, XHTML and JavaScript
application framework, watchdog, and the debugger, PostScript and XML
libraries.

PicoLisp has no compiler, everything starts up very quickly, and code
dynamically loaded at runtime (e.g. GUI pages) is immediately ready.
Yet, the interpreter is quite fast, usually three times a fast as
Python, for example. See also the article
\href{!wiki?NeedForSpeed}{Need For Speed}.
\end{description}


\vspace{1cm}
\begin{flushright}\noindent
Langweid, August 2012\hfill {\it Alexander Burger}\\
Berlin, August 2012\hfill {\it Thorsten Jolitz}\\\end{flushright}


% \vspace{\baselineskip}
% \begin{flushright}\noindent
% Langweid, September 2012 \hfill {\it Alexander Burger}
% \end{flushright}


